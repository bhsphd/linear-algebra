\documentclass[11pt]{article}
\usepackage[margin=1in]{geometry}
\usepackage{../linalgjh}

\setlength{\parindent}{0em}
\pagestyle{empty}
\begin{document}\thispagestyle{empty}
\makebox[\linewidth]{\textbf{Homework, MA~213}\hspace*{4in}\textbf{Due: 2014-Nov-17}}

\vspace*{3ex}
\textit{You may work with others to figure out how to do questions, 
and you are welcome to look for answers in the book, online, by talking
to someone who had the course before, etc.
However, you must write 
the answers on your own.
You must also show your work (you may, of course, 
quote any result from the book).}

\begin{enumerate}
\item
  Find the determinant.
  \begin{enumerate}
    \item 
      $\begin{vmat}
         1 &4 \\
         2 &8
       \end{vmat}$
    \item
      $
      \begin{vmat}
        2 &1 &1 \\
        1 &1 &0 \\
        6 &4 &1
      \end{vmat}$ 
  \end{enumerate}

\item
  Consider the linear transformation $\map{t}{\Re^3}{\Re^3}$
  represented with respect to the 
  standard bases by this matrix.
  \begin{equation*}
    \begin{mat}
      1 &0 &-1 \\
      3 &1 &1 \\
     -1 &0 &3
    \end{mat}
  \end{equation*}
  \begin{enumerate}
    \item Compute the determinant of the matrix.
    \item Find the size of the box defined by these vectors.
      \begin{equation*}
        \colvec{1 \\ -1 \\ 2}
        \quad
        \colvec{2 \\ 0 \\ -1}
        \quad
        \colvec{1 \\ 1 \\ 0}
      \end{equation*}
    Wht is its orientation?
  \item Find the image under $t$ of the vectors in the prior item and 
    find the size of the box that they define.
  \end{enumerate}

\item Consider this transformation of~$\Re^3$.
  \begin{equation*}
    t(\colvec{x \\ y \\ z})=\colvec{x-z \\ z \\ 2y}
  \end{equation*}
  Consider also these two bases.
  \begin{equation*}
    B=\sequence{\colvec{1 \\ 2 \\ 3}, 
                \colvec{0 \\ 1 \\ 0}, 
                \colvec{0 \\ 0 \\ 1}}
    \qquad
    D=\sequence{\colvec{1 \\ 0 \\ 0},
                \colvec{1 \\ 1 \\ 0},
                \colvec{1 \\ 0 \\ 1}}
  \end{equation*}
  We will represent the transformation using two similar matrices.
  \begin{enumerate}
  \item Draw the arrow diagram.
  \item Compute $T=\rep{t}{B,B}$.
  \item Compute $\hat{T}=\rep{t}{D,D}$.
  \item Compute the matrices for other two sides of the arrow square.
  \end{enumerate}
\end{enumerate}
\end{document}
