\documentclass[11pt]{examjh}
\usepackage{../../linalgjh}
\examhead{MA 213 Hef{}feron, \yearsemester}{Final Exam}
\printanswers
% \noprintanswers
\begin{document}
\begin{questions}
\question
  Represent the linear map $\map{d/dx}{\polyspace_4}{\polyspace_3}$
  with respect to $B=\sequence{1,x,x^2,x^3,x^4}$ 
  and~$D=\sequence{1,1+x,1+x+x^2,1+x+x^2+x^3}$.
\begin{solution}[2.5in]
The effect of the map on each element of the domain basis~$B$
\begin{equation*}
  1\mapsto 0
  \quad
  x\mapsto 1
  \quad
  x^2\mapsto 2x
  \quad
  x^3\mapsto 3x^2
  \quad
  x^4\mapsto 4x^3
\end{equation*}
represented with respect to the codomain basis~$D$
\begin{multline*}
  \rep{0}{D}=\colvec{0 \\ 0 \\ 0 \\ 0}
  \quad
  \rep{1}{D}=\colvec{1 \\ 0 \\ 0 \\ 0}
  \quad
  \rep{2x}{D}=\colvec{-2 \\ 2 \\ 0 \\ 0}
  \\
  \rep{3x^2}{D}=\colvec{0 \\ -3 \\ 3 \\ 0}
  \quad
  \rep{4x^3}{D}=\colvec{0 \\ 0 \\ -4 \\ 4}
\end{multline*}
give this matrix.
\begin{equation*}
  \rep{d/dx}{B,D}=
  \begin{mat}
    0 &1 &-2 &0  &0  \\
    0 &0 &2  &-3 &0  \\
    0 &0 &0  &3  &-4 \\
    0 &0 &0  &0  &4
  \end{mat}
\end{equation*}
\end{solution}



\question For this subspace of $\polyspace_3$
\begin{equation*}
  S=\set{f(x)=ax^3+bx^2+cx+d\suchthat
          \text{$f(1)=0$ and $f(2)=0$}}
\end{equation*}
find a basis.
What is the dimension of of $\polyspace_3$ and of~$S$? 
\begin{solution}[2in]
That $f(1)=0$ gives $a+b+c+d=0$ and that
$f(2)=0$ gives $8a+4b+2c+d=0$.
Gauss's Method
\begin{equation*}
  \begin{linsys}{4}
    a &+ &b  &+ &c  &+ &d &= &0 \\
   8a &+ &4b &+ &2c &+ &d &= &0
  \end{linsys}
  \grstep{-8\rho_1+\rho_2}
  \begin{linsys}{4}
    a &+ &b  &+ &c  &+ &d  &= &0 \\
      & &-4b &- &6c &- &7d &= &0
  \end{linsys}
\end{equation*}
followed by parametrizing gets a basis.
\begin{equation*}
  \set{\colvec{3/4 \\ -7/4 \\ 1 \\ 0}c+\colvec{1/2 \\ -3/2 \\ 0 \\ 1}d 
      \suchthat c,d\in\Re}
  \qquad
  \sequence{(3/4)x^3-(7/4)x^2+x,
            (1/2)x^3-(3/2)x^2+1}
\end{equation*}
The dimension of~$\polyspace_3$ is~$4$, while the dimension of~$S$ is~$2$.
\end{solution}



\question
  Find the determinant.
  \begin{equation*}
    \begin{vmat}
      1 &0 &2  &0  \\
      2 &1 &1  &1  \\
      0 &-1 &1 &-1 \\
      2 &2  &1 &0
    \end{vmat}
  \end{equation*}
\begin{solution}[1.5in]
  \begin{equation*}
    \grstep[-2\rho_1+\rho_4]{-2\rho_1+\rho_2}
    \grstep[-2\rho_2+\rho_4]{\rho_2+\rho_3}
    \grstep{(3/2)\rho_3+\rho_4}
    \begin{mat}
      1 &0 &2  &0  \\
      0 &1 &-3 &1  \\
      0 &0 &-2 &0  \\
      0 &0 &0 &-2
    \end{mat}
  \end{equation*}
  The determinant is $+4$.
\end{solution}


\question
Find the inverse of this matrix, or state ``does not exist.''.
\begin{equation*}
  \begin{mat}
    2 &1 \\
    4 &-1
  \end{mat}
\end{equation*}
\begin{solution}[0.75in]
  \begin{equation*}
    \frac{1}{-6}\cdot
    \begin{mat}
      -1 &-1 \\
      -4 &2
    \end{mat}
  \end{equation*}
\end{solution}


\question
Decide if this set is linearly independent.
If so then show that, and if not then exhibit a dependence.
Is it a basis for $\matspace_{\nbyn{2}}$?
\begin{equation*}
  \set{
    \begin{mat}
      1 &0 \\
      0 &3
    \end{mat},
    \begin{mat}
      3 &1 \\
      4 &1
    \end{mat},
    \begin{mat}
      2 &2 \\
      1 &1
    \end{mat}
  }
\end{equation*}
\begin{solution}[2in]
If you can't find the answer by eye, equations derived from the upper left, 
upper right, lower left, and lower right give this system.
\begin{equation*}
  \begin{linsys}{3}
    c_1  &+  &3c_2 &+ &2c_3  &=  &0 \\
         &   &c_2  &+ &2c_3  &=  &0 \\
         &   &4c_2 &+ &c_3   &=  &0 \\
    3c_1 &+  &c_2  &+ &c_3   &=  &0   
  \end{linsys}
  \grstep{-3\rho_1+\rho_4}
  \grstep[8\rho_2+\rho_4]{-4\rho_2+\rho_3}
  \grstep{(11/7)\rho_4+\rho_4}
  \begin{linsys}{3}
    c_1  &+  &3c_2 &+ &2c_3  &=  &0 \\
         &   &c_2  &+ &2c_3  &=  &0 \\
         &   &     &  &-7c_3 &=  &0 \\
         &   &     &   &0    &=  &0   
  \end{linsys}
\end{equation*}
So the set is linearly independent. 
It is, however, not a basis because it only has three elements. 
\end{solution}

\question
Solve this system using Gauss-Jordan reduction.
Parametrize the solution set.
\begin{equation*}
  \begin{linsys}{3}
    -x &+ &2y &  &   &= &1   \\
     x &- &y  &+ &2z &= &-1  \\
    2x &- &y  &+ &6z &= &-2 
  \end{linsys}
\end{equation*}
\begin{solution}[2in]
The Gauss-Jordan reduction
  \begin{equation*}
    \grstep[2\rho_1+\rho_3]{\rho_1+\rho_2}
    \grstep{-3\rho_2+\rho_3}
    \grstep{-\rho_1}
    \grstep{2\rho_2+\rho_1}
    \begin{linsys}{3}
       x &  &   &+ &4z &= &-1   \\
         &  &y  &+ &2z &= &0  \\
         &  &   &  &0  &= &0 
    \end{linsys}
  \end{equation*}
leads to the parameterization.
\begin{equation*}
  \set{\colvec{x \\ y \\ z}\suchthat \text{$x+4z=-1$ and $y+2z=0$}}
  =\set{\colvec{-1 \\ 0 \\ 0}+\colvec{-4 \\ -2 \\ 1}z \suchthat z\in\Re}
\end{equation*}
\end{solution}


\question
Find all eigenvalues and eigenvectors of this matrix.
\begin{equation*}
  \begin{mat}
    1 &2 \\
    2 &4
  \end{mat}
\end{equation*}
\begin{solution}[2.5in]
  Setting 
  \begin{equation*}
    \begin{vmat}
      1-x &2 \\
      2  &4-x
    \end{vmat}
    =x^2-5x
  \end{equation*}
  to zero gives the two eigenvalues $\lambda_1=0$ and~$\lambda_2=5$.

  The eigenvalues associated with $\lambda_1=0$ are the 
  nonzero solutions of
  \begin{equation*}
    \begin{mat}
      1-0  &2 \\
      2    &4-0 
    \end{mat}
    \colvec{x \\ y}=\colvec{0 \\ 0}
  \end{equation*}
  which are the vectors in this set.
  \begin{equation*}
    V_0=\set{\colvec{-2 \\ 1}y \suchthat y\neq 0}
  \end{equation*}
  The eigenvectors for $\lambda_2=5$ are the nonzero solutions of
  \begin{equation*}
    \begin{mat}
      1-5  &2 \\
      2    &4-5 
    \end{mat}
    \colvec{x \\ y}=\colvec{0 \\ 0}
  \end{equation*}
  namely these.
  \begin{equation*}
    V_5=\set{\colvec{1 \\ 1/2}y \suchthat y\neq 0}
  \end{equation*}
\end{solution}



\question
Find the projection of the vector onto the line.
\begin{equation*}
  \vec{v}=\colvec{1 \\ 2 \\ -2}
  \qquad
  \ell=\set{t\colvec{2  \\ 1  \\ 0}\suchthat t\in\Re}
\end{equation*}
\begin{solution}[1in]
\begin{equation*}
  \frac{\colvec{1 \\ 2 \\ -2}\dotprod\colvec{2  \\ 1  \\ 0}}{\colvec{2  \\ 1  \\ 0}\dotprod \colvec{2  \\ 1  \\ 0}}\cdot \colvec{2  \\ 1  \\ 0}
  =\frac{4}{5}\cdot \colvec{2  \\ 1  \\ 0}    
\end{equation*}
\end{solution}


\question
Let $T$ represent $\map{t}{\Re^2}{\Re^2}$ with respect to $B,B$.
\begin{equation*}
  T=
  \begin{mat}
    1 &-1 \\ 
    2 &1
  \end{mat}
  \qquad
  B=\sequence{\colvec{1 \\ 0},\colvec{1 \\ 1}},\,
  D=\sequence{\colvec{2 \\ 0},\colvec{0 \\ -2}}
\end{equation*}
We will convert to the matrix representing~$t$ with resepct to $D,D$.
\begin{parts}
\part Draw the arrow diagram.
\begin{solution}[1in]
\begin{equation*}
  \begin{CD}
    \Re^2_{\wrt{B}}                   @>t>T>        \Re^2_{\wrt{B}}       \\
    @V{\scriptstyle\identity} VV              @V{\scriptstyle\identity} VV \\
    \Re^2_{\wrt{D}}                   @>t>\hat{T}>        \Re^2_{\wrt{D}}
  \end{CD}
\end{equation*}  
\end{solution}
\part Give the matrix that represents the left and right
  sides of that diagram, in the
  direction that we traverse the diagram to make the conversion.
\begin{solution}[2in]
For the right side we find the effect of the identity map
\begin{equation*}
  \colvec{1 \\ 0}\mapsunder{\identity}\colvec{1 \\ 0}
  \qquad
  \colvec{1 \\ 1}\mapsunder{\identity}\colvec{1 \\ 1}
\end{equation*}
and represent with respect to~$D$
\begin{equation*}
  \rep{\colvec{1 \\ 0}}{D}=\colvec{1/2 \\ 0}
  \qquad
  \rep{\colvec{1 \\ 1}}{D}=\colvec{1/2 \\ -1/2}
\end{equation*}
so we have this on the right.
\begin{equation*}
  \rep{\identity}{B,D}=
  \begin{mat}
    1/2 &1/2 \\
    0   &-1/2
  \end{mat}
\end{equation*}

For the matrix on the left we can either compute it directly, as in the
prior paragraph, or we can take the inverse. 
\begin{equation*}
  \rep{\identity}{D,B}=
  \frac{1}{(-1/4)}\cdot
  \begin{mat}
    -1/2 &-1/2 \\
    0   &1/2
  \end{mat}
  =
  \begin{mat}
    2 &2 \\
    0 &-2
  \end{mat}
\end{equation*}
\end{solution}

\part Find $\rep{t}{D,D}$.
\begin{solution}[1in]
As with the prior item we can either compute it directly from the definitions,
or compute it using matrix operations.
\begin{equation*}
  \begin{mat}
    2 &2 \\
    0 &-2
  \end{mat}
  \begin{mat}
    1 &-1 \\ 
    2 &1
  \end{mat}
  \frac{-1}{4}\cdot
  \begin{mat}
    -1/2 &-1/2 \\
    0   &1/2
  \end{mat}
  =
  \begin{mat}
    3 &3  \\
   -2 &-1
  \end{mat}
\end{equation*}
\end{solution}
\end{parts}


\question
Let $T$ represent a map $\map{t}{\Re^2}{\Re^2}$
with respect to the standard bases. 
\begin{equation*}
  T=
  \begin{mat}
    2 &-1 \\
    6 &-3
  \end{mat}
\end{equation*}
Find the range space and null space of~$t$.
\begin{solution}[2in]
Because these are the standard bases, all vectors represent themselves.
Let
\begin{equation*}
  \begin{mat}
    2 &-1 \\
    6 &-3
  \end{mat}
  \colvec{x \\ y}
  =\colvec{a \\ b}  
\end{equation*}
and do the Gauss-Jordan reduction. 
\begin{equation*}
  \begin{amat}{2}
    2 &-1 &a \\
    6 &-3 &b   
  \end{amat}
  \grstep{-3\rho_1+\rho_2}
  \grstep{(1/2)\rho_1}
  \begin{amat}{2}
    1 &-1/2 &(1/2)a \\
    0 &0    &-3a+b   
  \end{amat}
\end{equation*}
The range space is this
\begin{equation*}
  \rangespace{t}=
  \set{\colvec{a \\ b}\suchthat -3a+b=0}
  =\set{\colvec{1/3 \\ 1}b\suchthat b\in\Re}   
\end{equation*}
and the null space is this.
\begin{equation*}
  \nullspace{t}=
  \set{\colvec{x \\ y}\suchthat x-(1/2)y=0}
  =\set{\colvec{1/2 \\ 1}y\suchthat y\in\Re}
\end{equation*}
\end{solution}
\end{questions}
\end{document}



sage: load "../lab/gauss_method.sage"
/usr/lib/sagemath/local/lib/python2.7/site-packages/sage/misc/sage_extension.py:371: DeprecationWarning: Use %runfile instead of load.
See http://trac.sagemath.org/12719 for details.
  line = f(line, line_number)
sage: m = matrix(QQ, [[1,0,2,0], [2,1,1,1], [0,-1,1,-1], [2,2,1,0]])
sage: det(m)
4
sage: gauss_method(m)
[ 1  0  2  0]
[ 2  1  1  1]
[ 0 -1  1 -1]
[ 2  2  1  0]
 take -2 times row 1 plus row 2
 take -2 times row 1 plus row 4
[ 1  0  2  0]
[ 0  1 -3  1]
[ 0 -1  1 -1]
[ 0  2 -3  0]
 take 1 times row 2 plus row 3
 take -2 times row 2 plus row 4
[ 1  0  2  0]
[ 0  1 -3  1]
[ 0  0 -2  0]
[ 0  0  3 -2]
 take 3/2 times row 3 plus row 4
[ 1  0  2  0]
[ 0  1 -3  1]
[ 0  0 -2  0]
[ 0  0  0 -2]


sage: m = matrix(QQ, [[-1,2,0,1], [1,-1,2,-1], [2,-1,6,-2]])
sage: gauss_jordan(m)
[-1  2  0  1]
[ 1 -1  2 -1]
[ 2 -1  6 -2]
 take 1 times row 1 plus row 2
 take 2 times row 1 plus row 3
[-1  2  0  1]
[ 0  1  2  0]
[ 0  3  6  0]
 take -3 times row 2 plus row 3
[-1  2  0  1]
[ 0  1  2  0]
[ 0  0  0  0]
 take -1 times row 1
[ 1 -2  0 -1]
[ 0  1  2  0]
[ 0  0  0  0]
 take 2 times row 2 plus row 1
[ 1  0  4 -1]
[ 0  1  2  0]
[ 0  0  0  0]

sage: m = matrix(QQ, [[1,1,1,1,0], [8,4,2,1,0]])
sage: m
[1 1 1 1 0]
[8 4 2 1 0]
sage: gauss_method(m)
[1 1 1 1 0]
[8 4 2 1 0]
 take -8 times row 1 plus row 2
[ 1  1  1  1  0]
[ 0 -4 -6 -7  0]
sage: var('a,b,c,d')
(a, b, c, d)
sage: eqns=[a+b+c+d==0, 8*a+4*b+2*c+d==0]
sage: solve(eqns, a, b,c,d)
[[a == 3/4*r1 + 1/2*r2, b == -7/4*r1 - 3/2*r2, c == r2, d == r1]]


sage: P=matrix(QQ, [[-1/2,-1/2], [0,1/2]])
sage: P
[-1/2 -1/2]
[   0  1/2]
sage: T=matrix(QQ, [[1,-1], [2,1]])
sage: T
[ 1 -1]
[ 2  1]
sage: P*T*P^(-1)
[ 3  3]
[-2 -1]

sage: m2=matrix(QQ, [[2,-1], [6,-3]])
sage: gauss_jordan(m2)
[ 2 -1]
[ 6 -3]
 take -3 times row 1 plus row 2
[ 2 -1]
[ 0  0]
 take 1/2 times row 1
[   1 -1/2]
[   0    0]
