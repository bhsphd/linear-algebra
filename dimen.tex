% Chapter 2, Topic _Linear Algebra_ Jim Hefferon
%  http://joshua.smcvt.edu/linearalgebra
%  2001-Jun-11
\topic{Dimensional Analysis}
``You can't add apples and oranges,'' the old saying goes.
It reflects our experience that in applications the quantities
have units and keeping track of those units can help.
Everyone has done calculations such as this one 
that use the units as a check.
\begin{equation*}
  60\,\frac{\text{sec}}{\text{min}}
  \cdot 60\,\frac{\text{min}}{\text{hr}}
  \cdot 24\,\frac{\text{hr}}{\text{day}}
  \cdot 365\,\frac{\text{day}}{\text{year}}
  = 31\,536\,000\,\frac{\text{sec}}{\text{year}}
\end{equation*}
We can take the idea of including the units beyond bookkeeping.
We can use units to draw conclusions
about what relationships are possible among the physical quantities.

To start, consider the falling body 
equation~$\text{distance}=16\cdot(\text{time})^2$.
If the distance is in feet and the time is in seconds then this is a 
true statement. 
However
it is not correct in other unit systems, such as meters and seconds,
because~$16$ isn't the
right constant in those systems.  
We can fix that by attaching units to the $16$, making it a 
\definend{dimensional constant}\index{dimensional!constant}.
\begin{equation*}
  \text{dist}=16\,\frac{\text{ft}}{\text{sec}^2}\cdot (\text{time})^2
\end{equation*}  
Now the equation holds also in the meter-second system   
because when we align the units (a foot is approximately $0.30$ meters), 
\begin{equation*}
  \text{distance in meters}=16\,\frac{0.30\,\text{m}}{\text{sec}^2}
                    \cdot (\text{time in sec})^2
                   =4.8\,\frac{\text{m}}{\text{sec}^2}
                    \cdot (\text{time in sec})^2
\end{equation*}
the constant gets adjusted.
So in order to have equations that are correct across unit systems, 
we restrict our attention
to those that use dimensional constants. 
Such an equation is \definend{complete}.\index{complete equation}

Moving away from a particular unit system allows us to just 
measure quantities in combinations
of some units of length~$L$, mass~$M$, and time~$T$.
These three are our
\definend{physical dimensions}\index{physical dimension}.
For instance, we could measure velocity 
in $\text{feet}/\text{second}$
or $\text{fathoms}/\text{hour}$ but at all events it involves
a unit of length divided by a unit of time
so the \definend{dimensional formula}\index{dimensional!formula} 
of velocity is $L/T$.
Similarly, density's dimensional formula is $M/L^3$.

To write the dimensional formula 
we shall use negative exponents instead of  
fractions and we shall include the dimensions with a
zero exponent.
Thus we will write the dimensional formula of
velocity as $L^1M^0T^{-1}$ and that of density
as $L^{-3}M^1T^0$.

With that, ``you can't add apples and oranges'' becomes
the advice to check that all of an equation's terms have 
the same dimensional formula.
An example is this version of the falling body equation~$d-gt^2=0$. 
The dimensional formula of the $d$ term is $L^1M^0T^0$. 
For the other term, the dimensional formula of $g$ is $L^1M^0T^{-2}$ 
($g$ is given above as $16\,\text{ft}/\text{sec}^2$) 
and the dimensional formula of $t$ is $L^0M^0T^1$ so that of  
the entire $gt^2$ term is
$L^1M^0T^{-2}(L^0M^0T^1)^2=L^1M^0T^0$. 
Thus the two terms have the same dimensional formula.
An equation with this property is \definend{dimensionally homogeneous}.

Quantities with dimensional formula $L^0M^0T^0$ are \definend{dimensionless}.
For example, we measure an angle by taking
the ratio of the subtended arc to the radius
\begin{center}
  \includegraphics{ch2.22}
\end{center}
which is the ratio of a length to a length $(L^1M^0T^0)(L^1M^0T^0)^{-1}$ and
thus angles have the dimensional formula $L^0M^0T^0$.
%(Of course, to measure the angle in degrees instead we would 
%multiply by a scalar, which would leave the dimensional formula unchanged.)

The classic example of using the units for more than bookkeeping, using them 
to draw conclusions, considers the formula for the period of a pendulum.
\begin{equation*}
  p=\text{--some expression involving the length of the string, etc.--}
\end{equation*}
The period is in units of time $L^0M^0T^1$.
So the quantities on the 
other side of the equation must have dimensional formulas that combine 
in such a way that their $L$'s and $M$'s cancel and only a 
single $T$ remains.
The table on page~\pageref{table:Dimen} has the quantities that
an experienced investigator would consider possibly relevant to the period of
a pendulum. 
The only dimensional formulas involving $L$ are for
the length of the string and the acceleration due to gravity.
For the $L$'s of these two to cancel when they appear 
in the equation they must be in ratio,
e.g., as $(\ell/g)^2$, or as $\cos(\ell/g)$, 
or as $(\ell/g)^{-1}$.
Therefore the period is a function of $\ell/g$. 

This is a remarkable result:~with a pencil and paper analysis,
before we ever took out the pendulum and made measurements,
we have determined something about what makes up its period. 

To do dimensional analysis systematically, we need two facts
(arguments for these are in \cite{Bridgman}, Chapter~II and~IV).
The first is that each equation relating physical 
quantities that we shall see involves a sum of terms, 
where each term has the form
\begin{equation*}
  m_1^{p_1}m_2^{p_2}\cdots m_k^{p_k}
\end{equation*}
for numbers $m_1$, \ldots, $m_k$ that measure the quantities.

For the second fact, observe that an easy way to construct a dimensionally
homogeneous expression is by taking a product of dimensionless quantities
or by adding such dimensionless terms.
Buckingham's Theorem states that any
complete relationship among quantities with 
dimensional formulas can be algebraically manipulated into a form where
there is some function $f$ such that
\begin{equation*}
  f(\Pi_1,\ldots,\Pi_n)=0
\end{equation*}
for a complete set $\set{\Pi_1,\ldots,\Pi_n}$ of dimensionless products.
(The first example below describes what makes a set of dimensionless 
products `complete'.)
We usually want to express one of the quantities, 
$m_1$ for instance, in terms of the others. 
For that we will assume that the above equality can be rewritten 
\begin{equation*}
  m_1=m_2^{-p_2}\cdots m_k^{-p_k}\cdot \hat{f}(\Pi_2,\ldots,\Pi_n)
\end{equation*}
where $\Pi_1=m_1m_2^{p_2}\cdots m_k^{p_k}$ is dimensionless
and the products $\Pi_2$, \ldots, $\Pi_n$ don't involve $m_1$
(as with $f$, here $\hat{f}$ is an arbitrary function, this time 
of $n-1$ arguments).
Thus, to do dimensional analysis we should
find which dimensionless products are possible.
%For that we will use vector spaces.

For example, again consider 
the formula for a pendulum's period.
\begin{center}
  \begin{tabular}{c}
    \includegraphics{ch2.23}
  \end{tabular}
 \qquad\quad
 \label{table:Dimen}
  \begin{tabular}{r|l} 
    \multicolumn{1}{r}{\textit{quantity}}
    &\multicolumn{1}{l}{\begin{tabular}[b]{@{}l} 
                           \textit{dimensional} \\[-.3ex] 
                           \textit{formula}
                         \end{tabular}}                   \\ \hline
    period $p$                      &$L^0M^0T^1$          \\
    length of string $\ell$         &$L^1M^0T^0$          \\
    mass of bob $m$                 &$L^0M^1T^0$          \\
    acceleration due to gravity $g$ &$L^1M^0T^{-2}$    \\
    arc of swing $\theta$           &$L^0M^0T^0$         
  \end{tabular}
\end{center} 
By the first fact cited above, we expect the formula to 
have (possibly sums of terms of) the form
$p^{p_1}\ell^{p_2}m^{p_3}g^{p_4}\theta^{p_5}$.
To use the second fact, to find which combinations of the 
powers~$p_1$, \ldots, $p_5$ 
yield dimensionless products, consider this equation.
\begin{equation*}
  (L^0M^0T^1)^{p_1}(L^1M^0T^0)^{p_2}(L^0M^1T^0)^{p_3}
     (L^1M^0T^{-2})^{p_4}(L^0M^0T^0)^{p_5}
  =L^0M^0T^0 
\end{equation*}
It gives three conditions on the powers. 
\begin{equation*}
  \begin{linsys}{5}
       &\   &p_2  &\   &    &+  &p_4   &  &  &=  &0  \\
       &    &     &    &p_3 &   &      &  &  &=  &0  \\
    p_1&    &     &    &    &-  &2p_4  &  &  &=  &0  
  \end{linsys}  
\end{equation*}
Note that $p_3=0$ so the mass of the bob does not affect the period.
Gaussian reduction and parametrization of that system gives this 
\begin{equation*}
  \set{\colvec{p_1 \\ p_2 \\ p_3 \\ p_4 \\ p_5}=
       \colvec[r]{1 \\  -1/2  \\  0  \\  1/2  \\  0}p_1+
       \colvec[r]{0   \\  0   \\  0  \\  0  \\  1}p_5
       \suchthat p_1,p_5\in\Re}
\end{equation*}
(we've taken $p_1$ as one of the parameters in order to express
the period in terms of the other quantities).

The set of dimensionless products 
contains all terms $p^{p_1}\ell^{p_2}m^{p_3}a^{p_4}\theta^{p_5}$ subject
to the conditions above.
This set forms a vector space under the 
`$+$' operation of multiplying two such products 
and the `$\cdot$' operation of raising 
such a product to the power of the scalar
(see \nearbyexercise{exer:DimLessProdsVecSp}).
The term `complete set of dimensionless products' 
in Buckingham's Theorem means a basis for this vector space.

We can get a basis 
by first taking $p_1=1$, $p_5=0$, and then taking $p_1=0$, $p_5=1$.
The associated dimensionless 
products are $\Pi_1=p\ell^{-1/2}g^{1/2}$ and $\Pi_2=\theta$.
Because the set $\set{\Pi_1,\Pi_2}$ is complete, 
Buckingham's Theorem says that
\begin{equation*}
  p = \ell^{1/2}g^{-1/2}\cdot\hat{f}(\theta) 
    = \sqrt{\ell/g}\cdot\hat{f}(\theta)
\end{equation*}
where $\hat{f}$ is a function that we cannot determine from this analysis
(a first year physics text will show by other means 
that for small angles it is approximately the constant
function $\hat{f}(\theta)=2\pi$).

Thus, analysis of the relationships that are possible between the
quantities with the given dimensional formulas 
has given us a fair amount of information:~a pendulum's period does not 
depend on the mass of the bob, 
and it rises with the square root of the length of the string.

For the next example we try to determine the period of revolution of
two bodies in space orbiting each other under mutual gravitational attraction.
An experienced investigator could expect that these are the relevant 
quantities.
\begin{center}
  \begin{tabular}{@{}c@{}}
    \includegraphics{ch2.24}
  \end{tabular}
  \qquad\quad
  \begin{tabular}{r|l} 
    \multicolumn{1}{r}{\textit{quantity}}
    &\multicolumn{1}{l}{\begin{tabular}[b]{@{}l@{}} 
                          \textit{dimensional} \\[-.3ex] 
                          \textit{formula}
                         \end{tabular}} \\ \hline
    period $p$       &$L^0M^0T^1$         \\
    mean separation $r$  &$L^1M^0T^0$          \\
    first mass $m_1$          &$L^0M^1T^0$          \\
    second mass $m_2$         &$L^0M^1T^0$          \\
    gravitational constant $G$     &$L^3M^{-1}T^{-2}$   
  \end{tabular}
\end{center} 
%(\cite{Bridgman} discusses why the investigator needs experience).
To get the complete set of dimensionless products we consider the
equation 
\begin{equation*}
  (L^0M^0T^1)^{p_1}(L^1M^0T^0)^{p_2}(L^0M^1T^0)^{p_3}(L^0M^1T^0)^{p_4}
        (L^3M^{-1}T^{-2})^{p_5}=L^0M^0T^0
\end{equation*}
which results in a system
\begin{equation*}
  \begin{linsys}{5}
          &  &p_2  &   &     &   &      &+  &3p_5 &=  &0  \\
          &  &     &   &p_3  &+  &p_4   &-  &p_5  &=  &0  \\
      p_1 &  &     &   &     &   &      &-  &2p_5 &=  &0  
  \end{linsys}
\end{equation*}
with this solution.
\begin{equation*}
  \set{\colvec[r]{1 \\ -3/2  \\ 1/2 \\ 0 \\ 1/2}p_1
       +\colvec[r]{0 \\ 0    \\  -1 \\ 1 \\ 0}p_4
    \suchthat p_1,p_4\in\Re}
\end{equation*}

As earlier, 
the set of dimensionless products of these quantities forms a vector space
and we want to produce a basis for that space, 
a `complete' set of dimensionless products.
One such set, gotten from setting $p_1=1$ and $p_4=0$ 
and also setting $p_1=0$ and $p_4=1$ is
$\set{\Pi_1=pr^{-3/2}m_1^{1/2}G^{1/2},\,\Pi_2=m_1^{-1}m_2}$.
With that, Buckingham's Theorem says that
any complete relationship among these quantities is stateable this form.
\begin{equation*}
  p = r^{3/2}m_1^{-1/2}G^{-1/2}\cdot\hat{f}(m_1^{-1}m_2)  
    = \frac{r^{3/2}}{\sqrt{Gm_1}}\cdot\hat{f}(m_2/m_1)
\end{equation*}

\textit{Remark.}
An important application of the prior formula is
when $m_1$ is the mass of the sun and $m_2$ is the mass  
of a planet.
Because $m_1$ is very much greater than $m_2$, 
the argument to $\hat{f}$ is approximately $0$, 
and we can wonder whether
this part of the formula remains approximately constant as $m_2$ varies.
One way to see that it does is this.
The sun is so much larger than the planet
that the mutual rotation is approximately about the sun's center.
If we vary the planet's mass $m_2$ by a factor of $x$ 
(e.g., Venus's mass is
$x=0.815$ times Earth's mass),
then the force of attraction is multiplied by $x$,
and $x$ times the force acting on $x$ times the mass gives, since $F=ma$, 
the same acceleration, about the same center (approximately).
Hence,
the orbit will be the same and so its period will be the same, 
and thus the right side of the above equation also remains unchanged 
(approximately).
Therefore,
$\hat{f}(m_2/m_1)$ is approximately constant as $m_2$ varies.
This is Kepler's Third Law:~the square of the
period of a planet is proportional to the cube of the mean radius of its orbit
about the sun.

The final example was one of the first explicit applications of
dimensional analysis.
Lord Raleigh considered the speed of a wave in deep water and
suggested these as the relevant quantities.
\begin{center}
  \begin{tabular}{r|l} 
    \multicolumn{1}{r}{\textit{quantity}}
    &\multicolumn{1}{l}{\begin{tabular}[b]{@{}l} 
                          \textit{dimensional} \\[-.3ex] 
                          \textit{formula}
                         \end{tabular}} \\ \hline
    velocity of the wave $v$           &$L^1M^0T^{-1}$         \\
    density of the water $d$           &$L^{-3}M^1T^0$         \\
    acceleration due to gravity $g$    &$L^1M^0T^{-2}$         \\
    wavelength $\lambda$               &$L^1M^0T^0$           
  \end{tabular}
\end{center} 
The equation
\begin{equation*}
  (L^1M^0T^{-1})^{p_1}(L^{-3}M^1T^0)^{p_2}
      (L^1M^0T^{-2})^{p_3}(L^1M^0T^0)^{p_4}=L^0M^0T^0
\end{equation*}
gives this system
\begin{equation*}
  \begin{linsys}{4}
    p_1  &-  &3p_2 &+  &p_3  &+  &p_4  &=  &0  \\
         &   &p_2  &   &     &   &     &=  &0  \\
    -p_1 &   &     &-  &2p_3 &   &     &=  &0  
  \end{linsys}
\end{equation*}
with this solution space.
\begin{equation*}
  \set{\colvec[r]{1 \\ 0 \\ -1/2  \\ -1/2}p_1
       \suchthat p_1\in \Re}
\end{equation*}
There is one dimensionless product, $\Pi_1=vg^{-1/2}\lambda^{-1/2}$, 
and so $v$ is $\sqrt{\lambda g}$ times a constant;
$\hat{f}$ is constant since it is a function of no arguments. 
The quantity $d$ is 
not involved in the relationship.

The three examples above show that dimensional analysis 
can bring us far toward expressing the relationship among the quantities. 
For further reading,
the classic reference is \cite{Bridgman}\Dash this brief book is delightful.
Another source is \cite{Giordano83}.
A description of dimensional analysis's place in
modeling is in \cite{Giordano82}.

\begin{exercises}
  \item 
    \cite{deMestre}
    Consider a projectile, launched with initial velocity $v_0$, at an angle
    $\theta$.
    To study its motion we may
    guess that these are the relevant quantities.
    \begin{center}
      \begin{tabular}{r|l} 
        \multicolumn{1}{r}{\textit{quantity}}
        &\multicolumn{1}{l}{\begin{tabular}[b]{@{}l} 
                               \textit{dimensional} \\[-.3ex] 
                               \textit{formula}
                             \end{tabular}} \\ \hline
        horizontal position $x$         &$L^1M^0T^0$         \\
        vertical position $y$           &$L^1M^0T^0$         \\
        initial speed $v_0$             &$L^1M^0T^{-1}$      \\
        angle of launch $\theta$        &$L^0M^0T^0$         \\         
        acceleration due to gravity $g$ &$L^1M^0T^{-2}$      \\
        time $t$                        &$L^0M^0T^1$          
      \end{tabular}
    \end{center} 
    \begin{exparts}
      \partsitem Show that $\set{gt/v_0,gx/v_0^2,gy/v_0^2,\theta}$ 
        is a complete set of dimensionless products.
        (\textit{Hint.}
        One way to go is to find the appropriate free variables 
        in the linear system that arises but there is a shortcut that
        uses the properties of a basis.)
      \partsitem These two equations of motion for projectiles are familiar:
        $x=v_0\cos(\theta) t$ and $y=v_0\sin(\theta) t- (g/2)t^2$.
        Manipulate each to rewrite it as a relationship among 
        the dimensionless products of the prior item.
    \end{exparts}
    \begin{answer}
      \begin{exparts}
        \partsitem Assuming that this
          \begin{equation*}
            (L^1M^0T^0)^{p_1}(L^1M^0T^0)^{p_2}(L^1M^0T^{-1})^{p_3}
             (L^0M^0T^0)^{p_4}(L^1M^0T^{-2})^{p_5}(L^0M^0T^1)^{p_6}
          \end{equation*}
          equals $L^0M^0T^0$
          gives rise to this linear system
          \begin{equation*}
            \begin{linsys}{5}
              p_1  &+  &p_2  &+  &p_3  &+  &p_5  &   &    &=  &0  \\
              %      &   &     &   &     &   &     &   &0   &=  &0  \\
                   &   &     &   &-p_3 &-  &2p_5 &+  &p_6 &=  &0  
            \end{linsys}
          \end{equation*}
          (there is no restriction on $p_4$).
          The natural parametrization uses the free variables to give
          $p_3=-2p_5+p_6$ and $p_1=-p_2+p_5-p_6$.
          The resulting description of the solution set
          \begin{equation*}
            \set{\colvec{p_1 \\ p_2 \\ p_3 \\ p_4 \\ p_5 \\ p_6}
                =p_2\colvec[r]{-1 \\ 1 \\ 0 \\ 0  \\ 0 \\ 0} 
                +p_4\colvec[r]{0 \\ 0 \\ 0 \\ 1 \\ 0 \\ 0} 
                +p_5\colvec[r]{1 \\ 0 \\ -2 \\ 0 \\ 1 \\ 0} 
                +p_6\colvec[r]{-1 \\ 0 \\ 1 \\ 0 \\ 0 \\ 1} 
                \suchthat p_2, p_4, p_5, p_6\in\Re  }
          \end{equation*}
          gives $\set{y/x,\theta,xt/{v_0}^2,v_0t/x}$ 
          as a complete set of dimensionless products
          (recall that ``complete'' in this context does not mean 
          that there are no other dimensionless products;
          it simply means that the set is a basis).
          This is, however, not the set of dimensionless products that
          the question asks for.

          There are two ways to proceed.
          The first is to fiddle with the choice of parameters, hoping to
          hit on the right set.
          For that, we can do the prior paragraph in reverse.
          Converting the given dimensionless products $gt/v_0$,
          $gx/v_0^2$, $gy/v_0^2$, and $\theta$
          into vectors gives this description (note the \textit{?}'s where the
          parameters will go).
          \begin{equation*}
            \set{\colvec{p_1 \\ p_2 \\ p_3 \\ p_4 \\ p_5 \\ p_6}
                =\underline{\ \text{\textit{?}}\ }
                      \colvec[r]{0  \\ 0 \\ -1 \\ 0  \\ 1 \\ 1} 
                +\underline{\ \text{\textit{?}}\ }
                   \colvec[r]{1 \\ 0 \\ -2 \\ 0 \\ 1 \\ 0} 
                +\underline{\ \text{\textit{?}}\ }
                   \colvec[r]{0 \\ 1 \\ -2 \\ 0 \\ 1 \\ 0} 
                +p_4\colvec[r]{0  \\ 0 \\ 0 \\ 1 \\ 0 \\ 0} 
                \suchthat p_2, p_4, p_5, p_6\in\Re  }
          \end{equation*}
          The $p_4$ is already in place.
          Examining the rows shows that we can also put in place $p_6$, $p_1$,
          and $p_2$.

          The second way to proceed, following the hint, is to note that
          the given set is of size four in a four-dimensional vector space
          and so we need only show that it is linearly independent.
          That is easily done by inspection, by considering the
          sixth, first, second, and fourth components of the vectors.
        \item The first equation can be rewritten
          \begin{equation*}
            \frac{gx}{{v_0}^2}=\frac{gt}{v_0}\cos\theta
          \end{equation*}
          so that Buckingham's function is 
          $f_1(\Pi_1,\Pi_2,\Pi_3,\Pi_4)=\Pi_2-\Pi_1\cos(\Pi_4)$.
          The second equation can be rewritten
          \begin{equation*}
            \frac{gy}{{v_0}^2}=\frac{gt}{v_0}\sin\theta
              -\frac{1}{2}\left(\frac{gt}{v_0}\right)^2
          \end{equation*}
          and Buckingham's function here is 
          $f_2(\Pi_1,\Pi_2,\Pi_3,\Pi_4)
             =\Pi_3-\Pi_1\sin(\Pi_4)+(1/2){\Pi_1}^2$.
      \end{exparts}
    \end{answer}
  \item \cite{Einstein1911}
    conjectured that the infrared characteristic frequencies of a solid
    might be determined by the same forces between atoms as determine
    the solid's ordinary elastic behavior.
    The relevant quantities are these.
    \begin{center}
      \begin{tabular}{r|l} 
        \multicolumn{1}{r}{\textit{quantity}}
        &\multicolumn{1}{l}{\begin{tabular}[b]{@{}l} 
                              \textit{dimensional} \\[-.3ex] 
                              \textit{formula}
                             \end{tabular}} \\ \hline
        characteristic frequency $\nu$       &$L^0M^0T^{-1}$         \\
        compressibility $k$                  &$L^1M^{-1}T^2$          \\
        number of atoms per cubic cm $N$     &$L^{-3}M^0T^0$          \\
        mass of an atom $m$                  &$L^0M^1T^0$         
      \end{tabular}
    \end{center} 
    Show that there is one dimensionless product.
    Conclude that, in any complete relationship among quantities
    with these dimensional formulas, $k$ is a constant times 
    $\nu^{-2}N^{-1/3}m^{-1}$. 
    This conclusion played an important role in the early study of
    quantum phenomena.
    \begin{answer}
      Consider
      \begin{equation*}
        (L^0M^0T^{-1})^{p_1}(L^1M^{-1}T^2)^{p_2}
          (L^{-3}M^0T^0)^{p_3}(L^0M^1T^0)^{p_4}=(L^0M^0T^0)
      \end{equation*}
      which gives these relations among the powers.
      \begin{equation*}
        \begin{linsys}{4}
               &    &p_2   &-  &3p_3  &   &    &=  &0  \\
               &    &-p_2  &   &      &+  &p_4 &=  &0  \\
         -p_1  &+   &2p_2  &   &      &   &    &=  &0  
        \end{linsys}
        \grstep{\rho_1\leftrightarrow\rho_3}
        \repeatedgrstep{\rho_2+\rho_3}
        \begin{linsys}{4}
         -p_1  &+   &2p_2  &   &      &   &    &=  &0  \\
               &    &-p_2  &   &      &+  &p_4 &=  &0  \\
               &    &      &\  &-3p_3 &+  &p_4 &=  &0  
        \end{linsys}
      \end{equation*}
      This is the solution space
      (because we wish to express $k$ as a function of the other quantities,
      we take $p_2$ as the parameter).
      \begin{equation*}
        \set{\colvec[r]{2 \\ 1 \\ 1/3 \\ 1}p_2
             \suchthat p_2\in\Re}
      \end{equation*}
      Thus, $\Pi_1=\nu^2kN^{1/3}m$ is the dimensionless combination,
      and we have that $k$ equals 
      $\nu^{-2}N^{-1/3}m^{-1}$ times a constant
      (the function $\hat{f}$ is constant since it has no arguments). 
    \end{answer}
  \item 
    \cite{Giordano83}
    The torque produced by an engine
    has dimensional formula $L^2M^1T^{-2}$.
    We may first guess that it depends on 
    the engine's rotation rate (with dimensional formula
    $L^0M^0T^{-1}$), and the volume of air displaced (with dimensional
    formula $L^3M^0T^0$).
    \begin{exparts}
      \partsitem Try to find a complete set of dimensionless products.
        What goes wrong?
      \partsitem Adjust the guess by adding the density of the air
        (with dimensional formula $L^{-3}M^1T^0$).
        Now find a complete set of dimensionless products.
    \end{exparts}
    \begin{answer}
      \begin{exparts}
        \partsitem Setting
          \begin{equation*}
            (L^2M^1T^{-2})^{p_1}(L^0M^0T^{-1})^{p_2}(L^{3}M^0T^0)^{p_3}
              =(L^0M^0T^0)
          \end{equation*}
          gives this
          \begin{equation*}
            \begin{linsys}{3}
              2p_1  &   &    &+  &3p_3  &=  &0  \\
              p_1   &   &    &   &      &=  &0  \\
              -2p_1 &-  &p_2 &   &      &=  &0  
            \end{linsys}
          \end{equation*}
          which implies that $p_1=p_2=p_3=0$.
          That is, among quantities with these dimensional formulas, the only
          dimensionless product is the trivial one.
        \partsitem Setting
          \begin{equation*}
            (L^2M^1T^{-2})^{p_1}(L^0M^0T^{-1})^{p_2}(L^{3}M^0T^0)^{p_3}
              (L^{-3}M^1T^0)^{p_4}=(L^0M^0T^0)
          \end{equation*}
          gives this.
          \begin{multline*}
            \begin{linsys}{4}
              2p_1  &   &    &+  &3p_3  &-  &3p_4 &=  &0  \\
              p_1   &   &    &   &      &+  &p_4  &=  &0  \\
              -2p_1 &-  &p_2 &   &      &   &     &=  &0  
            \end{linsys}                                         \\
            \grstep[\rho_1+\rho_3]{(-1/2)\rho_1+\rho_2}
            \repeatedgrstep{\rho_2\leftrightarrow\rho_3}
            \begin{linsys}{4}
              2p_1  &   &      &+  &3p_3       &-   &3p_4      &=  &0  \\
                    &   &-p_2  &+  &3p_3       &-   &3p_4      &=  &0  \\
                    &  &      &   &(-3/2)p_3  &+   &(5/2)p_4  &=  &0  
            \end{linsys}
          \end{multline*}
          Taking $p_1$ as parameter to express the torque gives this
          description of the solution set.
          \begin{equation*}
            \set{\colvec[r]{1 \\ -2 \\ -5/3 \\ -1}p_1
                 \suchthat p_1\in\Re}
          \end{equation*}
          Denoting the torque by $\tau$, the rotation rate by $r$, the volume
          of air by $V$, and the density of air by $d$ we have that
          $\Pi_1=\tau r^{-2} V^{-5/3} d^{-1}$, and so
          the torque is $r^2V^{5/3}d$ times a constant.
      \end{exparts}
    \end{answer}
  \item 
    \cite{Tilley} 
    Dominoes falling make a wave. 
    We may conjecture that the wave speed $v$ depends on the
    spacing $d$ between the dominoes,
    the height $h$ of each domino, 
    and the acceleration due to gravity $g$.
    \begin{exparts}
      \partsitem Find the dimensional formula for each of the four quantities.
      \partsitem Show that $\set{\Pi_1=h/d,\Pi_2=dg/v^2}$ 
        is a complete set of dimensionless products.
      \partsitem Show that if $h/d$ is fixed then
        the propagation speed is proportional to the square root
        of $d$.
    \end{exparts}
    \begin{answer}
          \begin{exparts}
            \partsitem These are the dimensional formulas.
              \begin{center}
                \begin{tabular}{r|l} 
                  \multicolumn{1}{r}{\textit{quantity}}
                    &\multicolumn{1}{l}{\begin{tabular}[b]{@{}l} 
                                           \textit{dimensional} \\[-.3ex] 
                                           \textit{formula}
                                       \end{tabular}} \\ \hline
                        speed of the wave $v$               &$L^1M^0T^{-1}$ \\
                        separation of the dominoes $d$      &$L^1M^0T^0$   \\
                        height of the dominoes $h$          &$L^1M^0T^0$ \\
                        acceleration due to gravity $g$    &$L^1M^0T^{-2}$ 
                \end{tabular}
              \end{center}     
            \partsitem The relationship
              \begin{equation*}
                (L^1M^0T^{-1})^{p_1}(L^1M^0T^0)^{p_2}(L^1M^0T^0)^{p_3}
                  (L^1M^0T^{-2})^{p_4}=(L^0M^0T^0)
              \end{equation*}
              gives this linear system.
              \begin{equation*}
                \begin{linsys}{4}
                  p_1  &+  &p_2  &+  &p_3  &+  &p_4  &=  &0  \\
                       &   &     &   &     &   &0    &=  &0  \\
                 -p_1  &   &     &   &     &-  &2p_4 &=  &0  
                \end{linsys}
                \grstep{\rho_1+\rho_4}
                \begin{linsys}{4}
                  p_1  &+  &p_2  &+  &p_3  &+  &p_4  &=  &0  \\
                       &   &p_2  &+  &p_3  &-  &p_4  &=  &0  
                \end{linsys}
              \end{equation*}
              Taking $p_3$ and $p_4$ as parameters, we can describe the
              solution set in this way.
              \begin{equation*}
                \set{\colvec[r]{0 \\ -1 \\ 1 \\ 0}p_3
                     +\colvec[r]{-2 \\ 1 \\ 0 \\ 1}p_4
                     \suchthat p_3,p_4\in\Re}
              \end{equation*}
              That gives $\set{\Pi_1=h/d,\Pi_2=dg/v^2}$ as a complete set.
            \partsitem Buckingham's Theorem says that 
              $v^2=dg\cdot\hat{f}(h/d)$ and so, since $g$ is a constant,
              if $h/d$ is fixed then $v$ is proportional to $\sqrt{d\,}$.
          \end{exparts}
        \end{answer}
   \item \label{exer:DimLessProdsVecSp}
      Prove that the dimensionless products form a vector space
      under the $\vec{+}$ operation of multiplying two such products 
      and the $\vec{\cdot}$ operation of raising 
      such the product to the power of the scalar.
      (The vector arrows are a precaution against confusion.)
      That is, prove that, for any particular homogeneous system,
      this set of products of powers of $m_1$, \ldots, $m_k$
      \begin{equation*}
        \set{m_1^{p_1}\dots m_k^{p_k}
             \suchthat 
              \text{$p_1$, \ldots, $p_k$ satisfy the system}}
      \end{equation*}
      is a vector space under:
      \begin{equation*}
        m_1^{p_1}\dots m_k^{p_k}
        \vec{+}
        m_1^{q_1}\dots m_k^{q_k}
        =
        m_1^{p_1+q_1}\dots m_k^{p_k+q_k}
      \end{equation*}
      and
      \begin{equation*}
        r\vec{\cdot} (m_1^{p_1}\dots m_k^{p_k})
        =
        m_1^{rp_1}\dots m_k^{rp_k}
      \end{equation*}
      (assume that all variables represent real numbers).
      \begin{answer}
        Checking the conditions in the definition of a vector space is 
        routine.
      \end{answer}
%  \item Buckingham's Theorem says there exists a complete set such that
%    the relationship can be stated in that way.
%    Why do we not have to worry that the particular complete set
%    we have produced might not be the right one?
%    (\textit{Hint.} 
%    Consider the prior exercise.)
%    \begin{answer}
%      Any `complete' set (any basis for the vector space under consideration)
%      can be expressed in terms of any other one.
%      So any function that is correct when given the members of a particular
%      complete set as arguments can
%      be rewritten to be correct for the  members of any other complete set.
%    \end{answer}
  \item \label{exer:AppsAndOranges}
    The advice about apples and oranges is not right.
    Consider the familiar equations for a circle $C=2\pi r$ and $A=\pi r^2$.
    \begin{exparts}
      \partsitem Check that $C$ and $A$ have different dimensional formulas.
      \partsitem Produce an equation that is not
        dimensionally homogeneous (i.e., 
        it adds apples and oranges) but is nonetheless true of any circle.
      \partsitem The prior item asks for an equation that is complete but
        not dimensionally homogeneous.
        Produce an equation that is dimensionally homogeneous but not complete.
    \end{exparts}
    (Just because the old saying isn't strictly right doesn't 
    keep it from being a useful strategy.
    Dimensional homogeneity is often used to check the plausibility of 
    equations used in models.
    For an argument that any complete equation can easily be made 
     dimensionally homogeneous, see \cite{Bridgman}, Chapter~I, 
     especially page~15.)
    \begin{answer}
      \begin{exparts}
        \partsitem The dimensional formula of the circumference is $L$,
          that is, $L^1M^0T^0$.
          The dimensional formula of the area is $L^2$.
        \partsitem One is $C+A=2\pi r + \pi r^2$.
        \partsitem One example is this formula relating the  
           the length of arc subtended by an angle to the radius and the
           angle measure in radians:~$\ell-r\theta=0$.
           Both terms in that formula have dimensional
           formula $L^1$.
           The relationship holds for some
           unit systems (inches and radians, for instance) but not for all
           unit systems (inches and degrees, for instance).  
      \end{exparts}
    \end{answer}
\end{exercises}
\endinput