% Chapter 5, Topic _Linear Algebra_ Jim Hefferon
%  http://joshua.smcvt.edu/linearalgebra
%  2016-Jun-17
% \documentclass{article}
% \usepackage{graphicx}
% \usepackage{amsmath}

% \newcommand{\definend}[1]{\textit{#1}}
\topic{Coupled Oscillators}
\index{Coupled Oscillators|(}
% \begin{document}

This is a \definend{Wilberforce pendulum}.\index{Wilberforce pendulum}
Hanging on the spring is a mass, or bob.
Push it up a bit and release and it will oscillate up and down.
\begin{center}
  \includegraphics{asy/wilber000.pdf}
\end{center}
But then, if the device is properly adjusted, something fascinating happens.
After a few seconds, in addition to going up
and down, the mass begins to rotate,
to spin about the axis of the spring.
This yaw increases until the mass's motion  
becomes almost entirely rotary, with very little up and down.
Perhaps ten seconds later the motion returns to a combination.
Then, after some more time, order reappears.
Amazingly, now
the motion is almost entirely vertical.
This continues, with the device trading off
periods of pure vertical motion with periods of pure rotational motion,
interspersed with mixtures. 
(Searching online for ``wilberforce pendulum video'' will return 
some excellent demonstrations.) 

Each pure motion state is a \definend{normal mode} of oscillation.
We will analyze this device's behavior when it is in a normal mode. 
It is all about the eigenvalues.

\begin{center}
  \includegraphics{asy/wilber001.pdf}
\end{center}

Write~$x(t)$ for the vertical motion over time and 
$\theta(t)$ for the rotational motion.
Fix the coordinate system so that in rest position $x=0$ and~$\theta=0$, 
so that positive $x$'s are up, and so that positive $\theta$'s
are counterclockwise when vewed from above.

We start by modelling the motion of a mass on a 
spring constrained to have no twist.
This is simpler because there is only one motion, one degree of freedom.
Put the mass in rest position and 
push it up to compress the spring.
Hooke's Law is that for small distances
the restoring force is 
proportional to the distance, $F=-k\cdot x$.
The constant~$k$ is the
\definend{stiffness} of the spring.

Newton's Law is that a force is proportional to the 
associated acceleration $F=m\cdot d^2\,x(t)/dt$.
The constant of proportionality, $m$, is the \definend{mass}
of the object.
Combining Hooke's Law with Newton's 
gives the differential equation expressing the 
mass's motion $m\cdot d^2\,x(t)/dt=-k\cdot x(t)$.
We prefer the from with the variables all on one side.
\begin{equation*}
  m\cdot \frac{d^2\,x(t)}{dt}+k\cdot x(t)=0
  \tag{$*$}
\end{equation*}

Our physical intuition is that over time the bob oscillates.
It started high so 
the graph should look like this.
\begin{center}
  \includegraphics{asy/wilber003.pdf}
\end{center}
Of course, this looks like a cosine graph and we recognize 
that the differential equation of motion~($*$) is satisfied by
$x(t)=\cos \omega t$, where $\omega=\sqrt{m/k}$,
since $dx/dt=\omega\cdot \sin \omega t$
and $d^2x/dt^2=-\omega^2\cdot \cos \omega t$.
Here, $\omega$ is the \definend{angular frequency}.
It governs the period of the oscillation since if $\omega=1$ then the
period is~$2\pi$, while if $\omega=2$ then the period is~$\pi$, etc.

We can give a more general solution of~($*$).
For a general amplitude we put a factor~$A$ in front
$x(t)=A\cdot \cos wt$.
And we can allow a phase shift, 
so we are not required to start the clock when the
mass is high, with 
$x(t)=A\cos (wt+\phi)$.
This is the equation of \definend{simple harmonic motion}.

Now back to consideration of the coupled pair of motions, 
vertical and rotational.
These two interact because a spring that is twisted will 
lengthen or shorten just a bit.
For instance, it could work as here, or could be reversed depending on the
spring.  
\begin{center}
  \includegraphics{asy/wilber002.pdf}
\end{center}
Likewise, a spring that is stretched or compressed from its rest
position will twist slightly.
The interaction of the two produces \definend{coupled oscillations}.

To see how the interaction can produce the dramatic behavior that
we see in a normal mode
imagine that the mass is twisting in the direction that will 
will make the spring longer.
If the spring's vertical motion at that moment is that it is getting shorter, 
then  
superimposing the shortening on the lengthening
could result in the two almost cancelling.
The bob ends up not moving vertically much at all, just twisting.
If the device is properly adjusted this could last for a number of seconds.

``Properly adjusted'' means that  
the period of the pure vertical motion is the same as, 
or close to, the period of the pure rotational motion.
With that, the cancellation will go on for some time so 
we will see the bob halt its vertical motion for seconds. 

The interaction between the motions can also produce the other normal mode 
behavior, where the bob moves mostly vertically without much twist,
if the spring's motion produces a twist that opposes the bob's twist.
The bob will not rotate, almost, so its motion is almost entirely vertical.

To get the equations of motion in this two degrees of freedom case,
we make the same assumption as we did for the one degree case,
that for small displacements the restoring force is
proportional to the displacement. 
But now we take that assumption both for
the vertical motion and for the rotation.
Let the constant of proportionality in the rotational motion be~$\kappa$.
Similarly we also use Newton's Law that force is proportional to 
acceleration for the rotational motion as well, and take
the constant of proportionality to be~$I$. 

Most crucially, we add a coupling between the two
motions, which we take to be proportional to each, with constant of 
proportionality~$\epsilon/2$.

That gives a system of two differential equations,
the first for vertical motion, the second
for rotation.
These equations describe the behavior of the coupled system at any time~$t$.
\begin{equation*}
  \begin{split}
  m\cdot\frac{d^2\,x(t)}{dt^2}
      +k\cdot x(t)+\frac{\epsilon}{2}\cdot \theta(t) 
      &=0  \\
  I\cdot\frac{d^2\,\theta(t)}{dt^2}
      +\kappa\cdot \theta(t)+\frac{\epsilon}{2}\cdot x(t)  
      &=0
  \end{split}
  \tag{$**$}
\end{equation*}
We will use those to analyze the
system's behavior at times when it is in a normal mode, when
its motion is almost entirely up and down 
or almost entirely rotational.

First consider the uncoupled motions, as given by the equations without the
$\epsilon$ terms.
Without those terms these describe simple harmonic functions,
and we write~$\omega_x^2$ for~$k/m$, and $\omega_\theta^2$ for $\kappa/I$.
We have argued above that to observe the stand-still behavior we should 
adjust the device so that the periods are the same $\omega_x^2=\omega_\theta^2$. 
Write~$\omega_0$ for that number.

Now consider the coupled motions $x(t)$ and~$\theta(t)$.
By the same principle, to observe the stand-still behavior we want them in 
in sync, for instance so that the rotation is at its
peak when the stretch is at its peak.
That is, in a normal mode the oscillations have the same
angular frequency~$\omega$.
As to phase shift, as we also discussed there are two
cases: when the twist imparted by the spring's motion is in the same direction
as the twist given by the rotational oscillation and when they are opposed.
In either case to get a normal mode the peaks should coincide so the
phase shifts are $\phi$ and $\phi+\pi$~radians.
We will work through the first case, leaving the second is an exercise.
\begin{equation*}
  x(t) = A_1\cos(\omega t+\phi)  
  \qquad
  \theta(t) = A_2\cos(\omega t+\phi)  
\end{equation*}
We want to find which $\omega$'s are possible.

Take the second derivatives 
\begin{equation*}
  \frac{d^2\,x(t)}{dt}=-A_1\omega^2\cos(\omega t+\phi)
  \qquad
  \frac{d^2\,\theta(t)}{dt}=-A_2\omega^2\cos(\omega t+\phi)
\end{equation*}
and plug into the equations of motion~($**$).
\begin{align*}
  m\cdot(-A_1\omega^2\cos(\omega t+\phi))
      +k\cdot (A_1\cos(\omega t+\phi))
      +\frac{\epsilon}{2}\cdot (A_2\cos(\omega t+\phi)) 
      &=0  \\
  I\cdot(-A_2\omega^2\cos(\omega t+\phi))
      +\kappa\cdot (A_2\cos(\omega t+\phi))
      +\frac{\epsilon}{2}\cdot (A_1\cos(\omega t+\phi))  
      &=0  
\end{align*}
Factor out $\cos(\omega t+\phi)$ and divide through by~$m$.
\begin{align*}
  \big(\frac{k}{m}-\omega^2\big)\cdot A_1 
      +\frac{\epsilon}{2m}\cdot A_2 
      &=0  \\
  \big(\frac{\kappa}{I}-\omega^2\big)\cdot A_2
      +\frac{\epsilon}{2m}\cdot A_1  
      &=0  
\end{align*}
We are assuming that $k/m=\omega_{x}^2$ and
replace $\kappa/I=\omega_{\theta}^2$ are equal, and writing $\omega_0^2$
for that number.
Make the substitution and restate it as a matrix equation.
\begin{equation*}
  \begin{pmatrix}
    \omega_0^2-\omega^2    &\epsilon/2m  \\
    \epsilon/2I &\omega_0^2-\omega^2 
  \end{pmatrix}
  \begin{pmatrix}
    A_1 \\
    A_2
  \end{pmatrix}
  =
  \begin{pmatrix}
    0 \\
    0
  \end{pmatrix}
\end{equation*}
Obviously this system has the trivial solution~$A_1=0$, $A_2=0$,
for the case where the mass is at rest.
We want to know for which frequencies~$\omega$ this system has a nontrivial
solution. 
\begin{equation*}
  \begin{pmatrix}
    \omega_0^2    &\epsilon/2m  \\
    \epsilon/2I  &\omega_0^2 
  \end{pmatrix}
  \begin{pmatrix}
    A_1 \\
    A_2
  \end{pmatrix}
  =
  \omega^2
  \begin{pmatrix}
    A_1 \\
    A_2
  \end{pmatrix}
\end{equation*}
The normal mode angular frequencies~$\omega$ are the eigenvalues of the matrix.

To calculate it
take the determinant and set it to zero.
\begin{equation*}
  \begin{vmatrix}
    \omega_0^2-\omega^2    &\epsilon/2m  \\
    \epsilon/2I           &\omega_0^2-\omega^2     
  \end{vmatrix}=0
  \quad\Longrightarrow\quad
  \omega^4-(2\omega_0^2)\,\omega^2
      +(\omega_0^4 -\frac{\epsilon^2}{4mI})=0
\end{equation*}
That equation is quadratic in~$\omega^2$.  
Apply the formula to solve quadratic equations, $(-b\pm\sqrt{b^2-4ac})/2a$.
\begin{equation*}   % \rule[-.3\baselineskip]{0pt}{\baselineskip}
  \omega^2 
  =\frac{2\omega_0^2
       \pm\sqrt{\smash[b]{\strut} 4\omega_0^4
                -4(\omega_0^4-\epsilon^2/4mI)}}{2}
  =\omega_0^2\pm\frac{\epsilon}{2\sqrt{mI}}
\end{equation*}
The value $\epsilon/\sqrt{mI}=\epsilon/\sqrt{\kappa k}$ is often written
$\omega_B$ so that
$
  \omega^2 =\omega_0^2\pm\omega_B/2
$
This is the \definend{beat frequency}, the difference
between the two normal mode frequencies.

\begin{exercises}
\item Use the formula for the cosine of a sum to give an even more
  general formula for simple harmonic motion.
  \begin{answer}
    The angle sum formula for the cosine function is 
    $\cos(\alpha+\beta)=cos(\alpha)\cos(\beta)-\sin(\alpha)\sin(\beta)$.
    Expand $A\cos(\omega t+\phi)$ to
    $A\cdot[cos(\omega t)\cos(\phi)-\sin(\omega t)\sin(\phi)]$.
    Then $\cos(\phi)$ and $\sin(\phi)$ do not vary with~$t$ so we get
    the general solution $x(t)=B\cos(\omega t)+C\sin(\omega t)$.
  \end{answer}
\item Find the values of $\omega$ in the case where the phase shift 
  is $\phi+\pi$.
\item Buld a Wilberforce pendulum out of a Slinky Jr and a soup can.
  You can drill holes in the can for bolts, either two or four of them,
  that you can use to adjust
  the moment of inertia of the can so the periods of vertical and rotational 
  motion coincide.
\end{exercises}
\index{Coupled Oscillators|)}
\endinput
% \end{document}