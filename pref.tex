% pref.tex  See http://joshua.smcvt.edu/linearalgebra
{\setlength{\parskip}{.7ex}  % note the group-starting open curly
% \bigskip
% \vspace*{1.25in plus .2in minus .1in}
\chapter*{Preface}
This book helps students to master the material of a standard 
US undergraduate first course in Linear Algebra.

The material is standard in that the subjects covered are
Gaussian reduction, 
vector spaces, linear maps,
determinants, and eigenvalues and eigenvectors.
Another standard is book's audience:
sophomores or juniors, usually with a background 
of at least one semester of calculus. 
The help that it gives to students comes from taking a developmental 
approach\Dash 
this book's presentation emphasizes motivation and naturalness, 
using many examples.

The developmental approach is what most recommends this book
so I will elaborate.
Courses at the beginning of a mathematics program
focus less on theory and more on calculating.
Later courses
ask for mathematical maturity:~the ability to follow different 
types of arguments, 
a familiarity with
the themes that underlie many mathematical investigations such as
elementary set and function facts,
and a capacity for some independent reading and thinking.
Some programs have a separate course devoted to developing maturity but
in any case a Linear Algebra course 
is an ideal spot to work on this transition.
It comes early in a program so that progress made here pays off later
but it also comes late enough so that the classroom contains only
students who are serious about mathematics.
The material is accessible, coherent, and elegant.
And, examples are plentiful.

Helping readers with their transition 
requires taking the mathematics seriously. 
All of the results here are proved.
On the other hand, we cannot
assume that students have already arrived
and so 
in contrast with more advanced texts 
this book is filled with illustrations of the theory,
often quite detailed illustrations.

Some texts that assume a not-yet sophisticated reader
begin with matrix multiplication and determinants.
Then, when 
vector spaces and linear maps finally appear
and definitions and proofs start, the abrupt change
brings the students to an abrupt stop.
% They've been sent the wrong signals.
While this book begins with
linear reduction, from the start
we do more than compute.
The first chapter
includes proofs, such as the proof that linear reduction gives a correct and
complete solution set.
With that as motivation
the second chapter does vector spaces over the reals.
In the schedule below this happens at the start of the third week.

% Another example of the emphasis here on motivation and naturalness
% is that the chapter on linear maps
% does not begin with the definition of homomorphism.
% Instead it begins with the definition of isomorphism, which
% is natural\Dash students themselves
% observe that some spaces are ``the same'' as others.
% After that,
% the next section takes the reasonable step of 
% isolating the operation-preservation idea
% to define homomorphism.
% This loses some mathematical slickness 
% but it is a good trade because it gives to students
% a large gain in sensibility.

A student progresses most in mathematics while doing exercises. 
The problem sets start with 
routine checks and range up to reasonably involved proofs.
% Since instructors often assign about a dozen exercises
I have aimed to typically put two dozen in each set, 
thereby giving a selection.
In particular there is a good selection of the medium-difficult problems
that stretch a learner, but not too far.
At the high end, there are even a few that are puzzles
taken from various journals, competitions, or
problems collections, which  
are marked with a
`\puzzlemark'  
(as part of the fun I have tried to keet the original wording).

That is, as with the rest of the book, 
the exercises are aimed to both build an ability at,
and help students experience the pleasure of, 
\emph{doing} mathematics.
Students should see how the ideas arise and should be able to 
picture themselves doing the same type of work.


%\vspace*{.5in}
\medskip
\noindent{\bf Applications.}
%\smallskip
% The point of view taken here, that students should think of 
% Linear Algebra as about vector spaces
% and linear maps, is not taken to the complete exclusion of others.
Applications and computing are interesting and vital aspects 
of the subject.
Consequently, each chapter closes with a selection of
topics in those areas.
These give a reader
a taste of the subject, discuss how Linear Algebra comes in,
point to some further reading, and give a few exercises. 
They are brief enough that an instructor can do one
in a day's class 
or can assign them as projects for individuals or small groups.
Whether they figure formally in a course or not, they help
readers see for themselves that Linear Algebra is a tool
that a professional must have. 




\medskip
\noindent{\bf Availability.}
This book is Free.
See this book's web page 
\url{http://joshua.smcvt.edu/linearalgebra}
for the license details.
That page also has the latest version, 
exercise answers, beamer slides, lab manual, additional material,
and \LaTeX\ source.
This book is also available in a professionally printed and bound edition,
from standard publishing sources, for very little cost.
See the web page.



\medskip
\noindent{\bf Acknowledgments.}
A lesson of software
development is that complex projects have bugs,
and need a process for bug fixes.
I am grateful for reports from both instructors and students.
I periodically issue revisions, and acknowledge in the book's source
all of the reports that I use. 
My current contact information is on the web page above.

I am grateful to Saint Michael's College 
for supporting this project over many years, even before the idea of 
open educational resources became familiar.
% I also thank Gabriel S Santiago for the cover colors.
I also thank Adobe Color~CC user \texttt{claflin61} for the cover colors.

And, I cannot thank my wife Lynne enough for her unflagging encouragement.



\newcommand{\classday}[1]{\textsc{#1}}
\newcommand{\colwidth}{1.25in}

%\vspace*{.5in}
\medskip
% \noindent{\bf If you are reading this on your own.}
\noindent{\bf Advice.}
%\smallskip
%
This book's emphasis on motivation and development,
and its availability, make it widely used for self-study.
If you are an independent student then good for you, I admire your industry.
However, you may find some advice useful.

While an experienced instructor knows what subjects and
pace suit their class, this semester's timetable 
(graciously shared by George Ashline)
may help you plan a sensible rate.
It
presumes Section~One.II, the elements of vectors.
\begin{center}   % George Ashline's
   \begin{tabular}{r|*{2}{p{\colwidth}}l}
      \textit{week}  
       &\textit{Monday}          
       &\textit{Wednesday}            
       &\textit{Friday}        \\ \hline
       1    &One.I.1         &One.I.1, 2        &One.I.2, 3         \\
       2    &One.I.3         &One.III.1          &One.III.2         \\
       3    &Two.I.1         &Two.I.1, 2         &Two.I.2         \\
       4    &Two.II.1         &Two.III.1         &Two.III.2         \\
       5    &Two.III.2        &Two.III.2, 3         &Two.III.3        \\
       6    &\classday{exam}   &Three.I.1         &Three.I.1       \\
       7    &Three.I.2         &Three.I.2          &Three.II.1         \\
       8    &Three.II.1        &Three.II.2          &Three.II.2          \\
       9    &Three.III.1       &Three.III.2         &Three.IV.1, 2       \\
      10    &Three.IV.2, 3   &Three.IV.4          &Three.V.1          \\
      11    &Three.V.1       &Three.V.2            &Four.I.1         \\
      12    &\classday{exam}  &Four.I.2            &Four.III.1       \\
      13    &Five.II.1    &\multicolumn{2}{c}{\classday{--Thanksgiving break--}} \\
      14    &Five.II.1, 2     &Five.II.2          &Five.II.3        
   \end{tabular}
\end{center}
As enrichment, you might pick one or two topics that appeal to you 
from the end of each chapter, such as the ones on 
Voting Paradoxes, 
Geometry of Linear Maps, and Coupled Oscillators, or from the lab manual.
(When I teach with this schedule as a target, I find that I often have room
for a couple of days of extras.)
You'll get more from these
if you have access to software for calculations.
I recommend \textit{Sage}, freely available 
from \url{http://sagemath.org}.

In the table of contents
I have marked some subsections as optional if
some instructors will pass over them in favor of spending more time elsewhere. 

Note that 
in addition to the in-class exams,
students in the above course do 
take-home problem sets that include proofs, such as a verification
that a set is a vector space.
Computations are important but so are the arguments.

My main advice is: do many exercises.
I have marked a good sample with \recommendationmark's in the margin.
Do not simply read the answers\Dash you must
try the problems and possibly struggle with them.
For all of the exercises, you must justify your answer either with a computation
or with a proof.
Be aware that few people can write correct proofs without training;
try to find a knowledgeable person to work with you.

Finally, a caution for all students, independent or not:~I 
cannot overemphasize that the 
statement, ``I understand the material but it is only 
that I have trouble with the problems''\spacefactor=1000\ %
shows a misconception.
Being able to do things with the ideas is their entire point.
The quotes below express this sentiment admirably
(I have taken the liberty of formatting them as poetry).
They capture the essence of both the beauty and the power
of mathematics and science in general, 
and of Linear Algebra in particular.

\bigskip
\par\noindent\begin{tabular}[t]{@{}l@{}}
  \textit{I know of no better tactic}                     \\
  \textit{\ than the illustration of exciting principles} \\
  \textit{by well-chosen particulars.}                    \\
  \hspace*{1in}\textit{--Stephen Jay Gould}
\end{tabular}

\bigskip
\par\noindent
\begin{tabular}[t]{@{}l@{}}   
\textit{If you really wish to learn}                     \\
   \textit{\ you must mount a machine}  \\ 
   \textit{\ and become acquainted with its tricks} \\
   \textit{by actual trial.}                    \\
   \hspace*{1in}\textit{--Wilbur Wright}
\end{tabular}

\vspace*{3ex}
\par\ \hfill\begin{tabular}[t]{@{}l@{}}
                       Jim Hef{}feron            \\
                       Mathematics, Saint Michael's College \\ 
                       Colchester, Vermont\ USA 05439  \\     
                       \url{http://joshua.smcvt.edu/linearalgebra} \\
                       % Date used in both book and answers for current version
2016-May-31
                    \end{tabular}

\vspace{3ex plus 1fill}
\par\noindent\textit{Author's Note.}
Inventing a good exercise, one that enlightens as well as tests, 
is a creative act, and hard work.
The inventor deserves recognition.
But texts have traditionally not given attributions for
questions.
I have changed that here where I was sure of the source.
I would be glad to hear from anyone who can help me to correctly
attribute others of the questions.   
} % ends the open curly for the parskip from the top of this file
