% Chapter 3, Topic _Linear Algebra_ Jim Hefferon
%  http://joshua.smcvt.edu/linearalgebra
%  2001-Jun-12
\topic{Geometry of Linear Maps} 
\index{Geometry of Linear Maps|(}
%The geometry of linear maps %$\map{h}{\Re^n}{\Re^m}$ 
%is appealing both for its simplicity and for its usefulness.
These pairs of pictures contrast the geometric action of the nonlinear maps
\( f_1(x)=e^x \) and \( f_2(x)=x^2 \) 
\begin{center}
  \includegraphics{ch3.46}
  \hspace*{4em}
  \includegraphics{ch3.47}
\end{center}
with the linear maps
\( h_1(x)=2x \) and \( h_2(x)=-x \).
\begin{center}
  \includegraphics{ch3.48}
  \hspace*{4em}
  \includegraphics{ch3.49}
\end{center}
Each of the four pictures shows the domain $\Re$ on the left 
mapped to the codomain $\Re$ on the right. 
Arrows trace where each map sends
$x=0$, $x=1$, $x=2$, $x=-1$, and $x=-2$.

The nonlinear maps distort
the domain in transforming it into the range.
For instance,
\( f_1(1) \) is further from
$f_1(2)$ than it is from $f_1(0)$ \Dash  this map spreads
the domain out unevenly so that a domain interval near $x=2$ is 
spread apart more 
than is a domain interval near $x=0$.
The linear maps are nicer, more regular, 
in that for each map all of the domain 
spreads by the same factor.
The map~$h_1$ on the left spreads all intervals apart to be twice as wide 
while on the right~$h_2$ keeps intervals the same length but reverses
their orientation, as with the rising interval from $1$ to $2$ 
being transformed 
to the falling interval from $-1$ to~$-2$.

The only linear maps from $\Re$ to $\Re$ are multiplications by a scalar but
in higher dimensions more can happen. 
For instance, this linear transformation of $\Re^2$
rotates vectors counterclockwise.\index{rotation}\index{linear map!rotation}
\begin{center}
  \includegraphics{ch3.50}
\end{center}
The transformation of $\Re^3$ 
that projects vectors into the $xz$-plane
is also not simply a rescaling.
\begin{center}
 \includegraphics{ch3.51}
\end{center}

Despite this additional variety, 
even in higher dimensions linear maps behave nicely.
Consider a linear $\map{h}{\Re^n}{\Re^m}$ and
use the standard bases to represent it by a matrix $H$.
Recall from Theorem~V.\ref{th:CanonFormForMatEquiv} 
that $H$ factors into $H=PBQ$ 
where $P$ and $Q$ are nonsingular and $B$ is a partial-identity matrix.
Recall also that nonsingular matrices
factor into elementary 
matrices\index{matrix!elementary reduction}\index{elementary reduction matrix}
$PBQ=T_nT_{n-1}\cdots T_sBT_{s-1}\cdots T_1$,
which are matrices that
come from the identity $I$ after one Gaussian row operation,
so each $T$ matrix is one of these three kinds
\begin{equation*}
  I\grstep{k\rho_i}M_i(k) 
  \qquad 
  I\grstep{\rho_i\leftrightarrow\rho_j}P_{i,j}  
  \qquad
  I\grstep{k\rho_i+\rho_j}C_{i,j}(k) 
\end{equation*}
with $i\neq j$, $k\neq 0$.
So if we understand the geometric effect of a linear map described
by a partial-identity matrix and the effect of the linear maps
described by the elementary matrices then we will in some sense
completely understand the effect of any linear map.
(The pictures below stick to transformations of $\Re^2$ for ease of drawing
but the principles extend for maps from any $\Re^n$ to any $\Re^m$.)

The geometric effect of the linear transformation represented by a  
partial-identity matrix is projection.
\begin{equation*}
  \colvec{x \\ y  \\ z}
  \quad\xrightarrow{{\text{\tiny$\displaystyle\left(\begin{array}[r]{@{}c@{\hspace{0.8em}}c@{\hspace{0.8em}}c@{}}
                1  &0  &0 \\
                0  &1  &0 \\
                0  &0  &0
          \end{array}\right)$}}}
% \begin{mat}[r]
%       1  &0  &0   \\
%       0  &1  &0   \\
%       0  &0  &0   
%     \end{mat}} % _{\stdbasis_3,\stdbasis_3}}
  \qquad
  \colvec{x \\ y  \\ 0}
\end{equation*}

The geometric effect of the $M_i(k)$ matrices
is to  
stretch vectors by a factor of $k$ along the $i$-th axis.
This map stretches by a factor of $3$ along the $x$-axis.
\begin{center}
  \includegraphics{ch3.52}
\end{center}
If $0\leq k<1$ or if $k<0$ then the $i$-th
component goes the other way, here to the left.
\begin{center}
  \includegraphics{ch3.53}
\end{center}
Either of these stretches is a 
\definend{dilation}.\index{dilation}\index{linear map!dilation}

A transformation represented by a $P_{i,j}$ matrix
interchanges the $i$-th and $j$-th axes.
This is \definend{reflection}\index{reflection}\index{linear map!reflection} 
about the line $x_i=x_j$.
\begin{center}
  \includegraphics{ch3.54}
\end{center}
Permutations involving more than two axes decompose into a combination 
of swaps of pairs of axes; see \nearbyexercise{exer:PermIsCompSwaps}.

The remaining  matrices have the form $C_{i,j}(k)$.
For instance $C_{1,2}(2)$ performs $2\rho_1+\rho_2$. 
\begin{equation*}
  \colvec{x  \\  y}\quad
  \xrightarrow{\text{\tiny$\displaystyle\left(\begin{array}[r]{@{}c@{\hspace{0.8em}}c@{}}
                1  &0  \\
                2  &1  
          \end{array}\right)$}}  % _{\stdbasis_2,\stdbasis_2}}
   \quad\colvec{x \\ 2x+y}
\end{equation*}
In the picture below, 
the vector $\vec{u}$ with the first component of $1$ is affected less 
than the vector $\vec{v}$ with the first component of $2$.
The vector $\vec{u}$ is mapped to a
$h(\vec{u})$ that is only $2$ higher than $\vec{u}$ while 
$h(\vec{v})$ is $4$ higher than $\vec{v}$.
\begin{center}
  \includegraphics{ch3.55}
\end{center}
Any vector with a first component of $1$ would be affected 
in the same way as $\vec{u}$:
it would slide up by $2$.
And any vector with a first component of $2$ would slide up $4$, 
as was $\vec{v}$.
That is, the transformation represented by 
$C_{i,j}(k)$ affects vectors depending on their $i$-th component.

Another way to see this point is to consider the action of this map 
on the unit square.
In the next picture,
vectors with a first component of $0$, such as the origin, are not pushed 
vertically at all but vectors with a positive first component slide up.
Here, all vectors with a first component of $1$, the entire 
right side of the square, slide to the same extent.
In general, vectors on the same vertical line slide by the same amount,
by twice their first component.
The resulting shape has the same base and height as the square
(and thus the same area) but the right angle corners are gone.
\begin{center}
  \includegraphics{ch3.56}
\end{center}

For contrast, the next picture shows the effect of the map represented by 
$C_{2,1}(2)$.
Here vectors are affected according to their  
second component:
$\binom{x}{y}$ slides horizontally by twice $y$.
\begin{center}
  \includegraphics{ch3.57}
\end{center}
In general, for any $C_{i,j}(k)$, the
sliding happens so that vectors with the same $i$-th component
are slid by the same amount.
This kind of map is a 
\definend{shear}.\index{shear}\index{linear map!shear}

With that we understand the geometric effect of the four types 
of matrices on the right-hand side of
$H=T_nT_{n-1}\cdots T_jBT_{j-1}\cdots T_1$
and so in some sense we understand  
the action of any matrix~$H$.
Thus, even in higher dimensions the geometry of linear maps is easy: it 
is built by putting
together a number of components, each of which acts in a simple way.

We will apply this understanding in two ways.
The first way is to prove something general about 
the geometry of linear maps. 
Recall that under a linear map, the image of a subspace is a subspace
and thus the linear transformation $h$ represented by $H$ maps lines 
through the origin to lines through the origin.
(The dimension of the image space cannot be greater than 
the dimension of the domain space, so a line can't map onto, say, a plane.)
We will show that $h$ maps any line\Dash not just one through the origin\Dash 
to a line.
The proof is simple:
the partial-identity projection $B$ and the elementary $T_i$'s
each turn a line input into a line output; 
verifying the four cases is \nearbyexercise{exer:ImageLinSurIsLinSur}.
Therefore their composition also preserves lines.
% Thus, by understanding its components we can understand arbitrary square 
% matrices $H$, in the sense that we can prove things about them.

The second way that we will apply 
the geometric understanding of linear maps 
is to elucidate a point from Calculus.
Below is a picture
of the action of the one-variable real function \( y(x)=x^2+x \).
As with the nonlinear functions pictured earlier, 
the geometric effect of this map is
irregular in that at different domain points it has different effects; for
example as the input~$x$ goes from $2$ to $-2$, the associated output~$f(x)$ 
at first decreases, then pauses for an instant,
and then increases.
\begin{center}
  \includegraphics{ch3.58}
\end{center}
But in Calculus we focus less on the map overall and more 
on the local effect of the map.
Below we look closely at what this map
does near $x=1$.
The derivative is $dy/dx=2x+1$
so that near \( x=1 \) 
we have \( \Delta y\approx 3\cdot\Delta x \).
%; in other words, \( (1.001^2+1.001)-(1^2+1)\approx 3\cdot (0.001) \).
That is, in a neighborhood of $x=1$,
in carrying the domain over this map causes it to grow by
a factor of $3$ \Dash  it is, locally, 
approximately, a dilation.
%The map $y$ is locally regular.
The picture below shows this as a small interval 
in the domain $(1-\Delta x\,..\,1+\Delta x)$
carried over to an interval in the codomain $(2-\Delta y\,..\,2+\Delta y)$
that is three times as wide. % $\Delta y \approx 3\cdot \Delta x$.
\begin{center}
  \includegraphics{ch3.59}
\end{center}
% (When the above picture is drawn in the traditional cartesian way
% then the prior sentence about the rate of growth of $y(x)$ is usually
% stated:~the derivative $3$ is the slope of the 
% line tangent to the graph at the point $(1,2)$.)

In higher dimensions the core idea is the same but more can happen.
For a function \( \map{y}{\Re^n}{\Re^m} \) and a point \( \vec{x}\in\Re^n \),
the derivative is defined to be the 
linear map \( \map{h}{\Re^n}{\Re^m} \) that best approximates
how \( y \) changes near \( y(\vec{x}) \).
So the geometry described above directly applies to the derivative.
%Calculus considers the map 
%that locally approximates the change \( \Delta x\mapsto 3\cdot\Delta x \)
%(instead of the actual change map
%\( \Delta x\mapsto y(1+\Delta x)-y(1) \)) because the local map 
%is easier.
%It is easier in that, if the
%input change is doubled, or tripled, etc., then the
%resulting output change will double, or triple, etc.
%\begin{equation*}
%  3(r\,\Delta x)=r\,(3\Delta x)
%\end{equation*}
%(for $r\in\Re$), 
%and it is easier in that 
%adding changes in input adds the resulting output changes.
%\begin{equation*}
%  3(\Delta x_1+\Delta x_2)=3\Delta x_1+3\Delta x_2
%\end{equation*}
%In short, what's
%easy about \( \Delta x\mapsto 3\cdot\Delta x \) is that
%it is linear.

We close by remarking how
this point of view makes clear an often misunderstood  
result about derivatives, the Chain Rule.
Recall that, under suitable conditions on the two functions,
the derivative of the composition is this.
\begin{equation*}
  \frac{d\,(\composed{g}{f})}{dx}(x) = 
  \frac{dg}{dx}(f(x))\cdot\frac{df}{dx}(x)
\end{equation*} 
For instance the derivative of $\sin(x^2+3x)$ is
$\cos(x^2+3x)\cdot(2x+3)$.

Where does this come from?
Consider $\map{f,g}{\Re}{\Re}$. 
\begin{center}
  \includegraphics{ch3.60}
\end{center}
The first map $f$ dilates the neighborhood of $x$ by a factor of 
\begin{equation*}
  \frac{df}{dx}(x) 
\end{equation*}
and the second map $g$ follows that by dilating 
a neighborhood of $f(x)$ by a factor of 
\begin{equation*}
  \frac{dg}{dx}(\,f(x)\,) 
\end{equation*}
and when combined, 
the composition dilates by the product of the two.
In higher dimensions 
the map expressing how a function changes near a point is a linear map,
and is represented by a matrix.
The Chain Rule multiplies the matrices.

% Thus, the geometry of linear maps $\map{h}{\Re^n}{\Re^m}$ 
% is appealing both for its simplicity and for its usefulness.

\begin{exercises}
  \item 
    Use the $H=PBQ$ decomposition to find
    the combination of dilations, flips, skews, and projections
    that produces the map $\map{h}{\Re^3}{\Re^3}$ 
    represented with respect to the standard bases by this matrix.
    \begin{equation*}
      H=\begin{mat}[r]
        1  &2  &1  \\
        3  &6  &0  \\
        1  &2  &2
      \end{mat}
    \end{equation*}
    \begin{answer}
      This Gaussian reduction
      \begin{multline*}
        \grstep[-\rho_1+\rho_3]{-3\rho_1+\rho_2}
        \begin{mat}[r]
          1  &2  &1  \\
          0  &0  &-3 \\
          0  &0  &1
        \end{mat}
        \grstep{(1/3)\rho_2+\rho_3}
        \begin{mat}[r]
          1  &2  &1  \\
          0  &0  &-3 \\
          0  &0  &0
        \end{mat}                                     \\
        \grstep{(-1/3)\rho_2}
        \begin{mat}[r]
          1  &2  &1  \\
          0  &0  &1 \\
          0  &0  &0
        \end{mat}
        \grstep{-\rho_2+\rho_1}
        \begin{mat}[r]
          1  &2  &0  \\
          0  &0  &1 \\
          0  &0  &0
        \end{mat}
      \end{multline*}
      gives the reduced echelon form of the matrix.
      Now the two column operations of taking $-2$ times the first column 
      and adding it to the second, and then of swapping columns two and three
      produce this partial identity. 
      \begin{equation*} 
        B=\begin{mat}[r]
          1  &0  &0  \\
          0  &1  &0  \\ 
          0  &0  &0
        \end{mat}
      \end{equation*}
      All of that translates into matrix terms as:~where
      \begin{equation*}
        P=
        \begin{mat}[r]
          1  &-1    &0  \\
          0  &1     &0  \\
          0  &0     &1         
        \end{mat}
        \begin{mat}[r]
          1  &0    &0  \\
          0  &-1/3 &0  \\
          0  &0    &1
        \end{mat}
        \begin{mat}[r]
          1  &0    &0  \\
          0  &1    &0  \\
          0  &1/3  &1         
        \end{mat}
        \begin{mat}[r]
          1  &0  &0  \\
          0  &1  &0  \\
         -1  &0  &1         
        \end{mat}
        \begin{mat}[r]
          1  &0  &0  \\
         -3  &1  &0  \\
          0  &0  &1         
        \end{mat}
      \end{equation*}
      and 
      \begin{equation*}
        Q=
        \begin{mat}[r]
          1  &-2    &0  \\
          0  &1     &0  \\
          0  &0     &1         
        \end{mat}
        \begin{mat}[r]
          0  &1     &0  \\
          1  &0     &0  \\
          0  &0     &1         
        \end{mat}
      \end{equation*}
      the given matrix factors as $PBQ$.
    \end{answer}
  \item 
    What combination of dilations, flips, skews, and projections
    produces a rotation counterclockwise by $2\pi/3$ radians?
    \begin{answer}
      We will first represent the map with a matrix $H$,
      perform the row operations and, if needed, column operations
      to reduce it to a partial-identity matrix.
      We will then translate that into a factorization $H=PBQ$.
      Substituting into the general matrix
          \begin{equation*}
            \rep{r_\theta}{\stdbasis_2,\stdbasis_2}
            \begin{mat}
              \cos\theta  &-\sin\theta  \\
              \sin\theta  &\cos\theta
            \end{mat}
          \end{equation*}
          gives this representation.
          \begin{equation*}
            \rep{r_{2\pi/3}}{\stdbasis_2,\stdbasis_2}
            \begin{mat}[r]
              -1/2        &-\sqrt{3}/2  \\
              \sqrt{3}/2  &-1/2
            \end{mat}
          \end{equation*}
          Gauss's Method is routine.
          \begin{equation*}
            \grstep{\sqrt{3}\rho_1+\rho_2}
            \begin{mat}[r]
              -1/2        &-\sqrt{3}/2  \\
               0          &-2
            \end{mat}
            \grstep[(-1/2)\rho_2]{-2\rho_1}
            \begin{mat}[r]
               1          &\sqrt{3}    \\
               0          &1
            \end{mat}
            \grstep{-\sqrt{3}\rho_2+\rho_1}
            \begin{mat}[r]
               1          &0   \\
               0          &1
            \end{mat}
          \end{equation*}
          That translates to a matrix equation in this way.
          \begin{equation*}
            \begin{mat}[r]
              1  &-\sqrt{3}  \\
              0  &1
            \end{mat}
            \begin{mat}[r]
              -2  &0    \\
               0  &-1/2
            \end{mat}
            \begin{mat}[r]
               1         &0  \\
               \sqrt{3}  &1
            \end{mat}
            \begin{mat}[r]
              -1/2        &-\sqrt{3}/2  \\
              \sqrt{3}/2  &-1/2
            \end{mat}
            =I
          \end{equation*}
          Taking inverses to solve for $H$ yields this factorization.
          \begin{equation*}
            \begin{mat}[r]
              -1/2        &-\sqrt{3}/2  \\
              \sqrt{3}/2  &-1/2
            \end{mat}
            =
            \begin{mat}[r]
                1         &0  \\
               -\sqrt{3}  &1
            \end{mat}
            \begin{mat}[r]
              -1/2  &0    \\
               0    &-2
            \end{mat}
            \begin{mat}[r]
              1  &\sqrt{3}  \\
              0  &1
            \end{mat}
            I
          \end{equation*}
    \end{answer}
  \item 
    If a map is nonsingular then to get from its representation to
    the identity matrix we do not need any column operations,
    so that in $H=PBQ$ the matrix $Q$ is the identity.
    An example of a nonsingular map is 
    the transformation $\map{t_{-\pi/4}}{\Re^2}{\Re^2}$ that rotates
    vectors clockwise by $\pi/4$~radians.
    \begin{exparts}
      \partsitem Find the matrix $H$ representing 
         this map with respect to the standard bases.
      \partsitem Use Gauss-Jordan to reduce $H$ to the identity,
        without column operations.
      \partsitem
        Translate that to a matrix equation
        $T_jT_{j-1}\cdots T_1H=I$.
      \partsitem Solve the matrix equation for $H$.
      \partsitem Describe $H$ as a 
        combination of dilations, flips, skews, and projections
        (the identity is a trivial projection). 
    \end{exparts}
    \begin{answer}
      \begin{exparts}
        \partsitem  Recall that rotation counterclockwise by 
          $\theta$~radians is represented with respect to the standard basis
          in this way.
          \begin{equation*}
            \rep{t_{\pi/4}}{\stdbasis_2,\stdbasis_2}
            =\begin{mat}
              \cos\theta  &-\sin\theta  \\
              \sin\theta  &\cos\theta
             \end{mat}
          \end{equation*}
          A clockwise angle is the negative of a counterclockwise
          one.  
          \begin{equation*}
            \rep{t_{-pi/4}}{\stdbasis_2,\stdbasis_2}
            =\begin{mat}
              \cos(-\pi/4)  &-\sin(-\pi/4)  \\
              \sin(-\pi/4)  &\cos(-\pi/4)
            \end{mat}
            =\begin{mat}[r]
              \sqrt{2}/2  &\sqrt{2}/2  \\
              -\sqrt{2}/2 &\sqrt{2}/2
            \end{mat}
          \end{equation*}
        \partsitem
          This Gauss-Jordan reduction
          \begin{equation*}
            \grstep{\rho_1+\rho_2}
            \begin{mat}[r]
              \sqrt{2}/2  &\sqrt{2}/2  \\
              0           &\sqrt{2}
            \end{mat}
            \grstep[(1/\sqrt{2})\rho_2]{(2/\sqrt{2})\rho_1}
            \begin{mat}[r]
              1  &1  \\
              0  &1
            \end{mat}
            \grstep{-\rho_2+\rho_1}
            \begin{mat}[r]
              1  &0  \\
              0  &1
            \end{mat}
          \end{equation*}
          produces the identity matrix. 
          Thus we do not need column-swapping operations
          to end with a partial-identity.
        \partsitem In matrix multiplication the reduction is
          \begin{equation*}
            \begin{mat}[r]
              1  &-1 \\
              0  &1
            \end{mat}
            \begin{mat}[r]
              2/\sqrt{2}  &0         \\
              0           &1/\sqrt{2}
            \end{mat}
            \begin{mat}[r]
              1  &0 \\
              1  &1
            \end{mat}
            H
            =I
          \end{equation*}
          (note that composition of the Gaussian operations is 
          from right to left).
        \partsitem  Taking inverses 
          \begin{equation*}
            H
            =
            \underbrace{
              \begin{mat}[r]
                1  &0 \\
                -1  &1
              \end{mat}
              \begin{mat}[r]
                \sqrt{2}/2  &0         \\
                0           &\sqrt{2}
              \end{mat}
              \begin{mat}[r]
                1  &1 \\
                0  &1
              \end{mat}
             }_P
            I
          \end{equation*}
          gives the desired factorization of $H$. 
          The partial
          identity is $I$.
          % , and $Q$ is trivial, that is, it is also an identity
          % matrix.
        \partsitem Reading the composition from right to left (and ignoring the
          identity matrices as trivial) gives that $H$ has the same
          effect as first performing this skew 
          \begin{center}
            \includegraphics{ch3.95}
         \end{center}
         followed by a dilation that multiplies all first components by 
         $\sqrt{2}/2$ (this is a shrink in that $\sqrt{2}/2\approx0.707$ 
         is less than $1$) 
         and all second components by $\sqrt{2}$,
         followed by another skew. 
          \begin{center}
            \includegraphics{ch3.96}
         \end{center}
         For an example we start with the unit vector whose angle with
         the $x$-axis is $\pi/6$ and apply the components of $H$ in turn.
          \begin{center}
            \includegraphics{ch3.97}
         \end{center}
         We can easily verify that the resulting vector has unit length 
         and forms an angle with the $x$-axis of $-\pi/12$, which is indeed
         a rotation clockwise of $\pi/4$ radians 
         since $(\pi/6)-(\pi/4)=-\pi/12$. 
      \end{exparts}
    \end{answer}
  \item \label{exer:RToRIsScalMult} 
    Show that any linear transformation of $\Re^1$ is a map $h_k$
    that multiplies by a scalar $x\mapsto kx$.
    \begin{answer}
      Represent it with respect to the 
      standard bases $\stdbasis_1,\stdbasis_1$. 
      That produces a $\nbyn{1}$ matrix.
      The only entry is the scalar~$k$.
    \end{answer}
  \item \label{exer:ImageLinSurIsLinSur} 
    Show that linear maps preserve the linear structures of a space.
    \begin{exparts}
      \partsitem Show that for any linear map from $\Re^n$ to $\Re^m$,
         the image of any line is a line.
         The image may be a degenerate line, that is, a single point.
      \partsitem Show that the image of any linear surface is a linear surface.
         This generalizes the result that under a linear map the image of
         a subspace is a subspace.
      \partsitem Linear maps preserve other linear ideas.
         Show that linear maps preserve ``betweeness'':~if the point
         $B$ is between $A$ and $C$ then the image of $B$ is between the
         image of $A$ and the image of $C$.
    \end{exparts}
    \begin{answer}
      \begin{exparts}
        \partsitem A line is a subset of $\Re^n$ of the form
          $\set{\vec{v}=\vec{u}+t\cdot\vec{w}\suchthat t\in\Re}$.
          The image of a point on that line is 
          $h(\vec{v})=h(\vec{u}+t\cdot\vec{w})=h(\vec{u})+t\cdot h(\vec{w})$,
          and the set of such vectors, as $t$ ranges over the reals, is
          a line (albeit, degenerate if $h(\vec{w})=\zero$).
        \partsitem This is an obvious extension of the prior argument.
        \partsitem If the point~$B$ is between the points~$A$ and~$C$ then the
          line from $A$ to $C$ has $B$ in it.
          That is, there is a $t\in (0\,..\,1)$ such that
          $\vec{b}=\vec{a}+t\cdot (\vec{c}-\vec{a})$ (where $B$ is the
          endpoint of $\vec{b}$, etc.).
          Now, as in the argument of the first item, linearity shows that
          $h(\vec{b})=h(\vec{a})+t\cdot h(\vec{c}-\vec{a})$.  
      \end{exparts}
    \end{answer}
  \item 
    Use a picture like the one 
    that appears in the discussion of the Chain Rule
    to answer:~if a function $\map{f}{\Re}{\Re}$ has an inverse,
    what's the relationship between how the function \Dash locally, 
    approximately \Dash  dilates space, and
    how its inverse dilates space (assuming, of course, that it has an 
    inverse)?
    \begin{answer}
      The two are inverse.
      For instance, for a fixed $x\in\Re$,
      if $f^\prime (x)=k$ (with $k\neq 0$) then 
      $(f^{-1})^\prime (x)=1/k$.
      \begin{center}
        \includegraphics{ch3.98}
     \end{center}
    \end{answer}
  \item \label{exer:PermIsCompSwaps}
    Show that any permutation, any reordering, $p$ of the numbers
    $1$, \ldots, $n$, the map 
    \begin{equation*}
      \colvec{x_1 \\ x_2 \\ \vdots\\ x_n}
      \mapsto
      \colvec{x_{p(1)} \\ x_{p(2)} \\ \vdots \\ x_{p(n)}}
    \end{equation*}
    can be done with a composition of maps, 
    each of which only swaps a single pair of coordinates.
    \textit{Hint:} you can use induction on $n$.
    (\textit{Remark:}~in the fourth chapter we will show this and we will also 
    show that the parity of the number of swaps used is determined by $p$.
    That is, although a particular
    permutation could be expressed in two different ways
    with two different numbers of swaps, either both ways use an even number of
    swaps, or both use an odd number.)
    \begin{answer}
      We can show this by induction on the number of components in the 
      vector.
      In the $n=1$ base case the only permutation is the trivial one,
      and the map
      \begin{equation*}
        \colvec{x_1}
        \mapsto
        \colvec{x_1}
      \end{equation*}
      is expressible as a composition of swaps\Dash as zero swaps.
      For the inductive step we assume that the map induced by 
      any permutation of fewer than
      $n$ numbers can be expressed with swaps only, and we consider the map
      induced by a 
      permutation $p$ of $n$ numbers.
      \begin{equation*}
        \colvec{x_1 \\ x_2 \\ \vdots \\ x_n}
        \mapsto
        \colvec{x_{p(1)} \\ x_{p(2)} \\ \vdots \\ x_{p(n)}}
      \end{equation*}
      Consider the number~$i$ such that $p(i)=n$.
      The map      
      \begin{equation*}
        \colvec{x_1      \\ x_2      \\ \vdots \\ x_i      \\ \vdots \\ x_n}
        \mapsunder{\hat{p}}
        \colvec{x_{p(1)} \\ x_{p(2)} \\ \vdots \\ x_{p(n)} \\ \vdots  \\ x_{n}}
      \end{equation*}
      will, when followed by the swap of the $i$-th and $n$-th components, 
      give the map~$p$.
      Now, the inductive hypothesis gives that $\hat{p}$ is achievable as 
      a composition of swaps.
    \end{answer}
\end{exercises}
\index{Geometry of Linear Maps|)}
\endinput


