% Chapter 3, Section 3 _Linear Algebra_ Jim Hefferon
%  http://joshua.smcvt.edu/linearalgebra
%  2001-Jun-12
\section{Computing Linear Maps}
The prior section shows that
a linear map is determined by its action on a basis.
The equation
\begin{equation*}
  h(\vec{v})
  =h(c_1\cdot\vec{\beta}_1+\dots+c_n\cdot\vec{\beta}_n)
  =c_1\cdot h(\vec{\beta}_1)+\dots +c_n\cdot h(\vec{\beta}_n)
\tag*{}\end{equation*}
describes how we get the value of the map on any vector $\vec{v}$
by starting 
from the value of the map on the vectors $\vec{\beta}_i$ in a basis 
and extending linearly.
% We just need to  
% find the $c$'s to express $\vec{v}$ with respect to the basis.

This section gives a convenient scheme based on matrices
to use the representations of 
\( h(\vec{\beta}_1) \), \ldots, \( h(\vec{\beta}_n) \)
to compute, from the representation of a vector in the domain 
$\rep{\vec{v}}{B}$,
the representation of that vector's image in the codomain 
$\rep{h(\vec{v})}{D}$.














\subsection{Representing Linear Maps with Matrices}
\index{homomorphism!matrix representing|(}
\begin{example}  \label{ex:TypLinMapRepByMat}
For the spaces $\Re^2$ and $\Re^3$ fix these bases.
\begin{equation*}
   B=\sequence{
               \colvec[r]{2 \\ 0},
               \colvec[r]{1 \\ 4}}
   \qquad
   D=\sequence{
               \colvec[r]{1 \\ 0 \\ 0},
               \colvec[r]{0 \\ -2 \\ 0},
               \colvec[r]{1 \\ 0 \\ 1}}
\end{equation*}
Consider the map $\map{h}{\Re^2}{\Re^3}$ that is determined by this association.
\begin{equation*}
  \colvec[r]{2 \\ 0}
    \mapsunder{h}
  \colvec[r]{1 \\ 1 \\ 1}
  \qquad
  \colvec[r]{1 \\ 4}
    \mapsunder{h}
  \colvec[r]{1 \\ 2 \\ 0}
\end{equation*}
To compute the action of this map on any vector at all from the domain
we first represent the vector $h(\vec{\beta}_1)$
\begin{equation*}
  \colvec[r]{1 \\ 1 \\ 1}=
         0\colvec[r]{1 \\ 0 \\ 0}
         -\frac{1}{2}\colvec[r]{0 \\ -2 \\ 0}
         +1\colvec[r]{1 \\ 0 \\ 1}
  \qquad
   \rep{ h(\vec{\beta}_1) }{D}=\colvec[r]{0 \\ -1/2 \\ 1}_D           
\end{equation*}
and the vector $h(\vec{\beta}_2)$.
\begin{equation*}
  \colvec[r]{1 \\ 2 \\ 0}=
        1\colvec[r]{1 \\ 0 \\ 0}
        -1\colvec[r]{0 \\ -2 \\ 0}
        +0\colvec[r]{1 \\ 0 \\ 1}
  \qquad
  \rep{ h(\vec{\beta}_2) }{D}=\colvec[r]{1 \\ -1 \\ 0}_D
\end{equation*}
With these, for any member $\vec{v}$ of the domain
we can compute $h(\vec{v})$.
\begin{align*}
  h(\vec{v})
  &=h(c_1\cdot \colvec[r]{2 \\ 0}+c_2\cdot \colvec[r]{1 \\ 4})   \\
  &=c_1\cdot h(\colvec[r]{2 \\ 0})+c_2\cdot h(\colvec[r]{1 \\ 4}) \\
  &=c_1\cdot (
        0\colvec[r]{1 \\ 0 \\ 0}
        \!-\frac{1}{2}\colvec[r]{0 \\ -2 \\ 0}
        \!+1\colvec[r]{1 \\ 0 \\ 1}\!  )
  +c_2\cdot (
        1\colvec[r]{1 \\ 0 \\ 0}
        \!-1\colvec[r]{0 \\ -2 \\ 0}
        \!+0\colvec[r]{1 \\ 0 \\ 1}\!  )      \\
  &=(0c_1+1c_2)\cdot \colvec[r]{1 \\ 0 \\ 0}
   +(-\frac{1}{2}c_1-1c_2)\cdot \colvec[r]{0 \\ -2 \\ 0}
   +(1c_1+0c_2)\cdot \colvec[r]{1 \\ 0 \\ 1}
\end{align*}
Thus,
\begin{center}
  if $\rep{\vec{v}}{B}=\colvec{c_1 \\ c_2}$
  then $\rep{\,h(\vec{v})\,}{D}
  =\colvec{0c_1+1c_2 \\ -(1/2)c_1-1c_2 \\ 1c_1+0c_2}$.
\end{center}
For instance, 
\begin{center}
  since $\rep{\colvec[r]{4 \\ 8}}{B}=\colvec[r]{1 \\ 2}_B$
  we have
  $\rep{\,h(\colvec[r]{4 \\ 8})\,}{D}
   =\colvec[r]{2 \\ -5/2 \\ 1}$.
\end{center}
\end{example}

We express computations like the one above with a matrix notation.
\begin{equation*}
    \begin{mat}[r]
      0             &1  \\
      -1/2          &-1  \\
      1             &0
    \end{mat}_{B,D}
  \colvec{c_1 \\ c_2}_B
  =
  \colvec{0c_1+1c_2 \\ (-1/2)c_1-1c_2 \\ 1c_1+0c_2}_D
\end{equation*}
In the middle is the argument $\vec{v}$ to the map, 
represented with respect to the domain's basis $B$
by the column vector with components $c_1$ and $c_2$.
On the right is the value of the map on that argument $h(\vec{v})$,
represented with respect to the codomain's basis $D$.
The matrix on the left is the new thing.
We will use it to represent the map and we will think of 
the above equation as representing an application of the map to the matrix.

That matrix consists of the coefficients from the vector on the right,
$0$ and $1$ from the first row, $-1/2$ and $-1$ from the
second row, and $1$ and $0$ from the third row.
That is, we make it by adjoining the vectors 
representing the $h(\vec{\beta}_i)$'s. 
\begin{equation*}
  \left(\begin{array}{c|c}
     \vdots                         &\vdots    \\
     \rep{\,h(\vec{\beta}_1)\,}{D}  &\rep{\,h(\vec{\beta}_2)\,}{D}  \\
     \vdots                         &\vdots
  \end{array}\right)
\end{equation*}

\begin{definition} \label{def:MatRepMap}
%<*df:MatRepMap>
Suppose that \( V \) and \( W \) are vector spaces of dimensions \( n \) and
\( m \) with bases \( B \) and \( D \),
and that \( \map{h}{V}{W} \) is a linear map.
If
\begin{equation*}
  \rep{h( \vec{\beta}_1 )}{D}
  =
  \colvec{h_{1,1} \\ h_{2,1} \\ \vdots \\ h_{m,1}}_D
  \quad\ldots\quad
  \rep{h( \vec{\beta}_n )}{D}
  =
  \colvec{h_{1,n} \\ h_{2,n} \\ \vdots \\ h_{m,n}}_D
\end{equation*}
then 
\begin{equation*}
  \rep{h}{B,D}=\generalmatrix{h}{n}{m}_{B,D}
\end{equation*}
is the \definend{matrix representation of \( h \) with respect to \( B, D \)}.%
\index{representation!of a matrix}\index{matrix!representation}%
\index{homomorphism!matrix representing}
%</df:MatRepMap>
\end{definition}

\noindent In that matrix the number of columns~$n$ is the 
dimension of the map's domain while  
the number of rows~$m$ is the dimension of the codomain.

We use lower case letters for a map, 
upper case for the matrix,
and lower case again for the entries of the matrix.
Thus for the map \( h \), the matrix representing it is \( H \), with
entries \( h_{i,j} \).

\begin{example}  \label{ex:PolyOneToRThree}
If \( \map{h}{\Re^3}{\polyspace_1} \) is
\begin{equation*}
  \colvec{a_1 \\ a_2 \\ a_3}
     \mapsunder{h}
   (2a_1+a_2)+(-a_3)x
\end{equation*}
then where
\begin{equation*}
  B=
  \sequence{\colvec[r]{0 \\ 0 \\ 1},
            \colvec[r]{0 \\ 2 \\ 0},
            \colvec[r]{2 \\ 0 \\ 0} }
  \qquad
  D=
  \sequence{1+x,-1+x}
\end{equation*}
the action of \( h \) on \( B \) is this.
\begin{equation*}
  \colvec[r]{0 \\ 0 \\ 1}\mapsunder{h}-x
  \qquad \colvec[r]{0 \\ 2 \\ 0}\mapsunder{h}2
  \qquad \colvec[r]{2 \\ 0 \\ 0}\mapsunder{h}4
\end{equation*}
A simple calculation
\begin{equation*}
  \rep{-x}{D}=\colvec[r]{-1/2 \\ -1/2}_D
  \quad \rep{2}{D}=\colvec[r]{1 \\ -1}_D 
  \quad \rep{4}{D}=\colvec[r]{2 \\ -2}_D 
\end{equation*}
shows that this is the matrix representing $h$ with respect to the bases.
\begin{equation*}
  \rep{h}{B,D}
   =
   \begin{mat}[r]
     -1/2  &1   &2  \\
     -1/2  &-1  &-2
   \end{mat}_{B,D}
\end{equation*}
\end{example}

\begin{theorem} \label{th:MatMultRepsFuncAppl}
%<*th:MatMultRepsFuncAppl>
Assume that \( V \) and \( W \) are vector spaces
of dimensions \( n \) and \( m \)
with bases \( B \) and \( D \),
and that \( \map{h}{V}{W} \) is a linear map.
If \( h \) is represented by
\begin{equation*}
  \rep{h}{B,D}=\generalmatrix{h}{n}{m}_{B,D}
\end{equation*}
and \( \vec{v}\in V \) is represented by
\begin{equation*}
  \rep{\vec{v}}{B}=\colvec{c_1 \\ c_2 \\ \vdots \\ c_n}_B
\end{equation*}
then the representation of the image of $\vec{v}$ is this.
\begin{equation*}
  \rep{\, h(\vec{v}) \,}{D}
  =
  \colvec{h_{1,1}c_1+h_{1,2}c_2+\dots+h_{1,n}c_n \\
          h_{2,1}c_1+h_{2,2}c_2+\dots+h_{2,n}c_n \\
          \vdots \\
          h_{m,1}c_1+h_{m,2}c_2+\dots+h_{m,n}c_n}_D
\end{equation*}
%</th:MatMultRepsFuncAppl>
\end{theorem}

\begin{proof}
%<*pf:MatMultRepsFuncAppl>
This formalizes \nearbyexample{ex:TypLinMapRepByMat}.
See \nearbyexercise{exer:MatVecMultRepLinMap}.
%</pf:MatMultRepsFuncAppl>
\end{proof}

\begin{definition} \label{def:MatrixVecProd}
%<*df:MatrixVecProd>
The \definend{matrix-vector product}\index{multiplication!matrix-vector}%
\index{matrix!matrix-vector product}
of a \( \nbym{m}{n} \) matrix and a
\( \nbym{n}{1} \) vector is this.
\begin{equation*}
  \generalmatrix{a}{n}{m}
  \colvec{c_1 \\ \vdots \\ c_n}
  =
  \colvec{a_{1,1}c_1+\dots+a_{1,n}c_n \\
             a_{2,1}c_1+\dots+a_{2,n}c_n \\
             \vdots \\ 
             a_{m,1}c_1+\dots+a_{m,n}c_n}
\end{equation*}
%</df:MatrixVecProd>
\end{definition}

% the product of the matrix $\rep{h}{B,D}$ with the vector $\rep{\vec{v}}{B}$ 
% is the vector $\rep{h(\vec{v})}{D}$.  
Briefly, 
application of a linear map is represented by the matrix-vector product 
of the map's representative and the vector's representative.

\begin{remark}  \label{rem:NotSurprising}
\nearbytheorem{th:MatMultRepsFuncAppl} is not surprising,
because we chose the matrix representative in \nearbydefinition{def:MatRepMap}
precisely to make the theorem true\Dash
if the theorem were not true then we would adjust the definition
to make it so.
Nonetheless, we need the verification. 
\end{remark}

\begin{example}
For the matrix from \nearbyexample{ex:PolyOneToRThree}
we can calculate where that map sends this vector.
\begin{equation*}
  \vec{v}=\colvec[r]{4 \\ 1 \\ 0}
\end{equation*}
With respect to the domain basis $B$ the representation of this vector is 
\begin{equation*}
  \rep{\vec{v}}{B}=\colvec[r]{0 \\ 1/2 \\ 2}_B
\end{equation*}
and so the matrix-vector product gives 
the representation of the value $h(\vec{v})$ with respect to
the codomain basis $D$.
\begin{align*}
  \rep{h(\vec{v})}{D}
  &=\begin{mat}[r]
      -1/2  &1   &2  \\
      -1/2  &-1  &-2
    \end{mat}_{B,D}
    \colvec[r]{0 \\ 1/2 \\ 2}_B                            \\
  &=\colvec{(-1/2)\cdot 0+1\cdot (1/2) + 2\cdot 2 \\ 
          (-1/2)\cdot 0-1\cdot (1/2) - 2\cdot 2}_D
  =\colvec[r]{9/2 \\ -9/2}_D
\end{align*}
To find $h(\vec{v})$ itself, not its representation,
take $(9/2)(1+x)-(9/2)(-1+x)=9$.
\end{example}

\begin{example}
Let \( \map{\pi}{\Re^3}{\Re^2} \) be projection onto the \( xy \)-plane.
To give a matrix representing this map, we first fix some bases. 
\begin{equation*}
  B=\sequence{
              \colvec[r]{1 \\ 0 \\ 0},
              \colvec[r]{1 \\ 1 \\ 0},
              \colvec[r]{-1 \\ 0 \\ 1} }
  \qquad
  D=\sequence{
              \colvec[r]{2 \\ 1},
              \colvec[r]{1 \\ 1} }
\end{equation*}
For each vector in the domain's basis, find its image under the map. 
\begin{equation*}
  \colvec[r]{1 \\ 0 \\ 0}\mapsunder{\pi}\colvec[r]{1 \\ 0}
  \quad
  \colvec[r]{1 \\ 1 \\ 0}\mapsunder{\pi}\colvec[r]{1 \\ 1}
  \quad
  \colvec[r]{-1 \\ 0 \\ 1}\mapsunder{\pi}\colvec[r]{-1 \\ 0}
\end{equation*}
Then find the representation of each image with respect to the codomain's
basis. 
\begin{equation*}
  \rep{\colvec[r]{1 \\ 0}}{D}=\colvec[r]{1 \\ -1}
  \quad
  \rep{\colvec[r]{1 \\ 1}}{D}=\colvec[r]{0 \\ 1}
  \quad
  \rep{\colvec[r]{-1 \\ 0}}{D}=\colvec[r]{-1 \\ 1}
\end{equation*}
Finally, adjoining these representations gives the matrix representing 
\( \pi \) with respect to \( B,D \).
\begin{equation*}
    \rep{\pi}{B,D}
    =\begin{mat}[r]
      1  &0  &-1  \\
      -1 &1  &1
    \end{mat}_{B,D}
\end{equation*}
We can illustrate \nearbytheorem{th:MatMultRepsFuncAppl} by computing
the matrix-vector product representing this action by 
the projection map.
\begin{equation*}
  \pi(
     \colvec[r]{2 \\ 2 \\ 1}
     )=\colvec[r]{2 \\ 2}
\end{equation*}
Represent the domain vector 
with respect to the domain's basis
\begin{equation*}
   \rep{\colvec[r]{2 \\ 2 \\ 1}}{B}=
        \colvec[r]{1 \\ 2 \\ 1}_B
\end{equation*}
to get this matrix-vector product.
\begin{equation*}
   \rep{ \,\pi(\colvec[r]{2 \\ 2 \\ 1})\,}{D}=
      \begin{mat}[r]
        1  &0  &-1  \\
        -1 &1  &1
      \end{mat}_{B,D}
   \colvec[r]{1 \\ 2 \\ 1}_B
   =
   \colvec[r]{0 \\ 2}_D
\end{equation*}
Expanding this into a linear combination of vectors from
\( D \) 
\begin{equation*}
   0\cdot\colvec[r]{2 \\ 1}
   +2\cdot\colvec[r]{1 \\ 1}
   =
   \colvec[r]{2 \\ 2}
\end{equation*}
checks that the map's action is indeed
reflected in the operation of the matrix.
We will sometimes compress these three displayed equations into one. 
\begin{equation*}
  \colvec[r]{2 \\ 2 \\ 1}=\colvec[r]{1 \\ 2 \\ 1}_B
    \;\overset{h}{\underset{H}{\longmapsto}}\;
  \colvec[r]{0 \\ 2}_D=\colvec[r]{2 \\ 2}    
\end{equation*}
\end{example}

We now have two ways to compute the effect of projection,
the straightforward formula that drops each three-tall vector's third component
to make a two-tall vector, 
and the above formula that uses representations and matrix-vector 
multiplication.
The second way may seem complicated compared to the first,
but it has advantages.
The next example shows that for some maps this new scheme simplifies
the formula.

\begin{example} \label{exam:RepsOfRigidPlaneMaps}
To represent a rotation\index{rotation (or turning)!represented}
map $\map{t_{\theta}}{\Re^2}{\Re^2}$ that 
turns all vectors in the plane counterclockwise through an angle $\theta$
\begin{center}  \small
  \includegraphics{map/mp/ch3.15}
\end{center}
we start by fixing the standard bases
$\stdbasis_2$ for both the domain and codomain basis,
Now find the image under the map of each
vector in the domain's basis.
\begin{equation*}
  \colvec[r]{1 \\ 0}\mapsunder{t_\theta}\colvec{\cos\theta \\ \sin\theta}
  \qquad
  \colvec[r]{0 \\ 1}\mapsunder{t_\theta}\colvec{-\sin\theta \\ \cos\theta}
  \tag{$*$}
\end{equation*}
Represent these images with respect to the codomain's basis.
Because this basis is $\stdbasis_2$, vectors represent themselves.
Adjoin the representations to get the matrix representing the map.
\begin{equation*}
  \rep{t_\theta}{\stdbasis_2,\stdbasis_2}
  =
  \begin{mat}
    \cos\theta  &-\sin\theta \\
    \sin\theta  &\cos\theta
  \end{mat}
\end{equation*}
The advantage of this scheme is that we get a formula for the 
image of any vector at 
all just by knowing in~($*$) how to
represent the image of the two basis vectors. 
For instance, here we rotate a vector by $\theta=\pi/6$.
\begin{equation*}
  \colvec[r]{3  \\ -2} = \colvec[r]{3  \\ -2}_{\stdbasis_2}\!\mapsunder{t_{\pi/6}}\;
  \begin{mat}[r]
    \sqrt{3}/2  &-1/2  \\
     1/2        &\sqrt{3}/2
  \end{mat}
  \colvec[r]{3  \\ -2}
  \approx
  \colvec[r]{3.598 \\ -0.232}_{\stdbasis_2}
  \!\!=  
  \colvec[r]{3.598 \\ -0.232}
\end{equation*}
More generally, we have a formula for rotation by $\theta=\pi/6$.
\begin{equation*}
  \colvec[r]{x  \\ y} \mapsunder{t_{\pi/6}}\;
  \begin{mat}[r]
    \sqrt{3}/2  &-1/2  \\
     1/2        &\sqrt{3}/2
  \end{mat}
  \colvec[r]{x  \\ y}
  =
  \colvec[r]{(\sqrt{3}/2)x-(1/2)y \\ (1/2)x+(\sqrt{3}/2)y }
\end{equation*}
\end{example}

\begin{example}
In the definition of matrix-vector product
the width of the matrix equals the height of the vector.
Hence, this product is not defined. 
\begin{equation*}
    \begin{mat}[r]
      1  &0  &0  \\
      4  &3  &1
    \end{mat}
  \colvec[r]{1 \\ 0}
\end{equation*}
It is undefined for a reason:~the three-wide
matrix represents a map with a three-dimensional domain while the 
two-tall vector represents a member of a two-dimensional space.
So the vector cannot be in the domain of the map.
\end{example}

Nothing in \nearbydefinition{def:MatrixVecProd} forces us to view
matrix-vector product in terms of representations.
We can get some insights by focusing on how the entries combine.

A good way to view matrix-vector product is that it is formed from 
the dot products of the rows of the matrix with the column vector.
\begin{equation*}
    \begin{mat}
               &\vdots                         \\
      a_{i,1}  &a_{i,2}  &\ldots   &a_{i,n}    \\
               &\vdots
    \end{mat}
  \colvec{c_1 \\ c_2 \\ \vdots \\ c_n}
  =
  \colvec{\vdots \\ a_{i,1}c_1+a_{i,2}c_2+\cdots+a_{i,n}c_n \\ \vdots}
\end{equation*}
Looked at in this row-by-row way,
this new operation generalizes dot product.

We can also view the operation column-by-column.
\begin{align*}
           \generalmatrix{h}{n}{m}
           \colvec{c_1 \\ c_2 \\ \vdots \\ c_n}
  &=\colvec{h_{1,1}c_1+h_{1,2}c_2+\dots+h_{1,n}c_n \\
               h_{2,1}c_1+h_{2,2}c_2+\dots+h_{2,n}c_n \\
               \vdots \\
               h_{m,1}c_1+h_{m,2}c_2+\dots+h_{m,n}c_n}    \\
  &=c_1\colvec{h_{1,1} \\ h_{2,1} \\ \vdots \\ h_{m,1}}
%   +c_2\colvec{h_{1,2} \\ h_{2,2} \\ \vdots \\ h_{m,2}}
   +\dots
   +c_n\colvec{h_{1,n} \\ h_{2,n} \\ \vdots \\ h_{m,n}}
\end{align*}
The result is the
columns of the matrix weighted by the entries of the vector.

\begin{example}
\begin{equation*}
    \begin{mat}[r]
      1  &0  &-1  \\
      2  &0  &3
    \end{mat}
  \colvec[r]{2 \\ -1 \\ 1}
  =
  2\colvec[r]{1 \\ 2}
  -1\colvec[r]{0 \\ 0}
  +1\colvec[r]{-1 \\ 3}
  =
  \colvec[r]{1 \\ 7}
\end{equation*}
\end{example}

This way of looking at matrix-vector product
brings us back to the objective stated at the start of this section, to compute
\( h(c_1\vec{\beta}_1+\dots+c_n\vec{\beta}_n) \)
as
\( c_1h(\vec{\beta}_1)+\dots+c_nh(\vec{\beta}_n) \).

We began this section
by noting that the equality of these two enables us to compute the action 
of $h$ on any
argument knowing only $h(\vec{\beta}_1)$, \ldots, $h(\vec{\beta}_n)$.
We have developed this into a scheme to
compute the action of the map by taking 
the matrix-vector product of the matrix representing the 
map with the vector representing the argument.
In this way, with respect to any bases, for any linear map there is 
a matrix representation.
The next subsection will show the converse, that if we fix bases then 
for any matrix there is an associated linear map.


\begin{exercises}
  \recommended \item  
    Multiply the matrix
    \begin{equation*}
      \begin{mat}[r]
        1  &3  &1  \\
        0  &-1 &2  \\
        1  &1  &0
      \end{mat}
    \end{equation*}
    by each vector, or state ``not defined.''
    \begin{exparts*}
      \partsitem \( \colvec[r]{2 \\ 1 \\ 0} \)
      \partsitem \( \colvec[r]{-2 \\ -2} \)
      \partsitem \( \colvec[r]{0 \\ 0 \\ 0} \)
    \end{exparts*}
    \begin{answer}
      \begin{exparts*}
        \partsitem \(
           \colvec{1\cdot 2+3\cdot 1+1\cdot 0         \\
                        0\cdot 2+(-1)\cdot 1+2\cdot 0 \\
                        1\cdot 2+1\cdot 1+0\cdot 0}
           =\colvec[r]{5 \\ -1 \\ 3}   \)
        \partsitem Not defined.
        \partsitem \(  \colvec[r]{0 \\ 0 \\ 0}  \)
      \end{exparts*}  
    \end{answer}
  \item Multiply this matrix by each vector or state ``not defined.''
    \begin{equation*}
      \begin{mat}
        3  &1  \\
        2  &4
      \end{mat}
    \end{equation*}
    \begin{exparts*}
      \partsitem $\colvec{1 \\ 2  \\ 1}$
      \partsitem $\colvec{0 \\ -1}$
      \partsitem $\colvec{0 \\ 0 \\ 0}$
    \end{exparts*}
    \begin{answer}
      \begin{exparts}
        \partsitem This is not defined.
        \partsitem This is defined.
          \begin{equation*}
            \begin{mat}
              3  &1  \\
              2  &4
            \end{mat}
            \colvec{0 \\ -1}
            =
            \colvec{-1 \\ -4}
          \end{equation*}
       \partsitem Not defined.
      \end{exparts}
    \end{answer}
  \item 
    Perform, if possible, each matrix-vector multiplication.
    \begin{exparts*}
      \partsitem $\begin{mat}[r]
                    2  &1  \\
                    3  &-1/2
                  \end{mat}
                  \colvec[r]{4  \\ 2}$
      \partsitem $\begin{mat}[r]
                    1  &1  &0 \\
                    -2 &1  &0
                  \end{mat}
                  \colvec[r]{1 \\ 3 \\ 1}$
      \partsitem $\begin{mat}[r]
                    1  &1  \\
                    -2  &1
                  \end{mat}
                  \colvec[r]{1 \\ 3 \\ 1}$
    \end{exparts*}
    \begin{answer}
      \begin{exparts*}
        \partsitem $\colvec{2\cdot 4 +1\cdot 2 \\
                            3\cdot 4-(1/2)\cdot 2}
                   =\colvec[r]{10 \\ 11}$
        \partsitem $\colvec[r]{4 \\ 1}$
        \partsitem Not defined.
      \end{exparts*}
    \end{answer}
  \item  
    This matrix equation expresses a linear system.
    Solve it.
    \begin{equation*}
      \begin{mat}[r]
        2  &1  &1  \\
        0  &1  &3  \\
        1  &-1 &2
      \end{mat}
      \colvec{x \\ y \\ z}
      =\colvec[r]{8 \\ 4 \\ 4}
    \end{equation*}
    \begin{answer}
      Matrix-vector multiplication gives rise to a linear system.
      \begin{equation*}
        \begin{linsys}{3}
          2x  &+  &y  &+  &z  &=  &8  \\
              &   &y  &+  &3z &=  &4  \\
           x  &-  &y  &+  &2z &=  &4 
        \end{linsys}
      \end{equation*}
      Gaussian reduction shows that \( z=1 \), \( y=1 \), and \( x=3 \).  
    \end{answer}
  \recommended \item 
    For a homomorphism from \( \polyspace_2 \) to \( \polyspace_3 \) that sends
    \begin{equation*}
      1\mapsto 1+x,
      \quad
      x\mapsto 1+2x,
      \quad\text{and}\quad
      x^2\mapsto x-x^3
    \end{equation*}
    where does \( 1-3x+2x^2 \) go?
    \begin{answer}
      Here are two ways to get the answer.

      First, obviously $1-3x+2x^2=1\cdot 1-3\cdot x+2\cdot x^2$, and so we can
      apply the general property of preservation of combinations to get
      $h(1-3x+2x^2)
       =h(1\cdot 1-3\cdot x+2\cdot x^2) 
       =1\cdot h(1)-3\cdot h(x)+2\cdot h(x^2) 
       =1\cdot (1+x)-3\cdot (1+2x)+2\cdot (x-x^3) 
       =-2-3x-2x^3$.

      The other way uses the computation scheme developed in this subsection.
      Because we know where these elements of the space go, we consider
      this basis \( B=\sequence{1,x,x^2} \) for the domain.
      Arbitrarily, we can take \( D=\sequence{1,x,x^2,x^3} \)
      as a basis for the codomain.
      With those choices, we have that
      \begin{equation*}
        \rep{h}{B,D}
        =\begin{mat}[r]
           1   &1  &0  \\
           1   &2  &1  \\
           0   &0  &0  \\
           0   &0  &-1
         \end{mat}_{B,D}
      \end{equation*}
      and, as
      \begin{equation*}
        \rep{1-3x+2x^2}{B}=\colvec[r]{1 \\ -3 \\ 2}_B
      \end{equation*}
      the matrix-vector multiplication calculation gives this.
      \begin{equation*}
        \rep{h(1-3x+2x^2)}{D}=
         \begin{mat}[r]
           1   &1  &0  \\
           1   &2  &1  \\
           0   &0  &0  \\
           0   &0  &-1
         \end{mat}_{B,D}
         \colvec[r]{1 \\ -3 \\ 2}_B
         =\colvec[r]{-2 \\ -3 \\ 0 \\ -2}_D
      \end{equation*}
      Thus, \( h(1-3x+2x^2)
              =-2\cdot 1-3\cdot x+0\cdot x^2-2\cdot x^3
              =-2-3x-2x^3 \),
      as above.  
    \end{answer}
  \item Let $\map{h}{\Re^2}{\matspace_{\nbyn{2}}}$ be the linear transformation
   with this action.
   \begin{equation*}
     \colvec{1 \\ 0}\mapsto
     \begin{mat}
       1  &2  \\
       0  &1
     \end{mat}
     \qquad
     \colvec{0 \\ 1}\mapsto
     \begin{mat}
       0  &-1  \\
       1  &0
     \end{mat}
   \end{equation*}
   What is its effect on the general vector with entries $x$ and~$y$?
   \begin{answer}
     Fix this natural basis for $\matspace_{\nbyn{2}}$.
     \begin{equation*}
       D=\sequence{
       \begin{mat}
          1 &0  \\
          0 &0
       \end{mat},
       \begin{mat}
          0 &1  \\
          0 &0
       \end{mat},
       \begin{mat}
          0 &0  \\
          1 &0
       \end{mat},
       \begin{mat}
          0 &0  \\
          0 &1
       \end{mat}}
     \end{equation*}
     The representation of the map~$h$ with respect to $\stdbasis_2,D$ is this.
     \begin{equation*}
       \rep{h}{\stdbasis_2,D}=
       \begin{mat}
         1  &0  \\
         2  &-1  \\
         0  &1  \\
         1  &0
       \end{mat}
     \end{equation*}
     Since the general vector is represented with respect to~$\stdbasis_2$ 
     by itself, 
     and similarly every matrix is represented with respect to~$D$ by itself, 
     we have this for the effect of the map.
     \begin{equation*}
       \rep{h(\colvec{x \\ y})}{D}
       =
       \begin{mat}
         1  &0  \\
         2  &-1  \\
         0  &1  \\
         1  &0
       \end{mat}
       \colvec{x \\ y}
       =
       \colvec{x \\ 2x-y \\ y \\ x}
     \end{equation*}
   \end{answer}
  \recommended \item  
    Assume that \( \map{h}{\Re^2}{\Re^3} \) is determined by 
    this action.
    \begin{equation*}
      \colvec[r]{1 \\ 0}\mapsto\colvec[r]{2 \\ 2 \\ 0}
      \qquad
      \colvec[r]{0 \\ 1}\mapsto\colvec[r]{0 \\ 1 \\ -1}
    \end{equation*}
    Using the standard bases, find
    \begin{exparts}
      \partsitem the matrix representing this map;
      \partsitem a general formula for \( h(\vec{v}) \).
    \end{exparts}
    \begin{answer}
      Again, as recalled in the subsection, 
      with respect to $\stdbasis_i$, a column vector represents itself.
      \begin{exparts}
        \partsitem To represent \( h \) with respect 
          to \( \stdbasis_2,\stdbasis_3 \) take 
          the images of the basis vectors from the domain,
          and represent them with respect to the basis for the codomain.
          The first is this
          \begin{equation*}
            \rep{\,h(\vec{e}_1)\,}{\stdbasis_3}
            =\rep{\colvec[r]{2 \\ 2 \\ 0}}{\stdbasis_3}
            =\colvec[r]{2 \\ 2 \\ 0}
          \end{equation*}
          while the second is this.
          \begin{equation*}
            \rep{\,h(\vec{e}_2)\,}{\stdbasis_3}
            =\rep{\colvec[r]{0 \\ 1 \\ -1}}{\stdbasis_3}
            =\colvec[r]{0 \\ 1 \\ -1}
          \end{equation*}
          Adjoin these to make the matrix.
          \begin{equation*}
            \rep{h}{\stdbasis_2,\stdbasis_3}=
            \begin{mat}[r]
              2  &0  \\
              2  &1  \\
              0  &-1
            \end{mat}
          \end{equation*}
        \partsitem For any \( \vec{v} \) in the domain \( \Re^2 \),
          \begin{equation*}
            \rep{\vec{v}}{\stdbasis_2}
            =\rep{\colvec{v_1 \\ v_2}}{\stdbasis_2}
            =\colvec{v_1 \\ v_2}
          \end{equation*}
          and so
          \begin{equation*}
            \rep{\,h(\vec{v})\,}{\stdbasis_3}
            =\begin{mat}[r]
              2  &0  \\
              2  &1  \\
              0  &-1
            \end{mat}
            \colvec{v_1 \\ v_2}
            =\colvec{2v_1 \\ 2v_1+v_2 \\ -v_2}
          \end{equation*}
          is the desired representation.
      \end{exparts}  
    \end{answer}
  \item Represent the homomorphism $\map{h}{\Re^3}{\Re^2}$ given by this formula
    and with respect to these bases.
    \begin{equation*}
      \colvec{x \\ y \\ z}\mapsto\colvec{x+y \\ x+z}
      \qquad
       B=\sequence{\colvec{1 \\ 1 \\ 1},
                   \colvec{1 \\ 1 \\ 0},
                   \colvec{1 \\ 0 \\ 0}}
       \quad
          D=\sequence{\colvec{1 \\ 0},
                  \colvec{0 \\ 2}}
    \end{equation*}
    \begin{answer}
      The action of the map on the domain's basis vectors is this.
      \begin{equation*}
          \colvec{1 \\ 1 \\ 1}\mapsto\colvec{2 \\ 2}
          \quad
          \colvec{1 \\ 1 \\ 0}\mapsto\colvec{2 \\ 1}
          \quad
          \colvec{1 \\ 0 \\ 0}\mapsto\colvec{1 \\ 1}
      \end{equation*}
      Represent those with respect to the codomain's basis.
      \begin{equation*}
        \rep{\colvec{2 \\ 2}}{D}=\colvec{2 \\ 1}_D
        \quad
        \rep{\colvec{2 \\ 1}}{D}=\colvec{2 \\ 1/2}_D
        \quad
        \rep{\colvec{1 \\ 1}}{D}=\colvec{1 \\ 1/2}_D
      \end{equation*}
      Concatenate them together into a matrix.
      \begin{equation*}
        \rep{h}{B,D}=
        \begin{mat}
          2 &2   &1 \\
          1 &1/2 &1/2
        \end{mat}
      \end{equation*}      
    \end{answer}
  \recommended \item  
    Let \( \map{d/dx}{\polyspace_3}{\polyspace_3} \) be the derivative
    transformation.
    \begin{exparts}
      \partsitem Represent \( d/dx \) with respect to \( B,B \) where
        \( B=\sequence{1,x,x^2,x^3} \).
      \partsitem Represent \( d/dx \) with respect to \( B,D \) where
        \( D=\sequence{1,2x,3x^2,4x^3} \).
    \end{exparts}
    \begin{answer}
      \begin{exparts*}
        \partsitem 
        We must first find the image of each vector from the domain's basis,
        and then represent that image with respect to the codomain's basis.
        \begin{multline*}
          \rep{\frac{d\,1}{dx}}{B}=\colvec[r]{0 \\ 0 \\ 0 \\ 0}
          \quad
          \rep{\frac{d\,x}{dx}}{B}=\colvec[r]{1 \\ 0 \\ 0 \\ 0}
          \quad
          \rep{\frac{d\,x^2}{dx}}{B}=\colvec[r]{0 \\ 2 \\ 0 \\ 0}
          \\
          \rep{\frac{d\,x^3}{dx}}{B}=\colvec[r]{0 \\ 0 \\ 3 \\ 0}
        \end{multline*}
        Those representations are then adjoined to make the matrix
        representing the map.
        \begin{equation*}
          \rep{\frac{d}{dx}}{B,B}=
          \begin{mat}[r]
            0  &1  &0  &0  \\
            0  &0  &2  &0  \\
            0  &0  &0  &3  \\
            0  &0  &0  &0 
          \end{mat}  
        \end{equation*}
        \partsitem Proceeding as in the prior item, we represent the images
          of the domain's basis vectors
        \begin{multline*}
          \rep{\frac{d\,1}{dx}}{D}=\colvec[r]{0 \\ 0 \\ 0 \\ 0}
          \quad
          \rep{\frac{d\,x}{dx}}{D}=\colvec[r]{1 \\ 0 \\ 0 \\ 0}
          \quad
          \rep{\frac{d\,x^2}{dx}}{D}=\colvec[r]{0 \\ 1 \\ 0 \\ 0}
          \\
          \rep{\frac{d\,x^3}{dx}}{D}=\colvec[r]{0 \\ 0 \\ 1 \\ 0}
        \end{multline*}
        and adjoin to make the matrix.
        \begin{equation*}
          \rep{\frac{d}{dx}}{B,D}=
          \begin{mat}[r]
            0  &1  &0  &0  \\
            0  &0  &1  &0  \\
            0  &0  &0  &1  \\
            0  &0  &0  &0
          \end{mat}  
        \end{equation*}
      \end{exparts*}  
    \end{answer}
  \recommended \item 
    Represent each linear map with respect to each pair of bases.
    \begin{exparts}
      \partsitem \( \map{d/dx}{\polyspace_n}{\polyspace_n} \) with respect to
        \( B,B \) where \( B=\sequence{1,x,\dots,x^n} \), given by
        \begin{equation*}
            a_0+a_1x+a_2x^2+\dots+a_nx^n
            \mapsto
            a_1+2a_2x+\dots+na_nx^{n-1}
        \end{equation*}
      \partsitem \( \map{\int}{\polyspace_n}{\polyspace_{n+1}} \) 
        with respect to
        \( B_n,B_{n+1} \) where \( B_i=\sequence{1,x,\dots,x^i} \), given by
        \begin{equation*}
            a_0+a_1x+a_2x^2+\dots+a_nx^n
            \mapsto
            a_0x+\frac{a_1}{2}x^2+\dots+\frac{a_n}{n+1}x^{n+1}
        \end{equation*}
      \partsitem \( \map{\int^1_0}{\polyspace_n}{\Re} \) with respect to
        \( B,\stdbasis_1 \) where \( B=\sequence{1,x,\dots,x^n} \)
        and \( \stdbasis_1=\sequence{1} \), given by
        \begin{equation*}
            a_0+a_1x+a_2x^2+\dots+a_nx^n
            \mapsto
            a_0+\frac{a_1}{2}+\dots+\frac{a_n}{n+1}
        \end{equation*}
      \partsitem \( \map{\text{eval}_3}{\polyspace_n}{\Re} \) with respect to
        \( B,\stdbasis_1 \) where \( B=\sequence{1,x,\dots,x^n} \)
        and \( \stdbasis_1=\sequence{1} \), given by
        \begin{equation*}
            a_0+a_1x+a_2x^2+\dots+a_nx^n
            \mapsto
            a_0+a_1\cdot 3+a_2\cdot 3^2+\dots+a_n\cdot 3^n
        \end{equation*}
      \partsitem \( \map{\text{slide}_{-1}}{\polyspace_n}{\polyspace_n} \) 
        with respect
        to \( B,B \) where \( B=\sequence{1,x,\ldots,x^n} \), given by
        \begin{equation*}
            a_0+a_1x+a_2x^2+\dots+a_nx^n
            \mapsto
            a_0+a_1\cdot (x+1)+\dots+a_n\cdot (x+1)^n
        \end{equation*}
    \end{exparts}
    \begin{answer}
      For each, we must find the image of each of the domain's basis vectors,
      represent each image with respect to the codomain's basis,
      and then adjoin those representations to get the matrix.
      \begin{exparts}
        \partsitem The basis vectors from the domain have these images
          \begin{equation*}
            1\mapsto 0  
            \quad x\mapsto 1  
            \quad x^2\mapsto 2x  
            \quad \ldots
          \end{equation*}
          and these images are represented with respect to the codomain's
          basis in this way.
          \begin{multline*}
            \rep{0}{B}=\colvec{0 \\ 0 \\ 0 \\ \vdots \\ \  \\  \ }
            \quad
            \rep{1}{B}=\colvec{1 \\ 0 \\ 0 \\ \vdots \\ \  \\ \ }
            \quad
            \rep{2x}{B}=\colvec{0 \\ 2 \\ 0 \\ \vdots \\ \  \\ \ }        \\
            \ldots\quad
            \rep{nx^{n-1}}{B}=\colvec{0 \\ 0 \\ 0 \\ \vdots \\ n \\ 0}
          \end{multline*}
          The matrix
          \begin{equation*}
            \rep{\frac{d}{dx}}{B,B}
            =\begin{mat}
              0  &1  &0  &\ldots  &0  \\
              0  &0  &2  &\ldots  &0  \\
                 &\vdots             \\
              0  &0  &0  &\ldots  &n  \\
              0  &0  &0  &\ldots  &0
            \end{mat}
          \end{equation*}
          has $n+1$ rows and columns.
       \partsitem Once the images under this map of the domain's basis
          vectors are determined
          \begin{equation*}
            1\mapsto x 
            \quad x\mapsto x^2/2  
            \quad x^2\mapsto x^3/3 
            \quad \ldots
          \end{equation*}
          then they can be represented with respect to the codomain's basis
          \begin{multline*}
            \rep{x}{B_{n+1}}=\colvec{0 \\ 1 \\ 0 \\ \vdots \\ \ }
            \quad
            \rep{x^2/2}{B_{n+1}}=\colvec{0 \\ 0 \\ 1/2 \\ \vdots \\ \ }   \\
            \ldots\quad
            \rep{x^{n+1}/(n+1)}{B_{n+1}}
                     =\colvec{0 \\ 0 \\ 0 \\ \vdots \\ 1/(n+1)}
          \end{multline*}
          and put together to make the matrix.
          \begin{equation*}
            \rep{\int}{B_{n},B_{n+1}}
            =\begin{mat}
              0  &0  &\ldots  &0      &0  \\
              1  &0  &\ldots  &0      &0  \\
              0  &1/2&\ldots  &0      &0  \\
                 &\vdots                  \\
              0  &0  &\ldots  &0      &1/(n+1)
            \end{mat}
          \end{equation*}
        \partsitem The images of the basis vectors of the domain are
          \begin{equation*} 
            1\mapsto 1 
            \quad x\mapsto 1/2 
            \quad x^2\mapsto 1/3 
            \quad \ldots
          \end{equation*}
          and they are represented with respect to the codomain's basis as
          \begin{equation*}
            \rep{1}{\stdbasis_1}=1
            \quad \rep{1/2}{\stdbasis_1}=1/2
            \quad \ldots
          \end{equation*}
          so the matrix is 
          \begin{equation*}
            \rep{\int}{B,\stdbasis_1}
            =\begin{mat}
              1  &1/2 &\cdots  &1/n    &1/(n+1)
            \end{mat}
          \end{equation*}
          (this is an $\nbym{1}{(n+1)}$ matrix).
        \partsitem Here, the images of the domain's basis vectors are
          \begin{equation*}  
            1\mapsto 1 
            \quad x\mapsto 3 
            \quad x^2\mapsto 9
            \quad \ldots 
          \end{equation*}
          and they are represented in the codomain as
          \begin{equation*}
            \rep{1}{\stdbasis_1}=1
            \quad\rep{3}{\stdbasis_1}=3
            \quad\rep{9}{\stdbasis_1}=9
            \quad \ldots 
          \end{equation*}
          and so the matrix is this.
          \begin{equation*}
            \rep{\int_0^1}{B,\stdbasis_1}
            =\begin{mat}[r]
              1  &3   &9  &\cdots  &3^n
            \end{mat}
          \end{equation*}
        \partsitem The images of the basis vectors from the domain are
          $1\mapsto 1$, 
          and $x\mapsto x+1=1+x$,  
          and $x^2\mapsto (x+1)^2=1+2x+x^2$,  
          and $x^3\mapsto (x+1)^3=1+3x+3x^2+x^3$, etc.  
          The representations are here.
          \begin{equation*}
            \rep{1}{B}=\colvec[r]{1 \\ 0 \\ 0 \\ 0 \\ \vdotswithin{0} \\ 0}
            \quad
            \rep{1+x}{B}=\colvec{1 \\ 1 \\ 0 \\ 0 \\ \vdotswithin{0} \\ 0}
            \quad
            \rep{1+2x+x^2}{B}=\colvec{1 \\ 2 \\ 1 \\ 0 \\ \vdotswithin{0} \\ 0}
            \quad\ldots
          \end{equation*}
          The resulting matrix
          \begin{equation*}
            \renewcommand{\arraystretch}{1.2}
            \rep{\text{slide}_{-1}}{B,B}
            =\begin{mat}
              1  &1   &1  &1  &\ldots  &1           \\
              0  &1   &2  &3  &\ldots  &\binom{n}{1}  \\
              0  &0   &1  &3  &\ldots  &\binom{n}{2}  \\
                 &\vdots                        \\
              0  &0   &0  &   &\ldots  &1
            \end{mat}
          \end{equation*}
          is \definend{Pascal's triangle}\index{Pascal's triangle}
          (recall that $\binom{n}{r}$ is the number of ways to choose $r$
          things, without order and without repetition,
          from a set of size $n$). 
      \end{exparts}  
    \end{answer}
  \item 
    Represent the identity map on any nontrivial
    space with respect to \( B,B \), where \( B \) is any basis.
    \begin{answer}
      Where the space is \( n \)-dimensional,
      \begin{equation*}
        \rep{\text{id}}{B,B}=
        \begin{mat}
          1  &0  \ldots  &0  \\
          0  &1  \ldots  &0  \\
             &\vdots         \\
          0  &0  \ldots  &1
        \end{mat}_{B,B}
      \end{equation*}
      is the $\nbyn{n}$ identity matrix.  
    \end{answer}
  \item 
    Represent, with respect to the natural basis, 
    the transpose transformation on the space 
    \( \matspace_{\nbyn{2}} \) of $\nbyn{2}$ matrices.
    \begin{answer}
      Taking this as the natural basis
      \begin{equation*}
        B=\sequence{\vec{\beta}_1,\vec{\beta}_2,\vec{\beta}_3,\vec{\beta}_4}
         =\sequence{
            \begin{mat}[r]
              1  &0  \\
              0  &0
            \end{mat},
            \begin{mat}[r]
              0  &1  \\
              0  &0
            \end{mat},
            \begin{mat}[r]
              0  &0  \\
              1  &0
            \end{mat},
            \begin{mat}[r]
              0  &0  \\
              0  &1
            \end{mat}   }
      \end{equation*}
      the transpose map acts in this way
      \begin{equation*}
        \vec{\beta}_1\mapsto\vec{\beta}_1 
        \quad \vec{\beta}_2\mapsto\vec{\beta}_3 
        \quad \vec{\beta}_3\mapsto\vec{\beta}_2 
        \quad \vec{\beta}_4\mapsto\vec{\beta}_4  
      \end{equation*}
      so that representing the images with respect to the codomain's
      basis and adjoining those column vectors together gives this.
      \begin{equation*}
        \rep{\text{trans}}{B,B}=
        \begin{mat}[r]
          1  &0  &0  &0  \\
          0  &0  &1  &0  \\
          0  &1  &0  &0  \\
          0  &0  &0  &1
        \end{mat}_{B,B}
      \end{equation*}   
     \end{answer}
  \item 
     Assume that 
     \( B=\sequence{\vec{\beta}_1,\vec{\beta}_2,\vec{\beta}_3,\vec{\beta}_4} \)
     is a basis for a vector space.
     Represent with respect to \( B,B \) the transformation that is determined
     by each.
     \begin{exparts}
       \partsitem \( \vec{\beta}_1\mapsto\vec{\beta}_2 \),
         \( \vec{\beta}_2\mapsto\vec{\beta}_3 \),
         \( \vec{\beta}_3\mapsto\vec{\beta}_4 \),
         \( \vec{\beta}_4\mapsto\zero \)
       \partsitem \( \vec{\beta}_1\mapsto\vec{\beta}_2 \),
         \( \vec{\beta}_2\mapsto\zero \),
         \( \vec{\beta}_3\mapsto\vec{\beta}_4 \),
         \( \vec{\beta}_4\mapsto\zero \)
       \partsitem \( \vec{\beta}_1\mapsto\vec{\beta}_2 \),
         \( \vec{\beta}_2\mapsto\vec{\beta}_3 \),
         \( \vec{\beta}_3\mapsto\zero \),
         \( \vec{\beta}_4\mapsto\zero \)
     \end{exparts}
     \begin{answer}
       \begin{exparts}
         \partsitem With respect to the basis of the codomain, the images of
           the members of the basis of the domain are represented as
           \begin{equation*}
             \rep{\vec{\beta}_2}{B}=\colvec[r]{0 \\ 1 \\ 0 \\ 0}
             \quad
             \rep{\vec{\beta}_3}{B}=\colvec[r]{0 \\ 0 \\ 1 \\ 0}
             \quad
             \rep{\vec{\beta}_4}{B}=\colvec[r]{0 \\ 0 \\ 0 \\ 1}
             \quad
             \rep{\zero}{B}=\colvec[r]{0 \\ 0 \\ 0 \\ 0}
           \end{equation*}
           and consequently, the matrix representing the transformation is 
           this.
           \begin{equation*}
             \begin{mat}[r]
                   0  &0  &0  &0  \\
                   1  &0  &0  &0  \\
                   0  &1  &0  &0  \\
                   0  &0  &1  &0
                 \end{mat}  
         \end{equation*}
         \partsitem 
              $\begin{mat}[r]
                   0  &0  &0  &0  \\
                   1  &0  &0  &0  \\
                   0  &0  &0  &0  \\
                   0  &0  &1  &0
                 \end{mat}$
         \partsitem 
               $\begin{mat}[r]
                   0  &0  &0  &0  \\
                   1  &0  &0  &0  \\
                   0  &1  &0  &0  \\
                   0  &0  &0  &0
                 \end{mat}$
       \end{exparts}  
     \end{answer}
  \item
    \nearbyexample{exam:RepsOfRigidPlaneMaps} shows how to represent
    the rotation transformation of the plane with respect to the 
    standard basis.
    Express these other transformations also with respect to the standard
    basis.
    \begin{exparts}
      \partsitem the \definend{dilation} map $d_s$, which multiplies 
        all vectors by the same scalar $s$\index{dilation!representing}
      \partsitem the \definend{reflection} map $f_\ell$, which reflects all
        all vectors across a line $\ell$ through the origin
    \end{exparts}
    \begin{answer}
      \begin{exparts}
        \partsitem The picture of $\map{d_s}{\Re^2}{\Re^2}$ is this.
          \begin{center}  \small
            \includegraphics{map/mp/ch3.75}
          \end{center}
         This map's effect on the vectors in the standard basis for the domain 
         is 
         \begin{equation*} 
            \colvec[r]{1 \\ 0}\mapsunder{d_s}\colvec{s \\ 0}
            \qquad
            \colvec[r]{0 \\ 1}\mapsunder{d_s}\colvec{0 \\ s}
         \end{equation*}
         and those images are represented with respect to the 
         codomain's basis (again, the standard basis) by themselves.
         \begin{equation*} 
           \rep{\colvec{s \\ 0}}{\stdbasis_2}=\colvec{s \\ 0}
           \qquad
           \rep{\colvec{0 \\ s}}{\stdbasis_2}=\colvec{0 \\ s}
         \end{equation*}
         Thus the representation of the dilation map is this.
         \begin{equation*}
           \rep{d_s}{\stdbasis_2,\stdbasis_2}
           =\begin{mat}
              s  &0  \\
              0  &s
           \end{mat}
        \end{equation*}
      \partsitem The picture of $\map{f_\ell}{\Re^2}{\Re^2}$ is this.
         \begin{center}  \small
           \includegraphics{map/mp/ch3.76}
        \end{center}
       Some calculation (see Exercise~I.\ref{exer:RigidPlaneMapsAutos})
       shows that when the line has slope $k$ 
       \begin{equation*}
         \colvec[r]{1 \\ 0}
             \mapsunder{f_\ell}\colvec{(1-k^2)/(1+k^2) \\ 2k/(1+k^2)}
         \qquad
         \colvec[r]{0 \\ 1}
             \mapsunder{f_\ell}\colvec{2k/(1+k^2) \\ -(1-k^2)/(1+k^2)}
       \end{equation*}
       (the case of a line with undefined slope is separate but easy) 
       and so the matrix representing reflection is this.
       \begin{equation*}
         \rep{f_\ell}{\stdbasis_2,\stdbasis_2}
         =\frac{1}{1+k^2}\cdot\begin{mat}
            1-k^2  &2k  \\
            2k     &-(1-k^2)
         \end{mat}
       \end{equation*}
      \end{exparts}  
    \end{answer}
  \recommended \item  
    Consider a linear transformation of \( \Re^2 \) determined
    by these two.
    \begin{equation*}
      \colvec[r]{1 \\ 1}\mapsto\colvec[r]{2 \\ 0}
      \qquad
      \colvec[r]{1 \\ 0}\mapsto\colvec[r]{-1 \\ 0}
    \end{equation*}
    \begin{exparts}
      \partsitem Represent this transformation with respect to the standard
        bases.
      \partsitem Where does the transformation send this vector?
        \begin{equation*}
          \colvec[r]{0 \\ 5}
        \end{equation*}
      \partsitem Represent this transformation with respect to these bases.
        \begin{equation*}
          B=\sequence{\colvec[r]{1 \\ -1},\colvec[r]{1 \\ 1}}
          \qquad
          D=\sequence{\colvec[r]{2 \\ 2},\colvec[r]{-1 \\ 1}}
        \end{equation*}
      \partsitem Using \( B \) from the prior item, 
        represent the transformation with respect to \( B,B \).
    \end{exparts}
    \begin{answer}
      Call the map \( \map{t}{\Re^2}{\Re^2} \).
      \begin{exparts}
        \partsitem To represent this map with respect to the standard bases, we
          must find, and then represent, the images of the vectors $\vec{e}_1$
          and $\vec{e}_2$ from the domain's basis.
          The image of $\vec{e}_1$ is given.

          One way to find the image of $\vec{e}_2$ is by 
          eye\Dash we can see this.
          \begin{equation*}
            \colvec[r]{1 \\ 1}-\colvec[r]{1 \\ 0}=\colvec[r]{0 \\ 1}
            \;\mapsunder{t}\;
            \colvec[r]{2 \\ 0}-\colvec[r]{-1 \\ 0}=\colvec[r]{3 \\ 0}
          \end{equation*}

          A more systematic way to find the image of $\vec{e}_2$ is to
          use the given information to represent the transformation, and then 
          use that representation to determine the image.  
          Taking this for a basis,
          \begin{equation*}
            C=\sequence{\colvec[r]{1 \\ 1},\colvec[r]{1 \\ 0}}
          \end{equation*}
          the given information says this.
          \begin{equation*}
            \rep{t}{C,\stdbasis_2}
            \begin{mat}[r]
              2  &-1  \\
              0  &0
            \end{mat}
          \end{equation*}
          As
          \begin{equation*}
            \rep{\vec{e}_2}{C}=\colvec[r]{1 \\ -1}_C
          \end{equation*}
          we have that
          \begin{equation*}
            \rep{t(\vec{e}_2)}{\stdbasis_2}
            =\begin{mat}[r]
               2  &-1  \\
               0  &0
             \end{mat}_{C,\stdbasis_2}
            \colvec[r]{1 \\ -1}_C
            =\colvec[r]{3 \\ 0}_{\stdbasis_2}
          \end{equation*}
          and consequently we know that $t(\vec{e}_2)=3\cdot\vec{e}_1$
          (since, with respect to the standard basis, this vector is 
          represented by itself).
          Therefore, this is the representation of $t$ with respect to
          $\stdbasis_2,\stdbasis_2$.
          \begin{equation*}
            \rep{t}{\stdbasis_2,\stdbasis_2}
            =\begin{mat}[r]
               -1 &3   \\
               0  &0
             \end{mat}_{\stdbasis_2,\stdbasis_2}
          \end{equation*}
        \partsitem To use the matrix developed in the prior item, note that
          \begin{equation*}
            \rep{\colvec[r]{0 \\ 5}}{\stdbasis_2}=\colvec[r]{0 \\ 5}_{\stdbasis_2}
          \end{equation*}
          and so we have this is the representation, with respect to the 
          codomain's basis, of the image of the given vector.
          \begin{equation*}
            \rep{t(\colvec[r]{0 \\ 5})}{\stdbasis_2}
            =\begin{mat}[r]
              -1  &3   \\
               0  &0
             \end{mat}_{\stdbasis_2,\stdbasis_2}
            \colvec[r]{0 \\ 5}_{\stdbasis_2}
            =\colvec[r]{15 \\ 0}_{\stdbasis_2}
          \end{equation*}
          Because the codomain's basis is the standard one, and so vectors
          in the codomain are represented by themselves, we have this.
          \begin{equation*}
            t(\colvec[r]{0 \\ 5})
            =\colvec[r]{15 \\ 0}
          \end{equation*}
        \partsitem We first find the image of each member of \( B \), and then
          represent those images with respect to \( D \).
          For the first step, we can use the matrix developed earlier.
          \begin{equation*}
            \rep{\colvec[r]{1 \\-1}}{\stdbasis_2}
            =\begin{mat}[r]
              -1  &3   \\
               0  &0
             \end{mat}_{\stdbasis_2,\stdbasis_2}
            \colvec[r]{1 \\ -1}_{\stdbasis_2}
            =\colvec[r]{-4 \\ 0}_{\stdbasis_2}
            \quad\text{so}\quad
            t(\colvec[r]{1 \\ -1})=\colvec[r]{-4 \\ 0}
          \end{equation*}
          Actually, for the second member of $B$ there is no need to apply the
          matrix because the problem statement gives its image.
          \begin{equation*}
            t(\colvec[r]{1 \\ 1})=\colvec[r]{2 \\ 0}
          \end{equation*}
          Now representing those images with respect to $D$ is routine.
          \begin{equation*}
            \rep{\colvec[r]{-4 \\ 0}}{D}=\colvec[r]{-1 \\ 2}_D
            \quad\text{and}\quad
            \rep{\colvec[r]{2 \\ 0}}{D}=\colvec[r]{1/2 \\ -1}_D
          \end{equation*}
          Thus, the matrix is this.
          \begin{equation*}
            \rep{t}{B,D}=
            \begin{mat}[r]
              -1  &1/2  \\
               2  &-1
            \end{mat}_{B,D}
          \end{equation*}
        \partsitem We know the images of the members of the domain's basis
          from the prior item.
          \begin{equation*}
             t(\colvec[r]{1 \\ -1})=\colvec[r]{-4 \\ 0}
             \qquad
             t(\colvec[r]{1 \\ 1})=\colvec[r]{2 \\ 0}
          \end{equation*}
          We can  compute the representation of those images with respect to
          the codomain's basis.
          \begin{equation*}
             \rep{\colvec[r]{-4 \\ 0}}{B}=\colvec[r]{-2 \\ -2}_B
             \quad\text{and}\quad
             \rep{\colvec[r]{2 \\ 0}}{B}=\colvec[r]{1 \\ 1}_B
          \end{equation*}
          Thus this is the matrix.
          \begin{equation*}
            \rep{t}{B,B}=
            \begin{mat}[r]
              -2  &1  \\
              -2  &1
            \end{mat}_{B,B}
          \end{equation*}
      \end{exparts}  
    \end{answer}
  \item 
    Suppose that \( \map{h}{V}{W} \) is one-to-one so that 
    by \nearbytheorem{th:OOHomoEquivalence}, for any basis
    \( B=\sequence{\vec{\beta}_1,\dots,\vec{\beta}_n}\subset V \) the image
    \( h(B)=\sequence{h(\vec{\beta}_1),\dots,h(\vec{\beta}_n)} \)
    is a basis for \( W \).
    \begin{exparts}
      \partsitem Represent the map $h$ with respect to $B,h(B)$.
      \partsitem For a member $\vec{v}$ of the domain, where
        the representation of $\vec{v}$ has components $c_1$, \ldots, $c_n$,
        represent the image vector \( h(\vec{v}) \) with respect to 
        the image basis $h(B)$.
    \end{exparts}
    \begin{answer}
      \begin{exparts}
        \partsitem The images of the members of the domain's basis are
          \begin{equation*}
            \vec{\beta}_1\mapsto h(\vec{\beta}_1)
            \quad 
            \vec{\beta}_2\mapsto h(\vec{\beta}_2)
            \quad\ldots\quad 
            \vec{\beta}_n\mapsto h(\vec{\beta}_n)
          \end{equation*}
          and those images are represented with respect to the codomain's
          basis in this way.
          \begin{multline*}
            \rep{\,h(\vec{\beta}_1)\,}{h(B)}=\colvec[r]{1 \\ 0 \\ \vdotswithin{0} \\ 0}
            \quad
            \rep{\,h(\vec{\beta}_2)\,}{h(B)}=\colvec[r]{0 \\ 1 \\ \vdotswithin{0} \\ 0}  \\
            \ldots\quad
            \rep{\,h(\vec{\beta}_n)\,}{h(B)}=\colvec[r]{0 \\ 0 \\ \vdotswithin{0} \\ 1}
          \end{multline*}
          Hence, the matrix is the identity.
          \begin{equation*}
            \rep{h}{B,h(B)}
            =\begin{mat}
               1  &0  &\ldots  &0  \\
               0  &1  &        &0  \\
                  &   &\ddots      \\
               0  &0  &        &1 
            \end{mat}
          \end{equation*}
        \partsitem Using the matrix in the prior item, 
          the representation is this.
          \begin{equation*}
            \rep{\,h(\vec{v})\,}{h(B)}
             =\colvec{c_1 \\ \vdots \\ c_n}_{h(B)}
          \end{equation*}
        \end{exparts}  
     \end{answer}
  \item 
    Give a formula for the product of a matrix and \( \vec{e}_i \), the
    column vector that is all zeroes except for a single one in the \( i \)-th
    position.
    \begin{answer}
      The product
      \begin{equation*}
        \begin{mat}
          h_{1,1} &\ldots  &h_{1,i} &\ldots &h_{1,n}  \\
          h_{2,1} &\ldots  &h_{2,i} &\ldots &h_{2,n}  \\
                  &\vdots                             \\
          h_{m,1} &\ldots  &h_{m,i} &\ldots &h_{1,n}
        \end{mat}
        \colvec{0 \\ \vdots \\ 1 \\ \vdots \\ 0}
        =
        \colvec{h_{1,i} \\ h_{2,i} \\ \vdots \\ h_{m,i}}
      \end{equation*}
      gives the \( i \)-th column of the matrix.  
    \end{answer}
  \recommended \item 
    For each vector space of functions of one real variable,
    represent the derivative transformation with respect to \( B,B \).
    \begin{exparts}
      \partsitem \( \set{a\cos x+b\sin x \suchthat a,b\in\Re} \),
         \( B=\sequence{\cos x,\sin x} \)
      \partsitem \( \set{ae^x+be^{2x} \suchthat a,b\in\Re} \),
         \( B=\sequence{e^x,e^{2x}} \)
      \partsitem \( \set{a+bx+ce^x+dxe^{x} \suchthat a,b,c,d\in\Re} \),
         \( B=\sequence{1,x,e^x,xe^{x}} \)
    \end{exparts}
    \begin{answer}
      \begin{exparts}
        \partsitem The images of the basis vectors for the domain are
         \( \cos x\mapsunder{d/dx}-\sin x \) and
          \( \sin x\mapsunder{d/dx}\cos x \).
          Representing those with respect to the codomain's basis (again, $B$)
          and adjoining the representations gives this matrix.
          \begin{equation*}
            \rep{\frac{d}{dx}}{B,B}=
            \begin{mat}[r]
                0  &1  \\
               -1  &0
            \end{mat}_{B,B}
          \end{equation*}
        \partsitem The images of the vectors in the domain's basis are 
          \( e^x\mapsunder{d/dx}e^x \) and
          \( e^{2x}\mapsunder{d/dx}2e^{2x} \).
          Representing with respect to the codomain's basis and adjoining
          gives this matrix.
          \begin{equation*}
            \rep{\frac{d}{dx}}{B,B}=
            \begin{mat}[r]
                1  &0  \\
                0  &2
            \end{mat}_{B,B}
          \end{equation*}
        \partsitem The images of the members of the domain's basis are 
          \( 1\mapsunder{d/dx}0 \), 
          \( x\mapsunder{d/dx}1 \),
          \( e^{x}\mapsunder{d/dx}e^{x} \), and
          \( xe^{x}\mapsunder{d/dx}e^x+xe^x \).
          Representing these images with respect to $B$ and adjoining
          gives this matrix.
          \begin{equation*}
            \rep{\frac{d}{dx}}{B,B}=
            \begin{mat}[r]
                0  &1  &0  &0 \\
                0  &0  &0  &0 \\
                0  &0  &1  &1 \\
                0  &0  &0  &1
            \end{mat}_{B,B}
          \end{equation*}
      \end{exparts}  
    \end{answer}
  \item 
    Find the range of the linear transformation of \( \Re^2 \) represented
    with respect to the standard bases by each matrix.
    \begin{exparts*}
      \partsitem $\begin{mat}[r]
            1  &0 \\
            0  &0
          \end{mat}$
      \partsitem $\begin{mat}[r]
          0  &0  \\
          3  &2  
        \end{mat}$
      \partsitem a matrix of the form 
        $\begin{mat}
            a   &b  \\
            2a  &2b
          \end{mat}$
    \end{exparts*}
    \begin{answer}
      \begin{exparts}
        \partsitem It is the set of vectors of the codomain represented with
          respect to the codomain's basis in this way.
          \begin{equation*}
            \set{
              \begin{mat}[r]
                1  &0  \\
                0  &0
              \end{mat}
              \colvec{x  \\ y}
              \suchthat x,y\in\Re}
            =\set{\colvec{x  \\ 0}
                  \suchthat x,y\in\Re}
          \end{equation*}
          As the codomain's basis is $\stdbasis_2$, 
          and so each vector is represented
          by itself, the range of this transformation is the $x$-axis.
        \partsitem It is the set of vectors of the codomain represented
          in this way.
          \begin{equation*}
            \set{
              \begin{mat}[r]
                0  &0  \\
                3  &2
              \end{mat}
              \colvec{x  \\ y}
              \suchthat x,y\in\Re}
            =\set{\colvec{0  \\ 3x+2y}
                  \suchthat x,y\in\Re}
          \end{equation*}
          With respect to $\stdbasis_2$ vectors represent 
          themselves so this range
          is the $y$~axis.
        \partsitem The set of vectors represented with 
          respect to $\stdbasis_2$ as
          \begin{align*}
            \set{
              \begin{mat}
                a   &b  \\
                2a  &2b
              \end{mat}
              \colvec{x  \\ y}
              \suchthat x,y\in\Re}
            &=\set{\colvec{ax+by  \\ 2ax+2by}
                  \suchthat x,y\in\Re}                    \\
            &=\set{(ax+by)\cdot\colvec[r]{1  \\ 2}
                  \suchthat x,y\in\Re}
          \end{align*}
          is the line $y=2x$, provided either $a$ or $b$ is not zero, and
          is the set consisting of just the origin if both are zero.
      \end{exparts}  
    \end{answer}
  \recommended \item  
    Can one matrix represent two different linear maps?
    That is, can \( \rep{h}{B,D}=\rep{\hat{h}}{\hat{B},\hat{D}} \)?
    \begin{answer}
      Yes, for two reasons.

      First, the two maps $h$ and $\hat{h}$ need not have the same domain
      and codomain.
      For instance,
      \begin{equation*}
        \begin{mat}[r]
          1  &2  \\
          3  &4
        \end{mat}
      \end{equation*}
      represents a map \( \map{h}{\Re^2}{\Re^2} \) with respect to the standard
      bases that sends
      \begin{equation*}
        \colvec[r]{1 \\ 0}\mapsto\colvec[r]{1 \\ 3}
        \quad\text{and}\quad
        \colvec[r]{0 \\ 1}\mapsto\colvec[r]{2 \\ 4}
      \end{equation*}
      and also represents a
      \( \map{\hat{h}}{\polyspace_1}{\Re^2} \) with respect to
      \( \sequence{1,x} \) and \( \stdbasis_2 \) that acts in this way.
      \begin{equation*}
        1\mapsto\colvec[r]{1 \\ 3}
        \quad\text{and}\quad
        x\mapsto\colvec[r]{2 \\ 4}
      \end{equation*}

      The second reason is that, even if the domain and
      codomain of \( h \) and \( \hat{h} \) coincide, different bases produce
      different maps.
      An example is the $\nbyn{2}$ identity matrix
      \begin{equation*}
        I=\begin{mat}[r]
          1  &0  \\
          0  &1
        \end{mat}
      \end{equation*}
      which represents the identity map on $\Re^2$ with respect to
      $\stdbasis_2,\stdbasis_2$.
      However, with respect to $\stdbasis_2$ for the domain but the basis 
      $D=\sequence{\vec{e}_2,\vec{e}_1}$ for the codomain,
      the same matrix $I$ represents the map that swaps the first and second
      components 
      \begin{equation*}
        \colvec{x \\ y}\mapsto\colvec{y \\ x}
      \end{equation*}
      (that is, reflection about the line $y=x$).
    \end{answer}
  \item \label{exer:MatVecMultRepLinMap} 
    Prove \nearbytheorem{th:MatMultRepsFuncAppl}.
    \begin{answer}
      We mimic \nearbyexample{ex:TypLinMapRepByMat}, just replacing the 
      numbers with letters.

      Write \( B \) as \( \sequence{\vec{\beta}_1,\ldots,\vec{\beta}_n} \)
      and  \( D \) as \( \sequence{\vec{\delta}_1,\dots,\vec{\delta}_m} \).
      By definition of representation of a map with respect to bases,
      the assumption that
      \begin{equation*}
        \rep{h}{B,D}
        =\begin{mat}
           h_{1,1} &\ldots  &h_{1,n}  \\
           \vdots  &        &\vdots   \\
           h_{m,1} &\ldots  &h_{m,n}
         \end{mat}
      \end{equation*}
      means that
      $h(\vec{\beta}_i)=h_{i,1}\vec{\delta}_1+\dots+h_{i,n}\vec{\delta}_n$.
      And, by the definition of the representation of a vector with respect to
      a basis, the assumption that
      \begin{equation*}
        \rep{\vec{v}}{B}=\colvec{c_1 \\ \vdots \\ c_n}
      \end{equation*}
      means that \( \vec{v}=c_1\vec{\beta}_1+\cdots+c_n\vec{\beta}_n \).
      Substituting gives
      \begin{align*}
        h(\vec{v})
        &=h(c_1\cdot\vec{\beta}_1+\dots+c_n\cdot\vec{\beta}_n)      \\
        &=c_1\cdot h(\vec{\beta}_1)+\dots+c_n\cdot \vec{\beta}_n    \\
        &=c_1\cdot (h_{1,1}\vec{\delta}_1+\dots+h_{m,1}\vec{\delta}_m) 
        +\dots                                         
        +c_n\cdot (h_{1,n}\vec{\delta}_1+\dots+h_{m,n}\vec{\delta}_m) \\
        &=(h_{1,1}c_1+\dots+h_{1,n}c_n)\cdot\vec{\delta}_1    
        +\cdots                                
        +(h_{m,1}c_1+\dots+h_{m,n}c_n)\cdot\vec{\delta}_m
      \end{align*}
      and so $h(\vec{v})$ is represented as required.   
    \end{answer}
  \recommended \item  
    \nearbyexample{exam:RepsOfRigidPlaneMaps} shows how to represent 
    rotation of all vectors in the plane through an angle
    \( \theta \) about the origin,
    with respect to the standard bases.
    \begin{exparts}
      \partsitem Rotation of all vectors in three-space through an angle
        \( \theta \) about the \( x \)-axis is a transformation of $\Re^3$.
        Represent it with respect to the standard bases.
        Arrange the rotation so that 
        to someone whose feet are at the origin and
        whose head is at \( (1,0,0) \), the movement appears clockwise.
      \partsitem Repeat the prior item, only rotate about the \( y \)-axis 
        instead.
        (Put the person's head at $\vec{e}_2$.)
      \partsitem Repeat, about the \( z \)-axis.
      \partsitem Extend the prior item to \( \Re^4 \).
        (\textit{Hint:} we can restate 
        `rotate about the \( z \)-axis' as `rotate parallel
        to the \( xy \)-plane'.)
    \end{exparts}
    \begin{answer}
      \begin{exparts}
        \partsitem The picture is this.
          \begin{center}  \small
            \includegraphics{map/mp/ch3.77}
         \end{center}
         The images of the vectors from the domain's basis 
         \begin{equation*}
           \colvec[r]{1 \\ 0 \\ 0}\mapsto\colvec[r]{1 \\ 0 \\ 0}
           \qquad
           \colvec[r]{0 \\ 1 \\ 0}\mapsto\colvec{0 \\ \cos\theta \\ -\sin\theta}
           \qquad
           \colvec[r]{0 \\ 0 \\ 1}\mapsto\colvec{0 \\ \sin\theta \\ \cos\theta}
         \end{equation*}
         are represented with respect to the codomain's basis
         (again, $\stdbasis_3$) by themselves, 
         so adjoining the representations to
         make the matrix gives this.
         \begin{equation*}
            \rep{r_\theta}{\stdbasis_3,\stdbasis_3}=
            \begin{mat}
              1  &0       &0                \\
              0  &\cos\theta   &\sin\theta   \\
              0  &-\sin\theta  &\cos\theta
            \end{mat}                  
         \end{equation*}
       \partsitem The picture is similar to the one in the prior answer. 
         The images of the vectors from the domain's basis 
         \begin{equation*}
           \colvec[r]{1 \\ 0 \\ 0}\mapsto\colvec{\cos\theta \\ 0 \\ \sin\theta}
           \qquad
           \colvec[r]{0 \\ 1 \\ 0}\mapsto\colvec[r]{0 \\ 1 \\ 0}
           \qquad
           \colvec[r]{0 \\ 0 \\ 1}\mapsto\colvec{-\sin\theta \\ 0 \\ \cos\theta}
         \end{equation*}
         are represented with respect to the codomain's basis $\stdbasis_3$
         by themselves, so this is the matrix.
         \begin{equation*}
            \begin{mat}
              \cos\theta  &0       &-\sin\theta   \\
              0           &1       &0             \\
              \sin\theta  &0       &\cos\theta
            \end{mat}                  
         \end{equation*}
       \partsitem To a person standing up, with the vertical $z$-axis,
         a rotation of the $xy$-plane that is clockwise proceeds from
         the positive $y$-axis to the positive $x$-axis.
         That is, it rotates opposite to the direction in  
         \nearbyexample{exam:RepsOfRigidPlaneMaps}.
         The images of the vectors from the domain's basis 
         \begin{equation*}
           \colvec[r]{1 \\ 0 \\ 0}\mapsto\colvec{\cos\theta \\ -\sin\theta \\ 0}
           \qquad
           \colvec[r]{0 \\ 1 \\ 0}\mapsto\colvec{\sin\theta \\ \cos\theta \\ 0}
           \qquad
           \colvec[r]{0 \\ 0 \\ 1}\mapsto\colvec[r]{0 \\ 0 \\ 1}
         \end{equation*}
         are represented with respect to $\stdbasis_3$
         by themselves, so the matrix is this.
         \begin{equation*}
            \begin{mat}
              \cos\theta    &\sin\theta  &0   \\
              -\sin\theta   &\cos\theta  &0    \\
              0             &0           &1
            \end{mat}                  
         \end{equation*}
        \partsitem 
            $\begin{mat}
              \cos\theta  &\sin\theta &0 &0 \\
              -\sin\theta  &\cos\theta  &0 &0 \\
              0           &0           &1 &0 \\
              0           &0           &0 &1
            \end{mat}$
      \end{exparts}   
    \end{answer}
  \item (Schur's Triangularization Lemma)\index{Triangularization}%
    \index{matrix!triangular}
    \begin{exparts}
      \partsitem Let \( U \) be a subspace of \( V \) and fix bases
        \( B_U\subseteq B_V \).
        What is the relationship between the representation of a vector
        from \( U \) with
        respect to \( B_U \) and the representation of that vector
        (viewed as a member of \( V \)) with
        respect to \( B_V \)?
      \partsitem What about maps?
      \partsitem Fix a basis 
        \( B=\sequence{\vec{\beta}_1,\dots,\vec{\beta}_n} \)
        for \( V \) and observe that the spans
        \begin{equation*}
          \spanof{\emptyset}=\set{\zero}\subset\spanof{\set{\vec{\beta}_1}}
                        \subset\spanof{\set{\vec{\beta}_1,\vec{\beta}_2}}
               \subset \quad\cdots\quad 
               \subset\spanof{B}=V
        \end{equation*}
        form a strictly increasing chain of subspaces.
        Show that for any linear map \( \map{h}{V}{W} \) there is a chain
        \( W_0=\set{\zero}\subseteq W_1\subseteq \dots \subseteq W_m =W \) of
        subspaces of \( W \) such that 
        \begin{equation*}
          h(\spanof{\set{\vec{\beta}_1,\dots,\vec{\beta}_i}})\subseteq W_i
        \end{equation*}
        for each \( i \).
      \partsitem Conclude that for every linear map 
        \( \map{h}{V}{W} \) there are
        bases \( B,D \) so the matrix representing \( h \) with respect to
        \( B,D \) is upper-triangular
        (that is, each entry \( h_{i,j} \) with \( i>j \) is zero).
      \item Is an upper-triangular representation unique?
    \end{exparts}
    \begin{answer}
      \begin{exparts}
        \partsitem 
          Write the basis \( B_U \) as 
          \( \sequence{\vec{\beta}_1,\dots,\vec{\beta}_k} \) and
          then write $B_V$ as the extension 
          \( \sequence{\vec{\beta}_1,\dots,\vec{\beta}_k,
                            \vec{\beta}_{k+1},\dots,\vec{\beta}_n} \).
          If 
          \begin{equation*}
            \rep{\vec{v}}{B_U}=\colvec{c_1 \\ \vdots \\ c_k}
          \end{equation*}
          so that $\vec{v}=c_1\cdot\vec{\beta}_1+\cdots+c_k\cdot\vec{\beta}_k$
          then
          \begin{equation*}
            \rep{\vec{v}}{B_V}=\colvec{c_1 \\ \vdots\\ c_k \\ 0 \\ \vdots \\ 0}
          \end{equation*}
          because $\vec{v}=c_1\cdot\vec{\beta}_1+\dots+c_k\cdot\vec{\beta}_k
                    +0\cdot\vec{\beta}_{k+1}+\dots+0\cdot\vec{\beta}_n$.
        \partsitem
          We must first decide what the question means.
          Compare \( \map{h}{V}{W} \) with its restriction to the subspace
          \( \map{\restrictionmap{h}{U}}{U}{W} \).
          The range space of the restriction is a subspace of \( W \), so fix a
          basis \( D_{h(U)} \) for this range space and extend it to a basis 
          \( D_V \) for \( W \).
          We want the relationship between these two.
          \begin{equation*}
            \rep{h}{B_V,D_V}
            \quad\text{and}\quad
            \rep{\restrictionmap{h}{U}}{B_U,D_{h(U)}}
          \end{equation*}
          The answer falls right out of the prior item:~if
          \begin{equation*}
            \rep{\restrictionmap{h}{U}}{B_U,D_{h(U)}}
             =\begin{mat}
                h_{1,1}  &\ldots  &h_{1,k}  \\
                \vdots   &        &\vdots   \\
                h_{p,1}  &\ldots  &h_{p,k}
              \end{mat}
          \end{equation*}
          then the extension is represented in this way.
          \begin{equation*}
            \rep{h}{B_V,D_V}
             =\begin{mat}
                h_{1,1}  &\ldots  &h_{1,k}  &h_{1,k+1}  &\ldots  &h_{1,n}  \\
                \vdots   &        &         &           &        &\vdots   \\
                h_{p,1}  &\ldots  &h_{p,k}  &h_{p,k+1}  &\ldots  &h_{p,n}  \\
                0        &\ldots  &0        &h_{p+1,k+1}&\ldots  &h_{p+1,n}  \\
                \vdots   &        &         &           &        &\vdots   \\
                0        &\ldots  &0        &h_{m,k+1}  &\ldots  &h_{m,n}
              \end{mat}
          \end{equation*}
        \partsitem Take \( W_i \) to be the span of
          \( \set{h(\vec{\beta}_1),\dots,h(\vec{\beta}_i)} \).
        \partsitem Apply the answer from the second item to the third item.
        \partsitem No.
          For instance \( \map{\pi_x}{\Re^2}{\Re^2} \), projection onto
          the \( x \)~axis, is represented by these two upper-triangular
          matrices 
          \begin{equation*}
             \rep{\pi_x}{\stdbasis_2,\stdbasis_2}=
             \begin{mat}[r]
               1  &0  \\
               0  &0
             \end{mat}
             \quad\text{and}\quad
             \rep{\pi_x}{C,\stdbasis_2}=
             \begin{mat}[r]
               0  &1  \\
               0  &0
             \end{mat}
          \end{equation*}
          where \( C=\sequence{\vec{e}_2,\vec{e}_1} \).
      \end{exparts}  
    \end{answer}
\index{homomorphism!matrix representing|)}
\end{exercises}















\subsection{Any Matrix Represents a Linear Map}

%<*MapDefinedByMat0>
The prior subsection shows that
the action of a linear map $h$ is described by a matrix $H$,
with respect to appropriate bases, in this way.
\begin{equation*}
 \vec{v}=\colvec{v_1 \\ \vdots \\ v_n}_B
  \quad\overset{h}{\underset{H}{\longmapsto}}\quad
  h(\vec{v})=
  \colvec{h_{1,1}v_1+\dots+h_{1,n}v_n \\ 
                     \vdots                      \\
                     h_{m,1}v_1+\dots+h_{m,n}v_n}_D
  \tag{$*$}
\end{equation*}
Here we will show the converse, that
each matrix represents a linear map.
%</MapDefinedByMat0>

%<*MapDefinedByMat1>
So we start with a matrix
\begin{equation*}
  H=\generalmatrix{h}{n}{m}
\end{equation*}
and we will describe how it defines a map~$h$.
We require that the map be represented by the matrix so 
first note that in ($*$) the dimension of the map's domain is the
number of columns~$n$ of the matrix 
and the dimension of the codomain is the number of rows~$m$.
Thus, for $h$'s domain 
fix an \( n \)-dimensional vector space $V$ and for the
codomain fix an \( m \)-dimensional space $W$.
Also fix bases
\( B=\sequence{\vec{\beta}_1,\dots,\vec{\beta}_n} \) and
\( D=\sequence{\vec{\delta}_1,\dots,\vec{\delta}_m} \) for those spaces.
%</MapDefinedByMat1>

%<*MapDefinedByMat2>
Now let \( \map{h}{V}{W} \) be:~where $\vec{v}$ in the domain
has the representation
\begin{equation*}
  \rep{\vec{v}}{B}
    =\colvec{v_1 \\ \vdots \\ v_n}_B
\end{equation*}
then its image \( h(\vec{v}) \) is the member of the codomain 
with this representation.
\begin{equation*}
  \rep{\,h(\vec{v})\,}{D}
    =\colvec{h_{1,1}v_1+\dots+h_{1,n}v_n \\ \vdots \\ 
                   h_{m,1}v_1+\dots+h_{m,n}v_n}_D
\end{equation*}
That is, to compute the action of $h$ on 
any $\vec{v}\in V$, first express $\vec{v}$ with respect to the
basis
$\vec{v}=v_1\vec{\beta}_1+\dots+v_n\vec{\beta}_n$ and then 
$h(\vec{v})=
 (h_{1,1}v_1+\dots+h_{1,n}v_n)\cdot\vec{\delta}_1
  +\dots+
  (h_{m,1}v_1+\dots+h_{m,n}v_n)\cdot\vec{\delta}_m$.
%</MapDefinedByMat2>

%<*MapDefinedByMat3>
Above we have made some choices; for instance $V$ can be any
$n$-dimensional space and $B$ could be any basis for $V$,
so $H$ does not define a unique function.
However, note   
once we have fixed $V$, $B$, $W$, and $D$ then 
$h$ is well-defined
since $\vec{v}$ has a unique representation with respect to the basis $B$
% by Theorem~II.\ref{th:BasisIffUniqueRepWRT} 
and the calculation of 
$\vec{w}$ from its representation is also uniquely determined.
%</MapDefinedByMat3>

\begin{example}
Consider this matrix.
\begin{equation*}
  H=
  \begin{mat}
    1 &2 \\
    3 &4 \\
    5 &6
  \end{mat}
\end{equation*}
It is $\nbym{3}{2}$ so any map that it defines must carry a dimension~$2$
domain to a dimension~$3$ codomain.
We can choose the domain and codomain to be $\Re^2$ and $\polyspace_2$, 
with these bases.
\begin{equation*}
  B=\sequence{\colvec{1 \\ 1}, \colvec[r]{1 \\ -1}}
  \qquad
  D=\sequence{x^2, x^2+x, x^2+x+1}
\end{equation*}
Then let $\map{h}{\Re^2}{\polyspace_2}$ be the function defined by~$H$.
We will compute the image under $h$ of this member of the domain.
\begin{equation*}
  \vec{v}=\colvec[r]{-3 \\ 2}
\end{equation*}
The computation is straightforward.
\begin{equation*}
  \rep{h(\vec{v})}{D}
  =H\cdot\rep{\vec{v}}{B}
  =
  \begin{mat}
    1  &2  \\
    3  &4  \\
    5 &6
  \end{mat}
  \colvec{-1/2 \\ -5/2}
  =\colvec{-11/2 \\ -23/2 \\ -35/2}
\end{equation*}
From its representation, computation of $\vec{w}$ is routine
$(-11/2)(x^2)-(23/2)(x^2+x)-(35/2)(x^2+x+1)
=(-69/2)x^2-(58/2)x-(35/2)$.  
\end{example}


\begin{theorem} \label{th:MatIsLinMap}\index{homomorphism!matrix representing}
%<*th:MatIsLinMap>
Any matrix represents a homomorphism between vector spaces of
appropriate dimensions, with respect to any pair of bases.
%</th:MatIsLinMap>
\end{theorem}

\begin{proof}
%<*pf:MatIsLinMap>
We must check that for any matrix $H$ and any domain and codomain bases $B,D$,
the defined map \( h \) is linear.
If $\vec{v}, \vec{u}\in V$ are such that
\begin{equation*}
  \rep{\vec{v}}{B}=\colvec{v_1 \\ \vdots \\ v_n}
    \qquad
  \rep{\vec{u}}{B}=\colvec{u_1 \\ \vdots \\ u_n}
\end{equation*}
and \( c,d\in\Re \) then the calculation
\begin{align*}
  h(c\vec{v}+d\vec{u})
  &=\bigl(h_{1,1}(cv_1+du_1)+\dots+
          h_{1,n}(cv_n+du_n)\bigr)\cdot\vec{\delta}_1+  \\
  & \hbox{}\quad\cdots+\bigl(h_{m,1}(cv_1+du_1)+\dots
         +h_{m,n}(cv_n+du_n)\bigr)\cdot\vec{\delta}_m  \\
  &=c\cdot h(\vec{v})+d\cdot h(\vec{u})
\end{align*}
supplies that check.
%</pf:MatIsLinMap>
\end{proof}

\begin{example} \label{ex:CngBasesChgMap}
Even if the domain and codomain are the same, the map that the 
matrix represents depends on the 
bases that we choose.
If
\begin{equation*}
  H=
  \begin{mat}[r]
    1  &0  \\
    0  &0
  \end{mat},
  \quad
  B_1=D_1=\sequence{\colvec[r]{1 \\ 0},\colvec[r]{0 \\ 1} },
  \quad\text{and}\quad
  B_2=D_2=\sequence{\colvec[r]{0 \\ 1},\colvec[r]{1 \\ 0} },
\end{equation*}
then \( \map{h_1}{\Re^2}{\Re^2} \) represented by \( H \)
with respect to \( B_1,D_1 \) maps
\begin{equation*}
  \colvec{c_1 \\ c_2}
  =\colvec{c_1 \\ c_2}_{B_1}
  \quad
  \mapsto
  \quad
  \colvec{c_1 \\ 0}_{D_1}
  =
  \colvec{c_1 \\ 0}
\end{equation*}
while \( \map{h_2}{\Re^2}{\Re^2} \) represented by \( H \)
with respect to \( B_2,D_2 \) is this map.
\begin{equation*}
  \colvec{c_1 \\ c_2}
  =\colvec{c_2 \\ c_1}_{B_2}
  \quad
  \mapsto
  \quad
  \colvec{c_2 \\ 0}_{D_2}
  =
  \colvec{0 \\ c_2}
\end{equation*}
These are different functions.
The first is projection onto the \( x \)-axis while the second 
is projection onto the $y$-axis.
\end{example}

% So not only is any linear map described by a
% matrix but any matrix describes a linear map.
This result means that when convenient we can 
work solely with matrices,
just doing the computations without having to
worry whether a matrix of interest represents 
a linear map on some pair of spaces.

When we are working with a matrix 
but we do not have particular spaces or bases in mind then
we can take the
domain and codomain to be $\Re^n$ and $\Re^m$, with the standard
bases.
This is convenient because with the standard bases
vector representation is transparent\Dash
the representation of $\vec{v}$ is $\vec{v}$.
(In this case the
column space of the matrix equals the range of the map and
consequently
the column space of \( H \) is often denoted by \( \rangespace{H} \).) 

% With the theorem, we have characterized that for fixed bases we can 
% pass back and forth between a linear map and its representation as a matrix.
% Each linear map is described by a matrix and each matrix describes a 
% linear map
Given a matrix, to come up with an associated map we can choose among many 
domain and codomain spaces, and many bases for those. 
So a matrix can represent many maps.
We finish this section by illustrating how the matrix can 
give us information
about the associated maps.

\begin{theorem} \label{th:RankMatEqRankMap}
%<*th:RankMatEqRankMap>
\index{rank}\index{homomorphism!rank}\index{matrix!rank}
The rank of a matrix equals the rank of any map that it represents.
%</th:RankMatEqRankMap>
\end{theorem}

\begin{proof}
%<*pf:RankMatEqRankMap0>
Suppose that the matrix \( H \) is \( \nbym{m}{n} \). 
Fix domain and codomain spaces $V$ and $W$ of dimension $n$ and~$m$ with
bases \( B=\sequence{\vec{\beta}_1,\dots,\vec{\beta}_n} \) and \( D \).
Then \( H \) represents some linear map $h$ between those spaces with respect 
to these bases whose range space
\begin{align*}
  \set{h(\vec{v})\suchthat \vec{v}\in V}
  &=\set{h(c_1\vec{\beta}_1+\dots+c_n\vec{\beta}_n)  
           \suchthat c_1,\dots,c_n\in\Re}                    \\
  &=\set{c_1h(\vec{\beta}_1)+\dots+c_nh(\vec{\beta}_n)
            \suchthat c_1,\dots,c_n\in\Re}
\end{align*}
is the span $\spanof{\set{h(\vec{\beta}_1),\dots,h(\vec{\beta}_n)}}$.
The rank of the map $h$ is the dimension of this range space.
%</pf:RankMatEqRankMap0>

%<*pf:RankMatEqRankMap1>
The rank of the matrix is the dimension of its column space,
the span of the set of its columns
$\spanof{\,\set{\rep{h(\vec{\beta}_1)}{D},\dots,\rep{h(\vec{\beta}_n)}{D}}\,}$.
%</pf:RankMatEqRankMap1>

%<*pf:RankMatEqRankMap2>
To see that the two spans have the same dimension, recall 
from the proof of Lemma~I.\ref{lem:EqDimImpIso} that if we fix a basis then
representation with respect to that basis gives an isomorphism 
$\map{\mbox{Rep}_D}{W}{\Re^m}$.
Under this isomorphism there is a linear relationship among members of the
range space if and only if the same relationship holds in the
column space, e.g,
$\zero=c_1\cdot h(\vec{\beta}_1)+\dots+c_n\cdot h(\vec{\beta}_n)$ if and only if
$\zero=c_1\cdot\rep{h(\vec{\beta}_1)}{D}+\dots+c_n\cdot\rep{h(\vec{\beta}_n)}{D}$.
Hence, a subset of the range space is linearly independent if and only if the
corresponding subset of the column space is linearly independent.
Therefore the size of the largest linearly independent subset of the
range space equals the size of the largest linearly independent subset of the
column space, and so the two spaces have the same dimension.
%</pf:RankMatEqRankMap2>
\end{proof}

That settles the apparent ambiguity in our use of the same
word `rank' to apply both to matrices and to maps.

\begin{example}
Any map represented by
\begin{equation*}
    \begin{mat}[r]
      1  &2  &2  \\
      1  &2  &1  \\
      0  &0  &3  \\
      0  &0  &2
    \end{mat}
\end{equation*}
must have three-dimensional domain and a four-dimensional codomain.
In addition, because the rank of this matrix is two 
(we can spot this by eye or get it with Gauss's Method),
any map represented by this matrix has a two-dimensional range space.
\end{example}

\begin{corollary} \label{cor:MatDescsMap}
%<*co:MatDescsMap>
Let $h$ be a linear map represented by a matrix $H$.
Then $h$ is onto if and only if the rank of $H$ equals the number 
of its rows, 
and $h$ is one-to-one if and only if the rank of $H$ equals the number of
its columns.
%</co:MatDescsMap>
\end{corollary}

\begin{proof}
%<*pf:MatDescsMap0>
For the onto half, the dimension of the range space of $h$ is the rank of $h$,
which equals the rank of $H$ by the theorem.
Since the dimension of the codomain of $h$ equals the number of rows in $H$,
if the rank of $H$ equals the number of rows then the dimension of the
range space equals the dimension of the codomain.
But a subspace with the same dimension as its superspace must equal
that superspace
(because any basis for the range space is a linearly independent subset 
of the codomain
whose size is equal to the dimension of the codomain, and thus so this 
basis for the range space must also be 
a basis for the codomain).
%</pf:MatDescsMap0>

%<*pf:MatDescsMap1>
For the other half, 
a linear map is one-to-one if and only if it is an isomorphism
between its domain and its range, that is, if and only if its domain has the
same dimension as its range.
The number of columns in $H$ is the dimension of $h$'s domain and
by the theorem the rank of $H$ equals the dimension of 
$h$'s range.
%</pf:MatDescsMap1>
\end{proof}

\begin{definition}  \label{df:NonsingularMap}
%<*df:NonsingularMap>
A linear map that is one-to-one and onto is 
\definend{nonsingular}\index{homomorphism!nonsingular}%
\index{nonsingular!homomorphism}, otherwise it is 
\definend{singular}\index{homomorphism!singular}%
\index{singular!homomorphism}.
That is, a linear map is nonsingular if and only if it is an isomorphism.
%</df:NonsingularMap>
\end{definition}

\begin{remark}
Some authors use `nonsingular' as a synonym for one-to-one  
while others use it the way that we have here.
The difference is slight because any map is onto its
range space, so a one-to-one map is an isomorphism with its range.
\end{remark}

In the first chapter we defined a matrix to be nonsingular 
if it is square and is
the matrix of coefficients of a linear system with a unique solution.
The next result justifies our dual use of the term.

\begin{lemma} \label{le:NonsingMatIffNonsingMap}
\index{homomorphism!nonsingular}\index{matrix!nonsingular}
\index{nonsingular}
%<*le:NonsingMatIffNonsingMap>
A nonsingular linear map is represented by a square matrix.
A square matrix represents nonsingular maps if and only if it is a nonsingular
matrix.
Thus, a matrix represents isomorphisms if and only if it is square and
nonsingular.
%</le:NonsingMatIffNonsingMap>
\end{lemma}

\begin{proof}
%<*pf:NonsingMatIffNonsingMap0>
Assume that the map $\map{h}{V}{W}$ is nonsingular.
\nearbycorollary{cor:MatDescsMap} says that
for any matrix $H$ representing that map, because $h$ is onto the number of rows
of $H$ equals the rank of $H$, 
and because $h$ is one-to-one the number of columns of~$H$
is also equal to the rank of~$H$.
Hence $H$ is square. 
%</pf:NonsingMatIffNonsingMap0>

%<*pf:NonsingMatIffNonsingMap1>
Next assume that $H$ is square, $\nbyn{n}$.
The matrix~$H$ is nonsingular if and only if its row rank is~$n$,
which is true if and only if $H$'s rank is~$n$ 
by Theorem~Two.III.\ref{th:RowRankEqualsColumnRank},
which is true if and only if $h$'s rank is~$n$
by \nearbytheorem{th:RankMatEqRankMap},
which is true if and only if $h$ is an isomorphism 
by Theorem~I.\ref{th:NDimSpaceIsoRN}.
(This last holds because the domain of $h$ is $n$-dimensional as it is the
number of columns in $H$.)
%</pf:NonsingMatIffNonsingMap1>
\end{proof}


\begin{example}
Any map from \( \Re^2 \) to \( \polyspace_1 \) represented 
with respect to any pair of bases by
\begin{equation*}
  \begin{mat}[r]
     1  &2  \\
     0  &3  
  \end{mat}
\end{equation*}
is nonsingular because this matrix has rank two.
\end{example}

\begin{example} \label{ex:NonSMatHasNonSMap}
Any map \( \map{g}{V}{W} \) represented by
\begin{equation*}
  \begin{mat}[r]
    1  &2  \\
    3  &6
  \end{mat}
\end{equation*}
is singular because this matrix is singular.
\end{example}

We've now seen that the relationship between maps and 
matrices goes both ways:~for a particular pair of bases, 
any linear map is represented by a
matrix and any matrix describes a linear map.
That is, by fixing spaces and bases we get
a correspondence between maps and matrices.
In the rest of this chapter we will explore this correspondence.
For instance, we've defined for linear maps the operations of addition 
and scalar multiplication and we shall see what the corresponding matrix 
operations are.
We shall also see the matrix operation that represent the map operation
of composition.
And, we shall see how to find the matrix that represents a map's inverse.


\begin{exercises}
  \item For each matrix, state the dimension of the domain and codomain
    of any map that the matrix represents.
    \begin{exparts*}
      \partsitem 
        $\begin{mat}
          2  &1  \\
          3  &4
        \end{mat}$
      \partsitem 
        $\begin{mat}
          1 &1 &-3 \\ 
          2 &5 &0
        \end{mat}$
      \partsitem 
        $\begin{mat}
           1 &3 \\
           1 &4 \\
           1 &-1
        \end{mat}$
      \partsitem 
        $\begin{mat}
           0 &0 &0 \\
           0 &0 &0
        \end{mat}$
      \partsitem 
        $\begin{mat}
           1 &-1 &4 &5 \\
           0 &0  &0 &0
        \end{mat}$
    \end{exparts*}
    \begin{answer}
      \begin{exparts}
        \partsitem domain: $2$, codomain: $2$
        \partsitem domain: $3$, codomain: $2$
        \partsitem domain: $2$, codomain: $3$
        \partsitem domain: $3$, codomain: $2$
        \partsitem domain: $4$, codomain: $2$
      \end{exparts}
    \end{answer}
  \item Consider a linear map $\map{f}{V}{W}$ represented with respect to 
    some bases $B,D$ by the matrix.
    Decide if that map is nonsingular.
    \begin{exparts*}
      \partsitem $\begin{mat}
           2  &1 \\
           3  &4
         \end{mat}$
      \partsitem $\begin{mat}
           1 &1 \\  
          -3 &-3
         \end{mat}$
      \partsitem $\begin{mat}
          3 &0 &0 \\
          2 &1 &0 \\
          4 &4 &4
         \end{mat}$
      \partsitem $\begin{mat}
          2 &0 &-2 \\
          1 &1 &0 \\
          4 &1 &-4
         \end{mat}$
    \end{exparts*}
    \begin{answer}
      For each we just have to decide if the matrix is nonsingular, perhaps by
      doing Gauss's Method.
      (In truth, we can do each of these by eye.)
      \begin{exparts}
        \partsitem This matrix is nonsingular,
          since the second row is not a multiple of the first, 
          so the map is nonsingular.
        \partsitem The matrix is singular so the map is singular.
        \partsitem Nonsingular.
        \partsitem Nonsingular.
      \end{exparts}
    \end{answer}
  \recommended \item
   Let $h$ be the linear map defined by this matrix on the domain~$\polyspace_1$
   and codomain~$\Re^2$
   with respect to the given bases.
   \begin{equation*}
     H=
     \begin{mat}
       2  &1  \\
       4  &2
     \end{mat}
     \quad
     B=\sequence{1+x,x},\,
     D=\sequence{\colvec{1 \\ 1},\colvec{1 \\ 0}}
   \end{equation*}
   What is the image under $h$ of the vector $\vec{v}=2x-1$?
   \begin{answer}
     With respect to $B$ the vector's representation is this.
     \begin{equation*}
       \rep{2x-1}{B}=\colvec{-1 \\ 3}
     \end{equation*}
     Using the matrix-vector product we can compute $\rep{h(\vec{v})}{D}$
     \begin{equation*}
       \rep{h(2x-1)}{D}
       =
       \begin{mat}
         2  &1  \\
         4  &2
       \end{mat}
       \colvec{-1 \\ 3}_B
       =
       \colvec{1 \\ 2}_D
     \end{equation*}
     From that representation we can compute 
     $h(\vec{v})$.
     \begin{equation*}
       h(2x-1)=1\cdot\colvec{1 \\ 1}+2\cdot\colvec{1 \\ 0}
       =\colvec{3 \\ 1}
     \end{equation*}
   \end{answer}
 \recommended \item 
   Decide if each vector lies in the range of the map from \( \Re^3 \)
   to \( \Re^2 \) represented with respect to the standard bases by the matrix.
   \begin{exparts*}
     \partsitem \( \begin{mat}[r]
                1  &1  &3  \\
                0  &1  &4
              \end{mat}  \),~\( \colvec[r]{1 \\ 3} \)
     \partsitem \( \begin{mat}[r]
                2  &0  &3  \\
                4  &0  &6
              \end{mat}  \),~\( \colvec[r]{1 \\ 1} \)
   \end{exparts*}
   \begin{answer} 
     As described in the subsection, with respect to the standard bases,
     representations are transparent,
     and so, for instance, the first matrix describes this map.
     \begin{equation*}
       \colvec[r]{1 \\ 0 \\ 0}=\colvec[r]{1 \\ 0 \\ 0}_{\stdbasis_3}
          \!\mapsto\colvec[r]{1 \\ 0}_{\stdbasis_2}=\colvec[r]{1 \\ 0}
       \qquad
       %\colvec[r]{0 \\ 1 \\ 0}=\colvec[r]{0 \\ 1 \\ 0}_{\stdbasis_3}
       %   \!\mapsto\colvec[r]{1 \\ 1}_{\stdbasis_2}=\colvec[r]{1 \\ 1}
       \colvec[r]{0 \\ 1 \\ 0}
          \!\mapsto\colvec[r]{1 \\ 1}
       \qquad
       %\colvec[r]{0 \\ 0 \\ 1}=\colvec[r]{0 \\ 0 \\ 1}_{\stdbasis_3}
       %   \!\mapsto\colvec[r]{3 \\ 4}_{\stdbasis_2}=\colvec[r]{3 \\ 4}
       \colvec[r]{0 \\ 0 \\ 1}
          \!\mapsto\colvec[r]{3 \\ 4}
     \end{equation*}
     So, for this first one, we are asking whether there are scalars such that
     \begin{equation*}
       c_1\colvec[r]{1 \\ 0}+c_2\colvec[r]{1 \\ 1}
           +c_3\colvec[r]{3 \\ 4}=\colvec[r]{1 \\ 3}
     \end{equation*}
     that is, whether the vector is in the column space of the matrix.
     \begin{exparts}
       \partsitem Yes.
         We can get this conclusion by setting up the resulting linear system
         and applying Gauss's Method, as usual.
         Another way to get it is to note by inspection of the equation of
         columns that taking $c_3=3/4$, and $c_1=-5/4$, and $c_2=0$ will do.  
         Still a third way to get this conclusion is to note that the rank
         of the matrix is two, which equals the dimension of the
         codomain, and so the map is onto\Dash the range is all of $\Re^2$ and
         in particular includes the given vector.
       \partsitem No; note that all of the columns in the matrix have a second
         component that is twice the first, while the vector does not.
         Alternatively, the column space of the matrix is 
         \begin{equation*}
           \set{c_1\colvec[r]{2 \\ 4}
                +c_2\colvec[r]{0 \\ 0}
                +c_3\colvec[r]{3 \\ 6} \suchthat c_1,c_2,c_3\in\Re}
           =\set{c\colvec[r]{1 \\ 2}\suchthat c\in\Re}
         \end{equation*}
         (which is the fact already noted, but we got it by calculation 
         rather than inspiration), and the given vector is not in this set.
     \end{exparts}  
    \end{answer}
  \recommended \item  
    Consider this matrix, representing a 
    transformation of $\Re^2$ with respect to the bases.
    \begin{equation*}
      \frac{1}{2}\cdot
      \begin{mat}[r]
        1  &1  \\
        -1 &1
      \end{mat}
      \qquad
      B=\sequence{\colvec[r]{0 \\ 1},\colvec[r]{1 \\ 0}}
      \quad 
      D=\sequence{\colvec[r]{1 \\ 1},\colvec[r]{1 \\ -1}}
    \end{equation*}
    \begin{exparts}
      \partsitem To what vector in the codomain 
        is the first member of $B$ mapped? 
      \partsitem The second member?
      \partsitem Where is a general vector from the domain (a vector with 
        components $x$ and $y$) mapped? 
        That is, what transformation of \( \Re^2 \) is represented with 
        respect to \( B,D \) by this matrix?
    \end{exparts}
    \begin{answer}
      \begin{exparts}
        \partsitem The first member of the basis
          \begin{equation*}
            \colvec[r]{0 \\ 1}=\colvec[r]{1 \\ 0}_B
          \end{equation*}
          maps to 
          \begin{equation*}
            \colvec[r]{1/2 \\ -1/2}_D
          \end{equation*}
          which is this member of the codomain.
          \begin{equation*}
            \frac{1}{2}\cdot\colvec[r]{1 \\ 1}
              -\frac{1}{2}\cdot\colvec[r]{1 \\ -1}
              =\colvec[r]{0 \\ 1}
          \end{equation*}
        \partsitem The second member of the basis maps
          \begin{equation*}
            \colvec[r]{1 \\ 0}=\colvec[r]{0 \\ 1}_B
            \mapsto
            \colvec[r]{(1/2 \\ 1/2}_D 
          \end{equation*}
          to this member of the codomain.
          \begin{equation*}
            \frac{1}{2}\cdot\colvec[r]{1 \\ 1}
              +\frac{1}{2}\cdot\colvec[r]{1 \\ -1}
              =\colvec[r]{1 \\ 0}
          \end{equation*}
        \partsitem Because the map that the matrix represents is the identity
          map on the basis, it must be the identity on all members of the 
          domain.
          We can come to the same conclusion in another way by considering
          \begin{equation*}
            \colvec{x \\ y}=\colvec{y \\ x}_B
          \end{equation*}
          which maps to
          \begin{equation*}
            \colvec{(x+y)/2 \\ (x-y)/2}_D
          \end{equation*}
          which represents this member of $\Re^2$. 
          \begin{equation*}
            \frac{x+y}{2}\cdot\colvec[r]{1 \\ 1}
              +\frac{x-y}{2}\cdot\colvec[r]{1 \\ -1}
            =\colvec{x \\ y}
          \end{equation*}
      \end{exparts}
    \end{answer}
  \item Consider a homomorphism  $\map{h}{\Re^2}{\Re^2}$ represented with 
    respect to the standard bases $\stdbasis_2,\stdbasis_2$ by this matrix.
    \begin{equation*}
      \begin{mat}
        1  &3  \\
        2  &4
      \end{mat}
    \end{equation*}
    Find the image under $h$ of each vector.
    \begin{exparts*}
      \partsitem $\colvec{2  \\ 3}$
      \partsitem $\colvec{0  \\ 1}$
      \partsitem $\colvec{-1 \\ 1}$
    \end{exparts*}
    \begin{answer}
      Since, with respect to the standard basis~$\stdbasis_2$, 
      a matrix is represented 
      by itself, we just need to do the matrix-vector multiplication.
      \begin{exparts}
        \partsitem 
          \begin{align*}
            \rep{h(\colvec{2 \\ 3})}{\stdbasis_2}       
             &=\rep{h}{\stdbasis_2,\stdbasis_2}\;
             \rep{\colvec{2 \\ 3}}{\stdbasis_2}        \\
             &=\begin{mat}
              1  &3  \\
              2  &4
             \end{mat}_{\stdbasis_2,\stdbasis_2}
             \colvec{2 \\ 3}_{\stdbasis_2}                 \\
             &=\colvec{11 \\ 16}_{\stdbasis_2}=
             \colvec{11 \\ 16}
          \end{align*}
        \partsitem
           $\begin{mat}
              1  &3  \\
              2  &4
             \end{mat}
             \colvec{0 \\ 1}=
             \colvec{3 \\ 4}$
        \partsitem
           $\begin{mat}
              1  &3  \\
              2  &4
             \end{mat}
             \colvec{-1 \\ 1}=
             \colvec{2 \\ 2}$
      \end{exparts}
    \end{answer}
  \item 
    What transformation of
    \( F=\set{a\cos\theta+b\sin\theta\suchthat a,b\in\Re} \)
    is represented with respect to
    \( B=\sequence{\cos\theta-\sin\theta,\sin\theta} \) and
    \( D=\sequence{\cos\theta+\sin\theta,\cos\theta} \) by this matrix?
    \begin{equation*}
      \begin{mat}[r]
          0  &0  \\
          1  &0
      \end{mat}
    \end{equation*}
    \begin{answer}
      A general member of the domain, represented with respect to the
      domain's basis as
      \begin{equation*}
        a\cos\theta+b\sin\theta=\colvec{a \\ a+b}_B
      \end{equation*}
      maps to 
      \begin{equation*}
        \colvec{0 \\ a}_D
          \quad\text{representing}\quad
        0\cdot(\cos\theta+\sin\theta)+a\cdot(\cos\theta)
      \end{equation*}
      and so the linear map represented by the matrix with respect to these
      bases 
      \begin{equation*}
        a\cos\theta+b\sin\theta  
            \mapsto
        a\cos\theta
      \end{equation*}
      is projection onto the first component.  
    \end{answer}
  \recommended \item  
     Decide whether $1+2x$ is in the range of the map from $\Re^3$ to 
     $\polyspace_2$ represented with respect to $\stdbasis_3$ and 
     $\sequence{1,1+x^2,x}$ by this matrix.
     \begin{equation*}
       \begin{mat}[r]
         1  &3  &0  \\
         0  &1  &0  \\
         1  &0  &1
       \end{mat}
     \end{equation*}
     \begin{answer}
       Denote the given basis of
       $\polyspace_2$
       by $B$.
       Application of the linear map is represented by matrix-vector 
       multiplication.
       Thus the first vector in $\stdbasis_3$ maps to the element
       of $\polyspace_2$ represented with respect to~$B$ by
       \begin{equation*}
       \begin{mat}[r]
         1  &3  &0  \\
         0  &1  &0  \\
         1  &0  &1
       \end{mat}
       \colvec[r]{1 \\ 0 \\ 0}
       =
       \colvec[r]{1 \\ 0 \\ 1}  
       \end{equation*}
       and that element is $1+x$.
       Calculate the other two images of basis vectors in the same way.
       \begin{equation*}
         \begin{mat}[r]
           1  &3  &0  \\
           0  &1  &0  \\
           1  &0  &1
         \end{mat}
         \colvec[r]{0 \\ 1 \\ 0}
         =
         \colvec[r]{3 \\ 1 \\ 0}=\rep{4+x^2}{B}
         \quad
         \begin{mat}[r]
           1  &3  &0  \\
           0  &1  &0  \\
           1  &0  &1
         \end{mat}
         \colvec[r]{0 \\ 0 \\ 1}
         =
         \colvec[r]{0 \\ 0 \\ 1}=\rep{x}{B}
       \end{equation*}
       So the range of $h$ is the span of three polynomials 
       $1+x$, $4+x^2$, and $x$. 
       We can thus decide if $1+2x$ is in the range of the map by 
       looking for scalars $c_1$, $c_2$, and $c_3$ such that
       \begin{equation*}
         c_1\cdot(1+x)+c_2\cdot(4+x^2)+c_3\cdot(x)=1+2x
       \end{equation*}
       and obviously $c_1=1$, $c_2=0$, and $c_3=1$ suffice.
       Thus $1+2x$ is in the range, since it is the image of
       this vector. 
       \begin{equation*}
         1\cdot\colvec[r]{1 \\ 0 \\ 0}+0\cdot\colvec[r]{0 \\ 1 \\ 0}
             +1\cdot\colvec[r]{0 \\ 0 \\ 1}
       \end{equation*}

       \textit{Comment.}
       A slicker argument is to note that the matrix is nonsingular,
       so it has rank~$3$, so the range has dimension~$3$,
       and since the codomain has dimension~$3$ the map is onto.
       Thus every polynomial is the image of some vector and in 
       particular~$1+2x$ 
       is the image of a vector in the domain.
     \end{answer}
  \item Find the map that this matrix represents with respect to $B,B$.
    \begin{equation*}
      \begin{mat}
        2  &1  \\
        -1 &0
      \end{mat}
      \qquad
      B=\sequence{\colvec{1 \\ 0},\colvec{1 \\ 1}}
    \end{equation*}
    \begin{answer}
        Where $B=\sequence{\vec{\beta}_1,\vec{\beta}_2}$ we
          can find $\rep{\vec{v}}{B}$ by eye, where $\vec{v}$ is the general 
          vector, with entries $x$ and~$y$.
          \begin{equation*}
            \colvec{x \\ y}=a\cdot\colvec{1 \\ 0}+b\cdot\colvec{1 \\ 1}
            \quad\text{gives}\quad  
            b=y, a=x-y
          \end{equation*}
          Thus the representation of general vector with respect to~$B$ is
          this.
          \begin{equation*}
            \rep{\colvec{x \\ y}}{B}=\colvec{x-y \\ y}
          \end{equation*}
          Compute the effect of the map with matrix-vector 
          multiplication.
          \begin{equation*}
            \begin{mat}
              2  &1  \\
             -1 &0
            \end{mat}_B
            \colvec{x-y \\ y}_B
            =
            \colvec{2(x-y)+y \\ -(x-y)}_B     
            =
            \colvec{2x-y \\ -x+y}_B     
          \end{equation*}
          Finish by converting back to the standard vector representation.
          \begin{equation*}
            \colvec{2x-y \\ -x+y}_B     
            =
            (2x-y)\cdot\colvec{1 \\ 0}+(-x+y)\cdot\colvec{1 \\ 1}
            =\colvec{x \\ -x+y}      
          \end{equation*}
    \end{answer}
  \item 
    \nearbyexample{ex:NonSMatHasNonSMap} gives a matrix that is
    singular and is therefore associated with maps that are singular.
    We cannot state the
    action of the associated map~$g$ on domain elements~$\vec{v}\in V$, 
    because do not know  the domain~$V$ or codomain~$W$ 
    or the starting and ending bases $B$ and~$D$.
    But we can compute what happens to the representations $\rep{\vec{v}}{B,D}$.
    \begin{exparts}
      \partsitem Find the set of column vectors representing the members of
        the null space of any map~$g$ represented by this matrix.
      \partsitem Find the nullity of any such map~$g$.
      \partsitem Find the set of column vectors representing the members of
        the range space of any map~$g$ represented by the matrix.
      \partsitem Find the rank of any such map~$g$.
      \partsitem Check that rank plus nullity equals the dimension of the
        domain.
    \end{exparts}
    \begin{answer}
      Let the matrix be $G$, and suppose that it represents $\map{g}{V}{W}$ 
      with respect to bases $B$ and $D$.
      Because $G$ has two columns, the domain $V$ is two-dimensional.
      Because $G$ has two rows, the codomain $W$ is two-dimensional.
      The action of $g$ on a representation $\rep{\vec{v}}{B,D}$ 
      of a general member of the domain is this.
      \begin{equation*}
        \colvec{x \\ y}_B 
         \;\mapsto\; 
        \colvec{x+2y \\ 3x+6y}_D
      \end{equation*}
      \begin{exparts}
        \partsitem No matter what is the codomain's basis~$D$,
           the only representation of the zero vector $\zero_{W}$ is  
           \begin{equation*}
             \rep{\zero}{D}=\colvec[r]{0 \\ 0}_D
           \end{equation*}
           and so the set of representations of members of the null space is
           this.
           \begin{equation*}
             \set{\colvec{x \\ y}_B\suchthat \text{$x+2y=0$ and $3x+6y=0$}}
             =\set{y\cdot\colvec[r]{-1/2 \\ 1}_D\suchthat y\in\Re}
           \end{equation*}
         \partsitem The nullity is~$1$.
           The representation map $\map{\mbox{Rep}_D}{W}{\Re^2}$
           and its inverse
           are isomorphisms and so preserve the dimension of subspaces.
           The subspace of $\Re^2$ that is in the prior item is
           one-dimensional.
           Therefore, the image of that subspace under the inverse of the
           representation map\Dash the null space of $G$, 
           is also one-dimensional.
         \partsitem The set of representations of members of the range space is
           this.
           \begin{align*}
             \set{\colvec{x+2y \\ 3x+6y}_D\suchthat x,y\in\Re}
             &=\set{x\cdot\colvec{1 \\ 3}_D+y\cdot\colvec{2 \\ 6}_D\suchthat x,y\in\Re}  \\
             &=\set{k\cdot\colvec[r]{1 \\ 3}_D\suchthat k\in\Re}
           \end{align*}
         \partsitem Of course, \nearbytheorem{th:RankMatEqRankMap} gives that
           the rank of the map equals the rank of the matrix, which is one.
           Alternatively, the same argument that we used above for the 
           null space gives here that the dimension of the range space is one.
         \partsitem One plus one equals two. 
      \end{exparts}
    \end{answer}
  \recommended \item
     Take each matrix to represent $\map{h}{\Re^m}{\Re^n}$
     with respect to the standard bases.
     For each (i)~state $m$ and~$n$.
     Then set up an augmented matrix with the given matrix on the left and
     a vector representing a range space element on the right
     (e.g., if the codomain is~$\Re^3$ then in the right-hand column put 
     the three entries $a$, $b$, and $c$).
     Perform Gauss-Jordan reduction.
     Use that to 
     (ii)~find $\rangespace{h}$ and $\rank(h)$ (and state whether the 
     underlying map is onto),
     and (iii)~find $\nullspace{h}$ and $\nullity(h)$
     (and state whether the underlying map is one-to-one).
     \begin{exparts}
       \partsitem     $
         \begin{mat}
           2  &1  \\ 
           -1 &3
         \end{mat}$
      \partsitem
        $\begin{mat}
          0  &1  &3  \\ 
          2  &3  &4  \\
         -2  &-1 &2 
        \end{mat}$
      \partsitem
        $\begin{mat}
          1  &1  \\ 
          2  &1  \\
          3  &1
       \end{mat}$
     \end{exparts}
     \begin{answer}
       \begin{exparts}
         \partsitem
           (i)~The dimension of the domain space is the number of columns~$m=2$.
           The dimension of the codomain space is the number of rows~$n=2$.

           For the rest, we consider this matrix-vector equation.
           \begin{equation*}
             \begin{mat}
               2  &1  \\ 
              -1  &3        
             \end{mat}
             \colvec{x \\ y}
             =
             \colvec{a \\ b}
             \tag{$*$}
           \end{equation*}
           We solve for $x$ and~$y$.
           \begin{equation*}
             \begin{amat}{2}
             2  &1 &a  \\ 
             -1 &3 &b       
             \end{amat}
             \grstep{(1/2)\rho_1+\rho_2}
             \grstep[(2/7)\rho_2]{(1/2)\rho_1}
             \grstep{-(1/2)\rho_2+\rho_1}
             \begin{amat}{2}
               1  &0 &(3/7)a-(1/7)b  \\ 
               0  &1 &(1/7)a+(2/7)b       
             \end{amat}
           \end{equation*}
           (ii)~For all
           \begin{equation*}
             \colvec{a \\ b}\in\Re^2
           \end{equation*}
           in equation~($*$) the system has a solution, by the calculation.
           So the range space is all of the codomain~$\rangespace{h}=\Re^2$.
           The map's rank is the dimension of the range,~$2$.
           The map is onto because the range space is all of the codomain.

           (iii)~Again by the calculation, to find the nullspace,
           setting $a=b=0$ in equation~($*$)
           gives that $x=y=0$.
           The null space is the trivial subspace of the domain.
           \begin{equation*}
             \nullspace{h}=\set{\colvec{0 \\ 0}}
           \end{equation*}
           The nullity is the dimension of that null space,~$0$.
           The map is one-to-one because the null space is trivial. 
         \partsitem
            (i)~The dimension of the domain space is the number of 
            matrix columns,~$m=3$,
            and the dimension of the codomain space is the number of 
            rows,~$n=3$.

           The calculation is this. 
            \begin{multline*}
              \begin{amat}{3}
                0  &1  &3  &a \\ 
                2  &3  &4  &b \\
               -2  &-1 &2  &c
              \end{amat}
              \grstep{\rho_1\leftrightarrow\rho_2}
              \grstep{\rho_1+\rho_3}
              \grstep{-2\rho_2+\rho_3}                 \\
              \grstep{(1/2)\rho_1}
              \grstep{-(3/2)\rho_2+\rho_1}
              \begin{amat}{3}
                1  &0 &-5/2 &-(3/2)a+(1/2)b  \\ 
                0  &1 &3    &a  \\
                0  &0 &0    &-2a+b+c      
              \end{amat}
            \end{multline*}
            (ii)~There are codomain triples
            \begin{equation*}
              \colvec{a \\ b \\ c}\in\Re^3
            \end{equation*}
            for which the system does not have a solution,
            specifically the system only has a solution if $-2a+b+c=0$.
            \begin{equation*}
              \rangespace{h}=\set{\colvec{a \\ b \\ c}\suchthat a=(b+c)/2}
                            =\set{\colvec{1/2 \\ 1 \\ 0}b
                                  +\colvec{1/2 \\ 0 \\ 1}c\suchthat b,c\in\Re} 
            \end{equation*}
            The map's rank is the range's dimension,~$2$.
            The map is not onto because the range space is not all of the 
            codomain.

            (iii)~Setting $a=b=c=0$ in the calculation
            gives infinitely many solutions.
            Parametrizing using the free variable~$z$ leads to this description
            of the nullspace.
            \begin{equation*}
              \nullspace{h}=\set{\colvec{x \\ y \\ z}\suchthat 
                               \text{$y=-3z$ and $x=(5/2)z$}}
                           =\set{\colvec{5/2 \\ -3 \\ 1}z\suchthat z\in\Re}
            \end{equation*}
            The nullity is the dimension of that null space,~$1$.
            The map is not one-to-one because the null space is not trivial. 
          \partsitem
            (i)~The domain has dimension $m=2$ while the codomain has 
            dimension~$n=3$.
            Here is the calculation.
              \begin{equation*}
                \begin{amat}{2}
                  1  &1  &a \\ 
                  2  &1  &b \\
                  3  &1  &c
                \end{amat}
                \grstep[-3\rho_1+\rho_3]{-2\rho_1+\rho_2}
                \grstep{-2\rho_2+\rho_3}
                \grstep{-\rho_2}                 
                \grstep{-\rho_2+\rho_1}
                \begin{amat}{2}
                  1  &0 &-a+b  \\ 
                  0  &1 &2a-b  \\
                  0  &0 &a-2b+c      
                \end{amat}
              \end{equation*}

            (ii)~The range is this subspace of the codomain.
            \begin{equation*}
              \rangespace{h}=\set{\colvec{2b-c \\ b \\ c}\suchthat b,c\in\Re}
                  =\set{\colvec{2 \\ 1 \\ 0}b+\colvec{-1 \\ 0 \\ 1}c\suchthat b,c\in\Re}
            \end{equation*}
            The rank is~$2$.
            The map is not onto.

            (iii)~The null space is the trivial subspace of the domain.
            \begin{equation*}
              \nullspace{h}=\set{\colvec{x \\ y}=\colvec{0 \\ 0}}
            \end{equation*}
            The nullity is~$0$.
            The map is one-to-one.
       \end{exparts}
     \end{answer}
  \item Use the method from the prior exercise on this matrix.
    \begin{equation*}
        \begin{mat}
          1 &0 &-1 \\
          2 &1 &0  \\
          2 &2 &2
        \end{mat}
    \end{equation*}
    \begin{answer}
          Here is the Gauss-Jordan reduction.
          \begin{align*}
            \begin{amat}{3}
                1 &0 &-1 &a \\
                2 &1 &0  &b \\
                2 &2 &2  &c  
            \end{amat}
            &\grstep[-2\rho_1+\rho_3]{-2\rho_1+\rho_2}
            \begin{amat}{3}
                1 &0 &-1 &a \\
                0 &1 &2  &-2a+b \\
                0 &2 &4  &-2a+c  
            \end{amat}                                    \\
            &\grstep{-2\rho_2+\rho_3}
            \begin{amat}{3}
                1 &0 &-1 &a \\
                0 &1 &2  &-2a+b \\
                0 &0 &0  &2a-2b+c  
            \end{amat}
          \end{align*}
          (i)~The dimensions are $m=n=3$.
          (ii)~The range space is the set containing all of the members 
          of the codomain 
          for which this system has a solution.
          \begin{equation*}
            \rangespace{h}=\set{\colvec{b-(1/2)c \\ b \\ c}\suchthat b,c\in\Re}
          \end{equation*}
          The rank is 2.
          Because the rank is less than the dimension~$n=3$ of the codomain,
          the map is not onto.
      
          (iii)~The null space is the set of members of the domain that map to 
          $a=0$, $b=0$, and~$c=0$.
          \begin{equation*}
            \nullspace{h}=\set{\colvec{z \\ -2z \\ z}\suchthat z\in\Re}
          \end{equation*}
          The nullity is~$1$.
          Because the nullity is not~$0$ the map is not one-to-one.      
    \end{answer}
  \item Verify that the map represented by this matrix
        is an isomorphism.
        \begin{equation*}
          \begin{mat}
            2  &1  &0  \\
            3  &1  &1  \\
            7  &2  &1
          \end{mat}
        \end{equation*}
    \begin{answer}
       For any map represented by this matrix, the domain and codomain are 
       each of dimension~$3$.
           To show that the map is an isomorphism, 
           we must show it is both onto and one-to-one.
           For that we don't need to augment the matrix with $a$, $b$, 
           and~$c$; this calculation
             \begin{multline*}
               \begin{mat}
                 2  &1  &0 \\ 
                 3  &1  &1 \\
                 7  &2  &1
               \end{mat}
               \grstep[-(7/2)\rho_1+\rho_3]{-(3/2)\rho_1+\rho_2}
               \grstep{-3\rho_2+\rho_3}
               \grstep[-2\rho_2 \\ -(1/2)\rho_3]{(1/2)\rho_1}                 
               \grstep{2\rho_3+\rho_2}                          \\
               \grstep{-(1/2)\rho_2+\rho_1}
               \begin{mat}
                 1  &0 &0  \\ 
                 0  &1 &0  \\
                 0  &0 &1      
               \end{mat}
             \end{multline*}
          gives that for each codomain vector there is one and only one 
          associated domain vector.
    \end{answer}
  \item This is an alternative proof of 
    \nearbylemma{le:NonsingMatIffNonsingMap}.
    Given an $\nbyn{n}$ matrix $H$,
    fix a domain~$V$ and codomain~$W$
    of appropriate dimension~$n$, and bases $B,D$ for those
    spaces, and consider
    the map~$h$ represented by the matrix.  
    \begin{exparts}
      \partsitem
        Show that $h$ is onto if and only if there is at least one
        $\rep{\vec{v}}{B}$ associated by $H$ with each $\rep{\vec{w}}{D}$.  
      \partsitem
        Show that $h$ is one-to-one if and only if there is at most one
        $\rep{\vec{v}}{B}$ associated by $H$ with each $\rep{\vec{w}}{D}$.  
      \partsitem
        Consider the linear system
        $H\cdot\rep{\vec{v}}{B}=\rep{\vec{w}}{D}$.
        Show that $H$ is nonsingular if and only if 
        there is exactly one solution~$\rep{\vec{v}}{B}$ for each
        $\rep{\vec{w}}{D}$.
    \end{exparts}
    \begin{answer}
      \begin{exparts}
        \partsitem
          The defined map $h$ is onto if and only if 
          for every $\vec{w}\in W$ there is a 
          $\vec{v}\in V$ such that $h(\vec{v})=\vec{w}$.
          Since for every vector there is exactly one representation,
          converting to representations gives that $h$ is onto 
          if and only if for every
          representation $\rep{\vec{w}}{D}$ there is a
          representation $\rep{\vec{v}}{B}$ such that
          $H\cdot\rep{\vec{v}}{B}=\rep{\vec{w}}{D}$.
       \partsitem
          This is just like the prior part.
       \partsitem 
          As described at the start of this subsection, by definition
          the map $h$ defined by the matrix $H$  
          associates this domain vector $\vec{v}$ with this 
          codomain vector $\vec{w}$.
          \begin{equation*}
            \rep{\vec{v}}{B}=\colvec{v_1 \\ \vdots \\ v_n}
            \qquad
            \rep{\vec{w}}{D}
               =
               H\cdot\rep{\vec{v}}{B}
               =\colvec{h_{1,1}v_1+\dots+h_{1,n}v_n  \\ 
                          \vdots \\ 
                       h_{m,1}v_1+\dots+h_{m,n}v_n}
          \end{equation*}
          Fix $\vec{w}\in W$ and 
          consider the linear system
          defined by the above equation.
          \begin{equation*}
            % H\cdot\rep{\vec{v}}{B}=\rep{\vec{w}}{D}
            % \qquad
            \begin{linsys}{3}
              h_{1,1}v_1 &+ &\cdots &+ &h_{1,n}v_n &= &w_1  \\
              h_{2,1}v_1 &+ &\cdots &+ &h_{2,n}v_n &= &w_2  \\
                       &  &        & &          &\vdotswithin{=} & \\
              h_{n,1}v_1 &+ &\cdots &+ &h_{n,n}v_n &= &w_n
            \end{linsys}
          \end{equation*}
          (Again, here the $w_i$ are fixed and
           the $v_j$ are unknowns.)
          Now, $H$ is nonsingular if and only if 
          for all $w_1$, \ldots, $w_n$ this system has a solution and the 
          solution is unique.
          By the first two parts of this exercise 
          this is true if and only if the map $h$ is onto and one-to-one.
          This in turn is true if and only if $h$ is an isomorphism. 
       \end{exparts}
      \end{answer}
  \recommended \item  
    Because
    the rank of a matrix equals the rank of any map it represents, if
    one matrix represents two different maps 
    \( H=\rep{h}{B,D}=\rep{\hat{h}}{\hat{B},\hat{D}} \) 
    (where \( \map{h,\hat{h}}{V}{W} \))
    then the dimension of the range space of
    \( h \) equals the dimension of the range space of \( \hat{h} \).
    Must these equal-dimensional range spaces actually be the same?
    \begin{answer}
      No, the range spaces may differ.
      \nearbyexample{ex:CngBasesChgMap} shows this.
    \end{answer}
  \item 
    Let \( V \) be an \( n \)-dimensional space with bases \( B \) and
    \( D \).
    Consider a map that sends, for \( \vec{v}\in V\), 
    the column vector representing \( \vec{v} \) with
    respect to \( B \) to the column vector representing \( \vec{v} \) with
    respect to \( D \).
    Show that map is a linear transformation of \( \Re^n \).
    \begin{answer}
      Recall that the representation map
      \begin{equation*}
        V\mapsunder{\text{Rep}_{B}}\Re^n
      \end{equation*}
      is an isomorphism.
      Thus, its inverse map $\map{\mbox{Rep}_B^{-1}}{\Re^n}{V}$
      is also an isomorphism.
      The desired transformation of $\Re^n$ is then this composition.
      \begin{equation*}
        \Re^n\mapsunder{\text{Rep}_{B}^{-1}}
        V\mapsunder{\text{Rep}_{D}}\Re^n
      \end{equation*}
      Because a composition of isomorphisms is also an isomorphism, 
      this map $\composed{\mbox{Rep}_{D}}{\mbox{Rep}_{B}^{-1}}$
      is an isomorphism.
    \end{answer}
  \item 
    \nearbyexample{ex:CngBasesChgMap} shows that changing the pair of
    bases can change the map that a matrix
    represents, even though the domain and codomain remain the same.
    Could the map ever not change?
    Is there a matrix \( H \), vector spaces \( V \) and \( W \), and
    associated pairs of bases \( B_1,D_1 \) and \( B_2,D_2 \) (with
    \( B_1\neq B_2 \) or \( D_1\neq D_2 \) or both) 
    such that the map represented
    by \( H \) with respect to \( B_1,D_1 \) equals the map represented
    by \( H \) with respect to \( B_2,D_2 \)?
    \begin{answer}
      Yes.
      Consider
      \begin{equation*}
        H=\begin{mat}[r]
            1  &0  \\
            0  &1
          \end{mat}
      \end{equation*}
      representing a map from \( \Re^2 \) to \( \Re^2 \).
      With respect to the standard bases 
      \( B_1=\stdbasis_2, D_1=\stdbasis_2 \) this matrix
      represents the identity map.
      With respect to
      \begin{equation*}
        B_2=D_2=\sequence{\colvec[r]{1 \\ 1},\colvec[r]{1 \\ -1}}
      \end{equation*}
      this matrix again represents the identity.
      In fact, as long as the starting and ending bases 
      are equal\Dash as long as
      \( B_i=D_i \)\Dash then the map represented by $H$ is the identity.
    \end{answer}
  \recommended \item 
    A square matrix is a 
    \definend{diagonal}\index{matrix!diagonal}\index{diagonal matrix} %
    matrix if it is all zeroes
    except possibly for the entries on its upper-left to lower-right
    diagonal\Dash its \( 1,1 \) entry, its \( 2,2 \) entry, etc.
    Show that a linear map is an isomorphism if there are bases such that,
    with respect to those bases, the map is represented by a diagonal matrix 
    with no zeroes on the diagonal.
    \begin{answer}
      This is immediate from \nearbylemma{le:NonsingMatIffNonsingMap}.
    \end{answer}
  \item 
    Describe geometrically the action on \( \Re^2 \) of
    the map represented with respect to the standard 
    bases $\stdbasis_2,\stdbasis_2$ by this matrix.
    \begin{equation*}
      \begin{mat}[r]
        3  &0  \\
        0  &2
      \end{mat}
    \end{equation*}
    Do the same for these.
    \begin{equation*}
      \begin{mat}[r]
        1  &0  \\
        0  &0
      \end{mat}
      \quad
      \begin{mat}[r]
        0  &1  \\
        1  &0
      \end{mat}
      \quad
      \begin{mat}[r]
        1  &3  \\
        0  &1
      \end{mat}
    \end{equation*}
    \begin{answer}
      The first map 
      \begin{equation*}
        \colvec{x \\ y}=\colvec{x \\ y}_{\stdbasis_2}
        \mapsto
        \colvec{3x \\ 2y}_{\stdbasis_2}=\colvec{3x \\ 2y}
      \end{equation*}
      stretches vectors by a factor of three in the
      \( x \)~direction and by a factor of two in the \( y \)~direction.
      The second map
      \begin{equation*}
        \colvec{x \\ y}=\colvec{x \\ y}_{\stdbasis_2}
        \mapsto
        \colvec{x \\ 0}_{\stdbasis_2}=\colvec{x \\ 0}
      \end{equation*}
      projects vectors onto the \( x \)~axis.
      The third 
      \begin{equation*}
        \colvec{x \\ y}=\colvec{x \\ y}_{\stdbasis_2}
        \mapsto
        \colvec{y \\ x}_{\stdbasis_2}=\colvec{y \\ x}
      \end{equation*}
      interchanges first and second components
      (that is, it is a reflection about the line \( y=x \)).
      The last 
      \begin{equation*}
        \colvec{x \\ y}=\colvec{x \\ y}_{\stdbasis_2}
        \mapsto
        \colvec{x+3y \\ y}_{\stdbasis_2}=\colvec{x+3y \\ y}
      \end{equation*}
      stretches vectors parallel to the \( y \)~axis, by an amount
      equal to three times their distance from that axis 
      (this is a \definend{skew}.)  
     \end{answer}
  \item 
     The fact that for any linear map the rank plus the nullity
     equals the dimension of the domain shows that a necessary
     condition for the existence of a homomorphism between two spaces, onto
     the second space, is that there be no gain in dimension.
     That is, where $\map{h}{V}{W}$ is onto, the dimension of $W$ must
     be less than or equal to the dimension of $V$.
     \begin{exparts}
       \partsitem Show that this (strong) converse holds: 
          no gain in dimension implies that 
          there is a homomorphism and, further, 
          any matrix with the correct size and correct rank 
          represents such a map.
       \partsitem Are there bases for $\Re^3$ such that
          this matrix
          \begin{equation*}
            H=\begin{mat}[r]
                1  &0  &0 \\
                2  &0  &0 \\
                0  &1  &0 
              \end{mat}
          \end{equation*}
          represents a map from $\Re^3$ to $\Re^3$ whose range is
          the $xy$~plane subspace of $\Re^3$?
     \end{exparts}
     \begin{answer}
       \begin{exparts}
         \partsitem  This is immediate from
           \nearbytheorem{th:RankMatEqRankMap}.
         \partsitem Yes.
            This is immediate from the prior item.

            To give a specific example, we can 
            start with $\stdbasis_3$ as the basis for the domain, and then
            we require a basis $D$ for the codomain $\Re^3$.
            The matrix $H$ gives the action of the map as this
            \begin{multline*}
              \colvec[r]{1 \\ 0 \\ 0}=\colvec[r]{1 \\ 0 \\ 0}_{\stdbasis_3}
                 \mapsto\colvec[r]{1 \\ 2 \\ 0}_D
              \quad       
              \colvec[r]{0 \\ 1 \\ 0}=\colvec[r]{0 \\ 1 \\ 0}_{\stdbasis_3}
                 \mapsto\colvec[r]{0 \\ 0 \\ 1}_D
              \\  %\quad       
              \colvec[r]{0 \\ 0 \\ 1}=\colvec[r]{0 \\ 0 \\ 1}_{\stdbasis_3}
                 \mapsto\colvec[r]{0 \\ 0 \\ 0}_D
            \end{multline*}
            and there is no harm in finding a basis $D$ so that
            \begin{equation*}
              \rep{\colvec[r]{1 \\ 0 \\ 0}}{D}=\colvec[r]{1 \\ 2 \\ 0}_D
              \quad\mbox{and}\quad       
              \rep{\colvec[r]{0 \\ 1 \\ 0}}{D}=\colvec[r]{0 \\ 0 \\ 1}_D
            \end{equation*}
            that is, so that the map represented by $H$ with respect to
            $\stdbasis_3,D$ is projection down onto the $xy$~plane.
            The second condition gives that the third member of $D$
            is $\vec{e}_2$.
            The first condition gives that the first member of $D$ plus twice
            the second equals $\vec{e}_1$, and so this basis will do.
            \begin{equation*}
              D=\sequence{\colvec[r]{0 \\ -1 \\ 0},
                          \colvec[r]{1/2 \\ 1/2 \\ 0},
                          \colvec[r]{0 \\ 1 \\ 0}}
            \end{equation*}
       \end{exparts} 
     \end{answer}
  \item 
    Let \( V \) be an \( n \)-dimensional space and suppose
    that \( \vec{x}\in\Re^n \).
    Fix a basis \( B \) for \( V \) and consider the map
    \( \map{h_{\vec{x}}}{V}{\Re} \) given 
    $\vec{v}\mapsto\vec{x}\dotprod\rep{\vec{v}}{B}$ by the dot product.
    \begin{exparts}
      \partsitem Show that this map is linear.
      \partsitem Show that for any linear map \( \map{g}{V}{\Re} \) there is 
        an \( \vec{x}\in\Re^n \) such that \( g=h_{\vec{x}} \).
      \partsitem In the prior item we fixed the basis and varied the 
        \( \vec{x} \) to get all possible linear maps.
        Can we get all possible linear maps by fixing an \( \vec{x} \) and
        varying the basis?
    \end{exparts}
    \begin{answer}
      \begin{exparts}
        \partsitem Recall that the representation map
          $\map{\mbox{Rep}_{B}}{V}{\Re^n}$ is linear (it is actually
          an isomorphism, but we do not need that it is one-to-one or onto
          here).
          Considering the column vector $x$ to be a $\nbym{n}{1}$ matrix
          gives that the map from $\Re^n$ to $\Re$ that takes a column vector
          to its dot product with $\vec{x}$ is linear (this is a matrix-vector
          product and so \nearbytheorem{th:MatIsLinMap} applies).
          Thus the map under consideration $h_{\vec{x}}$ is linear because 
          it is the composition of two linear maps.
          \begin{equation*}
            \vec{v}\mapsto \rep{\vec{v}}{B}
                   \mapsto \vec{x}\cdot\rep{\vec{v}}{B}     
          \end{equation*}
       \partsitem Any linear map $\map{g}{V}{\Re}$ is represented by some
          matrix
          \begin{equation*}
            \begin{mat}
              g_1  &g_2 &\cdots &g_n
            \end{mat}
          \end{equation*}
          (the matrix has $n$ columns because $V$ is $n$-dimensional and it
          has only one row because $\Re$ is one-dimensional).
          Then taking $\vec{x}$ to be the column vector that is the transpose
          of this matrix
          \begin{equation*}
            \vec{x}=\colvec{g_1 \\ \vdots \\ g_n}
          \end{equation*}
          has the desired action.
          \begin{equation*}
            \vec{v}=\colvec{v_1 \\ \vdots \\ v_n}
             \mapsto
            \colvec{g_1 \\ \vdots \\ g_n}\dotprod\colvec{v_1 \\ \vdots \\ v_n}
            =g_1v_1+\dots+g_nv_n
          \end{equation*}
        \partsitem No.
          If \( \vec{x} \) has any nonzero entries then \( h_{\vec{x}} \)
          cannot be the zero map (and if \( \vec{x} \) is the zero vector
          then \( h_{\vec{x}} \) can only be the zero map).
      \end{exparts}  
    \end{answer}
  \item
    Let \( V,W,X \) be vector spaces with bases \( B,C,D \).
    \begin{exparts}
      \partsitem Suppose that \( \map{h}{V}{W} \) 
        is represented with respect to \( B,C \) by the matrix \( H \).
        Give the matrix representing the scalar multiple
        \( rh \) (where \( r\in\Re \)) with
        respect to \( B,C \) by expressing it in terms of \( H \).
      \partsitem Suppose that \( \map{h,g}{V}{W} \) are represented with 
        respect to \( B,C \) by \( H \) and \( G \).
        Give the matrix representing \( h+g \) with
        respect to \( B,C \) by expressing it in terms of \( H \) and \( G \).
      \partsitem Suppose that \( \map{h}{V}{W} \) is represented 
        with respect to \( B,C \) by \( H \) and
        \( \map{g}{W}{X} \) is represented with respect to
        \( C,D \) by \( G \).
        Give the matrix representing \( \composed{g}{h} \) with
        respect to \( B,D \) by expressing it in terms of \( H \) and \( G \).
    \end{exparts}
    \begin{answer}
       See the following section.
    \end{answer}
\end{exercises}
