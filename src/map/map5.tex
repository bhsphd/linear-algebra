% Chapter 3, Section 5 _Linear Algebra_ Jim Hefferon
%  http://joshua.smcvt.edu/linearalgebra
%  2001-Jun-12
\section{Change of Basis}
\index{change of basis|(}
Representations vary with the bases.
For instance, 
with respect to the bases $\stdbasis_2$ and
\begin{equation*}
  B=\sequence{\colvec{1 \\ 1},\colvec{1 \\ -1}}
\end{equation*}
$\vec{e}_1\in\Re^2$ has these different representations.
\begin{equation*}
  \rep{\vec{e}_1}{\stdbasis_2}=\colvec{1 \\ 0}
  \qquad
  \rep{\vec{e}_1}{B}=\colvec{1/2 \\ 1/2}
\end{equation*}
The same holds for maps:
with respect to the basis pairs $\stdbasis_2,\stdbasis_2$ and $\stdbasis_2,B$, 
the identity map
has these representations.
\begin{equation*}
  \rep{\text{id}}{\stdbasis_2,\stdbasis_2}=
   \begin{mat}[r]
     1  &0  \\
     0  &1
   \end{mat}
   \qquad
  \rep{\text{id}}{\stdbasis_2,B}=
   \begin{mat}[r]
     1/2  &1/2  \\
     1/2  &-1/2
   \end{mat}
\end{equation*}
This section shows how to translate among the representations.
That is, 
we will compute how the representations vary as the bases vary.












\subsection{Changing Representations of Vectors}
In converting 
$\rep{\vec{v}}{B}$ to $\rep{\vec{v}}{D}$
the underlying vector $\vec{v}$ doesn't change.
Thus,  
the translation between these two ways of expressing the vector
is accomplished by the identity map on the space,
described so that
the domain space vectors are represented with respect to $B$ and
the codomain space vectors are represented with respect to $D$. 
\index{arrow diagram}
%<*ChangeRepresentationOfVectorArrowDiagram>
\begin{equation*}
  \begin{CD}
    V_{\wrt{B}}                      \\
    @V{\text{\scriptsize$\identity$}} VV   \\
    V_{\wrt{D}}
  \end{CD}
\end{equation*}
(This diagram is
vertical to fit with the ones in the next subsection.)
%</ChangeRepresentationOfVectorArrowDiagram>

\begin{definition}  \label{df:ChangeOfBasisMatrix}
%<*df:ChangeOfBasisMatrix>
The \definend{change of basis matrix}\index{vector!representation of}%
\index{matrix!change of basis}\index{basis!change of}
for bases \( B,D\subset V \) is the representation of the identity
map  \( \map{\identity}{V}{V} \) with respect to those bases.
\begin{equation*}
  \rep{\identity}{B,D}=
  \begin{pmat}{c@{\hspace{1em}}c@{\hspace{1em}}c}
     \vdots                  &             &\vdots                  \\
     \rep{\vec{\beta}_1}{D}  &\;\cdots\;   &\rep{\vec{\beta}_n}{D}  \\
     \vdots                  &             &\vdots
  \end{pmat}
\end{equation*}
%\( B=\basis{\beta}{n} \).
%</df:ChangeOfBasisMatrix>
\end{definition}

\begin{remark}
A better name would be `change of representation matrix' but the above name
is standard.  
\end{remark}

The next result supports the definition.

\begin{lemma}  \label{le:ChBasisMatDoesChBases}
%<*lm:ChBasisMatDoesChBases>
To convert from the representation of a vector~$\vec{v}$ 
with respect to \( B \) 
to its representation with respect to
\( D \) use the change of basis matrix.
\begin{equation*}
  \rep{\identity}{B,D}\,\rep{\vec{v}}{B}=\rep{\vec{v}}{D}
\end{equation*}
Conversely, if left-multiplication by a matrix changes bases 
$M\cdot\rep{\vec{v}}{B}=\rep{\vec{v}}{D}$
then $M$ is a change of basis matrix.
%</lm:ChBasisMatDoesChBases>
\end{lemma}

\begin{proof}
%<*pf:ChBasisMatDoesChBases>
The first sentence holds 
because matrix-vector multiplication represents a map application and so
\( \rep{\identity}{B,D}\cdot\rep{\vec{v}}{B}=\rep{\,\identity(\vec{v})\,}{D}
  =\rep{\vec{v}}{D} \) for each \( \vec{v} \). 
For the second sentence,
with respect to $B,D$ the matrix $M$ represents a linear
map whose action is to map each vector to itself, and is therefore
the identity map.
%</pf:ChBasisMatDoesChBases>
\end{proof}

\begin{example}
With these bases for \( \Re^2 \),
\begin{equation*}
  B=
  \sequence{
            \colvec[r]{2 \\ 1},
            \colvec[r]{1 \\ 0} }
  \qquad
  D=
  \sequence{
            \colvec[r]{-1 \\ 1},
            \colvec[r]{1 \\ 1} }
\end{equation*}
because
\begin{equation*}
  \rep{\,\identity(\colvec[r]{2 \\ 1})}{D}
  =\colvec[r]{-1/2 \\ 3/2}_D
  \qquad
  \rep{\,\identity(\colvec[r]{1 \\ 0})}{D}
  =\colvec[r]{-1/2 \\ 1/2}_D
\end{equation*}
the change of basis matrix is this.
\begin{equation*}
  \rep{\rm id}{B,D}
  =
    \begin{mat}[r]
       -1/2  &-1/2  \\
        3/2  &1/2
    \end{mat}
\end{equation*}
For instance, this 
is the representation of $\vec{e}_2$
\begin{equation*}
  \rep{\colvec[r]{0 \\ 1} }{B}
  =\colvec[r]{1 \\ -2}
  % \qquad
  % \rep{\colvec[r]{0 \\ 1} }{D}
  % =\colvec[r]{1/2 \\ 1/2}
\end{equation*}
and the matrix does the conversion. 
\begin{equation*}
    \begin{mat}[r]
       -1/2  &-1/2  \\
        3/2  &1/2
    \end{mat}
  \colvec[r]{1 \\ -2}
  =
  \colvec[r]{1/2 \\ 1/2}
\end{equation*}
Checking that vector on the right is $\rep{\vec{e}_2 }{D}$ is easy.
\end{example}

We finish this subsection by recognizing 
the change of basis matrices as a familiar set.

\begin{lemma}    \label{le:NonSingIsChBasis}
%<*lm:NonSingIsChBasis>
A matrix changes bases if and only if it is nonsingular.
%</lm:NonSingIsChBasis>
\end{lemma}

\begin{proof}
%<*pf:NonSingIsChBasis0>
For the `only if' direction, 
if left-multiplication by a matrix changes bases then
the matrix represents an invertible function,
simply because we can invert the function by changing the bases back.
Because it represents a function that is invertible, 
the matrix itself is invertible, 
and so is nonsingular.
%</pf:NonSingIsChBasis0>

%<*pf:NonSingIsChBasis1>
For `if' we will show that any nonsingular matrix $M$ 
performs a change of basis operation from any given starting basis $B$ 
(having~$n$ vectors, where the matrix is $\nbyn{n}$) 
to some ending basis.

% The matrix is nonsingular so it will Gauss-Jordan reduce to the
% identity.
If the matrix is the identity~$I$ then the statement is obvious.
Otherwise because the matrix is nonsingular 
Corollary~IV.\ref{cor:ReducViaMatrices} says
there are elementary reduction matrices such that
$R_r\cdots R_1\cdot M=I$ with $r\geq 1$.
Elementary matrices are invertible and their inverses are also elementary
so multiplying both sides of that equation from the left 
by ${R_r}^{-1}$, then by ${R_{r-1}}^{-1}$, etc., gives 
$M$ as a product of elementary matrices
$M={R_1}^{-1}\cdots {R_r}^{-1}$.
% (We've combined ${R_r}^{-1}I$ to make ${R_r}^{-1}$; 
% because $r\geq 1$ we can always have the $I$ disappear in this way.)
%</pf:NonSingIsChBasis1>

%<*pf:NonSingIsChBasis2>
We will be done if we show that elementary matrices 
change a given basis to another basis, since then 
${R_r}^{-1}$ changes $B$ to some other basis $B_r$ and
${R_{r-1}}^{-1}$ changes $B_r$ to some $B_{r-1}$, etc.
We will cover the three types of elementary matrices
separately; recall the notation for the three.
%</pf:NonSingIsChBasis2>
% This equation doesn't fit well on slide, so I copied it over an tweaked it.
\begin{equation*}
  M_{i}(k)
    \colvec{
       c_1     \\
       \vdots  \\
       c_i    \\
       \vdots  \\
       c_n  }
  =
    \colvec{
       c_1     \\
       \vdots  \\
       kc_i    \\
       \vdots  \\
       c_n  }
  \quad
  P_{i,j}
    \colvec{
       c_1     \\
       \vdots  \\
       c_i    \\
       \vdots  \\
       c_j    \\
       \vdots  \\
       c_n  }
  =
    \colvec{
       c_1     \\
       \vdots  \\
       c_j    \\
       \vdots  \\
       c_i    \\
       \vdots  \\
       c_n  }
  \quad
  C_{i,j}(k)
    \colvec{
       c_1     \\
       \vdots  \\
       c_i    \\
       \vdots  \\
       c_j    \\
       \vdots  \\
       c_n  }
  =
    \colvec{
       c_1     \\
       \vdots  \\
       c_i    \\
       \vdots  \\
       kc_i+c_j    \\
       \vdots  \\
       c_n  }
\end{equation*}
%<*pf:NonSingIsChBasis3>
Applying a row-multiplication matrix~$M_{i}(k)$
changes a representation with respect to
\( \sequence{\vec{\beta}_1,\dots,\vec{\beta}_i,\dots,\vec{\beta}_n } \)
to one with respect to
\( \sequence{\vec{\beta}_1,\dots,(1/k)\vec{\beta}_i,\dots,\vec{\beta}_n } \).
\begin{multline*}
   \vec{v}= c_1\cdot\vec{\beta}_1+\dots+c_i\cdot\vec{\beta}_i
                                   +\dots+c_n\cdot\vec{\beta}_n  
   \\  \mapsto\;                                                       
    c_1\cdot\vec{\beta}_1+\dots+kc_i\cdot(1/k)\vec{\beta}_i+\dots
                                  +c_n\cdot\vec{\beta}_n=\vec{v}     
\end{multline*}
The second one is a basis because the first is a 
basis and because of the $k\neq 0$ restriction in the definition of a 
row-multiplication matrix.
%</pf:NonSingIsChBasis3>
%<*pf:NonSingIsChBasis4>
Similarly, left-multiplication by a row-swap matrix $P_{i,j}$
changes a representation with respect to the basis
\( \sequence{\vec{\beta}_1,\dots,\vec{\beta}_i,\dots,\vec{\beta}_j,
  \dots,\vec{\beta}_n } \)
into one with respect to this basis
\( \sequence{\vec{\beta}_1,\dots,\vec{\beta}_j,\dots,\vec{\beta}_i,
  \dots,\vec{\beta}_n } \).
\begin{multline*}
   \vec{v}= c_1\cdot\vec{\beta}_1+\dots+c_i\cdot\vec{\beta}_i
                   +\dots+c_j\vec{\beta}_j+\dots+c_n\cdot\vec{\beta}_n  
   \\  \mapsto\;                                                       
    c_1\cdot\vec{\beta}_1+\dots+c_j\cdot\vec{\beta}_j+\dots
               +c_i\cdot\vec{\beta}_i+\dots+c_n\cdot\vec{\beta}_n=\vec{v}     
\end{multline*}
%</pf:NonSingIsChBasis4>
%<*pf:NonSingIsChBasis5>
And, a representation with respect to
\( \sequence{\vec{\beta}_1,\dots,\vec{\beta}_i,\dots,\vec{\beta}_j,
  \dots,\vec{\beta}_n } \)
changes via left-multiplication by a row-combination matrix
$C_{i,j}(k)$ into a representation with respect to 
\( \sequence{\vec{\beta}_1,\dots,\vec{\beta}_i-k\vec{\beta}_j,
  \dots,\vec{\beta}_j,\dots,\vec{\beta}_n } \)
\begin{multline*}
   \vec{v}= c_1\cdot\vec{\beta}_1+\dots+c_i\cdot\vec{\beta}_i
                       +c_j\vec{\beta}_j+\dots+c_n\cdot\vec{\beta}_n  
   \\  \mapsto\;                                                       
    c_1\cdot\vec{\beta}_1+\dots+c_i\cdot(\vec{\beta}_i-k\vec{\beta}_j)+\dots
          +(kc_i+c_j)\cdot\vec{\beta}_j+\dots+c_n\cdot\vec{\beta}_n=\vec{v} 
\end{multline*}
(the definition of $C_{i,j}(k)$ specifies that 
\( i\neq j \) and \( k\neq 0 \)).
%</pf:NonSingIsChBasis5>
\end{proof}

\begin{corollary}  \label{co:MatrixNonsingularIffChangesBasis}
%<*co:MatrixNonsingularIffChangesBasis>
A matrix is nonsingular if and only if it represents the identity map
with respect to some pair of bases.
%</co:MatrixNonsingularIffChangesBasis>
\end{corollary}

% Just as this subsection shows how to translate among representations of
% vectors 
% the next subsection will
% show how to translate among representations of 
% maps, that is, how to change
% $\rep{h}{B,D}$ to $\rep{h}{\hat{B},\hat{D}}$.
% The above corollary is a special case of this, where the domain and range are
% the same space and where the map is the identity map.


\begin{exercises}
  \recommended \item 
    In \( \Re^2 \), where
    \begin{equation*}
      D=\sequence{\colvec[r]{2 \\ 1},\colvec[r]{-2 \\ 4}}
    \end{equation*}
    find the change of basis matrices from \( D \) to \( \stdbasis_2 \) and 
    from \( \stdbasis_2 \) to \( D \).
    Multiply the two.
    \begin{answer}
      For the matrix to change bases from $D$ to $\stdbasis_2$ we need that
      $\rep{\identity(\vec{\delta}_1)}{\stdbasis_2}
        =\rep{\vec{\delta}_1}{\stdbasis_2}$ 
      and that 
      $\rep{\identity(\vec{\delta}_2)}{\stdbasis_2}
        =\rep{\vec{\delta}_2}{\stdbasis_2}$.
      Of course, the representation of a vector in $\Re^2$ with respect to 
      the standard basis is easy.
      \begin{equation*}
        \rep{\vec{\delta}_1}{\stdbasis_2}=\colvec[r]{2 \\ 1}
        \qquad
        \rep{\vec{\delta}_2}{\stdbasis_2}=\colvec[r]{-2 \\ 4}
      \end{equation*}
      Concatenating those two together to make the columns of the change of
      basis matrix gives this. 
      \begin{equation*}
        \rep{\identity}{D,\stdbasis_2}
        =\begin{mat}[r]
          2     &-2    \\
          1     &4
        \end{mat}
      \end{equation*}
      For the change of basis matrix in the other direction we can
      calculate $\rep{\identity(\vec{e}_1)}{D}=\rep{\vec{e}_1}{D}$ and 
      $\rep{\identity(\vec{e}_2)}{D}=\rep{\vec{e}_2}{D}$ (this job is routine)
      or we can take the inverse of the above matrix.
      Because of the formula for the inverse of a $\nbyn{2}$ matrix, 
      this is easy. 
      \begin{equation*}
        \rep{\identity}{\stdbasis_2,D}=
        \frac{1}{10}\cdot\begin{mat}[r]
          4  &2  \\
         -1  &2
        \end{mat}
        =\begin{mat}[r]
          4/10  &2/10  \\
         -1/10  &2/10
        \end{mat}
      \end{equation*}
    \end{answer}
 \item Which of these matrices could be used to change bases?
   \begin{exparts*}
     \partsitem 
       $\begin{mat}
          1 &2 \\
          3 &4 
        \end{mat}$
     \partsitem 
       $\begin{mat}
          0 &-1  \\
          1 &-1
        \end{mat}$
     \partsitem 
       $\begin{mat}
          2 &3 &-1 \\
          0 &1 &0
        \end{mat}$
     \partsitem 
       $\begin{mat}
          2 &3 &-1 \\
          0 &1 &0  \\
          4 &7 &-2
        \end{mat}$
     \partsitem 
       $\begin{mat}
          0 &2 &0 \\
          0 &0 &6 \\
          1 &0 &0
        \end{mat}$
   \end{exparts*}
   \begin{answer}
     If the matrix is nonsingular then it can be a change of basis matrix. 
     For all of these matrices we can check that by eye.
     \begin{exparts}
       \partsitem This is nonsingular since the second row is not a multiple
          of the first.
       \partsitem This is nonsingular.
       \partsitem This matrix is singular, since to be nonsingular a matrix
          must be square.
       \partsitem This matrix is singular since twice the first row plus
          the second row equals the third row.
       \partsitem Nonsingular.
     \end{exparts}
   \end{answer}
  \recommended \item \label{exer:ChngeBasMatsRTwo}
    Find the change of basis matrix for \( B,D\subseteq\Re^2 \).
    \begin{exparts*}
      \partsitem \( B=\stdbasis_2 \),
        \( D=\sequence{\vec{e}_2,\vec{e}_1} \)
      \partsitem \( B=\stdbasis_2 \),
        \( D=\sequence{\colvec[r]{1 \\ 2},\colvec[r]{1 \\ 4}} \)
      \partsitem  \( B=\sequence{\colvec[r]{1 \\ 2},\colvec[r]{1 \\ 4}} \),
        \( D=\stdbasis_2 \)
      \partsitem  \( B=\sequence{\colvec[r]{-1 \\ 1},\colvec[r]{2 \\ 2}} \),
        \( D=\sequence{\colvec[r]{0 \\ 4},\colvec[r]{1 \\ 3}} \)
    \end{exparts*}
    \begin{answer}
      Concatenate 
      $\rep{\identity(\vec{\beta}_1)}{D}=\rep{\vec{\beta}_1}{D}$
      and $\rep{\identity(\vec{\beta}_2)}{D}=\rep{\vec{\beta}_2}{D}$
      to make the change of basis matrix
      $\rep{\identity}{B,D}$.
      \begin{exparts*}
        \partsitem \( \begin{mat}[r]
                   0  &1  \\
                   1  &0
                 \end{mat} \)
        \partsitem \( \begin{mat}[r]
                   2  &-1/2  \\
                   -1 &1/2
                 \end{mat} \)
        \partsitem \( \begin{mat}[r]
                   1  &1     \\
                   2  &4
                 \end{mat} \)
        \partsitem \( \begin{mat}[r]
                   1  &-1    \\
                  -1  &2
                 \end{mat} \)
      \end{exparts*}  
    \end{answer}
  \recommended \item
    Find the change of basis matrix for each \( B,D\subseteq\polyspace_2 \).
    \begin{exparts*}
      \partsitem \( B=\sequence{1,x,x^2},
               D=\sequence{x^2,1,x} \)
      \partsitem \( B=\sequence{1,x,x^2},
               D=\sequence{1,1+x,1+x+x^2} \)
      \partsitem \( B=\sequence{2,2x,x^2},
               D=\sequence{1+x^2,1-x^2,x+x^2} \)
    \end{exparts*}
    \begin{answer}
      The vectors
      $\rep{\identity(\vec{\beta}_1)}{D}=\rep{\vec{\beta}_1}{D}$,
      $\rep{\identity(\vec{\beta}_2)}{D}=\rep{\vec{\beta}_2}{D}$,
      and $\rep{\identity(\vec{\beta}_3)}{D}=\rep{\vec{\beta}_3}{D}$
      make the change of basis matrix
      $\rep{\identity}{B,D}$.
      \begin{exparts*}
        \partsitem \( \begin{mat}[r]
                   0  &0  &1  \\
                   1  &0  &0  \\
                   0  &1  &0
                 \end{mat} \)
        \partsitem \( \begin{mat}[r]
                   1  &-1 &0  \\
                   0  &1  &-1 \\
                   0  &0  &1
                 \end{mat} \)
        \partsitem \( \begin{mat}[r]
                   1  &-1 &1/2  \\
                   1  &1  &-1/2 \\
                   0  &2  &0
                 \end{mat} \)
      \end{exparts*}  
      E.g., for the first column of the first matrix,
      $1=0\cdot x^2+1\cdot 1+0\cdot x$.
     \end{answer}
  \item 
    For the bases in \nearbyexercise{exer:ChngeBasMatsRTwo}, 
    find the change of basis matrix in the other direction, from $D$ to $B$.
    \begin{answer}
       One way to go is to find 
       $\rep{\vec{\delta}_1}{B}$ and $\rep{\vec{\delta}_2}{B}$,
       and then concatenate them into the columns of the desired 
       change of basis matrix.
       Another way is to find the inverse of the matrices that answer
       \nearbyexercise{exer:ChngeBasMatsRTwo}. 
       \begin{exparts*}
        \partsitem
          $\begin{mat}[r]
            0  &1  \\
            1  &0
          \end{mat}$
        \partsitem \( \begin{mat}[r]
            1  &1  \\
            2  &4
          \end{mat} \)
        \partsitem \( \begin{mat}[r]
            2  &-1/2  \\
            -1 &1/2
          \end{mat} \)
        \partsitem \( \begin{mat}[r]
            2  &1  \\
            1  &1
          \end{mat} \)
      \end{exparts*} 
    \end{answer}
  \recommended \item 
    Decide if each changes bases on \( \Re^2 \).
    To what basis is \( \stdbasis_2 \) changed?
    \begin{exparts*}
      \partsitem \( \begin{mat}[r]
                 5  &0  \\
                 0  &4
               \end{mat}  \)
      \partsitem \( \begin{mat}[r]
                 2  &1  \\
                 3  &1
               \end{mat}  \)
      \partsitem \( \begin{mat}[r]
                -1  &4  \\
                 2  &-8
               \end{mat}  \)
      \partsitem \( \begin{mat}[r]
                 1  &-1 \\
                 1  &1
               \end{mat}  \)
    \end{exparts*}
    \begin{answer}
       A matrix changes bases if and only if it is nonsingular.
       \begin{exparts}
        \partsitem This matrix is nonsingular and so changes bases.
          Finding to what basis \( \stdbasis_2 \) is changed 
          means finding $D$ such that 
          \begin{equation*}
            \rep{\identity}{\stdbasis_2,D}=
            \begin{mat}[r]
                 5  &0  \\
                 0  &4
            \end{mat}
          \end{equation*}
          and by the definition of how a matrix represents a linear map,
          we have this.
          \begin{equation*}
            \rep{\identity(\vec{e}_1)}{D}=\rep{\vec{e}_1}{D}=\colvec[r]{5 \\ 0}
            \qquad
            \rep{\identity(\vec{e}_2)}{D}=\rep{\vec{e}_2}{D}=\colvec[r]{0 \\ 4}
          \end{equation*}
          Where
          \begin{equation*}
            D=\sequence{\colvec{x_1 \\ y_1},\colvec{x_2 \\ y_2}}
          \end{equation*}
          we can either solve the system
          \begin{equation*}
            \colvec[r]{1 \\ 0}
            =5\colvec{x_1 \\ y_1}+0\colvec{x_2 \\ y_1}
            \qquad
            \colvec[r]{0 \\ 1}
            =0\colvec{x_1 \\ y_1}+4\colvec{x_2 \\ y_1}
          \end{equation*}
          or else just spot the answer
          (thinking of the proof of \nearbylemma{le:NonSingIsChBasis}).
          \begin{equation*}
            D=\sequence{\colvec[r]{1/5 \\ 0},
                        \colvec[r]{0 \\ 1/4}}
          \end{equation*}
        \partsitem Yes, this matrix is nonsingular and so changes bases.
          To calculate $D$, we proceed as above with 
          \begin{equation*}
            D=\sequence{\colvec{x_1 \\ y_1},
                        \colvec{x_2 \\ y_2}}
          \end{equation*}
          to solve 
          \begin{equation*}
            \colvec[r]{1 \\ 0}
            =2\colvec{x_1 \\ y_1}+3\colvec{x_2 \\ y_1}
            \quad\text{and}\quad
            \colvec[r]{0 \\ 1}
            =1\colvec{x_1 \\ y_1}+1\colvec{x_2 \\ y_1}
          \end{equation*}
          and get this.
          \begin{equation*}
            D=\sequence{\colvec[r]{-1 \\ 3},
                        \colvec[r]{1 \\ -2}}
          \end{equation*}
        \partsitem No, this matrix does not change bases because it
           is singular.
        \partsitem Yes, this matrix changes bases because it is nonsingular.
          The calculation of the changed-to basis is as above. 
          \begin{equation*}
            D=\sequence{\colvec[r]{1/2 \\ -1/2},
                        \colvec[r]{1/2 \\ 1/2}}
          \end{equation*}
      \end{exparts}  
    \end{answer}
  \item   For each space find the matrix changing a vector representation with 
    respect to $B$ to one with respect to~$D$.
    \begin{exparts}
      \partsitem $V=\Re^3$, $B=\stdbasis_3$, 
          $D=\sequence{\colvec{1 \\ 2 \\ 3},
                   \colvec{1 \\ 1 \\ 1},
                   \colvec{0 \\ 1 \\ -1}}$
      \partsitem $V=\Re^3$,  
        $B=\sequence{\colvec{1 \\ 2 \\ 3},
                     \colvec{1 \\ 1 \\ 1},
                     \colvec{0 \\ 1 \\ -1}}$,
        $D=\stdbasis_3$
      \partsitem $V=\polyspace_2$,
        $B=\sequence{x^2, x^2+x, x^2+x+1}$,
        $D=\sequence{2, -x, x^2}$
    \end{exparts}
    \begin{answer}
      \begin{exparts}
        \partsitem 
          Start by computing the effect of the identity function 
          on each element of the
          starting basis~$B$.  Obviously this is the effect.
          \begin{equation*}
            \colvec{1 \\ 0 \\ 0}\mapsunder{\identity}\colvec{1 \\ 0 \\ 0}
            \quad
            \colvec{0 \\ 1 \\ 0}\mapsunder{\identity}\colvec{0 \\ 1 \\ 0}
            \quad
            \colvec{0 \\ 0 \\ 1}\mapsunder{\identity}\colvec{0 \\ 0 \\ 1}
          \end{equation*}
          Now represent the three outputs with respect to the ending basis.
          \begin{equation*}
            \rep{\colvec{1 \\ 0 \\ 0}}{D}=\colvec{-2/3 \\ 5/3 \\ -1/3}
            \quad
            \rep{\colvec{0 \\ 1 \\ 0}}{D}=\colvec{1/3 \\ -1/3 \\ 2/3}
            \quad
            \rep{\colvec{0 \\ 0 \\ 1}}{D}=\colvec{1/3 \\ -1/3 \\ -1/3}
          \end{equation*}
          Concatenate them into a basis.
          \begin{equation*}
            \rep{\identity}{B,D}=
            \begin{mat}
              -2/3 &1/3  &1/3  \\
               5/3 &-1/3 &-1/3 \\
              -1/3 &2/3  &-1/3
            \end{mat}
          \end{equation*}
        \partsitem
          One way to find this is to take the inverse of the prior matrix, 
          since it 
          converts bases in the other direction.
          Alternatively, we can compute these three
          \begin{equation*}
            \rep{\colvec{1 \\ 2 \\ 3}}{\stdbasis_3}
              =\colvec{1 \\ 2 \\ 3}
            \quad
            \rep{\colvec{1 \\ 1 \\ 1}}{\stdbasis_3}
              =\colvec{1 \\ 1 \\ 1}
            \quad
            \rep{\colvec{0 \\ 1 \\ -1}}{\stdbasis_3}
               =\colvec{0 \\ 1 \\ -1}
          \end{equation*}
          and put them in a matrix.
          \begin{equation*}
            \rep{\identity}{B,D}=
            \begin{mat}
              1 &1 &0 \\
              2 &1 &1 \\
              3 &1 &-1
            \end{mat}
          \end{equation*}
        \partsitem
          Representing $\identity(x^2)$, $\identity(x^2+x)$, and~$\identity(x^2+x+1)$
          with respect to the ending basis gives this.
          \begin{equation*}
            \rep{x^2}{D}=\colvec{0 \\ 0 \\ 1}
            \quad
            \rep{x^2+x}{D}=\colvec{0 \\ -1 \\ 1}
            \quad
            \rep{x^2+x+1}{D}=\colvec{1/2 \\ -1 \\ 1}
          \end{equation*}
          Put them together.
          \begin{equation*}
            \rep{\identity}{B,D}=
            \begin{mat}
              0 &0 &1/2 \\
              0 &-1 &-1 \\
              1 &1  &1
            \end{mat}
          \end{equation*}
      \end{exparts}
    \end{answer}
  \item 
    Find bases such that this matrix represents the identity map
    with respect to those bases.
    \begin{equation*}
      \begin{mat}[r]
        3  &1  &4  \\
        2  &-1 &1  \\
        0  &0  &4
      \end{mat}
    \end{equation*}
    \begin{answer}
      This question has many different solutions.
      One way to proceed is to make up any basis $B$ for any space,
      and then compute the appropriate $D$ (necessarily for the same space,
      of course).
      Another, easier, way to proceed is to fix the codomain as $\Re^3$ and
      the codomain basis as $\stdbasis_3$.
      This way (recall that the representation of any vector with respect
      to the standard basis is just the vector itself),
      we have this. 
      \begin{equation*}
        B=\sequence{\colvec[r]{3 \\ 2 \\ 0},
                    \colvec[r]{1 \\ -1 \\ 0},
                    \colvec[r]{4 \\ 1 \\ 4}  }
        \qquad
        D=\stdbasis_3
      \end{equation*}  
    \end{answer}
  \item 
    Consider the vector space of real-valued functions with basis
    \( \sequence{\sin(x),\cos(x)} \).
    Show that \( \sequence{2\sin(x)+\cos(x),3\cos(x)} \)
    is also a basis for this space.
    Find the change of basis matrix in each direction.
    \begin{answer}
      Checking that \( B=\sequence{2\sin(x)+\cos(x),3\cos(x)} \) is a basis
      is routine.
      Call the natural basis $D$.
      To compute the change of basis matrix $\rep{\identity}{B,D}$ we must
      find $\rep{2\sin(x)+\cos(x)}{D}$ and $\rep{3\cos(x)}{D}$, that is,
      we need $x_1,y_1, x_2,y_2$ such that these equations hold.
      \begin{align*}
        x_1\cdot \sin(x)+y_1\cdot\cos(x) &= 2\sin(x)+\cos(x) \\
        x_2\cdot \sin(x)+y_2\cdot\cos(x) &= 3\cos(x) 
      \end{align*}
      Obviously this is the answer.
      \begin{equation*}
        \rep{\identity}{B,D}
        =\begin{mat}[r]
          2  &0  \\
          1  &3
        \end{mat}
      \end{equation*}
      For the change of basis matrix in the other direction we could 
      look for $\rep{\sin(x)}{B}$ and $\rep{\cos(x)}{B}$ by solving these. 
      \begin{align*}
        w_1\cdot (2\sin(x)+\cos(x))+z_1\cdot(3\cos(x)) &= \sin(x) \\
        w_2\cdot (2\sin(x)+\cos(x))+z_2\cdot(3\cos(x)) &= \cos(x) 
      \end{align*}
      An easier method is to find the inverse of the matrix found above.
      \begin{equation*}
        \rep{\identity}{D,B}
        =\begin{mat}[r]
          2  &0  \\
          1  &3  
        \end{mat}^{-1}
        =\frac{1}{6}\cdot\begin{mat}[r]
          3  &0  \\
          -1  &2  
        \end{mat}
        =\begin{mat}[r]
           1/2  &0  \\
          -1/6  &1/3
         \end{mat}
      \end{equation*}
    \end{answer}
  \item 
    Where does this matrix
    \begin{equation*}
        \begin{mat}
           \cos(2\theta)   &\sin(2\theta)   \\
           \sin(2\theta)   &-\cos(2\theta)
        \end{mat}
    \end{equation*}
    send the standard basis for \( \Re^2 \)?
    Any other bases?
    \textit{Hint.}
    Consider the inverse.
    \begin{answer} 
      We start by taking
      the inverse of the matrix, that is, by deciding what is the inverse to
      the map of interest.
      \begin{align*}
        \rep{\identity}{D,\stdbasis_2}
        \rep{\identity}{D,\stdbasis_2}^{-1}
        &=\frac{1}{-\cos^2(2\theta)-\sin^2(2\theta)}\cdot\begin{mat}
           -\cos(2\theta)   &-\sin(2\theta)  \\
           -\sin(2\theta)   &\cos(2\theta)
        \end{mat}                                         \\
        &=\begin{mat}
           \cos(2\theta)   &\sin(2\theta)  \\
           \sin(2\theta)   &-\cos(2\theta)
        \end{mat}
      \end{align*}
      This is more tractable than the representation the other way
      because this matrix is the concatenation of these two column vectors
      \begin{equation*}
        \rep{\vec{\delta}_1}{\stdbasis_2}
           =\colvec{\cos(2\theta) \\ \sin(2\theta)}
        \qquad 
        \rep{\vec{\delta}_2}{\stdbasis_2}
           =\colvec{\sin(2\theta) \\ -\cos(2\theta)}
      \end{equation*}
      and representations with respect to $\stdbasis_2$ are transparent.
      \begin{equation*}
        \vec{\delta}_1=\colvec{\cos(2\theta) \\ \sin(2\theta)}
        \qquad 
        \vec{\delta}_2=\colvec{\sin(2\theta) \\ -\cos(2\theta)}
      \end{equation*}
      This pictures the action of the map that transforms $D$ to $\stdbasis_2$
      (it is, again, the inverse of the map that is the answer to
      this question).
      The line lies at an angle $\theta$ to the $x$~axis.
      \begin{center}  \small
        \includegraphics{map/mp/ch3.81}
      \end{center}
      This map reflects vectors over that line.
      Since reflections are self-inverse, the answer to the question is:~the
      original map reflects about the line
      through the origin with angle of elevation $\theta$. 
      (Of course, it does this to any basis.)
    \end{answer}
  \recommended \item
    What is the change of basis matrix with respect to \( B,B \)?
    \begin{answer}
       The appropriately-sized identity matrix.  
     \end{answer}
  \item 
    Prove that a matrix changes bases if and only if it is invertible.
    \begin{answer}
       Each is true if and only if the matrix is nonsingular.
    \end{answer}
  \item 
    Finish the proof of \nearbylemma{le:NonSingIsChBasis}.
    \begin{answer}
      What remains is to show that 
      left multiplication by a reduction matrix represents a
      change from another basis to \( B=\basis{\beta}{n} \).

      Application of a row-multiplication matrix \( M_i(k) \) translates a
      representation with respect to the basis
      \( \sequence{\vec{\beta}_1,\dots,k\vec{\beta}_i,\dots,\vec{\beta}_n} \)
      to one with respect to \( B \), as here.
      \begin{equation*}
         \vec{v}=c_1\cdot\vec{\beta}_1+\dots+c_i\cdot(k\vec{\beta}_i)
                   +\dots+c_n\cdot\vec{\beta}_n  
         \;\mapsto\;                                                       
         c_1\cdot\vec{\beta}_1+\dots
             +(kc_i)\cdot\vec{\beta}_i+\dots+c_n\cdot\vec{\beta}_n=\vec{v}
      \end{equation*}
      Apply a row-swap matrix \( P_{i,j} \) to translates a representation
      with respect to the basis
      \( \sequence{\vec{\beta}_1,\dots,\vec{\beta}_j,\dots,
        \vec{\beta}_i,\dots,\vec{\beta}_n} \)
      to one with respect to
            \( \sequence{\vec{\beta}_1,\dots,\vec{\beta}_i,\dots,
        \vec{\beta}_j,\dots,\vec{\beta}_n} \).
      Finally, applying a row-combination matrix \( C_{i,j}(k) \) changes a
      representation with respect to
      \( \sequence{\vec{\beta}_1,\dots,\vec{\beta}_i+k\vec{\beta}_j,\dots,
        \vec{\beta}_j,\dots,\vec{\beta}_n} \)
      to one with respect to \( B \).  
      \begin{multline*}
         \vec{v}= c_1\cdot\vec{\beta}_1+\dots
                     +c_i\cdot(\vec{\beta}_i+k\vec{\beta}_j)
                       +\dots+c_j\vec{\beta}_j+\dots+c_n\cdot\vec{\beta}_n  
         \\  \mapsto\;                                                       
         c_1\cdot\vec{\beta}_1+\dots+c_i\cdot\vec{\beta}_i
               +\dots+(kc_i+c_j)\cdot\vec{\beta}_j
               +\dots+c_n\cdot\vec{\beta}_n=\vec{v}     
      \end{multline*}
      (As in the part of the proof in the body of this subsection, 
      the various conditions on the
      row operations, e.g., that the scalar $k$ is nonzero, assure that these
      are all bases.)
    \end{answer}
  \recommended \item 
    Let \( H \) be an \( \nbyn{n} \) nonsingular matrix.
    What basis of \( \Re^n \) does \( H \) change to the standard basis?
    \begin{answer}
       Taking $H$ as a change of basis matrix 
       $H=\rep{\identity}{B,\stdbasis_n}$, 
       its columns are
       \begin{equation*}
         \colvec{h_{1,i} \\ \vdots \\ h_{n,i}}
         =\rep{\identity(\vec{\beta}_i)}{\stdbasis_n}
         =\rep{\vec{\beta}_i}{\stdbasis_n}
       \end{equation*}
       and, because representations with respect to the standard basis
       are transparent, we have this.
       \begin{equation*}
         \colvec{h_{1,i} \\ \vdots \\ h_{n,i}}
         =\vec{\beta}_i
       \end{equation*}
       That is, the basis is the one composed of the columns of \( H \).  
    \end{answer}
  \recommended \item \label{exer:AnyNonZeroRepChgTOAnyOther}
    \begin{exparts}
      \partsitem  In \( \polyspace_3 \) with basis
        \( B=\sequence{1+x,1-x,x^2+x^3,x^2-x^3} \) we have this
        representation.
        \begin{equation*}
          \rep{1-x+3x^2-x^3}{B}=
            \colvec[r]{0 \\ 1 \\ 1 \\ 2}_B
        \end{equation*}
        Find a basis \( D \) 
        giving this different representation for the same
        polynomial. 
        \begin{equation*}
          \rep{1-x+3x^2-x^3}{D}=
            \colvec[r]{1 \\ 0 \\ 2 \\ 0}_D
        \end{equation*}
      \partsitem State and prove that we can change any nonzero vector
        representation to any other.
    \end{exparts}
    \noindent\textit{Hint.}
    The proof of \nearbylemma{le:NonSingIsChBasis}
    is constructive\Dash it not only says the bases change, it shows
    how they change.
    \begin{answer}
      \begin{exparts}
        \partsitem We can change the starting vector representation
          to the ending one through a sequence of row operations.
          The proof tells us what how the bases change. 
          We start by swapping the first and second rows
          of the representation with respect to $B$ to get a representation
          with respect to a new basis $B_1$.
          \begin{equation*}
            \rep{1-x+3x^2-x^3}{B_1}=
              \colvec[r]{1 \\ 0 \\ 1 \\ 2}_{B_1}
            \qquad
            B_1=\sequence{1-x,1+x,x^2+x^3,x^2-x^3}
          \end{equation*}
          We next add \( -2 \) times the third row of the vector
          representation to the fourth row.
          \begin{equation*}
            \rep{1-x+3x^2-x^3}{B_3}=
              \colvec[r]{1 \\ 0 \\ 1 \\ 0}_{B_2}
            \qquad
            B_2=\sequence{1-x,1+x,3x^2-x^3,x^2-x^3}
          \end{equation*}
          (The third element of \( B_2 \) is the third element of 
          \( B_1 \) minus \( -2 \) times the fourth element of $B_1$.)
          Now we can finish by doubling the third row.
          \begin{equation*}
            \rep{1-x+3x^2-x^3}{D}=
              \colvec[r]{1 \\ 0 \\ 2 \\ 0}_{D}
            \qquad
            D=\sequence{1-x,1+x,(3x^2-x^3)/2,x^2-x^3}
          \end{equation*}
        \partsitem 
          Here are three different approaches to stating such a result.
          The first is the assertion:~where $V$ is a vector space with
          basis $B$ and $\vec{v}\in V$ is nonzero, for any nonzero column
          vector $\vec{z}$ 
          (whose number of components equals the dimension of $V$) 
          there is a change of basis matrix $M$ such that
          $M\cdot \rep{\vec{v}}{B}=\vec{z}$. 
          The second possible statement:~for any ($n$-dimensional)
          vector space $V$ and any nonzero
          vector \( \vec{v}\in V \), where \( \vec{z}_1, \vec{z}_2\in\Re^n \)
          are nonzero, there are bases \( B, D\subset V \) such that
          \( \rep{\vec{v}}{B}=\vec{z}_1 \) and 
          \( \rep{\vec{v}}{D}=\vec{z}_2 \).
          The third is:~for any nonzero $\vec{v}$ member of 
          any vector space (of dimension~$n$) and any nonzero column vector
          (with $n$ components) there is a basis such that $\vec{v}$ is 
          represented with respect to that basis by that column vector.

          The first and second statements follow easily from the third.
          The first follows because the third statement gives a basis $D$
          such that $\rep{\vec{v}}{D}=\vec{z}$ and then 
          $\rep{\identity}{B,D}$ is the desired~$M$.
          The second follows from the third because it is just a
          doubled application of it.

          A way to prove the third is as in the answer to the first part
          of this question.
          Here is a sketch.
          Represent $\vec{v}$ with respect to any basis $B$ with a column
          vector $\vec{z}_1$.
          This column vector must have a nonzero component because $\vec{v}$
          is a nonzero vector.
          Use that component in a sequence of row operations to convert 
          $\vec{z}_1$ to $\vec{z}$.
          (We could fill out this sketch as an induction 
          argument on the dimension of $V$.)  
       \end{exparts}  
     \end{answer}
  \item 
    Let \( V,W \) be vector spaces, and let \( B,\hat{B} \) be bases for
    \( V \) and \( D,\hat{D} \) be bases for \( W \).
    Where \( \map{h}{V}{W} \) is linear, find a formula relating
    \( \rep{h}{B,D} \) to \( \rep{h}{\hat{B},\hat{D}} \).
    \begin{answer}
      This is the topic of the next subsection.
    \end{answer}
  \recommended \item
    Show that the columns of an \( \nbyn{n} \) change of basis matrix
    form a basis for \( \Re^n \).
    Do all bases appear in that way:~can 
    the vectors from any $\Re^n$ basis make the columns of a change of 
    basis matrix?
    \begin{answer}
      A change of basis matrix is nonsingular and thus
      has rank equal to the number of its columns.
      Therefore its set of columns is a linearly independent subset of size 
      $n$ in $\Re^n$ and it is thus a basis.
      The answer to the second half is also `yes'; all implications in the 
      prior sentence reverse
      (that is, all of the `if \ldots then~\ldots' parts of the prior sentence
      convert to `if and only if' parts).
    \end{answer}
  \recommended \item 
    Find a matrix having this effect.
    \begin{equation*}
      \colvec[r]{1 \\ 3}
      \;\mapsto\;
      \colvec[r]{4 \\ -1}
    \end{equation*}
    That is, find a $M$ that left-multiplies the 
    starting vector to yield the ending vector.
    Is there a matrix having these two effects?
    \begin{exparts*}
      \partsitem
        $
          \colvec[r]{1 \\ 3}\mapsto\colvec[r]{1 \\ 1}
          \quad
          \colvec[r]{2 \\ -1}\mapsto\colvec[r]{-1 \\ -1}
        $
      \partsitem $
          \colvec[r]{1 \\ 3}\mapsto\colvec[r]{1 \\ 1}
          \quad
          \colvec[r]{2 \\ 6}\mapsto\colvec[r]{-1 \\ -1}
        $
   \end{exparts*}
   Give a necessary and sufficient condition for there to be a
   matrix such that
   $\vec{v}_1\mapsto\vec{w}_1$ and $\vec{v}_2\mapsto\vec{w}_2$.
    \begin{answer}
      In response to the first half of the question, 
      there are infinitely many such matrices. 
      One of them  represents with respect to
      \( \stdbasis_2 \) the transformation of \( \Re^2 \) with this action.
      \begin{equation*}
        \colvec[r]{1 \\ 0}\mapsto\colvec[r]{4 \\ 0}
        \qquad
        \colvec[r]{0 \\ 1}\mapsto\colvec[r]{0 \\ -1/3}
      \end{equation*}  
     The problem of specifying two distinct input/output pairs is a bit 
     trickier.
     The fact that matrices have a linear action precludes some possibilities.
     \begin{exparts}
       \partsitem Yes, there is such a matrix.
         These conditions
         \begin{equation*}
           \begin{mat}
             a  &b  \\
             c  &d
           \end{mat}
           \colvec[r]{1 \\ 3}
           =
           \colvec[r]{1 \\ 1}
           \qquad
           \begin{mat}
             a  &b  \\
             c  &d
           \end{mat}
           \colvec[r]{2 \\ -1}
           =
           \colvec[r]{-1 \\ -1}
         \end{equation*}
         can be solved
         \begin{equation*}
           \begin{linsys}{4}
             a  &+  &3b  &   &   &   &   &=  &1  \\
                &   &    &   &c  &+  &3d &=  &1  \\
            2a  &-  &b   &   &   &   &   &=  &-1 \\
                &   &    &   &2c &-  &d  &=  &-1 
           \end{linsys}
         \end{equation*}
         to give this matrix.
         \begin{equation*}
           \begin{mat}[r]
             -2/7  &3/7 \\
             -2/7  &3/7 
           \end{mat}
         \end{equation*}
       \partsitem No, because
         \begin{equation*}
           2\cdot\colvec[r]{1 \\ 3}=\colvec[r]{2 \\ 6}
           \quad\text{but}\quad
           2\cdot\colvec[r]{1 \\ 1}\neq\colvec[r]{-1 \\ -1}
         \end{equation*}
         no linear action can produce this effect.
       \partsitem A sufficient condition is that 
         \( \set{\vec{v}_1,\vec{v}_2} \) be linearly independent, but
         that's not a necessary condition.
         A necessary and sufficient condition is that any linear dependences
         among the starting vectors appear also among the ending vectors.
         That is,
         \begin{equation*}
           c_1\vec{v}_1+c_2\vec{v}_2=\zero
           \quad\text{implies}\quad
           c_1\vec{w}_1+c_2\vec{w}_2=\zero.
         \end{equation*}
         The proof of this condition is routine.
     \end{exparts} 
   \end{answer}
\end{exercises}

















\subsection{Changing Map Representations}
\index{matrix equivalence|(}
The first subsection shows how to convert the representation
of a vector with respect to one basis to the representation of that
same vector with respect to another basis.
We next
convert the representation of a map  with
respect to one pair of bases  to the representation 
with respect to a different pair\Dash we convert from
$\rep{h}{B,D}$ to $\rep{h}{\hat{B},\hat{D}}$.
Here is the 
arrow diagram.\index{arrow diagram}
%<*ChangeRepresentationOfMapArrowDiagram>
\begin{equation*}
  \begin{CD}
    V_{\wrt{B}}                   @>h>H>        W_{\wrt{D}}       \\
    @V{\text{\scriptsize$\identity$}} VV                @V{\text{\scriptsize$\identity$}} VV \\
    V_{\wrt{\hat{B}}}             @>h>\hat{H}>  W_{\wrt{\hat{D}}}
  \end{CD}
\end{equation*}
To move from the lower-left
to the lower-right we can either go straight over, or
else up to $V_B$ then over to $W_D$ and then down.
So 
we can calculate $\hat{H}=\rep{h}{\hat{B},\hat{D}}$ 
either by directly using $\hat{B}$ and $\hat{D}$,
or else by first changing bases with $\rep{\identity}{\hat{B},B}$ 
then multiplying by \( H=\rep{h}{B,D} \)
and then changing bases with $\rep{\identity}{D,\hat{D}}$.

\begin{theorem}
To convert from the matrix~$H$ representing a map~$h$ with respect to $B,D$
to the matrix~$\hat{H}$ 
representing it with respect to~$\hat{B},\hat{D}$ use this formula.
\begin{equation*}
   \hat{H}=
   \rep{\identity}{D,\hat{D}}\cdot H\cdot \rep{\identity}{\hat{B},B}
   \tag*{\text{($*$})}
\end{equation*}
\end{theorem}

\begin{proof}
This is evident from the diagram.
\end{proof}

%</ChangeRepresentationOfMapArrowDiagram>
% (To compare the equation with the sentence before it
% remember to read its right hand side
% from right to left, because we read
% function composition from right to left and matrix multiplication 
% represents composition).

\begin{example}
The matrix
\begin{equation*}
  T=\begin{mat}[r]
     \cos(\pi/6)  &-\sin(\pi/6)  \\
     \sin(\pi/6)  &\cos(\pi/6)
  \end{mat}
  =
  \begin{mat}[r]
     \sqrt{3}/2  &-1/2  \\
     1/2         &\sqrt{3}/2
  \end{mat}
\end{equation*}
represents, with respect to \( \stdbasis_2,\stdbasis_2 \),
the transformation \( \map{t}{\Re^2}{\Re^2} \) that rotates vectors
through the counterclockwise angle of \( \pi/6 \) radians.
\begin{center}
  \includegraphics{map/mp/ch3.22}
\end{center}
We can translate $T$ to a representation 
with respect to these
\begin{equation*}
  \hat{B}=\sequence{
              \colvec[r]{1 \\ 1}
              \colvec[r]{0 \\ 2} }
  \qquad
  \hat{D}=\sequence{
              \colvec[r]{-1 \\ 0}
              \colvec[r]{2 \\ 3} }
\end{equation*}
by using the arrow diagram above.
\begin{equation*}
  \begin{CD}
    \Re^2_{\wrt{\stdbasis_2}}              @>t>T>        \Re^2_{\wrt{\stdbasis_2}}     \\
    @V\text{\scriptsize$\identity$} VV                @V\text{\scriptsize$\identity$} VV \\
    \Re^2_{\wrt{\hat{B}}}          @>t>\hat{T}>  \Re^2_{\wrt{\hat{D}}}
  \end{CD}
\end{equation*}
The picture illustrates that we can compute~$\hat{T}$ either directly
by going along the square's bottom, or as in formula~($*$) 
by going up on the left, then
across the top, and then down on the right, with 
$\hat{T}=
   \rep{\identity}{\stdbasis_2,\hat{D}}\cdot T\cdot \rep{\identity}{\hat{B},\stdbasis_2}$.
(Note again that the matrix multiplication reads right to left, as the three 
functions are composed and function composition reads right to left.)

Find the matrix for the left-hand side, 
the matrix~$\rep{\identity}{\hat{B},\stdbasis_2}$, in the usual 
way:~find the effect of the identity matrix on the 
starting basis~$\hat{B}$\Dash 
which of course is no effect at all\Dash and then represent 
those basis elements with respect to the ending basis~$\stdbasis_2$.
\begin{equation*}
  \rep{\identity}{\hat{B},\stdbasis_2}
  =
  \begin{mat}
    1 &0 \\
    1 &2
  \end{mat}
\end{equation*}
This calculation is easy when the ending basis is the standard one.

There are two ways to compute the matrix for going down the
square's right side, $\rep{\identity}{\stdbasis_2,\hat{D}}$.
We could calculate it directly as we did for the other change of basis 
matrix.
Or, we could instead calculate it as
the inverse of the matrix for going up $\rep{\identity}{\hat{D},\stdbasis_2}$.
Find that matrix is easy, and 
we have a formula for the $\nbyn{2}$ inverse so that's what is in the
equation below.
\begin{align*}
   \rep{t}{\hat{B},\hat{D}}
  &=\begin{mat}[r]
     -1     &2   \\
     0      &3
  \end{mat}^{-1}
  \begin{mat}[r]
     \sqrt{3}/2  &-1/2  \\
     1/2         &\sqrt{3}/2
  \end{mat}
  \begin{mat}[r]
     1      &0   \\
     1      &2
  \end{mat}                              \\                            
  &=\begin{mat}[r]
     (5-\sqrt{3})/6   &(3+2\sqrt{3})/3 \\
     (1+\sqrt{3})/6  &\sqrt{3}/3
  \end{mat}
\end{align*}

The matrix is messier but
the map that it represents is the same. 
For instance, to replicate the effect of $t$ in the picture, 
start with $\hat{B}$,
\begin{equation*}
  \rep{\colvec[r]{1 \\ 3}}{\hat{B}}=\colvec[r]{1 \\ 1}_{\hat{B}}
\end{equation*}
apply $\hat{T}$,
\begin{equation*}
  \begin{mat}[r]
     (5-\sqrt{3})/6   &(3+2\sqrt{3})/3 \\
     (1+\sqrt{3})/6  &\sqrt{3}/3
  \end{mat}_{\hat{B},\hat{D}}
  \colvec[r]{1 \\ 1}_{\hat{B}}
  =
  \colvec[r]{(11+3\sqrt{3})/6 \\ (1+3\sqrt{3})/6}_{\hat{D}}
\end{equation*}
and check it against $\hat{D}$.
\begin{equation*}
  \frac{11+3\sqrt{3}}{6}\cdot\colvec[r]{-1 \\ 0}
  +\frac{1+3\sqrt{3}}{6}\cdot\colvec[r]{2 \\ 3}
  =\colvec[r]{(-3+\sqrt{3})/2 \\ (1+3\sqrt{3})/2}
\end{equation*}
\end{example}

\begin{example} \label{ex:DiagizedMat}
Changing bases can make the matrix simpler.
On \( \Re^3 \) the map
\begin{equation*}
  \colvec{x \\ y \\ z}\mapsunder{t}\colvec{y+z \\ x+z \\ x+y}
\end{equation*}
is represented with respect to the standard basis in this way.
\begin{equation*}
  \rep{t}{\stdbasis_3,\stdbasis_3}=
  \begin{mat}[r]
    0  &1  &1  \\
    1  &0  &1  \\
    1  &1  &0
  \end{mat}
\end{equation*}
Representing it with respect to
\begin{equation*}
  B=\sequence{\colvec[r]{1 \\ -1 \\ 0},
                             \colvec[r]{1 \\ 1 \\ -2},
                             \colvec[r]{1 \\ 1 \\ 1}}
\end{equation*} 
gives a matrix that is diagonal.
\begin{equation*} 
  \rep{t}{B,B}=
  \begin{mat}[r]
   -1  &0  &0  \\
    0  &-1 &0  \\
    0  &0  &2
  \end{mat}
\end{equation*}
\end{example}

Naturally we usually prefer 
representations that are easier to understand.
We say that a map or matrix
has been \definend{diagonalized}\index{matrix!diagonalized}
when we find a basis~$B$ such that 
the representation is diagonal
with respect to $B,B$, that is,
with respect to the same starting basis as ending basis.
Chapter Five finds which maps and matrices are diagonalizable.

The rest of this subsection develops the easier case 
of finding two bases $B,D$ such that a representation is simple. 
Recall that the prior subsection 
shows that a matrix is a change of basis matrix 
if and only if it is nonsingular.
% That gives us another version of the above  arrow diagram
% and equation~($*$) from the start of this subsection.

\begin{definition}  \label{def:MatEquiv}
%<*df:MatEquiv>
Same-sized matrices \( H \) and \( \hat{H} \) are
\definend{matrix equivalent\/}\index{matrix equivalence!definition}%
\index{matrix!equivalent}\index{equivalence relation!matrix equivalence}
if there are nonsingular matrices \( P \) and \( Q \) such that
$\hat{H}=PHQ$.
%</df:MatEquiv>
\end{definition}

\begin{corollary} \label{le:MatEqIsSameMap}
%<*co:MatEqIsSameMap>
Matrix equivalent matrices represent the same map, with respect to appropriate
pairs of bases.
%</co:MatEqIsSameMap>
\end{corollary}

\begin{proof}
This is immediate from equation~($*$) above. 
\end{proof}

\nearbyexercise{exer:MatEqIsEqRel} checks that
matrix equivalence is an equivalence relation.
Thus it  partitions\index{partition!matrix equivalence classes} 
the set of matrices into matrix equivalence classes.
\begin{center}
  \raisebox{.5in}{\begin{tabular}{l}
                    All matrices:
                  \end{tabular}}
  \includegraphics{map/mp/ch3.23}
  \raisebox{.3in}{\begin{tabular}{l}
                     $H$ matrix equivalent \\ to $\hat{H}$
                  \end{tabular}}
\end{center}
We can get insight into the classes by comparing matrix equivalence
with row equivalence
(remember that matrices are row equivalent when they can be reduced to each
other by row operations).
In $\hat{H}=PHQ$, the matrices $P$ and $Q$ are nonsingular and 
thus each is a product of elementary reduction matrices
by Lemma~IV.\ref{lem:ComputeInvMat}.
Left-multiplication by the reduction matrices making up $P$
performs row operations.
Right-multiplication by the reduction matrices making up $Q$
performs column operations.
Hence, matrix equivalence is a generalization of row equivalence\Dash two
matrices are row equivalent if one can be converted to the other by
a sequence of row reduction steps, while
two matrices are matrix equivalent if one can be converted to the other by a 
sequence of row reduction steps followed by a sequence of column reduction
steps. 

Consequently, if matrices are row equivalent then they are also
matrix equivalent since we can take $Q$ to be the identity matrix.
The converse, however, does not hold:
two matrices can be matrix equivalent but not row equivalent.

\begin{example}
These two are matrix equivalent
\begin{equation*} 
  \begin{mat}[r]
    1  &0  \\
    0  &0
  \end{mat}
  \qquad
  \begin{mat}[r]
    1  &1  \\
    0  &0
  \end{mat}
\end{equation*}
because the second reduces to the first by
the column operation of taking $-1$ times the first column and adding
to the second.
They are not row equivalent because they have different reduced echelon
forms (both are already in reduced form).
\end{example}

We close this section by giving
a set of representatives
for the matrix equivalence classes. % \appendrefs{class representatives}%
\index{representative!of matrix equivalence classes}

\begin{theorem}  \label{th:CanonFormForMatEquiv}
\index{matrix equivalence!canonical form}
\index{canonical form!for matrix equivalence}
%<*th:CanonFormForMatEquiv>
Any \( \nbym{m}{n} \) matrix of rank \( k \) is matrix equivalent to
the \( \nbym{m}{n} \) matrix that is all zeros except that
the first \( k \) diagonal entries are ones.
\begin{equation*}
    \begin{mat}
      1  &0      &\ldots &0  &0  &\ldots  &0  \\
      0  &1      &\ldots &0  &0  &\ldots  &0  \\
         &\vdots                              \\
      0  &0      &\ldots &1  &0  &\ldots  &0  \\
      0  &0      &\ldots &0  &0  &\ldots  &0  \\
         &\vdots                              \\
      0  &0      &\ldots &0  &0  &\ldots  &0
    \end{mat}
\end{equation*}
%</th:CanonFormForMatEquiv>
\end{theorem}

%<*BlockPartialIdentityForm>
\noindent This is a 
\definend{block partial-identity}\index{matrix!block} 
form.
\begin{equation*}
    \begin{pmat}{c|c}
      I  &Z  \\  \hline
      Z  &Z
    \end{pmat}
\end{equation*}
%</BlockPartialIdentityForm>

\begin{proof}
%<*pf:CanonFormForMatEquiv>
Gauss-Jordan reduce the given matrix
and combine all the row reduction matrices to make \( P \).
Then use the leading entries to do column reduction and
finish by swapping the columns to put the leading ones on the diagonal.
Combine the column reduction matrices into
\( Q \). 
%</pf:CanonFormForMatEquiv>
\end{proof}

\begin{example}
We illustrate the proof by finding $P$ and $Q$ for this matrix. 
\begin{equation*}
    \begin{mat}[r]
       1  &2  &1  &-1  \\
       0  &0  &1  &-1  \\
       2  &4  &2  &-2
    \end{mat}
\end{equation*}
First Gauss-Jordan row-reduce.
\begin{equation*}
    \begin{mat}[r]
       1  &-1 &0    \\
       0  &1  &0    \\
       0  &0  &1
    \end{mat}
    \begin{mat}[r]
       1  &0  &0    \\
       0  &1  &0    \\
       -2 &0  &1
    \end{mat}
    \begin{mat}[r]
       1  &2  &1  &-1  \\
       0  &0  &1  &-1  \\
       2  &4  &2  &-2
    \end{mat}
  =
    \begin{mat}[r]
       1  &2  &0  &0   \\
       0  &0  &1  &-1  \\
       0  &0  &0  &0
    \end{mat}
\end{equation*}
Then column-reduce, which involves right-multiplication.
\begin{equation*}
    \begin{mat}[r]
       1  &2  &0  &0   \\
       0  &0  &1  &-1  \\
       0  &0  &0  &0
    \end{mat}
    \begin{mat}[r]
       1  &-2 &0  &0   \\
       0  &1  &0  &0   \\
       0  &0  &1  &0   \\
       0  &0  &0  &1
    \end{mat}
    \begin{mat}[r]
       1  &0  &0  &0   \\
       0  &1  &0  &0   \\
       0  &0  &1  &1   \\
       0  &0  &0  &1
    \end{mat}
  =
    \begin{mat}[r]
       1  &0  &0  &0   \\
       0  &0  &1  &0  \\
       0  &0  &0  &0
    \end{mat}
\end{equation*}
Finish by swapping columns.
\begin{equation*}
    \begin{mat}[r]
       1  &0  &0  &0   \\
       0  &0  &1  &0  \\
       0  &0  &0  &0
    \end{mat}
    \begin{mat}[r]
       1  &0  &0  &0   \\
       0  &0  &1  &0   \\
       0  &1  &0  &0   \\
       0  &0  &0  &1
    \end{mat}
  =
    \begin{mat}[r]
       1  &0  &0  &0   \\
       0  &1  &0  &0  \\
       0  &0  &0  &0
    \end{mat}
\end{equation*}
Finally, combine the left-multipliers together as $P$ and the
right-multipliers together as $Q$ to get $PHQ$.
\begin{equation*}
    \begin{mat}[r]
       1  &-1 &0    \\
       0  &1  &0    \\
       -2 &0  &1
    \end{mat}
    \begin{mat}[r]
       1  &2  &1  &-1  \\
       0  &0  &1  &-1  \\
       2  &4  &2  &-2
    \end{mat}
    \begin{mat}[r]
       1  &0  &-2  &0   \\
       0  &0  &1  &0   \\
       0  &1  &0  &1   \\
       0  &0  &0  &1
    \end{mat}
    =
    \begin{mat}[r]
       1  &0  &0  &0   \\
       0  &1  &0  &0  \\
       0  &0  &0  &0
    \end{mat}
\end{equation*}
\end{example}

\begin{corollary}  \label{co:MatrixEquivalentIffSameRank}
%<*co:MatrixEquivalentIffSameRank>
Matrix equivalence classes are
characterized\index{characterizes} by rank:
two same-sized matrices are matrix equivalent if and only if they
have the same rank.
%</co:MatrixEquivalentIffSameRank>
\end{corollary}

\begin{proof}
%<*pf:MatrixEquivalentIffSameRank>
Two same-sized matrices with the same rank
are equivalent to the same block partial-identity matrix.
%</pf:MatrixEquivalentIffSameRank>
\end{proof}

\begin{example}
The $\nbyn{2}$ matrices have
only three possible ranks:~zero, one, or~two.
Thus there are three matrix equivalence classes.
\index{partition!matrix equivalence classes}
\begin{center} % make an exercise comparing these classes with the ones for row-equivalence?
  \raisebox{.5in}{\begin{tabular}{l}
                    All $\nbyn{2}$ matrices:
                  \end{tabular}}
  \includegraphics{map/mp/ch3.24}
  \raisebox{.3in}{\begin{tabular}{l}
                    Three equivalence \\ classes
                  \end{tabular}}
\end{center}
Each class consists of all of the $\nbyn{2}$ matrices with the same rank. 
There is only one rank~zero matrix. 
The other two classes have infinitely many members; we've shown only the
canonical representative.
\end{example}

One nice thing about the representative in
\nearbytheorem{th:CanonFormForMatEquiv} is that we can completely 
understand the linear map when it is expressed 
in this way:
where the bases are
\( B=\sequence{\vec{\beta}_1,\dots,\vec{\beta}_n} \)
and
\( D=\sequence{\vec{\delta}_1,\dots,\vec{\delta}_m} \)
then the map's action is
\begin{equation*}
  c_1\vec{\beta}_1+\dots+c_k\vec{\beta}_k+c_{k+1}\vec{\beta}_{k+1}+\dots
     +c_n\vec{\beta}_n
  \;\mapsto\;
  c_1\vec{\delta}_1+\dots+c_k\vec{\delta}_k+\zero+\cdots+\zero
\end{equation*}
where \( k \) is the rank.
Thus we can view any linear map as a projection.
\begin{equation*}
  \colvec{c_1 \\ \vdots \\ c_k \\ c_{k+1} \\ \vdots \\ c_n}_B
  \quad\longmapsto\quad
  \colvec{c_1 \\ \vdots \\ c_k \\ 0 \\ \vdots \\ 0}_D
\end{equation*}
% Of course, ``understanding'' a map expressed in this way 
% requires that we understand the relationship between \( B \) and \( D \).
% Nonetheless,
% this is a good classification of linear maps.

\begin{exercises}
  \recommended \item 
    Decide if these are matrix equivalent.
    \begin{exparts}
      \partsitem \( \begin{mat}[r]
                 1  &3  &0  \\
                 2  &3  &0
               \end{mat} \),
            \( \begin{mat}[r]
                 2  &2  &1  \\
                 0  &5  &-1
               \end{mat} \)
      \partsitem \( \begin{mat}[r]
                 0  &3  \\
                 1  &1
               \end{mat} \),
            \( \begin{mat}[r]
                 4  &0  \\
                 0  &5
               \end{mat} \)
      \partsitem \( \begin{mat}[r]
                 1  &3  \\
                 2  &6
               \end{mat} \),
            \( \begin{mat}[r]
                 1  &3  \\
                 2  &-6
               \end{mat} \)
    \end{exparts}
    \begin{answer}
      \begin{exparts}
        \partsitem Yes, each has rank two.
        \partsitem Yes, they have the same rank.
        \partsitem No, they have different ranks.
      \end{exparts}  
    \end{answer}
  \item Which of these are matrix equivalent to each other?
    \begin{exparts*}
      \partsitem 
         $\begin{mat}
            1 &2 &3 \\
            4 &5 &6 \\
            7 &8 &9
         \end{mat}$
      \partsitem 
         $\begin{mat}
           1 &3 \\
          -1 &-3
         \end{mat}$
      \partsitem 
         $\begin{mat}
            -5 &1 &0 \\
            -1 &0 &1
         \end{mat}$
      \partsitem 
         $\begin{mat}
            0 &-1 \\
            0 &5
         \end{mat}$
      \partsitem 
         $\begin{mat}
            1 &0 &1 \\
            2 &0 &2 \\
            1 &3 &1
         \end{mat}$
      \partsitem 
         $\begin{mat}
            3  &1   &0 \\
            9  &3   &0 \\
            -3 &-1  &0
         \end{mat}$
    \end{exparts*}
    \begin{answer}
      Group the matrices into classes characterized by the condition that 
      all matrices in the same class are the same size and the same rank.
      \begin{exparts}
        \partsitem $\nbyn{3}$, rank~$2$
        \partsitem $\nbyn{2}$, rank~$1$
        \partsitem $\nbym{2}{3}$, rank~$2$
        \partsitem $\nbyn{3}$, rank~$2$
        \partsitem $\nbyn{3}$, rank~$1$
      \end{exparts}
    \end{answer}
  \recommended \item
    Find the canonical representative of the matrix equivalence class of
    each matrix.
    \begin{exparts*}
      \partsitem \( \begin{mat}[r]
                 2  &1  &0  \\
                 4  &2  &0
               \end{mat} \)
      \partsitem \( \begin{mat}[r]
                 0  &1  &0  &2  \\
                 1  &1  &0  &4  \\
                 3  &3  &3  &-1
               \end{mat} \)
    \end{exparts*}
    \begin{answer}
      We need only compute the rank of each.
      \begin{exparts*}
        \partsitem \( \begin{mat}[r]
                   1  &0  &0  \\
                   0  &0  &0
                 \end{mat} \)
        \partsitem \( \begin{mat}[r]
                   1  &0  &0  &0  \\
                   0  &1  &0  &0  \\
                   0  &0  &1  &0
                 \end{mat} \)
      \end{exparts*}  
    \end{answer}
  \item 
    Suppose that, with respect to
    \begin{equation*}
      B=\stdbasis_2
      \qquad
      D=\sequence{\colvec[r]{1 \\ 1},\colvec[r]{1 \\ -1}}
    \end{equation*}
    the transformation \( \map{t}{\Re^2}{\Re^2} \) is represented by
    this matrix.
    \begin{equation*}
      \begin{mat}[r]
        1  &2  \\
        3  &4
      \end{mat}
    \end{equation*}
    Use change of basis matrices to represent \( t  \) with respect
    to each pair.
    \begin{exparts}
      \partsitem \( \hat{B}=\sequence{\colvec[r]{0 \\ 1},\colvec[r]{1 \\ 1}} \),
        \( \hat{D}=\sequence{\colvec[r]{-1 \\ 0},\colvec[r]{2 \\ 1}} \)
      \partsitem \( \hat{B}=\sequence{\colvec[r]{1 \\ 2},\colvec[r]{1 \\ 0}} \),
        \( \hat{D}=\sequence{\colvec[r]{1 \\ 2},\colvec[r]{2 \\ 1}} \)
    \end{exparts}
    \begin{answer}
      Recall the diagram
      and the formula.
      \begin{equation*}
        \begin{CD}
          \Re^2_{\wrt{B}}                   @>t>T>        \Re^2_{\wrt{D}}       \\
          @V\text{\scriptsize$\identity$} VV   @V\text{\scriptsize$\identity$} VV \\
          \Re^2_{\wrt{\hat{B}}}             @>t>\hat{T}>  \Re^2_{\wrt{\hat{D}}}
        \end{CD}
        \qquad \hat{T}=
         \rep{\identity}{D,\hat{D}}\cdot T\cdot \rep{\identity}{\hat{B},B}
      \end{equation*}
      \begin{exparts}
        \partsitem These two 
          \begin{equation*}
            \colvec[r]{1 \\ 1}=1\cdot\colvec[r]{-1 \\ 0}
                            +1\cdot\colvec[r]{2 \\ 1}
            \qquad
            \colvec[r]{1 \\ -1}=(-3)\cdot\colvec[r]{-1 \\ 0}
                            +(-1)\cdot\colvec[r]{2 \\ 1}
          \end{equation*}
          show that
          \begin{equation*}
            \rep{\identity}{D,\hat{D}}=\begin{mat}[r]
              1  &-3  \\
              1  &-1
            \end{mat}
          \end{equation*}
          and similarly these two
          \begin{equation*}
            \colvec[r]{0 \\ 1}=0\cdot\colvec[r]{1 \\ 0}
                            +1\cdot\colvec[r]{0 \\ 1}
            \qquad
            \colvec[r]{1 \\  1}=1\cdot\colvec[r]{1 \\ 0}
                            +1\cdot\colvec[r]{0 \\ 1}
          \end{equation*}
          give the other nonsingular matrix.
          \begin{equation*}
            \rep{\identity}{\hat{B},B}=\begin{mat}[r]
              0  &1  \\
              1  &1
            \end{mat}
          \end{equation*}
          Then the answer is this.
          \begin{equation*}
            \hat{T}=
            \begin{mat}[r]
              1  &-3  \\
              1  &-1
            \end{mat}
            \begin{mat}[r]
              1  &2  \\
              3  &4
            \end{mat}
            \begin{mat}[r]
              0  &1  \\
              1  &1
            \end{mat}
            =\begin{mat}[r]
              -10  &-18  \\
              -2   &-4 
            \end{mat}
          \end{equation*}
          Although not strictly necessary, a check is reassuring.
          Arbitrarily fixing 
          \begin{equation*}
            \vec{v}=\colvec[r]{3 \\ 2}
          \end{equation*}
          we have that
          \begin{equation*}
            \rep{\vec{v}}{B}=\colvec[r]{3 \\ 2}_B
            \qquad
            \begin{mat}[r]
              1  &2  \\
              3  &4
            \end{mat}_{B,D}
            \colvec[r]{3 \\ 2}_B
            =\colvec[r]{7 \\ 17}_D
          \end{equation*}
          and so $t(\vec{v})$ is this.
          \begin{equation*}
            7\cdot\colvec[r]{1 \\ 1}+17\cdot\colvec[r]{1 \\ -1}=\colvec[r]{24 \\ -10}
          \end{equation*}
          Doing the calculation with respect to $\hat{B},\hat{D}$ starts with
          \begin{equation*}
            \rep{\vec{v}}{\hat{B}}=\colvec[r]{-1 \\ 3}_{\hat{B}}
            \qquad
            \begin{mat}[r]
              -10  &-18  \\
               -2  &-4
            \end{mat}_{\hat{B},\hat{D}}
            \colvec[r]{-1 \\ 3}_{\hat{B}}
            =\colvec[r]{-44 \\ -10}_{\hat{D}}
          \end{equation*}
          and then checks that this is the same result.
          \begin{equation*}
            -44\cdot\colvec[r]{-1 \\ 0}-10\cdot\colvec[r]{2 \\ 1}=\colvec[r]{24 \\ -10}
          \end{equation*}
    \partsitem These two
      \begin{equation*}
        \colvec[r]{1 \\ 1}=\frac{1}{3}\cdot\colvec[r]{1 \\ 2}
                  +\frac{1}{3}\cdot\colvec[r]{2 \\ 1}
        \qquad
        \colvec[r]{1 \\ -1}=-1\cdot\colvec[r]{1 \\ 2}
                  +1\cdot\colvec[r]{2 \\ 1}
      \end{equation*}
      show that
      \begin{equation*}
        \rep{\identity}{D,\hat{D}}=\begin{mat}[r]
          1/3  &-1  \\
          1/3  &1
        \end{mat}
      \end{equation*}
      and these two
      \begin{equation*}
        \colvec[r]{1 \\ 2}=1\cdot\colvec[r]{1 \\ 0}
                  +2\cdot\colvec[r]{0 \\ 1}
        \qquad
        \colvec[r]{1 \\ 0}=-1\cdot\colvec[r]{1 \\ 0}
                  +0\cdot\colvec[r]{0 \\ 1}
      \end{equation*}
      show this.
      \begin{equation*}
        \rep{\identity}{\hat{B},B}=\begin{mat}[r]
          1  &1  \\
          2  &0
        \end{mat}
      \end{equation*}
      With those, the conversion goes in this way.
      \begin{equation*}
        \hat{T}=\begin{mat}[r]
          1/3  &-1  \\
          1/3  &1
        \end{mat}
        \begin{mat}[r]
          1  &2  \\
          3  &4
        \end{mat}
        \begin{mat}[r]
          1  &1  \\
          2  &0
        \end{mat}
        =\begin{mat}[r]
          -28/3  &-8/3  \\
          38/3   &10/3
        \end{mat}
      \end{equation*}
      As in the prior item, a check provides some confidence that we did
      this calculation without mistakes.
      We can for instance, fix the vector
      \begin{equation*}
        \vec{v}=\colvec[r]{-1 \\ 2}
      \end{equation*}
      (this is arbitrary, taken from thin air).
      Now we have
      \begin{equation*}
        \rep{\vec{v}}{B}=\colvec[r]{-1 \\ 2}
        \qquad
        \begin{mat}[r]
          1  &2  \\
          3  &4
        \end{mat}_{B,D}
        \colvec[r]{-1  \\ 2}_{B}
        =\colvec[r]{3  \\ 5}_D
      \end{equation*}
      and so $t(\vec{v})$ is this vector.
      \begin{equation*}
        3\cdot\colvec[r]{1 \\ 1}+5\cdot\colvec[r]{1 \\ -1}=\colvec[r]{8 \\ -2}
      \end{equation*}
      With respect to $\hat{B},\hat{D}$ we first calculate
      \begin{equation*}
        \rep{\vec{v}}{\hat{B}}=\colvec[r]{1 \\ -2}
        \qquad
        \begin{mat}[r]
          -28/3  &-8/3  \\
          38/3   &10/3
        \end{mat}_{\hat{B},\hat{D}}
        \colvec[r]{1 \\ -2}_{\hat{B}}
        =\colvec[r]{-4 \\ 6}_{\hat{D}}
      \end{equation*}
      and, sure enough, that is the same result for $t(\vec{v})$.
      \begin{equation*}
        -4\cdot\colvec[r]{1 \\ 2}+6\cdot\colvec[r]{2 \\ 1}=\colvec[r]{8 \\ -2}
      \end{equation*}
      \end{exparts} 
    \end{answer}
  \item 
    What sizes are \( P \) and \( Q \) in the equation $\hat{H}=PHQ$?
    \begin{answer}
      Where \( H \) and \( \hat{H} \) are \( \nbym{m}{n} \), the
      matrix \( P \) is \( \nbyn{m} \) while \( Q \) is \( \nbyn{n} \).  
    \end{answer}
  \recommended \item
    Consider the spaces $V=\polyspace_2$ and~$W=\matspace_{\nbyn{2}}$, 
    with these bases.
    \begin{gather*}
      B=\sequence{1, 1+x, 1+x^2}
      \qquad
      D=\sequence{
        \begin{mat}
          0 &0 \\
          0 &1
        \end{mat},
        \begin{mat}
          0 &0 \\
          1 &1
        \end{mat},
        \begin{mat}
          0 &1 \\
          1 &1
        \end{mat},
        \begin{mat}
          1 &1 \\
          1 &1
        \end{mat}
         }                        \\
      \hat{B}=\sequence{1, x, x^2}
      \qquad
      \hat{D}=\sequence{
        \begin{mat}
          -1 &0 \\
          0 &0
        \end{mat},
        \begin{mat}
          0 &-1 \\
          0 &0
        \end{mat},
        \begin{mat}
          0 &0 \\
          1 &0
        \end{mat},
        \begin{mat}
          0 &0 \\
          0 &1
        \end{mat}
         }
    \end{gather*}
    We will find $P$ and~$Q$ to convert the representation of a map with 
    respect to $B,D$ to one with respect to $\hat{B},\hat{D}$
    \begin{exparts}
      \partsitem Draw the appropriate arrow diagram.
      \partsitem Compute $P$ and~$Q$.
    \end{exparts}
    \begin{answer}
      \begin{exparts}
        \partsitem
          This is the arrow diagram.
          \begin{equation*}
            \begin{CD}
              V_{\wrt{B}}                   @>h>H>        W_{\wrt{D}}       \\
              @V{\text{\scriptsize$\identity$}} VV      @V{\text{\scriptsize$\identity$}} VV \\
              V_{\wrt{\hat{B}}}             @>h>\hat{H}>   W_{\wrt{\hat{D}}}
            \end{CD}
          \end{equation*}
          The diagram gives $\hat{H}=PHQ$ where 
           $Q=\rep{\identity}{\hat{B},B}$ and $P=\rep{\identity}{D,\hat{D}}$
           (remembering that the operation done first is written on the right).
        \partsitem
          We want $Q=\rep{\identity}{\hat{B},B}$ and 
          $P=\rep{\identity}{D,\hat{D}}$.
          For $Q$ we do these calculations (done here by eye).
          \begin{equation*}
            \rep{\identity(1)}{B}=\colvec{1 \\ 0 \\ 0}
            \quad
            \rep{\identity(x)}{B}=\colvec{-1 \\ 1 \\ 0}
            \quad
            \rep{\identity(x^2)}{B}=\colvec{-1 \\ 0 \\ 1}
          \end{equation*}
          These calculations give $P$.
          \begin{multline*}
            \rep{\identity(
              \begin{mat}
                0 &0 \\
                0 &1
              \end{mat})}{\hat{D}}=\colvec{0 \\ 0 \\ 0 \\ 1} 
            \quad
            \rep{\identity(    
              \begin{mat}
                0 &0 \\
                1 &1
              \end{mat})}{\hat{D}}=\colvec{0 \\ 0 \\ 1 \\ 1} 
            \\
            \rep{\identity(    
              \begin{mat}
                0 &1 \\
                1 &1
              \end{mat})}{\hat{D}}=\colvec{0 \\ -1 \\ 1 \\ 1} 
            \rep{\identity(    
              \begin{mat}
                1 &1 \\
                1 &1
              \end{mat})}{\hat{D}}=\colvec{-1 \\ -1 \\ 1 \\ 1} 
          \end{multline*}
          This is the answer.
          \begin{equation*}
            P=
            \begin{mat}
              0 &0 &0  &-1 \\
              0 &0 &-1 &-1  \\
              0 &1 &1  &1  \\
              1 &1 &1  &1
            \end{mat}
            \qquad
            Q=\begin{mat}
              1 &-1 &-1 \\
              0 &1  &0  \\
              0 &0  &1
            \end{mat}
          \end{equation*}      
      \end{exparts}
    \end{answer}
  \recommended \item Find the change of basis matrices that will convert the 
   representation of a $\map{t}{\Re^2}{\Re^2}$
   with respect to $B,D$ to one with respect to $\hat{B},\hat{D}$.
   \begin{equation*}
     B=\sequence{\colvec{1 \\ 0}, \colvec{1 \\ 1}}\quad
     D=\sequence{\colvec{0 \\ 1}, \colvec{-1 \\ 0}}\quad
     \hat{B}=\stdbasis_2\quad
     \hat{D}=\sequence{\colvec{1 \\ -1}, \colvec{0 \\ 1}}
   \end{equation*}
    \begin{answer}
          This is the arrow diagram.
          \begin{equation*}
            \begin{CD}
              V_{\wrt{B}}                   @>t>T>        W_{\wrt{D}}       \\
              @V{\text{\scriptsize$\identity$}} VV      @V{\text{\scriptsize$\identity$}} VV \\
              V_{\wrt{\hat{B}}}             @>t>\hat{T}>   W_{\wrt{\hat{D}}}
            \end{CD}
          \end{equation*}
          Where $Q=\rep{\identity}{\hat{B},B}$ and 
          $P=\rep{\identity}{D,\hat{D}}$
          the equation is $\hat{T}=PTQ$.
 
          These are the calculations for $Q=\rep{\identity}{\hat{B},B}$
          (done by eye).
          \begin{equation*}
            \rep{\identity(\colvec{1 \\ 0})}{B}=\colvec{1 \\ 0}
            \quad
            \rep{\identity(\colvec{0 \\ 1})}{B}=\colvec{-1 \\ 1}
          \end{equation*}
          So we get this.
          \begin{equation*}
            Q=\begin{mat}
              1 &-1 \\
              0 &1  
             \end{mat}
          \end{equation*}
          These calculations give $P=\rep{\identity}{D,\hat{D}}$.
          \begin{equation*}
            \rep{\identity(\colvec{0 \\ 1})}{\hat{D}}=\colvec{0 \\ 1}
            \quad
            \rep{\identity(\colvec{-1 \\ 0})}{\hat{D}}=\colvec{-1 \\ -1}
          \end{equation*}
          Concatenate them to make the other matrix.
          \begin{equation*}
            P=\begin{mat}
              0 &-1 \\
              1 &-1  
             \end{mat}
          \end{equation*}
    \end{answer}
  \recommended \item Find the $P$ and~$Q$ to express $H$ via $PHQ$ 
    as a block partial identity matrix.
    \begin{equation*}
      H=
      \begin{mat}
        2 &1  &1  \\
        3 &-1 &0  \\
        1 &3  &2 
      \end{mat}
    \end{equation*}
    \begin{answer}
      Gauss's Method gives this.
      \begin{equation*}
          \begin{mat}
            2 &1  &1  \\
            3 &-1 &0  \\
            1 &3  &2 
          \end{mat}
          \grstep[-(1/2)\rho_1+\rho_3]{-(3/2)\rho_1+\rho_2}
          \grstep{\rho_2+\rho_3}
          \grstep[-(2/5)\rho_2]{(1/2)\rho_1}
          \begin{mat}
            1 &1/2  &1/2  \\
            0 &1    &3/5  \\
            0 &0    &0 
          \end{mat}
      \end{equation*}
      Column operations complete the job of reaching the canonical form
      for matrix equivalence.
      \begin{equation*}
        \grstep{-(3/5)\text{col}_2+\text{col}_3}
        \grstep[-(1/5)\text{col}_1+\text{col}_3]{-(1/2)\text{col}_1+\text{col}_2}
          \begin{mat}
            1 &0    &0  \\
            0 &1 &0  \\
            0 &0  &0 
          \end{mat}
      \end{equation*}
      Then these are the two matrices.
      \begin{align*}
        P&=
        \begin{mat}
          1 &0 &0 \\
          0 &-2/5 &0 \\
          0 &0 &1
         \end{mat}
        \begin{mat}
          1/2 &0 &0 \\
          0 &1 &0 \\
          0 &0 &1
         \end{mat}
        \begin{mat}
          1 &0 &0 \\
          0 &1 &0 \\
          0 &1 &1
         \end{mat}
        \begin{mat}
          1    &0 &0 \\
          0    &1 &0 \\
          -1/2 &0 &1
         \end{mat}
        \begin{mat}
          1    &0 &0 \\
          -3/2 &1 &0 \\
          0    &0 &1
         \end{mat}                                 \\ 
         &=
         \begin{mat}
           1/2   &0     &0 \\
           3/5   &-2/5  &0 \\
           -2    &1     &1
         \end{mat}                              \\
        Q&=
        \begin{mat}
          1 &0 &0 \\
          0 &1 &-3/5 \\
          0 &0 &1
        \end{mat}
        \begin{mat}
          1 &-1/2 &0 \\
          0 &1    &0 \\
          0 &0    &1
        \end{mat}
        \begin{mat}
          1 &0 &-1/5 \\
          0 &1 &0 \\
          0 &0 &1
        \end{mat}
        =
        \begin{mat}
          1 &-1/2 &-1/5 \\
          0 &1    &-3/5 \\
          0 &0    &1
        \end{mat}
      \end{align*}      
    \end{answer}
  \recommended \item
    Use \nearbytheorem{th:CanonFormForMatEquiv} to show that a square matrix
    is nonsingular if and only if it is equivalent to an identity matrix.
    \begin{answer}
        Any \( \nbyn{n} \) matrix is nonsingular if and only if it has 
        rank \( n \), that is, by  \nearbytheorem{th:CanonFormForMatEquiv},
        if and only if it is matrix equivalent to 
        the $\nbyn{n}$ matrix whose diagonal is all ones.  
    \end{answer}
  \item
    Show that, where \( A \) is a nonsingular square matrix, if
    \( P \) and \( Q \) are nonsingular square matrices such that \( PAQ=I \)
    then \( QP=A^{-1} \).
    \begin{answer}
      If \( PAQ=I \) then \( QPAQ=Q \), so \( QPA=I \), and so
      \( QP=A^{-1} \).  
    \end{answer}
  \item
    Why does \nearbytheorem{th:CanonFormForMatEquiv} not show that every matrix
    is diagonalizable (see \nearbyexample{ex:DiagizedMat})?
    \begin{answer}
      By the definition following  \nearbyexample{ex:DiagizedMat}, a matrix
      $M$ is diagonalizable if it represents $M=\rep{t}{B,D}$
      a transformation with the property that there is some basis
      $\hat{B}$ such that $\rep{t}{\hat{B},\hat{B}}$ is a diagonal 
      matrix\Dash the starting and ending bases must be equal. 
      But \nearbytheorem{th:CanonFormForMatEquiv}  says only that there are 
      $\hat{B}$ and $\hat{D}$ such that we can 
      change to a representation $\rep{t}{\hat{B},\hat{D}}$ and get a diagonal
      matrix.
      We have no reason to suspect that we could pick the two
      $\hat{B}$ and $\hat{D}$ so that they are equal.
    \end{answer}
  \item 
    Must matrix equivalent matrices have matrix equivalent transposes?
    \begin{answer}
      Yes.
      Row rank equals column rank, so the rank of the transpose equals
      the rank of the matrix.
      Same-sized matrices with equal ranks are matrix equivalent.  
    \end{answer}
  \item 
    What happens in \nearbytheorem{th:CanonFormForMatEquiv} if \( k=0 \)?
    \begin{answer}
      Only a zero matrix has rank zero.  
    \end{answer}
  \item \label{exer:MatEqIsEqRel}
    Show that matrix equivalence is an equivalence relation.
    \begin{answer}
      For reflexivity, to show that any matrix is matrix equivalent to
      itself, 
      take \( P \) and \( Q \) to be identity matrices.
      For symmetry, if \( H_1=PH_2Q \) then
      \( H_2=P^{-1}H_1Q^{-1} \) (inverses exist because $P$ and $Q$ are
      nonsingular).
      Finally, for transitivity, assume that \( H_1=P_2H_2Q_2 \) and 
      that \( H_2=P_3H_3Q_3 \).
      Then substitution gives 
      \( H_1=P_2(P_3H_3Q_3)Q_2=(P_2P_3)H_3(Q_3Q_2) \).
      A product of nonsingular matrices is nonsingular (we've shown that
      the product of invertible matrices is invertible;~in fact, we've shown
      how to calculate the inverse) and so \( H_1 \) is therefore
      matrix equivalent to \( H_3 \).  
     \end{answer}
  \recommended \item 
    Show that a zero matrix is alone in its matrix equivalence
    class.
    Are there other matrices like that?
    \begin{answer}
      By \nearbytheorem{th:CanonFormForMatEquiv}, a zero matrix is alone 
      in its class because it is the only \( \nbym{m}{n} \) of rank zero.
      No other matrix is alone in its class; 
      any nonzero scalar product of a matrix
      has the same rank as that matrix.  
    \end{answer}
  \item 
    What are the matrix equivalence classes of matrices of
    transformations on \( \Re^1 \)?
    \( \Re^3 \)?
    \begin{answer}
      There are two matrix equivalence classes of \( \nbyn{1} \)
      matrices\Dash those of rank zero and those of rank one.
      The  \( \nbyn{3} \) matrices fall into four matrix equivalence
      classes.
    \end{answer}
  \item 
    How many matrix equivalence classes are there?
    \begin{answer}
      For \( \nbym{m}{n} \) matrices there are classes for each possible
      rank: where \( k \)  is the minimum of \( m \) and \( n \) there are
      classes for the matrices of rank \( 0 \), \( 1 \), \ldots, \( k \).
      That's \( k+1 \) classes.  
      (Of course, totaling over all sizes of matrices we get infinitely
      many classes.)
    \end{answer}
  \item 
    Are matrix equivalence classes closed under scalar
    multiplication?
    Addition?
    \begin{answer}
      They are closed under nonzero scalar multiplication since
      a nonzero scalar multiple of a matrix has the same rank as does the
      matrix.
      They are not closed under addition,
      for instance, \( H+(-H) \) has rank zero.  
    \end{answer}
  \item 
    Let \( \map{t}{\Re^n}{\Re^n} \) represented by
    \( T \) with respect to \( \stdbasis_n,\stdbasis_n \).
    \begin{exparts}
      \partsitem Find $\rep{t}{B,B}$ in this specific case.
        \begin{equation*}
          T=\begin{mat}[r]
            1  &1  \\
            3  &-1
          \end{mat}
          \qquad
          B=\sequence{\colvec[r]{1  \\ 2},
                      \colvec[r]{-1 \\ -1}}
        \end{equation*}
      \partsitem Describe $\rep{t}{B,B}$ in the general case where
        \( B=\basis{\beta}{n} \).
    \end{exparts}
    \begin{answer}
        Here is the picture.
        \begin{equation*}
          \begin{CD}
            \Re^2_{\wrt{\stdbasis_2}}         @>t>T>     \Re^2_{\wrt{\stdbasis_2}}    \\
            @V\text{\scriptsize$\identity$} VV   @V\text{\scriptsize$\identity$} VV \\
           \Re^2_{\wrt{B}}        @>t>\hat{T}>  \Re^2_{\wrt{B}}
          \end{CD}
        \end{equation*}
        There are two ways to move from the lower left to the lower right.
        The first is direct, using $\hat{T}=\rep{t}{B,B}$.
        The second moves up, then over, then down, using
        $\rep{\identity}{\stdbasis_2,B}
                          \cdot T
                          \cdot \rep{\identity}{B,\stdbasis_2}$
        (remember that they get written right-to-left, so the ``up'' 
        matrix is on the right, in order to have that when
        applied to a $\rep{\vec{v}}{B}$ the matrix applied first is 
        $\rep{\identity}{B,\stdbasis_2}$).
        We write $S$ for $\rep{\identity}{B,\stdbasis_2}$; 
        of the two we choose this one because it is easier to calculate.
        So we have $\hat{T}=S^{-1}TS$.
      \begin{exparts}
        \partsitem We have
          \begin{equation*}
            \rep{\identity}{B,\stdbasis_2}=S=
            \begin{mat}[r]
              1  &-1  \\
              2  &-1
            \end{mat}
          \end{equation*}
          and
          \begin{equation*}
            \rep{\identity}{\stdbasis_2,B}
            =\rep{\identity}{B,\stdbasis_2}^{-1}=S^{-1}=
            \begin{mat}[r]
              1  &-1  \\
              2  &-1
            \end{mat}^{-1}
            =\begin{mat}[r]
              -1  &1  \\
              -2  &1 
            \end{mat}
          \end{equation*}
          and thus the answer is this.
          \begin{equation*}
            \rep{t}{B,B}
            =
            \begin{mat}[r]
              -1  &1  \\
              -2  &1 
            \end{mat}
            \begin{mat}[r]
              1  &1  \\
              3  &-1   
            \end{mat}
            \begin{mat}[r]
              1  &-1  \\
              2  &-1
            \end{mat}
            =\begin{mat}[r]
              -2  &0   \\
              -5  &2 
            \end{mat}
          \end{equation*}
          As a quick check, we can take a vector at random
          \begin{equation*}
            \vec{v}=\colvec[r]{4  \\  5}
          \end{equation*}
          giving
          \begin{equation*}
            \rep{\vec{v}}{\stdbasis_2}=\colvec[r]{4 \\ 5}
            \qquad
            \begin{mat}[r]
              1  &1  \\
              3  &-1   
            \end{mat}
            \colvec[r]{4 \\ 5}
            =\colvec[r]{9 \\ 7}=t(\vec{v})
          \end{equation*}
          while the calculation with respect to $B,B$ 
          \begin{equation*}
            \rep{\vec{v}}{B}=\colvec[r]{1 \\ -3}
            \qquad
            \begin{mat}[r]
              -2  &0   \\
              -5  &2 
            \end{mat}_{B,B}
            \colvec[r]{1 \\ -3}_B
            =\colvec[r]{-2 \\ -11}_B
          \end{equation*}
          yields the same result.
          \begin{equation*}
            -2\cdot\colvec[r]{1 \\ 2}-11\cdot\colvec[r]{-1 \\ -1}
               =\colvec[r]{9 \\ 7}
          \end{equation*}
       \partsitem As in the first item of this question
          \begin{equation*}
            S=\rep{\identity}{B,\stdbasis_2}
            =\begin{pmat}{c|c|c}
              \vec{\beta}_1 &\;\cdots\; &\vec{\beta}_n
            \end{pmat}
            \qquad
            \rep{\identity}{\stdbasis_2,B}=S^{-1}=
             \rep{\identity}{B,\stdbasis_2}^{-1}
          \end{equation*}
          so, writing $S$ for the matrix whose columns are the basis vectors,
          we have that $\rep{t}{B,B}=\hat{T}=S^{-1}TS$.
      \end{exparts}
    \end{answer}
  \item 
    \begin{exparts}
       \partsitem Let \( V \) have bases \( B_1 \) and \( B_2 \) and 
         suppose that \( W \) has the basis \( D \).
         Where \( \map{h}{V}{W} \), find the formula that computes
         \( \rep{h}{B_2,D} \) from \( \rep{h}{B_1,D} \).
       \partsitem Repeat the prior question with one basis 
         for \( V \) and two bases for \( W \).
     \end{exparts}
     \begin{answer}
       \begin{exparts}
         \partsitem The adapted form of the arrow diagram is this.
           \begin{equation*}
             \begin{CD}
               V_{\wrt{B_1}}                   @>h>H>        W_{\wrt{D}}       \\
               @V\text{\scriptsize$\identity$} VQV      @V\text{\scriptsize$\identity$} VPV \\
               V_{\wrt{B_2}}             @>h>\hat{H}>  W_{\wrt{D}}
             \end{CD}
           \end{equation*}
           Since there is no need to change bases in 
           \( W \) (or we can
           say that the change of basis matrix $P$ is the identity), we have
           \( \rep{h}{B_2,D}=\rep{h}{B_1,D}\cdot Q \) where
           \( Q=\rep{\identity}{B_2,B_1} \).
         \partsitem Here, this is the arrow diagram. 
           \begin{equation*}
             \begin{CD}
               V_{\wrt{B}}                   @>h>H>   W_{\wrt{D_1}}       \\
               @V\text{\scriptsize$\identity$} VQV  @V\text{\scriptsize$\identity$} VPV \\
               V_{\wrt{B}}             @>h>\hat{H}>  W_{\wrt{D_2}}
             \end{CD}
           \end{equation*}
           We have that \( \rep{h}{B,D_2}=P\cdot \rep{h}{B,D_1} \) where
           \( P=\rep{\identity}{D_1,D_2} \).
       \end{exparts}  
      \end{answer}
  \item 
    \begin{exparts}
      \partsitem If two matrices are matrix equivalent and invertible,
        must their
        inverses be matrix equivalent?
      \partsitem If two matrices have matrix equivalent inverses, must the two
        be matrix equivalent?
      \partsitem If two matrices are square and matrix equivalent, must their
        squares be matrix equivalent?
      \partsitem If two matrices are square and have matrix equivalent squares,
        must they be matrix equivalent?
    \end{exparts}
    \begin{answer}
      \begin{exparts}
        \partsitem Here is the arrow diagram, and a version of that diagram
          for inverse functions.
          \begin{equation*}
           \begin{CD}
             V_{\wrt{B}}                   @>h>H>      W_{\wrt{D}}       \\
             @V\text{\scriptsize$\identity$} VQV  @V\text{\scriptsize$\identity$} VPV \\
             V_{\wrt{\hat{B}}}             @>h>\hat{H}> W_{\wrt{\hat{D}}}
            \end{CD}
            \hspace*{4em}\qquad
           \begin{CD}
             V_{\wrt{B}}              @<h^{-1}<H^{-1}<    W_{\wrt{D}}       \\
             @V\text{\scriptsize$\identity$} VQV  @V\text{\scriptsize$\identity$} VPV \\
             V_{\wrt{\hat{B}}}        @<h^{-1}<\hat{H}^{-1}< W_{\wrt{\hat{D}}}
            \end{CD}
           \end{equation*}
           Yes, the inverses of the matrices represent the 
           inverses of the maps.
           That is, we can move from the lower right to the lower left by
           moving up, then left, then down.
           In other words, where \( \hat{H}=PHQ \) (and  \( P,Q \) invertible)
           and \( H,\hat{H} \) are invertible then
           \( \hat{H}^{-1}=Q^{-1}H^{-1}P^{-1} \).
        \partsitem Yes; this is the prior part repeated in different terms.
        \partsitem No, we need another assumption:~if \( H \) represents 
          \( h \) with respect to the same starting as ending bases \( B,B \), 
          for some \( B \) then \( H^2 \) represents
          \( \composed{h}{h} \).
          As a specific example, 
          these two matrices are both rank one and so they are
          matrix equivalent
          \begin{equation*}
             \begin{mat}[r]
               1  &0  \\
               0  &0
             \end{mat}
             \qquad
             \begin{mat}[r]
               0  &0  \\
               1  &0
             \end{mat}
          \end{equation*}
          but the squares are not matrix equivalent\Dash the square of the 
          first has rank one while the square of the second has rank zero.
        \partsitem No.
          These two are not matrix equivalent but have matrix equivalent
          squares.
          \begin{equation*}
             \begin{mat}[r]
               0  &0  \\
               0  &0
             \end{mat}
             \qquad
             \begin{mat}[r]
               0  &0  \\
               1  &0
             \end{mat} 
          \end{equation*}
      \end{exparts}  
    \end{answer}
  \item
    Square matrices are \definend{similar} if they represent the same
    transformation, but each with respect to the same ending as starting
    basis.
    That is, \( \rep{t}{B_1,B_1} \) is similar to \( \rep{t}{B_2,B_2} \).
    \begin{exparts}
      \partsitem Give a  definition of matrix similarity like that of
        \nearbydefinition{def:MatEquiv}.
      \partsitem Prove that similar matrices are matrix equivalent.
      \partsitem Show that similarity is an equivalence relation.
      \partsitem Show that if \( T \) is similar to \( \hat{T} \) then
        \( T^2 \) is similar to \( \hat{T}^2 \), the cubes are similar, etc.
        % \textit{Contrast with the prior exercise.}
      \partsitem Prove that there are matrix equivalent matrices 
        that are not similar.
    \end{exparts}
    \begin{answer}
      \begin{exparts}
        \partsitem The 
          arrow diagram suggests the definition.
          \begin{equation*}
            \begin{CD}
              V_{\wrt{B_1}}                   @>t>T>        V_{\wrt{B_1}}       \\
              @V\text{\scriptsize$\identity$} VV  @V\text{\scriptsize$\identity$} VV \\
              V_{\wrt{B_2}}             @>t>\hat{T}>  V_{\wrt{B_2}}
            \end{CD}
          \end{equation*}
          Call matrices \( T, \hat{T} \) \definend{similar} if there
          is a nonsingular matrix \( P \) such that
          \( \hat{T}=P^{-1}TP \).
        \item Take \( P^{-1} \) to be \( P \) and take \( P \) to be \( Q \).
        \item \textit{This is as in \nearbyexercise{exer:MatEqIsEqRel}.}
          Reflexivity is obvious: \( T=I^{-1}TI \).
          Symmetry is also easy: \( \hat{T}=P^{-1}TP \) implies that 
          \( T=P\hat{T}P^{-1} \) (multiply the first equation from the right
          by $P^{-1}$ and from the left by $P$).
          For transitivity, assume that \( T_1={P_2}^{-1}T_2P_2 \) and that 
          \( T_2={P_3}^{-1}T_3P_3 \).
          Then \( T_1={P_2}^{-1}({P_3}^{-1}T_3P_3)P_2
                     =({P_2}^{-1}{P_3}^{-1})T_3(P_3P_2) \) and we are finished
          on noting that \( P_3P_2 \) is an invertible matrix with inverse
          \( {P_2}^{-1}{P_3}^{-1} \).
        \partsitem Assume  \( \hat{T}=P^{-1}TP \).
          For squares,
          \( \hat{T}^2=(P^{-1}TP)(P^{-1}TP)
                      =P^{-1}T(PP^{-1})TP=P^{-1}T^2P \).
          Higher powers follow by induction.
        \partsitem These two are matrix equivalent but their squares are not
          matrix equivalent.
          \begin{equation*}
             \begin{mat}[r]
               1  &0  \\
               0  &0
             \end{mat}
             \qquad
             \begin{mat}[r]
               0  &0  \\
               1  &0
             \end{mat}
          \end{equation*}
          By the prior item, matrix similarity and matrix equivalence are thus
          different.
      \end{exparts}  
   \end{answer}
\index{matrix equivalence|)}
\index{change of basis|)}
\end{exercises}
