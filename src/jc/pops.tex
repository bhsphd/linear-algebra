% Chapter 4, Topic _Linear Algebra_ Jim Hefferon
%  http://joshua.smcvt.edu/linearalgebra
%  2001-Jun-12
\topic{Stable Populations}
\index{stable populations|(}
\index{populations, stable|(}
Imagine a reserve park with animals from a species that we are 
protecting.
The park doesn't have a fence so animals cross the boundary, 
both from the inside out and from the outside in.
Every year, $10\%$ of the animals from inside of the park leave and  
$1\%$ of the animals from the outside 
find their way in.
Can we reach a stable level;
are there populations for the park and the rest of the world that
will stay constant over time,
with the number of animals leaving equal to the number of animals entering?

Let $p_n$ be the year~$n$ population in the park and 
let $r_n$ be the  population in the rest of the world.
\begin{align*}
  p_{n+1} 
  &=.90p_n+.01r_n    \\
  r_{n+1}
  &=.10p_n+.99r_n 
\end{align*}
We have this matrix equation.
\begin{equation*}
  \colvec{p_{n+1} \\ r_{n+1}}
  =\begin{mat}[r]
    .90  &.01  \\
    .10  &.99
  \end{mat}
  \colvec{p_{n} \\ r_{n}}
\end{equation*}
The population will be stable if $p_{n+1}=p_n$ and $r_{n+1}=r_n$ so that the 
matrix equation $\vec{v}_{n+1}=T\vec{v}_{n}$  becomes $\vec{v}=T\vec{v}$.
We are therefore looking for eigenvectors for $T$ that are associated with
the eigenvalue $\lambda=1$.
The  equation $\zero=(\lambda I-T)\vec{v}=(I-T)\vec{v}$ is
\begin{equation*}
  \begin{mat}[r]
      0.10  &-0.01  \\
      -0.10  &0.01
  \end{mat}
  \colvec{p \\ r}=
  \colvec[r]{0 \\ 0}
\end{equation*}
and gives the eigenspace of vectors with the restriction that $p=.1r$.
For example,
if we start with a park population $p=10\,000$~animals and a
rest of the world population of $r=100\,000$~animals then every year 
ten percent of those inside leave the park
(this is a thousand animals), 
and every year one percent of those from the rest of
the world enter the park (also a thousand animals).
The population is stable, self-sustaining. 

Now imagine that we are trying to raise 
the total world population of this species.
We are trying to have the world population grow at
$1\%$ per year.
This makes the population level stable in some sense,
although it is a dynamic stability, in contrast to the static 
population level of the $\lambda=1$ case.
The equation $\vec{v}_{n+1}=1.01\cdot\vec{v}_n=T\vec{v}_{n}$ leads to 
$((1.01 I-T)\vec{v}=\zero$, which gives this system. 
\begin{equation*}
  \begin{mat}[r]
     0.11   &-0.01  \\
     -0.10  &0.02
  \end{mat}
  \colvec{p \\ r}=
  \colvec[r]{0 \\ 0}
\end{equation*}
This matrix is nonsingular and so the only solution is $p=0$,
$r=0$.
Thus there is no nontrivial initial population
that would lead to a regular annual one percent growth rate
in $p$ and $r$.

We can look for the rates that 
allow an initial population for
the park that results in a steady growth behavior.
We consider $\lambda\vec{v}=T\vec{v}$ and solve for $\lambda$.
\begin{equation*}
  0=\begin{vmat}
    \lambda-.9  &.01  \\
    .10         &\lambda-.99
  \end{vmat}
  =(\lambda-.9)(\lambda-.99)-(.10)(.01)
  =\lambda^2-1.89\lambda+.89
\end{equation*}
We already know that $\lambda=1$ is one solution of this characteristic 
equation.
The other is $0.89$.
Thus there are two ways to have a dynamically stable 
$p$ and $r$, where the two grow at the same rate despite
the leaky park boundaries:~have a world population that is does not 
grow or shrink, and have a world population that shrinks by $11\%$ every year.
 
So one way to look at eigenvalues and eigenvectors is that they give a 
stable state for a system.
If the eigenvalue is one then the system is static and
if the eigenvalue isn't one then it is a dynamic stability.





\begin{exercises}
  \item 
    For the park discussed above, what should be the initial park population
    in the
    case where the populations decline by $11\%$ every year?
    \begin{answer}
      The equation 
      \begin{equation*}
         0.89I-T
         =
         \begin{mat}
           0.89 &0 \\
           0 &0.89
         \end{mat}
         -
         \begin{mat}
           .90  &.01  \\
           .10  &.99
         \end{mat}
         =
         \begin{mat}
           -.01  &-.01  \\
           -.10  &-.10
         \end{mat}
      \end{equation*} 
      leads to this system.
      \begin{equation*}
        \begin{mat}
          -.01  &-.01  \\
          -.10  &-.10
        \end{mat}
        \colvec{p \\ r}
        =
        \colvec{0  \\ 0}
      \end{equation*}\
      So the eigenvectors have $p=-r$.
    \end{answer}
  \item 
    What will happen to the population of the park in the event of
    a growth in world population of $1\%$ per year?
    Will it lag the world growth, or lead it?
    Assume that the initial park population is ten thousand, and the
    world population is one hundred thousand, 
    and calculate over a ten year span.
    \begin{answer}
      \Sage{} finds this.
\begin{lstlisting}
sage: M = matrix(RDF, [[0.90,0.01], [0.10,0.99]])
sage: v = vector(RDF, [10000, 100000])
sage: for y in range(10):
....:     v[1] = v[1]*(1+.01)^y
....:     print "pop vector year",y," is",v
....:     v = M*v
....:     
pop vector year 0  is (10000.0, 100000.0)
pop vector year 1  is (10000.0, 101000.0)
pop vector year 2  is (10010.0, 103019.899)
pop vector year 3  is (10039.19899, 106111.421211)
pop vector year 4  is (10096.3933031, 110360.453787)
pop vector year 5  is (10190.3585107, 115891.187687)
pop vector year 6  is (10330.2345365, 122872.349786)
pop vector year 7  is (10525.9345807, 131525.973067)
pop vector year 8  is (10788.6008533, 142139.351965)
pop vector year 9  is (11131.1342876, 155081.09214)
\end{lstlisting}
       So inside the park the population grows by about eleven percent while
       outside the park the population grows by about fifty five percent.
    \end{answer}
  \item 
    The park discussed above is partially fenced so that now,
    every year, only $5\%$ of the animals from inside of the park leave (still,
    about $1\%$ of the animals from the outside 
    find their way in).
    Under what conditions can the park maintain a stable population now?
    \begin{answer}
      The matrix equation 
      \begin{equation*}
        \colvec{p_{n+1}  \\ r_{n+1}}
        =
        \begin{mat}
          0.95  &0.01  \\
          0.05  &0.99
        \end{mat}
        \colvec{p_n  \\ r_n}
      \end{equation*}
      means that to find eigenvalues we want to solve this.
      \begin{equation*}
        0
        =
        \begin{vmat}
          \lambda-0.95  &0.01  \\
          0.10          &\lambda-0.99
        \end{vmat}
        =\lambda^2-1.94\lambda-0.9404
      \end{equation*}
      \Sage{} gives this.
\begin{lstlisting}
sage: a,b,c = var('a,b,c') 
sage: qe = (x^2 - 1.94*x -.9404 == 0)
sage: print solve(qe, x)
[
x == -1/100*sqrt(18813) + 97/100,
x == 1/100*sqrt(18813) + 97/100
]
sage: n(-1/100*sqrt(18813) + 97/100)
-0.401604899378826
sage: n(1/100*sqrt(18813) + 97/100)
2.34160489937883       
\end{lstlisting}
    \end{answer}
  \item 
    Suppose that a species of bird only lives in Canada, the United States,
    or in Mexico.
    Every year, $4\%$ of the Canadian birds travel to the US, and $1\%$ of them
    travel to Mexico.
    Every year, $6\%$ of the US birds travel to Canada, and $4\%$ go to Mexico.
    From Mexico, every year $10\%$ travel to the US, and $0\%$ go to Canada.
    \begin{exparts}
      \partsitem Give the transition matrix.
      \partsitem Is there a way for the three countries to have constant
         populations?  
    \end{exparts}
    \begin{answer}
      \begin{exparts}
        \partsitem This is the recurrence.
          \begin{equation*}
            \colvec{c_{n+1} \\ u_{n+1} \\ m_{n+1}}
            =
            \begin{mat}
              .95  &.06  &0  \\
              .04  &.90  &.10  \\
              .01  &.04  &.90
            \end{mat}
            \colvec{c_n \\ u_n  \\ m_n}
          \end{equation*}
        \partsitem The system
          \begin{equation*}
            \begin{linsys}{3}
              .05c &- &.06u &  &     &= &0  \\
             -.04c &+ &.10u &- &.10m &= &0  \\
             -.01c &- &.04u &+ &.10m &= &0
            \end{linsys}
          \end{equation*}
          has infinitely many solutions.
\begin{lstlisting}
sage: var('c,u,m')
(c, u, m)
sage: eqns = [ .05*c-.06*u  == 0,
....:         -.04*c+.10*u-.10*m == 0,
....:         -.01*c-.04*u+.10*m == 0 ]
sage: solve(eqns, c,u,m)
[[c == 30/13*r1, u == 25/13*r1, m == r1]]            
\end{lstlisting}
      \end{exparts}
    \end{answer}
\end{exercises}
\index{populations, stable|)}
\index{stable populations|)}
\endinput






