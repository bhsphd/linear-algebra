% Chapter 2, Section 2 _Linear Algebra_ Jim Hefferon
%  http://joshua.smcvt.edu/linearalgebra
%  2001-Jun-10
\section{Linear Independence}
The prior section shows how to understand a vector space
as a span, 
as an unrestricted linear combination of some of its elements.
For example, the space of linear polynomials $\set{a+bx\suchthat a,b\in\Re}$ 
is spanned by the set $\set{1,x}$.
The prior section also showed that a space can have many sets that span it.
Two more sets that span the space of linear polynomials are 
$\set{1,2x}$ and $\set{1,x,2x}$.

At the end of that section we described some spanning sets as `minimal'
but we never precisely defined that word.
% We could mean one of two things.
We could mean that a spanning set is minimal if it 
contains the smallest number of members of any set with the same span,
so that $\set{1,x,2x}$ is not minimal because it has 
three members while we can give two-element sets spanning the same space.
Or we could mean that a spanning set is minimal when it has no elements 
that we can remove without changing the span.
Under this meaning $\set{1,x,2x}$ is not minimal because 
removing the \( 2x \) to get \( \set{1,x} \) leaves the
span unchanged.

The first sense of minimality appears to be a global requirement, 
in that to check if a spanning set is minimal 
we seemingly must look at all the sets that span 
and find one with the least number of elements. 
The second sense of minimality is local since
we need to look only at the set and consider the
span with and without various elements.
For instance, using the second sense
we could compare the span of $\set{1,x,2x}$ 
with the span of $\set{1,x}$ and
note that $2x$ is a ``repeat'' in that 
its removal doesn't shrink the span.

In this section we will use the second sense of `minimal spanning set'
because of this technical convenience.
However, the most important result of this book is that the two senses 
coincide.
We will prove that in the next section.


 






\subsection{Definition and Examples}

We saw ``repeats'' in the first chapter.
There, Gauss's Method turned them into 
$0=0$ equations.

\begin{example}  \label{ex:StaticsLIAndLD}
Recall the \hyperlink{ex:Statics}{Statics} example from
Chapter One's opening.
We got two balances with the pair of unknown-mass objects, one 
at \( 40 \)~cm and \( 15 \)~cm and another 
at \( -50 \)~cm and \( 25 \)~cm, and
we then computed the value of those masses.
Had we instead gotten the second balance at
\( 20 \)~cm and \( 7.5 \)~cm
then Gauss's Method on the resulting two-equations, two-unknowns system
would not have yielded a solution, it would have yielded a $0=0$ equation 
along with an equation containing a free variable.
Intuitively, the problem is that \( \rowvec{20 &7.5} \) is half of 
$\rowvec{40 &15}$, that is,
$\rowvec{20 &7.5}$ is in the span of the set $\set{\rowvec{40 &15}}$
and so is repeated data.
We would have been trying to 
solve a two-unknowns problem with essentially only one piece of information.
\end{example}
 
We take $\vec{v}$ to be a ``repeat''
of the vectors in a set~$S$ if $\vec{v}\in\spanof{S}$ so
that it depends on, that is, is expressible in terms of, 
elements of the set 
$\vec{v}=c_1\vec{s}_1+\cdots+c_n\vec{s}_n$.

\begin{lemma} \label{lm:AddVecSpanNoGrowIffVecInSpan}
%<*lm:AddVecSpanNoGrowIffVecInSpan>
Where $V$ is a vector space, $S$ is a subset of that space, and $\vec{v}$
is an element of that space, 
$\spanof{S\union\set{\vec{v}}} = \spanof{S}$
if and only if 
$\vec{v}\in\spanof{S}$.
%</lm:AddVecSpanNoGrowIffVecInSpan>
\end{lemma}

\begin{proof}
%<*pf:AddVecSpanNoGrowIffVecInSpan0>
Half of the if and only if is immediate: if $\vec{v}\notin\spanof{S}$ then 
the sets are not equal because $\vec{v}\in\spanof{S\union\set{\vec{v}}}$.

For the other half assume that $\vec{v}\in\spanof{S}$
so that $\vec{v}=c_1\vec{s}_1+\cdots+c_n\vec{s}_n$ for some scalars~$c_i$
and vectors $\vec{s}_i\in S$.
We will use mutual containment to show that the sets
$\spanof{S\union\set{\vec{v}}}$ and $\spanof{S}$ are equal.
The containment $\spanof{S\union\set{\vec{v}}}\supseteq\spanof{S}$ is clear.
%</pf:AddVecSpanNoGrowIffVecInSpan0>

%<*pf:AddVecSpanNoGrowIffVecInSpan1>
To show containment in the other direction let $\vec{w}$ be an element 
of~$\spanof{S\union\set{\vec{v}}}$.
Then $\vec{w}$ is a linear combination of elements of 
$S\union\set{\vec{v}}$, which we can write as 
$\vec{w}=c_{n+1}\vec{s}_{n+1}+\cdots+c_{n+k}\vec{s}_{n+k}+c_{n+k+1}\vec{v}$.
(Possibly some of the $\vec{s}_i$'s from $\vec{w}$'s equation are the same as
some of those from $\vec{v}$'s equation but that does not matter.)
Expand $\vec{v}$.
\begin{equation*}
  \vec{w}=c_{n+1}\vec{s}_{n+1}+\cdots+c_{n+k}\vec{s}_{n+k}
            +c_{n+k+1}\cdot(c_1\vec{s}_1+\cdots+c_n\vec{s}_n)
\end{equation*}
Recognize the right hand side as a linear combination of linear
combinations of vectors from~$S$.
Thus $\vec{w}\in\spanof{S}$.
%</pf:AddVecSpanNoGrowIffVecInSpan1>
\end{proof}

The discussion at the section's opening involved removing vectors instead of
adding them. 

\begin{corollary}  \label{cor:OmitVecNotShrinkSpanIffVecIsDep}
%<*cor:OmitVecNotShrinkSpanIffVecIsDep>
For $\vec{v}\in S$,
omitting that vector does not shrink the span
$\spanof{S}=\spanof{S-\set{\vec{v}}}$
if and only if it is dependent on other vectors in the 
set~$\vec{v}\in\spanof{S}$. 
%</cor:OmitVecNotShrinkSpanIffVecIsDep>
\end{corollary}

The corollary says that to know whether removing a vector will decrease
the span, we need to know whether the vector is a linear combination
of others in the set.

\begin{definition}\label{def:LinInd}
%<*df:LinInd>
In any vector space, a set of vectors is
\definend{linearly independent}\index{linearly independent}%
\index{sets!dependent, independent}
if none of its elements is a linear combination of the 
others from the set.\footnote{See also \nearbyremark{rem:WhyLIUsesMultiset}.}\spacefactor=1000\ 
Otherwise the set is 
\definend{linearly dependent}\index{linearly dependent}.
%</df:LinInd>
\end{definition}
% A multiset subset of a vector space is
% \definend{linearly independent}\index{linearly independent}%
% \index{sets!dependent, independent}
% if none of its elements is a linear combination of the 
% others.\footnote{More information on multisets is in the appendix. 
% Most of the time we won't need the 
% set-multiset distinction and we will
% follow the standard terminology of referring to a linearly independent or 
% dependent `set'.
% \nearbyremark{rem:WhyLIUsesMultiset} explains why 
% the definition requires a multiset.}\spacefactor=1000\ 
% Otherwise it is \definend{linearly dependent}\index{linearly dependent}.

Thus the set $\set{\vec{s}_0,\ldots,\vec{s}_n}$ is independent
if there is no equality
$\vec{s}_i=c_0\vec{s}_0+\ldots+c_{i-1}\vec{s}_{i-1}+c_{i+1}\vec{s}_{i+1}+\ldots+c_n\vec{s}_n$.
The definition's use of the word `others' means that
writing $\vec{s}_i$ as a linear combination via $\vec{s}_i=1\cdot\vec{s}_i$ 
does not count.

%<*LinInd>
Observe that, 
although this way of writing one vector as a combination of the others
\begin{equation*}
   \vec{s}_0=\lincombo{c}{\vec{s}}
\end{equation*}
visually sets off \( \vec{s}_0 \), algebraically
there is nothing special about that vector in that equation.
For any \( \vec{s}_i \) with a coefficient $c_i$ that is non-$0$
we can rewrite to isolate \( \vec{s}_i \).
\begin{equation*}
   \vec{s}_i=(1/c_i)\vec{s}_0+\dots
              +(-c_{i-1}/c_i)\vec{s}_{i-1}+(-c_{i+1}/c_i)\vec{s}_{i+1}
              +\dots+(-c_n/c_i)\vec{s}_n
\end{equation*}
When we don't want to single out any vector
we will instead say that
\( \vec{s}_0,\vec{s}_1,\dots,\vec{s}_n \) are in a
\definend{linear relationship}\index{linear relationship}%
\index{relationship!linear} 
and put all of the vectors on the same side.
%</LinInd>
The next result rephrases the linear independence definition in this style.
It is how we usually compute whether
a finite set is dependent or independent.

\begin{lemma}   \label{le:LDIffANonTrivLinRel}
%<*lm:LDIffANonTrivLinRel>
A subset \( S \) of a vector space is linearly independent if and only if 
among its elements  
the only linear relationship 
$ c_1\vec{s}_1+\dots+c_n\vec{s}_n=\zero$
is the trivial one, \( c_1=0,\dots,\,c_n=0 \)
(where $\vec{s}_i\neq\vec{s}_j$ when $i\neq j$) .
%</lm:LDIffANonTrivLinRel>
\end{lemma}

\begin{proof}
%<*pf:LDIffANonTrivLinRel>
If \( S \) is linearly independent then no vector
$\vec{s}_i$ 
is a linear combination of other vectors from $S$
so there is no linear relationship where some of the
$\vec{s}\,$'s have nonzero coefficients.

If \( S \) is not linearly independent then some \( \vec{s}_i \) is a linear
combination 
$\vec{s}_i=c_1\vec{s}_1+\dots+c_{i-1}\vec{s}_{i-1}
    +c_{i+1}\vec{s}_{i+1}+\dots+c_n\vec{s}_n$
of other vectors from \( S \). 
Subtracting $\vec{s}_i$ from both sides
gives a relationship
involving a nonzero coefficient,
the \( -1 \) in front of \( \vec{s}_i \).
%</pf:LDIffANonTrivLinRel>
\end{proof}

\begin{example}  \label{ex:StaticsLI}
In the vector space of two-wide row vectors, the two-element set
\( \set{ \rowvec{40 &15},\rowvec{-50 &25}} \) is linearly independent.
To check this, take
\begin{equation*}
  c_1\cdot\rowvec{40 &15}+c_2\cdot\rowvec{-50 &25}=\rowvec{0 &0}
\end{equation*}
and solve the resulting system.
\begin{equation*}
  \begin{linsys}{2}
    40c_1  &-  &50c_2  &=  &0  \\
    15c_1  &+  &25c_2  &=  &0  
   \end{linsys}
  \grstep{-(15/40)\rho_1+\rho_2}
  \begin{linsys}{2}
     40c_1  &- &50c_2       &=  &0  \\
            &  &(175/4)c_2  &=  &0  
   \end{linsys}
\end{equation*}
Both \( c_1 \) and \( c_2 \) are zero. 
So the only linear relationship between the two given row vectors
is the trivial relationship.

In the same vector space, the set
\( \set{ \rowvec{40 &15},\rowvec{20 &7.5}} \) is linearly dependent since
we can satisfy
% \begin{equation*}
$c_1\cdot \rowvec{40 &15}+c_2\cdot\rowvec{20 &7.5}=\rowvec{0 &0}$
% \end{equation*}
with \( c_1=1 \) and \( c_2=-2 \).
\end{example}

\begin{example}
The set \( \set{1+x,1-x} \) is linearly independent in \( \polyspace_2 \), the
space of quadratic polynomials with real coefficients, because
\begin{equation*}
   0+0x+0x^2
   =
   c_1(1+x)+c_2(1-x)
   =
   (c_1+c_2)+(c_1-c_2)x+0x^2
\end{equation*}
gives 
\begin{equation*}
  \begin{linsys}{2}
    c_1  &+  &c_2  &=  &0  \\
    c_1  &-  &c_2  &=  &0  
   \end{linsys}
  \grstep{-\rho_1+\rho_2}
  \begin{linsys}{2}
     c_1  &+  &c_2  &=  &0  \\
          &   &2c_2 &=  &0
  \end{linsys}
\end{equation*}
since polynomials are equal only if their coefficients are equal.
Thus, the only linear relationship between these two members of
$\polyspace_2$ is the trivial one.
\end{example}

\begin{remark}
The lemma specifies that $\vec{s}_i\neq\vec{s}_j$ 
when $i\neq j$ because of course if some vector~$\vec{s}$ appears 
twice then we can get a nontrivial
$c_1\vec{s}_1+\dots+c_n\vec{s}_n=\zero$ by taking
the associated coefficients to be $1$ and~$-1$.
However, note that the lemma allows the 
opposite of appearing more than once, that some 
vectors from~$S$ don't appear at all.
For instance, if~$S$ is infinite then because
linear relationships involve only finitely many vectors, 
any such relationship leaves out many of $S$'s vectors.

Note also that if~$S$ is finite then in the lemma 
we can take the combination
$c_1\vec{s}_1+\dots+c_n\vec{s}_n$ to contain each of $S$'s vectors once and
only once.
If a vector is missing from a combination then we can add it by using a 
coefficient of~$0$.
\end{remark}

\begin{example}
The rows of this matrix
\begin{equation*}
  A=
  \begin{mat}[r]
    2  &3  &1  &0  \\
    0  &-1 &0  &-2 \\
    0  &0  &0  &1
  \end{mat}
\end{equation*}
form a linearly independent set.
This is easy to check for this case but also 
recall that Lemma~One.III.\ref{le:EchFormNoLinCombo}
shows that the rows of any echelon form matrix form
a linearly independent set.
\end{example}

\begin{example}
In \( \Re^3 \), where
\begin{equation*}
   \vec{v}_1=\colvec{3 \\ 4 \\ 5}
   \quad
   \vec{v}_2=\colvec{2 \\ 9 \\ 2}
   \quad
   \vec{v}_3=\colvec{4 \\ 18 \\ 4}
\end{equation*}
the set \( S=\set{\vec{v}_1,\vec{v}_2,\vec{v}_3} \)
is linearly dependent because this is a relationship
\begin{equation*}
  0\cdot\vec{v}_1
  +2\cdot\vec{v}_2
  -1\cdot\vec{v}_3
  =\zero
\end{equation*}
where not all of the scalars are zero 
(the fact that some 
of the scalars are zero doesn't matter).
\end{example}

% \begin{remark}  \label{rem:WhyLIIsNonTrivLinRel}
That example illustrates why, 
although \nearbydefinition{def:LinInd} is a clearer
statement of what independence means,
\nearbylemma{le:LDIffANonTrivLinRel} is better for
computations.
Working straight from the definition, someone trying to compute whether $S$
is linearly independent would start by setting
\( \vec{v}_1=c_2\vec{v}_2+c_3\vec{v}_3 \)
and concluding that there are no such $c_2$ and $c_3$.
But knowing that the first vector is not
dependent on the other two is not enough.
This person would have to go on to try
\( \vec{v}_2=c_1\vec{v}_1+c_3\vec{v}_3 \), in order  
to find the dependence $c_1=0$, \( c_3=1/2 \).
\nearbylemma{le:LDIffANonTrivLinRel} 
gets the same conclusion with only one computation.
% \end{remark}

\begin{example} \label{ex:EmSetLI}
The empty subset\index{sets!empty} of a vector space is linearly independent.
There is no nontrivial linear relationship among its members as it has
no members.
\end{example}

\begin{example} \label{ex:SetWithZeroVecLD}
In any vector space, any subset containing the zero vector is linearly 
dependent.
One example is, in the space $\polyspace_2$ of quadratic polynomials, 
the subset $\set{1+x,x+x^2,0}$.
It is linearly 
dependent because 
$0\cdot\vec{v}_1+0\cdot\vec{v}_2+1\cdot\zero=\zero$ is a nontrivial
relationship, since not all of the coefficients are zero.

A subtler way to see that this subset is dependent is to remember that
the zero vector is equal to the trivial sum, 
the sum of the empty set.
So any set containing the zero vector has an element that
is a combination of a subset of other vectors from the 
set, specifically,
the zero vector is a combination of the empty subset.
\end{example}

\begin{remark} \label{rem:WhyLIUsesMultiset} %\cite{Velleman} 
The definition of linear independence,
\nearbydefinition{def:LinInd}, refers to a `set' of vectors.
Sets are the most familiar kind of collection and
in practice everyone refers to these collections as sets.
But to be complete we will note that sets are not quite the right kind
of collection for this purpose.

Recall that a set is a collection with two properties: 
(i)~order does not matter,
so that the set $\set{1,2}$ equals the set $\set{2,1}$, and 
(ii)~duplicates collapse, so that the set $\set{1,1,2}$ equals the set
$\set{1,2}$.

Now consider this matrix reduction.
\begin{equation*}
  \begin{mat}
    1 &1 &1 \\
    2 &2 &2 \\
    1 &2 &3
  \end{mat}
  \grstep{(1/2)\rho_2}
  \begin{mat}
    1 &1 &1 \\
    1 &1 &1 \\
    1 &2 &3
  \end{mat}
\end{equation*}
On the left the set of matrix rows 
$\set{\rowvec{1 &1 &1}, \rowvec{2 &2 &2}, \rowvec{1 &2 &3}}$ is
linearly dependent.
On the right the set of rows is  
$\set{\rowvec{1 &1 &1}, \rowvec{1 &1 &1}, \rowvec{1 &2 &3}}$. 
Because duplicates collapse, that equals
the set
$\set{\rowvec{1 &1 &1}, \rowvec{1 &2 &3}}$, 
which is linearly independent.
That's a problem because Gauss's Method should preserve linear dependence.

That is, strictly speaking,
we need a type of collection where duplicates do not collapse.
A collection where order does not matter and duplicates don't collapse
is a \definend{multiset}.\index{sets!multiset}%
\index{multiset}\index{linear independence!multiset}

However, the standard terminology here is `set'
and departing from a standard has its own pitfalls, so we will 
use that word also. 
Later we shall occasionally need to take combinations without
letting duplicates collapse and we shall do that without 
further comment.
\end{remark}

\begin{corollary} \label{cor:LiSetMinSpanSet}
%<*cor:LiSetMinSpanSet>
A set $S$ is linearly independent if and only if
for any $\vec{v}\in S$, its removal shrinks the span
$\spanof{S-\set{v}}\subsetneq\spanof{S}$.
%</cor:LiSetMinSpanSet>
\end{corollary}

\begin{proof}
%<*pf:LiSetMinSpanSet>
This follows from \nearbycorollary{cor:OmitVecNotShrinkSpanIffVecIsDep}.
If $S$ is linearly independent then none of its vectors is dependent on 
the other elements, so removal of any vector will shrink the span.
If $S$ is not linearly independent then it contains a vector that is 
dependent on other elements of the set, and removal of that vector
will not shrink the span.
%</pf:LiSetMinSpanSet>
\end{proof}

So a spanning set is minimal if and only if it is linearly independent.

The prior result addresses removing elements from a linearly independent set.
The next one adds elements.

\begin{lemma}  \label{lm:AddVecLiSetIsLiIffVecNotInSpan}
%<*lm:AddVecLiSetIsLiIffVecNotInSpan>
Suppose that $S$ is linearly independent and that $\vec{v}\notin S$.
Then
the set $S\union\set{\vec{v}}$ is linearly independent if and only if 
$\vec{v}\notin\spanof{S}$.   
%</lm:AddVecLiSetIsLiIffVecNotInSpan>
\end{lemma}

\begin{proof}
%<*pf:AddVecLiSetIsLiIffVecNotInSpan0>
% Assume that $S$ is linearly independent and that $\vec{v}\notin S$.
We will show that $S\union\set{\vec{v}}$ is not linearly independent 
if and only if $\vec{v}\in\spanof{S}$.

Suppose first that $v\in\spanof{S}$.
Express $\vec{v}$ as a combination $\vec{v}=c_1\vec{s}_1+\cdots+c_n\vec{s}_n$.
Rewrite that $\zero=c_1\vec{s}_1+\cdots+c_n\vec{s}_n-1\cdot\vec{v}$.
Since $v\notin S$, it does not equal any of the $\vec{s}_i$ so this is 
a nontrivial linear dependence among the elements of $S\union\set{\vec{v}}$.
Thus that set is not linearly independent.
%</pf:AddVecLiSetIsLiIffVecNotInSpan0>

%<*pf:AddVecLiSetIsLiIffVecNotInSpan1>
Now suppose that $S\union\set{\vec{v}}$ is not linearly independent and
consider a nontrivial dependence among its members
$\zero=c_1\vec{s}_1+\cdots+c_n\vec{s}_n+c_{n+1}\cdot\vec{v}$.
If $c_{n+1}=0$ then that is a dependence among the elements of~$S$, but 
we are assuming that~$S$ is independent, so $c_{n+1}\neq 0$.
Rewrite the equation  
as $\vec{v}=(c_1/c_{n+1})\vec{s}_1+\cdots+(c_n/c_{n+1})\vec{s}_n$
to get $\vec{v}\in\spanof{S}$
%</pf:AddVecLiSetIsLiIffVecNotInSpan1>
\end{proof}



\begin{example} \label{ex:LinindSetsAndSuper}
This subset of $\Re^3$ is linearly independent.
\begin{equation*}
  S
   =\set{\colvec{1 \\ 0 \\ 0}}
\end{equation*}
The span of $S$ is the $x$-axis.
Here are two supersets, one that is linearly dependent and the other
independent.
\begin{center}
     dependent:
     \( \set{
         \colvec{1 \\ 0 \\ 0},
         \colvec{-3 \\ 0 \\ 0}} \)      
     \qquad
     independent:
     \( \set{
         \colvec{1 \\ 0 \\ 0},
         \colvec{0 \\ 1 \\ 0}} \)      
\end{center}
We got the
dependent superset by adding a vector from the $x$-axis
and so the span did not grow.
We got the independent superset by adding a vector 
that isn't in $\spanof{S}$, because it has a nonzero $y$~component,
causing the span to grow.

For the independent set
\begin{equation*}
  S
   =\set{\colvec{1 \\ 0 \\ 0},
          \colvec{0 \\ 1 \\ 0}
                 }
\end{equation*}
the span \( \spanof{S} \) is the \( xy \)-plane.
Here are two supersets.
\begin{center}
     dependent:
     \( \set{
         \colvec{1 \\ 0 \\ 0},
         \colvec{0 \\ 1 \\ 0},
         \colvec{3 \\ -2 \\ 0} } \)       
     \qquad
     independent:
     \( \set{
         \colvec{1 \\ 0 \\ 0},
         \colvec{0 \\ 1 \\ 0},
         \colvec{0 \\ 0 \\ 1} } \)    
\end{center}
As above, the additional member of the 
dependent superset comes from $\spanof{S}$, the $xy$-plane, while the 
added member of the
independent superset comes from outside of that span.

Finally, consider this independent set 
\begin{equation*}
   S =\set{\colvec{1 \\ 0 \\ 0},
          \colvec{0 \\ 1 \\ 0},
          \colvec{0 \\ 0 \\ 1}        } 
\end{equation*}
with \( \spanof{S}=\Re^3 \).  
We can get a linearly dependent superset.
\begin{center}
     dependent:
     \( \set{
         \colvec{1 \\ 0 \\ 0},
         \colvec{0 \\ 1 \\ 0},
         \colvec{0 \\ 0 \\ 1},
         \colvec{2 \\ -1 \\ 3} } \)     
\end{center}
But there is no linearly independent superset of $S$.
One way to see that is to note that
for any vector that we would add to $S$, the equation
\begin{equation*}
  \colvec{x \\ y \\ z}
  =c_1\colvec{1 \\ 0 \\ 0}
   +c_2\colvec{0 \\ 1 \\ 0}
   +c_3\colvec{0 \\ 0 \\ 1}
\end{equation*}
has a solution $c_1=x$, $c_2=y$, and $c_3=z$.
Another way to see it is that
we cannot add any vectors from outside of the span $\spanof{S}$ because that
span is $\Re^3$.
\end{example}

\begin{corollary} \label{th:AlwaysAnLDSubset}
%<*th:AlwaysAnLDSubset>
In a vector space,
any finite set has a linearly independent subset with the same span.
%</th:AlwaysAnLDSubset>
\end{corollary}

\begin{proof}
%<*pf:AlwaysAnLDSubset0>
If \( S=\set{ \vec{s}_1,\dots,\vec{s}_n} \) is linearly independent
then $S$ itself satisfies the statement, so 
assume that it is linearly dependent.
%</pf:AlwaysAnLDSubset0>

%<*pf:AlwaysAnLDSubset1>
By the definition of dependent, $S$ contains
a vector $\vec{v}_1$ that is a linear combination of
the others.
Define the set \( S_1=S-\set{\vec{v}_1} \).
By \nearbycorollary{cor:OmitVecNotShrinkSpanIffVecIsDep} % {lm:ShrinkSpanByRemovingNonRepeat} 
the span does not shrink \( \spanof{S_1}=\spanof{S} \).
%</pf:AlwaysAnLDSubset1>

%<*pf:AlwaysAnLDSubset2>
If \( S_1 \) is linearly independent then we are done.
Otherwise iterate: 
take a vector $\vec{v}_2$ 
that is a linear combination of
other members of $S_1$ and discard it
to derive \( S_2=S_1-\set{\vec{v}_2} \)
such that \( \spanof{S_2}=\spanof{S_1} \).
Repeat this until a linearly independent set $S_j$ appears;
one must appear eventually because \( S \) is finite
and the empty set is linearly independent.
%</pf:AlwaysAnLDSubset2>
(Formally, this argument uses
induction on the number of elements in $S$.
\nearbyexercise{exer:FillIndDetProofSetHasLISub} asks for the details.)
\end{proof}

Thus if we have a set that is linearly dependent then we can, without changing
the span, pare down by   
discarding what we have called ``repeat'' vectors.

\begin{example}  \label{ex:ShrinkSetSameSpan}
This set spans \( \Re^3 \) (the check is routine)
but is not linearly independent. 
\begin{equation*}
  S=\set{\colvec[r]{1 \\ 0 \\ 0},
     \colvec[r]{0 \\ 2 \\ 0},
     \colvec[r]{1 \\ 2 \\ 0},
     \colvec[r]{0 \\ -1 \\ 1},
     \colvec[r]{3 \\ 3 \\ 0}  }
\end{equation*}
We will calculate which vectors to drop in order
to get a subset that is independent but
has the same span.
This linear relationship
\begin{equation*}
  c_1\colvec[r]{1 \\ 0 \\ 0}
  +c_2\colvec[r]{0 \\ 2 \\ 0}
  +c_3\colvec[r]{1 \\ 2 \\ 0}
  +c_4\colvec[r]{0 \\ -1 \\ 1}
  +c_5\colvec[r]{3 \\ 3 \\ 0}
  =\colvec[r]{0 \\ 0 \\ 0}
  \qquad\tag{$*$}
\end{equation*}
gives a system 
\begin{equation*}
  \begin{linsys}{5}
     c_1  &   &      &+  &c_3   &+  &    &+  &3c_5 &= &0  \\
          &   &2c_2  &+  &2c_3  &-  &c_4 &+  &3c_5 &= &0  \\
          &   &      &   &     &   &c_4  &   &     &= &0  
\end{linsys}
\end{equation*}
whose solution set has this parametrization.
\begin{equation*}
  \set{\colvec{c_1 \\ c_2 \\ c_3 \\ c_4 \\ c_5}=
     c_3\colvec[r]{-1 \\ -1 \\ 1 \\ 0 \\ 0}
     +c_5\colvec[r]{-3 \\ -3/2 \\ 0 \\ 0 \\ 1}
     \suchthat c_3,c_5\in\Re }
\end{equation*}
Set $c_5=1$ and~$c_3=0$ 
to get an instance of ($*$).
\begin{equation*}
  -3\cdot\colvec[r]{1 \\ 0 \\ 0}
  -\frac{3}{2}\cdot\colvec[r]{0 \\ 2 \\ 0}
  +0\cdot\colvec[r]{1 \\ 2 \\ 0}
  +0\cdot\colvec[r]{0 \\ -1 \\ 1}
  +1\cdot\colvec[r]{3 \\ 3 \\ 0}
  =\colvec[r]{0 \\ 0 \\ 0}
\end{equation*}
This shows that the vector from $S$ that we've 
associated with~$c_5$
is in the span of the set of $c_1$'s vector and $c_2$'s vector.
We can discard $S$'s fifth vector without shrinking the span.

Similarly, set $c_3=1$, and $c_5=0$
to get an instance of~($*$) that
shows we can discard $S$'s third vector without shrinking the span. 
 Thus this set has the same span as $S$.
\begin{equation*}
  % \hat{S}=
     \set{\colvec[r]{1 \\ 0 \\ 0},
     \colvec[r]{0 \\ 2 \\ 0},
     \colvec[r]{0 \\ -1 \\ 1}  }
\end{equation*}
The check that it is linearly independent is routine.
\end{example}

\begin{corollary} \label{cor:LDMeansLC}
%<*co:LDMeansLC>
A subset \( S=\set{\vec{s}_1,\dots,\vec{s}_n} \) of a vector space
is linearly dependent if and only if some \( \vec{s_i} \)
is a linear combination of the vectors 
\( \vec{s}_1 \), \ldots, \( \vec{s}_{i-1} \)
listed before it.
%</co:LDMeansLC>
\end{corollary}

\begin{proof}
%<*pf:LDMeansLC>
Consider \( S_0=\set{} \), \( S_1=\set{\vec{s_1}} \),
\( S_2=\set{\vec{s}_1,\vec{s}_2 } \), etc.
Some index \( i\geq 1 \) is the first one with
\( S_{i-1}\union\set{\vec{s}_i } \)
linearly dependent, and there \( \vec{s}_i\in\spanof{ S_{i-1} } \).
%</pf:LDMeansLC>
\end{proof}

The proof of 
\nearbycorollary{th:AlwaysAnLDSubset} describes producing a linearly 
independent set by shrinking, by taking subsets.
And the proof of \nearbycorollary{cor:LDMeansLC} describes finding a 
linearly dependent set by taking supersets. 
We finish this subsection by considering
how linear independence and dependence interact
with the subset relation between sets.

\begin{lemma}  \label{le:SubsetPreserveLI}
%<*lm:SubsetPreserveLI>
Any subset of a linearly independent set is also linearly independent.
Any superset of a linearly dependent set is also linearly dependent.
%</lm:SubsetPreserveLI>
\end{lemma}

\begin{proof}
%<*pf:SubsetPreserveLI>
Both are clear.
%</pf:SubsetPreserveLI>
\end{proof}

%<*SubsetPreserveLI>
Restated, subset preserves independence  
and superset preserves dependence.
%</SubsetPreserveLI>

Those are two of the four possible cases.
The third case, whether subset preserves linear dependence,
is covered by \nearbyexample{ex:ShrinkSetSameSpan}, which gives
a linearly dependent set $S$ with one subset 
that is linearly dependent and another 
that is independent.
The fourth case, whether superset preserves linear independence,
is covered by \nearbyexample{ex:LinindSetsAndSuper}, which gives cases
where a linearly independent set has both an independent
and a dependent superset.
This table summarizes.
\smallskip
%<*IndependenceAndSubsetTable>
\begin{center} \renewcommand{\arraystretch}{1.1}
  \begin{tabular}[b]{r|cc} 
                        \multicolumn{1}{c}{}
                        &\multicolumn{1}{c}{\( \hat{S}\subset S \)}
                        &\multicolumn{1}{c}{\( \hat{S}\supset S \)}      \\
     \cline{2-3}\rule{0em}{12pt} % make box a bit taller
          \textit{$S$ independent} 
              &\( \hat{S} \) must be independent   &\( \hat{S} \) may be either\\
     % \cline{2-3}\noalign{\smallskip}
          \textit{$S$ dependent} 
              &\( \hat{S} \) may be either &\( \hat{S} \) must be dependent    \\
     % \cline{2-3}
   \end{tabular}
\end{center}
%</IndependenceAndSubsetTable>
\smallskip

\nearbyexample{ex:LinindSetsAndSuper} has something else to say about the
interaction between linear independence and superset.
It names a
linearly independent set that is maximal in
that it has no supersets that are linearly independent.
By \nearbylemma{lm:AddVecLiSetIsLiIffVecNotInSpan} 
a linearly independent set is maximal if and only if it
spans the 
entire space, because that is when 
all the vectors in the space are already in the span.
This nicely
complements \nearbylemma{cor:LiSetMinSpanSet}, that
a spanning set is minimal if and only if it is linearly independent.

% In summary,
% we have introduced the definition of linear independence to 
% formalize the idea of the minimality of a spanning set.
% We have developed some properties of this idea.
% The most important is \nearbylemma{lm:AddVecLiSetIsLiIffVecNotInSpan}, which 
% tells us that a linearly independent set is maximal when it spans the space.


\begin{exercises}
  \recommended \item
    Decide whether each subset of \( \Re^3 \) is linearly dependent or
    linearly independent. 
    \begin{exparts*}
      \partsitem \( \set{\colvec{1 \\ -3 \\ 5},
                    \colvec{2 \\ 2 \\ 4},
                    \colvec{4 \\ -4 \\ 14} }  \)
      \partsitem \( \set{\colvec{1 \\ 7 \\ 7},
                    \colvec{2 \\ 7 \\ 7},
                    \colvec{3 \\ 7 \\ 7} }  \)
      \partsitem \( \set{\colvec{0 \\ 0 \\ -1},
                    \colvec{1 \\ 0 \\ 4} }  \)
      \partsitem \( \set{\colvec{9 \\ 9 \\ 0},
                    \colvec{2 \\ 0 \\ 1},
                    \colvec{3 \\ 5 \\ -4},
                    \colvec{12 \\ 12 \\ -1} }  \)
    \end{exparts*}
    \begin{answer}
      For each of these, when the subset is independent you must prove it, and
      when the subset is dependent you must give an example of a dependence.
      \begin{exparts}
        \partsitem It is dependent.
          Considering
          \begin{equation*}
             c_1\colvec{1 \\ -3 \\ 5}
             +c_2\colvec{2 \\ 2 \\ 4}
             +c_3\colvec{4 \\ -4 \\ 14}
             =\colvec{0 \\ 0 \\ 0}
          \end{equation*}
          gives this linear system.
          \begin{equation*}
            \begin{linsys}{3}
              c_1  &+  &2c_2  &+  &4c_3  &=  &0  \\
              -3c_1&+  &2c_2  &-  &4c_3  &=  &0  \\
              5c_1 &+  &4c_2  &+  &14c_3 &=  &0  
            \end{linsys}
          \end{equation*}
          Gauss's Method 
          \begin{equation*}
            \begin{amat}[r]{3}
              1  &2  &4  &0  \\
              -3 &2  &-4 &0  \\
              5  &4  &14 &0
            \end{amat}
            \grstep[-5\rho_1+\rho_3]{3\rho_1+\rho_2}
            \repeatedgrstep{(3/4)\rho_2+\rho_3}            
            \begin{amat}[r]{3}
              1  &2  &4  &0  \\
              0  &8  &8  &0  \\
              0  &0  &0  &0
            \end{amat}
          \end{equation*}
          yields a free variable, so there are infinitely many solutions.
          For an example of a particular dependence we can set $c_3$ to be,
          say, $1$.  Then we get
          \( c_2=-1 \) and \( c_1=-2 \).
        \partsitem It is dependent.
          The linear system that arises here
          \begin{equation*}
            \begin{amat}[r]{3}
              1  &2  &3  &0  \\
              7  &7  &7  &0  \\
              7  &7  &7  &0
            \end{amat}
            \grstep[-7\rho_1+\rho_3]{-7\rho_1+\rho_2}
            \repeatedgrstep{-\rho_2+\rho_3}
            \begin{amat}[r]{3}
              1  &2  &3   &0  \\
              0  &-7 &-14 &0  \\
              0  &0  &0   &0
            \end{amat}
          \end{equation*}
          has infinitely many solutions.
          We can get a particular solution by taking $c_3$ to be, say,
          $1$, and back-substituting to get the resulting $c_2$ and $c_1$.
        \partsitem It is linearly independent.
          The system
          \begin{equation*}
            \begin{amat}[r]{2}
              0  &1  &0  \\
              0  &0  &0  \\
              -1 &4  &0
            \end{amat}
            \grstep{\rho_1\leftrightarrow\rho_2}
            \repeatedgrstep{\rho_3\leftrightarrow\rho_1}
            \begin{amat}{2}
              -1 &4  &0  \\
              0  &1  &0  \\
              0  &0  &0  
            \end{amat}
          \end{equation*}
          has only the solution $c_1=0$ and $c_2=0$.
          (We could also have gotten the answer by inspection\Dash the second
          vector is obviously not a multiple of the first, and vice versa.)
        \partsitem It is linearly dependent.
          The linear system
          \begin{equation*}
            \begin{amat}[r]{4}
              9  &2  &3  &12  &0  \\
              9  &0  &5  &12  &0  \\
              0  &1  &-4 &-1  &0
            \end{amat}
          \end{equation*}
          has more unknowns than equations, and so Gauss's Method
          must end with at least one variable free (there can't be a 
          contradictory equation because the system is homogeneous, and so
          has at least the solution of all zeroes).
          To exhibit a combination, we can do the reduction 
          \begin{equation*}
            \grstep{-\rho_1+\rho_2}
            \grstep{(1/2)\rho_2+\rho_3}
            \begin{amat}[r]{4}
              9  &2  &3  &12  &0  \\
              0  &-2 &2  &0   &0  \\
              0  &0  &-3 &-1  &0
            \end{amat}
          \end{equation*}
          and take, say,  $c_4=1$.
          Then we have that $c_3=-1/3$, $c_2=-1/3$, and $c_1=-31/27$.
      \end{exparts}  
    \end{answer}
  \recommended \item 
    Which of these subsets of \( \polyspace_3 \) are
    linearly dependent and which are independent?
    \begin{exparts}
      \partsitem \( \set{3-x+9x^2,5-6x+3x^2,1+1x-5x^2} \)
      \partsitem \( \set{-x^2,1+4x^2} \)
      \partsitem \( \set{2+x+7x^2,3-x+2x^2,4-3x^2} \)
      \partsitem \( \set{8+3x+3x^2,x+2x^2,2+2x+2x^2,8-2x+5x^2} \)
    \end{exparts}
    \begin{answer}
      In the cases of independence, you must prove that it is independent.
      Otherwise, you must exhibit a dependence.
      (Here we give a specific dependence but others are possible.)
      \begin{exparts}
        \partsitem This set is independent.
          Setting up the relation
          \( c_1(3-x+9x^2)+c_2(5-6x+3x^2)+c_3(1+1x-5x^2)=0+0x+0x^2 \)
          gives a linear system 
          \begin{equation*}
            \begin{amat}[r]{3}
              3  &5  &1  &0  \\
              -1 &-6 &1  &0  \\
              9  &3  &-5 &0  
            \end{amat}
            \grstep[-3\rho_1+\rho_3]{(1/3)\rho_1+\rho_2}
            \repeatedgrstep{3\rho_2}
            \repeatedgrstep{-(12/13)\rho_2+\rho_3}
            \begin{amat}[r]{3}
              3  &5   &1        &0  \\
              0  &-13 &4        &0  \\
              0  &0   &-128/13  &0  
            \end{amat}
          \end{equation*}
          with only one solution: \( c_1=0 \), \( c_2=0 \), and \( c_3=0 \).
        \partsitem This set is independent.
           We can see this by inspection, straight from the definition
           of linear independence.
           Obviously neither is a multiple of the other.
        \partsitem This set is linearly independent.
           The linear system reduces in this way
           \begin{equation*}
             \begin{amat}[r]{3}
               2  &3  &4  &0  \\
               1  &-1 &0  &0  \\
               7  &2  &-3 &0  
             \end{amat}
             \grstep[-(7/2)\rho_1+\rho_3]{-(1/2)\rho_1+\rho_2}
             \repeatedgrstep{-(17/5)\rho_2+\rho_3}
             \begin{amat}[r]{3}
               2  &3    &4      &0  \\
               0  &-5/2 &-2     &0  \\
               0  &0    &-51/5  &0  
             \end{amat}
           \end{equation*}
           to show that there is only the solution $c_1=0$, 
           $c_2=0$, and $c_3=0$.
        \partsitem This set is linearly dependent.
           The linear system
           \begin{equation*}
             \begin{amat}[r]{4}
               8  &0  &2  &8  &0  \\ 
               3  &1  &2  &-2 &0  \\
               3  &2  &2  &5  &0
             \end{amat}
           \end{equation*}
           must, after reduction, end with at least one variable free
           (there are more variables than equations, and there is no
           possibility of a contradictory equation because the system is
           homogeneous).
           We can take the free variables as parameters to describe the
           solution set.
           We can then set the parameter to a nonzero value to get a
           nontrivial linear relation. 
      \end{exparts}  
     \end{answer}
  \item Determine if each set is linearly independent in the natural 
    space.
    \begin{exparts*}
      \partsitem $\set{\colvec{1 \\ 2 \\ 0}, 
              \colvec{-1 \\ 1 \\ 0}}$
      \partsitem $\set{\rowvec{1 &3 &1}, \rowvec{-1 &4 &3}, \rowvec{-1 &11 &7}}$
      \partsitem $\set{\begin{mat}
                5 &4 \\
                1 &2 
              \end{mat},
              \begin{mat}
                0 &0 \\
                0 &0
              \end{mat},
              \begin{mat}
                1 &0 \\
               -1 &4
              \end{mat}
           }$
    \end{exparts*}
    \begin{answer}
      \begin{exparts}
        \partsitem   The natural vector space is $\Re^3$.
          Set up the equation
          \begin{equation*}
            \colvec{0 \\ 0 \\ 0}=c_1\colvec{1 \\ 2 \\ 0}
                                 +c_2\colvec{-1 \\ 1 \\ 0}
          \end{equation*}
          and consider the resulting homogeneous system.
          \begin{equation*}
             \begin{amat}{2}
               1  &-1 &0  \\
               2  &1  &0  \\
               0  &0  &0
             \end{amat}
             \grstep{-2\rho_1+\rho_2}
             \begin{amat}{2}
              1  &-1 &0  \\
              0  &3  &0  \\
              0  &0  &0
            \end{amat} 
          \end{equation*}
          This has the unique solution, that $c_1=0$, $c_2=0$.
          So it is linearly independent.
        \partsitem   The natural vector space is the set of three-wide 
          row vectors.
          The equation
          \begin{equation*}
            \rowvec{0 &0 &0}=c_1\rowvec{1 &3 &1}
                            +c_2\rowvec{-1 &4 &3}
                            +c_3\rowvec{-1 &11 &7}
          \end{equation*}
          gives rise to a linear system
          \begin{align*}
            \begin{amat}{3}
              1  &-1  &-1  &0  \\
              3  &4   &11  &0  \\
              1  &3   &7   &0
           \end{amat}
           &\grstep[-\rho_1+\rho_3]{-3\rho_1+\rho_2}
           \begin{amat}{3}
             1  &-1  &-1  &0  \\
             0  &7   &14  &0  \\
             0  &4   &8   &0
          \end{amat}                                   \\
          &\grstep{(4/7)\rho_2+\rho_3}
          \begin{amat}{3}
            1  &-1  &-1  &0  \\
            0  &7   &14  &0  \\
            0  &0   &0   &0
         \end{amat}
       \end{align*}
       with infinitely many solutions, that is, more than just the 
       trivial solution.
       \begin{equation*}
         \set{\colvec{c_1 \\ c_2 \\ c_3}
              =\colvec{-1 \\ -2 \\ 1}c_3
              \suchthat c_3\in\Re}
       \end{equation*}
       So the set is linearly dependent.
       One dependence comes from setting $c_3=2$, giving 
       $c_1=-2$ and $c_2=-4$.
      \partsitem   Without having to set up a system we can see 
        that the second element of the set is a multiple of the first 
        (namely, $0$ times the first). 
      \end{exparts}
    \end{answer}
  \recommended \item
    Prove that each set \( \set{f,g} \) is linearly independent in the
    vector space of all functions from \( \Re^+ \) to \( \Re \).
    \begin{exparts}
      \partsitem \( f(x)=x \) and \( g(x)=1/x \)
      \partsitem \( f(x)=\cos(x) \) and \( g(x)=\sin(x) \)
      \partsitem \( f(x)=e^x \) and \( g(x)=\ln(x) \)
    \end{exparts}
    \begin{answer}
      Let $Z$ be the zero function $Z(x)=0$, which is the additive identity in
      the vector space under discussion.
      \begin{exparts}
        \partsitem This set is linearly independent.  
          Consider \( c_1\cdot f(x)+c_2\cdot g(x)=Z(x) \).
          Plugging in \( x=1 \) and \( x=2 \) gives a linear system 
          \begin{equation*}
            \begin{linsys}{2}
              c_1\cdot 1  &+  &c_2\cdot 1     &=  &0  \\
              c_1\cdot 2  &+  &c_2\cdot (1/2) &=  &0
            \end{linsys}
          \end{equation*}
          with the unique solution \( c_1=0 \), \( c_2=0 \).
        \partsitem This set is linearly independent.  
          Consider \( c_1\cdot f(x)+c_2\cdot g(x)=Z(x) \) and 
          plug in \( x=0 \) and \( x=\pi/2 \) to get 
          \begin{equation*}
            \begin{linsys}{2}
              c_1\cdot 1  &+  &c_2\cdot 0     &=  &0  \\
              c_1\cdot 0  &+  &c_2\cdot 1     &=  &0
            \end{linsys}
          \end{equation*}
          which obviously gives that \( c_1=0 \), \( c_2=0 \).
        \partsitem This set is also linearly independent.  
          Considering \( c_1\cdot f(x)+c_2\cdot g(x)=Z(x) \) and 
          plugging in \( x=1 \) and \( x=e \) 
          \begin{equation*}
            \begin{linsys}{2}
              c_1\cdot e    &+  &c_2\cdot 0     &=  &0  \\
              c_1\cdot e^e  &+  &c_2\cdot 1     &=  &0
            \end{linsys}
          \end{equation*}
          gives that \( c_1=0 \) and \( c_2=0 \).
      \end{exparts}  
     \end{answer}
  \recommended \item 
    Which of these subsets of the space of real-valued functions
    of one real variable is linearly dependent and which is linearly
    independent?
    (We have abbreviated some constant functions;~e.g., 
     in the first item, 
     the `$2$' stands for the constant function $f(x)=2$.)
    \begin{exparts*}
      \partsitem \( \set{2,4\sin^2(x),\cos^2(x)} \)
      \partsitem \( \set{1,\sin(x),\sin(2x)} \)
      \partsitem \( \set{x,\cos(x)} \)
      \partsitem \( \set{(1+x)^2,x^2+2x,3} \)
      \partsitem \( \set{\cos(2x),\sin^2(x),\cos^2(x)} \)
      \partsitem \( \set{0,x,x^2} \)
    \end{exparts*}
    \begin{answer}
      In each case, if the set is independent then you must prove that 
      and if it is
      dependent then you must exhibit a dependence.
      \begin{exparts}
        \partsitem This set is dependent.
          The familiar relation $\sin^2(x)+\cos^2(x)=1$ shows that
          $2=c_1\cdot(4\sin^2(x))+c_2\cdot(\cos^2(x))$ is satisfied by
          $c_1=1/2$ and $c_2=2$.
        \partsitem This set is independent.
          Consider the relationship
          $c_1\cdot 1+c_2\cdot\sin(x)+c_3\cdot\sin(2x)=0$
          (that `$0$' is the zero function).
          Taking three suitable points such as $x=\pi$, $x=\pi/2$, $x=\pi/4$ 
          gives a system
          \begin{equation*}
            \begin{linsys}{3}
               c_1  &   &                &   &      &=  &0  \\
               c_1  &+  &c_2             &   &      &=  &0  \\
               c_1  &+  &(\sqrt{2}/2)c_2 &+  &c_3   &=  &0  
            \end{linsys}
          \end{equation*}
          whose only solution is 
          $c_1=0$, $c_2=0$, and $c_3=0$. 
        \partsitem By inspection, this set is independent.
          Any dependence $\cos(x)=c\cdot x$ is not possible since the cosine
          function is not a multiple of the identity function
          (we are applying \nearbycorollary{cor:LDMeansLC}).
        \partsitem By inspection, we spot that there is a dependence.
          Because $(1+x)^2=x^2+2x+1$, we get that
          $c_1\cdot(1+x)^2+c_2\cdot(x^2+2x)=3$ is satisfied by 
          $c_1=3$ and $c_2=-3$.
        \partsitem This set is dependent.
          The easiest way to see that is to recall the trigonometric
          relationship $\cos^2(x)-\sin^2(x)=\cos(2x)$.
          (\textit{Remark.} 
          A person who doesn't recall this, and tries some $x$'s,
          simply never gets a system leading to a unique solution, and
          never gets to conclude that the set is independent.
          Of course, this person might wonder if they simply never tried the
          right set of $x$'s, but a few tries will lead most people to 
          look instead for a dependence.)
        \partsitem This set is dependent, because it contains the 
          zero object in the vector space, the zero polynomial.
      \end{exparts}  
     \end{answer}
  \item 
    Does the equation \( \sin^2(x)/\cos^2(x)=\tan^2(x) \) show that
    this set of functions
    \( \set{\sin^2(x),\cos^2(x),\tan^2(x)} \) is a linearly dependent
    subset of the set of all real-valued functions with domain
    the interval \( (-\pi/2..\pi/2) \) of real numbers between 
    \( -\pi/2 \) and \( \pi/2) \)?
    \begin{answer}
      No, that equation is not a linear relationship.
      In fact this set is independent, as the system arising from taking
      \( x \) to be \( 0 \), \( \pi/6 \) and \( \pi/4 \) shows.  
    \end{answer}
  \item  
    Is the $xy$-plane subset of the vector space $\Re^3$ linearly independent?
    \begin{answer}
      No.  
      Here are two members of the plane where the second is a multiple of the 
      first.
      \begin{equation*}
        \colvec{1  \\ 0 \\ 0},\,\colvec{2 \\ 0 \\ 0}
      \end{equation*}
      (Another reason that the answer is ``no'' is the the zero vector is a 
      member of the plane and no set containing the zero vector is linearly
      independent.)
    \end{answer}
  \recommended \item
    Show that the nonzero rows of an echelon form matrix form a linearly
    independent set.
    \begin{answer}
      We have already showed this: the Linear Combination
      Lemma and its corollary state that in an echelon form matrix, 
      no nonzero row is a linear combination of the others.  
    \end{answer}
  \item
     \begin{exparts}
       \partsitem Show that if the set \( \set{\vec{u},\vec{v},\vec{w}} \)
          is linearly independent then so is the set 
          \( \set{\vec{u},\vec{u}+\vec{v},\vec{u}+\vec{v}+\vec{w}} \).
       \partsitem  What is the relationship between the linear independence
         or dependence of \( \set{\vec{u},\vec{v},\vec{w}} \) and the
         independence or dependence of
         \( \set{\vec{u}-\vec{v},\vec{v}-\vec{w},\vec{w}-\vec{u}} \)?
     \end{exparts}
     \begin{answer}
       \begin{exparts}
         \partsitem Assume that 
           \( \set{\vec{u},\vec{v},\vec{w}} \) is linearly 
           independent, so that any relationship
           $d_0\vec{u}+d_1\vec{v}+d_2\vec{w}=\zero$ leads to the conclusion 
           that $d_0=0$, $d_1=0$, and $d_2=0$.

           Consider the relationship
           \( c_1(\vec{u})+c_2(\vec{u}+\vec{v})+c_3(\vec{u}+\vec{v}+\vec{w})
           =\zero \).
           Rewrite it to get
           \( (c_1+c_2+c_3)\vec{u}+(c_2+c_3)\vec{v}+(c_3)\vec{w}=\zero \).
           Taking $d_0$ to be $c_1+c_2+c_3$, taking $d_1$ to be $c_2+c_3$, 
           and taking $d_2$ to be $c_3$ we have this system.
           \begin{equation*}
             \begin{linsys}{3}
               c_1  &+  &c_2  &+  &c_3  &=  &0  \\
                    &   &c_2  &+  &c_3  &=  &0  \\
                    &   &     &   &c_3  &=  &0
             \end{linsys}
           \end{equation*}
           Conclusion:~the $c$'s are all zero, and so the set is linearly
           independent.
        \partsitem The second set is dependent
           \begin{equation*}
             1\cdot(\vec{u}-\vec{v})
             +1\cdot(\vec{v}-\vec{w})
             +1\cdot(\vec{w}-\vec{u})
             =\zero
           \end{equation*}
           whether or not the first set is independent.
      \end{exparts}
    \end{answer}
  \item 
    \nearbyexample{ex:EmSetLI} shows that the empty set is 
    linearly independent.   
    \begin{exparts}
      \partsitem When is a one-element set linearly independent?
      \partsitem How about a set with two elements?
    \end{exparts}
    \begin{answer}
      \begin{exparts}
        \partsitem A singleton set $\set{\vec{v}}$ is linearly independent 
          if and only if $\vec{v}\neq\zero$.
          For the `if' direction, with $\vec{v}\neq\zero$, 
          we can apply \nearbylemma{le:LDIffANonTrivLinRel} by considering 
          the relationship
          \( c\cdot\vec{v}=\zero \) and noting that the only solution
          is the trivial one:~$c=0$.
          For the `only~if' direction, just recall that 
          \nearbyexample{ex:SetWithZeroVecLD}
          shows that $\set{\zero}$ is linearly dependent, and so if the set
          $\set{\vec{v}}$ is linearly independent then $\vec{v}\neq\zero$. 

          (\textit{Remark.} 
          Another answer is to say that this is the special case of
          \nearbylemma{lm:AddVecLiSetIsLiIffVecNotInSpan} 
          where \( S=\emptyset \).)
        \partsitem A set with two elements is linearly independent 
          if and only if neither member is a  multiple of the other 
          (note that if one is the zero vector then it is a multiple of the
          other).
          This is an equivalent statement:~a set is linearly dependent if and
          only if one element is a multiple of the other.

          The proof is easy.
          A set $\set{\vec{v}_1,\vec{v}_2}$ is linearly dependent if and only
          if there is a relationship $c_1\vec{v}_1+c_2\vec{v}_2=\zero$ 
          with either $c_1\neq 0$ or $c_2\neq 0$ (or both).
          That holds if and only if $\vec{v}_1=(-c_2/c_1)\vec{v}_2$
          or $\vec{v}_2=(-c_1/c_2)\vec{v}_1$ (or both).
       \end{exparts}   
     \end{answer}
  \item  
    In any vector space \( V \), the empty set is linearly independent.
    What about all of \( V \)?
    \begin{answer}
      This set is linearly dependent set because it contains the zero vector.  
    \end{answer}
  \item 
    Show that if \( \set{\vec{x},\vec{y},\vec{z}} \) is linearly
    independent then so are all of its proper 
    subsets:~\( \set{\vec{x},\vec{y}} \),
    \( \set{\vec{x},\vec{z}} \), \( \set{\vec{y},\vec{z}} \),
    \( \set{\vec{x}} \),\( \set{\vec{y}} \), \( \set{\vec{z}} \),
    and \( \set{} \).
    Is that `only if' also?
    \begin{answer}
      \nearbylemma{le:SubsetPreserveLI} gives the `if' half.
      The converse (the `only if' statement) does not hold. 
      An example is to consider the vector space \( \Re^2 \) and
      these vectors.
      \begin{equation*}
         \vec{x}=\colvec{1 \\ 0},\quad
         \vec{y}=\colvec{0 \\ 1},\quad
         \vec{z}=\colvec{1 \\ 1}
      \end{equation*} 
    \end{answer}
  \item 
    \begin{exparts}
      \partsitem Show that this 
        \begin{equation*}
          S=\set{\colvec{1 \\ 1 \\ 0},\colvec{-1 \\ 2 \\ 0}}
        \end{equation*}
        is a linearly independent subset of \( \Re^3 \).
      \partsitem Show that
        \begin{equation*}
          \colvec{3 \\ 2 \\ 0}
        \end{equation*}
        is in the span of $S$ by finding \( c_1 \) and \( c_2 \) 
        giving a linear relationship.
        \begin{equation*}
          c_1\colvec{1 \\ 1 \\ 0}
          +c_2\colvec{-1 \\ 2 \\ 0}
          =\colvec{3 \\ 2 \\ 0}
        \end{equation*}
        Show that the pair \( c_1,c_2 \) is unique.
      \partsitem Assume that \( S \) is a subset of a vector space and that
        \( \vec{v} \) is in \( \spanof{S} \), so that \( \vec{v} \) is a
        linear combination of vectors from \( S \).
        Prove that if \( S \) is linearly independent then a linear combination
        of vectors from \( S \) adding to \( \vec{v} \)
        is unique (that is, unique up to reordering
        and adding or taking away terms of the form \( 0\cdot\vec{s} \)).
        Thus \( S \) as a spanning set is minimal in this strong sense:
        each vector in \( \spanof{S} \) is a combination of elements
        of $S$ a minimum number of
        times\Dash only once.
      \partsitem
        Prove that it can happen when \( S \) is not linearly 
        independent that distinct linear combinations sum to the same vector.
    \end{exparts}
    \begin{answer}
      \begin{exparts}
        \partsitem The linear system arising from
          \begin{equation*}
            c_1\colvec{1 \\ 1 \\ 0}
            +c_2\colvec{-1 \\ 2 \\ 0}
            =\colvec{0 \\ 0 \\ 0}
          \end{equation*}
          has the unique solution \( c_1=0 \) and \( c_2=0 \).
        \partsitem The linear system arising from
          \begin{equation*}
            c_1\colvec{1 \\ 1 \\ 0}
            +c_2\colvec{-1 \\ 2 \\ 0}
            =\colvec{3 \\ 2 \\ 0}
          \end{equation*}
          has the unique solution \( c_1=8/3 \) and \( c_2=-1/3 \).
        \partsitem Suppose that \( S \) is linearly independent.
          Suppose that we have both $\vec{v}=c_1\vec{s}_1+\dots+c_n\vec{s}_n$
          and $\vec{v}=d_1\vec{t}_1+\dots+d_m\vec{t}_m$
          (where the vectors are members of $S$).
          Now, 
          \begin{equation*}
            c_1\vec{s}_1+\dots+c_n\vec{s}_n
            =\vec{v}
            =d_1\vec{t}_1+\dots+d_m\vec{t}_m
          \end{equation*}
          can be rewritten in this way.
          \begin{equation*}
            c_1\vec{s}_1+\dots+c_n\vec{s}_n
            -d_1\vec{t}_1-\dots-d_m\vec{t}_m
            =\zero
          \end{equation*}
          Possibly some of the $\vec{s}\,$'s equal some of the $\vec{t}\,$'s;
          we can combine the associated coefficients 
          (i.e., if $\vec{s}_i=\vec{t}_j$ then
          $\cdots+c_i\vec{s}_i+\dots-d_j\vec{t}_j-\cdots$ can be rewritten
          as $\cdots+(c_i-d_j)\vec{s}_i+\cdots$).
          That equation is a linear relationship among  
          distinct (after the combining is done) members of the set $S$.
          We've assumed that $S$ is linearly independent, so all of the 
          coefficients are zero.
          If $i$ is such that $\vec{s}_i$ does not equal any $\vec{t}_j$
          then $c_i$ is zero.
          If $j$ is such that $\vec{t}_j$ does not equal any $\vec{s}_i$
          then $d_j$ is zero.
          In the final case, we have that $c_i-d_j=0$ and so $c_i=d_j$.  

          Therefore, the original two sums are the same, except perhaps for
          some $0\cdot\vec{s}_i$ or $0\cdot\vec{t}_j$ terms that we can
          neglect.
        \partsitem
          This set is not linearly independent:
          \begin{equation*}
            S=\set{\colvec{1 \\ 0},\colvec{2 \\ 0}}\subset\Re^2
          \end{equation*}
          and these two linear combinations give the same result
          \begin{equation*}
            \colvec{0 \\ 0}=2\cdot\colvec{1 \\ 0}-1\cdot\colvec{2 \\ 0}
                            =4\cdot\colvec{1 \\ 0}-2\cdot\colvec{2 \\ 0}
          \end{equation*}
          Thus, a linearly dependent set might have indistinct sums.

          In fact, this stronger statement holds:~if a set is linearly 
          dependent then it must have the property that there are two 
          distinct linear combinations that sum to the same vector.
          Briefly, where \( c_1\vec{s}_1+\dots+c_n\vec{s}_n=\zero \) then
          multiplying both sides of the relationship by two gives another 
          relationship.
          If the first relationship is nontrivial then the second is also.
      \end{exparts}  
    \end{answer}
  \item  \label{exer:PolyZeroFcnOnlyIfZeroPol}
     Prove that a polynomial gives rise to the zero function if and only if
     it is the zero polynomial.
     (\textit{Comment.}
     This question is not a Linear Algebra matter but we often use the result.
     A polynomial gives rise to a function in the natural 
     way:~$x\mapsto c_nx^n+\dots+c_1x+c_0$.)
     \begin{answer}
       In this `if and only if' statement, the `if' half is clear\Dash if 
       the polynomial is the zero polynomial then the function that arises 
       from the action of the polynomial must be the zero 
       function $x\mapsto 0$. 
       For `only if' we write $p(x)=c_nx^n+\dots+c_0$. 
       Plugging in zero $p(0)=0$ gives that $c_0=0$.
       Taking the derivative and plugging in zero $p^\prime(0)=0$ gives 
       that $c_1=0$.
       Similarly we get that each $c_i$ is zero, and $p$ is the zero 
       polynomial.
     \end{answer}
  \item
    Return to Section 1.2 and redefine point, line, plane,
    and other linear surfaces to avoid degenerate cases.
    \begin{answer}
      The work in this section suggests that we should define 
      an \( n \)-dimensional
      non-degenerate linear surface as the span of a linearly
      independent set of \( n \) vectors.  
    \end{answer}
  \item 
    \begin{exparts}  
      \partsitem Show that any set of four vectors in \( \Re^2 \) is 
         linearly dependent.
      \partsitem Is this true for any set of five?
         Any set of three?
      \partsitem What is the most number of elements that a 
         linearly independent subset of $\Re^2$ can have?
    \end{exparts}
    \begin{answer}
      \begin{exparts}
        \partsitem For any $a_{1,1}$, \ldots, $a_{2,4}$,
          \begin{equation*}
            c_1\colvec{a_{1,1} \\ a_{2,1}}
            +c_2\colvec{a_{1,2} \\ a_{2,2}}
            +c_3\colvec{a_{1,3} \\ a_{2,3}}
            +c_4\colvec{a_{1,4} \\ a_{2,4}}
            =\colvec{0 \\ 0}
          \end{equation*}
          yields a linear system
          \begin{equation*}
             \begin{linsys}{4}
              a_{1,1}c_1 &+ &a_{1,2}c_2 &+ &a_{1,3}c_3 &+ &a_{1,4}c_4 &= &0  \\
              a_{2,1}c_1 &+ &a_{2,2}c_2 &+ &a_{2,3}c_3 &+ &a_{2,4}c_4 &= &0  
             \end{linsys}
          \end{equation*}
        that has infinitely many solutions (Gauss's Method leaves at least
        two variables free).
        Hence there are nontrivial linear relationships among the given
        members of $\Re^2$.
      \partsitem Any set five vectors is a superset of a set of four vectors,
        and so is linearly dependent.

        With three vectors from $\Re^2$, the argument from the prior item 
        still applies, with the slight change that Gauss's Method now only 
        leaves at least one variable free (but that still gives infinitely many
        solutions).
      \partsitem The prior item shows that no three-element subset of $\Re^2$
        is independent.
        We know that there are two-element subsets of $\Re^2$ that are 
        independent\Dash one is
        \begin{equation*}
          \set{\colvec{1  \\ 0},\colvec{0  \\ 1}}
        \end{equation*} 
        and so the answer is two.
     \end{exparts}  
    \end{answer}
  \item
    Is there a set of four vectors in \( \Re^3 \) such that any three 
    form a linearly independent set?
    \begin{answer}
      Yes; here is one.
      \begin{equation*}
         \set{\colvec{1 \\ 0 \\ 0},
              \colvec{0 \\ 1 \\ 0},
              \colvec{0 \\ 0 \\ 1},
              \colvec{1 \\ 1 \\ 1} }
      \end{equation*}  
    \end{answer}
  \item  
    Must every linearly dependent set have a subset that is dependent and
    a subset that is independent?
    \begin{answer}
      Yes.
      Fix any vector space~$V$, and let $S$ be a linearly dependent subset
      of~$V$.
      For a subset of~$S$ that is dependent we can take~$S$ itself.
      For a subset of~$S$ that is independent we can take the empty set.
    \end{answer}
  \item  
    In \( \Re^4 \) what is the biggest linearly independent
    set you can find?
    The smallest?
    The biggest linearly dependent set?
    The smallest?
    (`Biggest' and `smallest' mean that there are no supersets or subsets 
    with the same property.)
    \begin{answer} 
      In \( \Re^4 \) the biggest linearly independent set has
      four vectors.
      There are many examples of such sets, this is one.
      \begin{equation*}
        \set{\colvec{1 \\ 0 \\ 0 \\ 0},
             \colvec{0 \\ 1 \\ 0 \\ 0},
             \colvec{0 \\ 0 \\ 1 \\ 0},
             \colvec{0 \\ 0 \\ 0 \\ 1}  }
      \end{equation*}
      To see that no set with five or more vectors can be independent, set up
      \begin{equation*}
             c_1\colvec{a_{1,1} \\ a_{2,1} \\ a_{3,1} \\ a_{4,1}}
            +c_2\colvec{a_{1,2} \\ a_{2,2} \\ a_{3,2} \\ a_{4,2}}
            +c_3\colvec{a_{1,3} \\ a_{2,3} \\ a_{3,3} \\ a_{4,3}}
            +c_4\colvec{a_{1,4} \\ a_{2,4} \\ a_{3,4} \\ a_{4,4}}
            +c_5\colvec{a_{1,5} \\ a_{2,5} \\ a_{3,5} \\ a_{4,5}}
             =\colvec{0 \\ 0 \\ 0 \\ 0}
      \end{equation*}
      and note that the resulting linear system 
      \begin{equation*}
        \begin{linsys}{5}
          a_{1,1}c_1 &+ &a_{1,2}c_2 &+ &a_{1,3}c_3 
              &+ &a_{1,4}c_4 &+ &a_{1,5}c_5 &=  &0   \\
          a_{2,1}c_1 &+ &a_{2,2}c_2 &+ &a_{2,3}c_3 
              &+ &a_{2,4}c_4 &+ &a_{2,5}c_5 &=  &0   \\
          a_{3,1}c_1 &+ &a_{3,2}c_2 &+ &a_{3,3}c_3 
              &+ &a_{3,4}c_4 &+ &a_{3,5}c_5 &=  &0   \\
          a_{4,1}c_1 &+ &a_{4,2}c_2 &+ &a_{4,3}c_3 
              &+ &a_{4,4}c_4 &+ &a_{4,5}c_5 &=  &0     
        \end{linsys}
      \end{equation*}
      has four equations and five unknowns, 
      so Gauss's Method must end with at least one \( c \) variable free,
      so there are infinitely many solutions,
      and so the above linear relationship among the four-tall vectors has
      more solutions than just the trivial solution.

      The smallest linearly independent set is the empty set.

      The biggest linearly dependent set is \( \Re^4 \).
      The smallest is \( \set{\zero} \).  
    \end{answer}
  \recommended \item 
    Linear independence and linear dependence are properties of sets.
    We can thus naturally ask how the properties of linear independence
    and dependence act with respect to
    the familiar elementary set relations and operations.  
    In this body of this subsection we have covered the subset and superset
    relations.
    We can also consider the operations of intersection, complementation, 
    and union.
    \begin{exparts}
       \partsitem How does linear independence relate to intersection:~can
         an intersection of linearly independent sets be independent?
         Must it be?
       \partsitem How does linear independence relate to complementation?
       \partsitem Show that the union of two linearly independent sets
          can be linearly independent.
       \partsitem Show that
          the union of two linearly independent sets need not be
          linearly independent.
    \end{exparts}
    \begin{answer}
      \begin{exparts}
        \partsitem The intersection of two linearly independent sets
          $S\intersection T$ must be linearly
          independent as it is a subset of the linearly independent set $S$
          (as well as the linearly independent set $T$ also, of course).
        \partsitem The complement of a linearly independent set is linearly
          dependent as it contains the zero vector.
        \partsitem A simple example in \( \Re^2 \) is these two sets.
          \begin{equation*}
            S=\set{\colvec{1 \\ 0}}
            \qquad
            T=\set{\colvec{0 \\ 1}}          
          \end{equation*}
          A somewhat subtler example, again in \( \Re^2 \), is these two.
          \begin{equation*}
            S=\set{\colvec{1 \\ 0}}
            \qquad
            T=\set{\colvec{1 \\ 0}, \colvec{0 \\ 1}}
          \end{equation*}
        \partsitem We must produce an example.
          One, in \( \Re^2 \), is
          \begin{equation*}
            S=\set{\colvec{1 \\ 0}}
            \qquad
            T=\set{\colvec{2 \\ 0}}
          \end{equation*}
          since the linear dependence of \( S_1\union S_2 \) is easy to see.
      \end{exparts}
    \end{answer}
  \item \textit{Continued from prior exercise.} 
    What is the interaction between the property of linear independence
    and the operation of union?
    \begin{exparts}
       \partsitem We might conjecture that the union $S\union T$ of 
          linearly independent sets 
          is linearly independent if and only if their spans have a trivial 
          intersection $\spanof{S}\intersection \spanof{T}=\set{\zero}$.
          What is wrong with this argument for the `if' direction
          of that conjecture? 
          ``If the union $S\union T$ is linearly independent then
          the only solution to          
          $c_1\vec{s}_1+\cdots+c_n\vec{s}_n
            +d_1\vec{t}_1+\cdots+d_m\vec{t}_m
            =\zero$
          is the trivial one $c_1=0$, \ldots, $d_m=0$. 
          So any member of the intersection of the spans
          must be the zero vector because in 
          $c_1\vec{s}_1+\cdots+c_n\vec{s}_n
            =d_1\vec{t}_1+\cdots+d_m\vec{t}_m$
          each scalar is zero.''
       \partsitem Give an example showing that the conjecture is false.
       \partsitem Find linearly independent sets \( S \) and \( T \)
          so that the union of \( S-(S\cap T)\) and \( T-(S\cap T) \)
          is linearly independent, but the union \( S\cup T \) is
          not linearly independent.
       \partsitem Characterize when the union of two linearly independent sets
          is linearly independent, in terms of the intersection of spans.
    \end{exparts}
    \begin{answer}
      \begin{exparts}
        \partsitem \nearbylemma{le:LDIffANonTrivLinRel} 
          requires that the vectors 
          $\vec{s}_1,\ldots,\vec{s}_n,\vec{t}_1,\ldots,\vec{t}_m$
          be distinct. 
          But we could have that the union $S\cup T$ is linearly 
          independent with some $\vec{s}_i$ equal to some $\vec{t}_j$.
        \partsitem One example in \( \Re^2 \) is these two.
          \begin{equation*}
            S=\set{\colvec{1 \\ 0}}
            \qquad
            T=\set{\colvec{1 \\ 0}, \colvec{0 \\ 1}}
          \end{equation*}
        \partsitem 
          An example from \( \Re^2 \) is these sets.
          \begin{equation*}
            S=\set{\colvec{1 \\ 0}, \colvec{0 \\ 1}}
            \qquad
            T=\set{\colvec{1 \\ 0}, \colvec{1 \\ 1}}
          \end{equation*}
        \partsitem The union of two linearly independent sets $S\union T$
          is linearly independent if and only if their spans of \( S \) 
          and \( T-(S\cap T)\) have a trivial intersection 
          $\spanof{S}\intersection \spanof{T-(S\cap T)}=\set{\zero}$.
          To prove that, assume that \( S \) and \( T \) are linearly 
          independent subsets of some vector space.

          For the `only if' direction, assume that the intersection of
          the spans is trivial 
          \( \spanof{S}\intersection \spanof{T-(S\cap T)}=\set{\zero} \).
          Consider the set $S\union (T-(S\cap T))=S\union T$ and
          consider the linear relationship 
          $c_1\vec{s}_1+\dots+c_n\vec{s}_n
            +d_1\vec{t}_1+\dots+d_m\vec{t}_m=\zero$.
          Subtracting gives
          $c_1\vec{s}_1+\dots+c_n\vec{s}_n=
            -d_1\vec{t}_1-\dots-d_m\vec{t}_m$.
          The left side of that equation sums to a vector in $\spanof{S}$, and
          the right side is a vector in $\spanof{T-(S\cap T)}$.
          Therefore, since the intersection of the spans is trivial, both
          sides equal the zero vector.
          Because $S$ is linearly independent, all of the $c$'s are zero.
          Because $T$ is linearly independent so also is  
          $T-(S\cap T)$ linearly independent, 
          and therefore all of the $d$'s are zero.
          Thus, the original linear relationship among members of 
          $S\union T$ only holds if all of the coefficients are zero.
          Hence, $S\union T$ is linearly independent.

          For the `if' half we can make the same argument in reverse.
          Suppose that the union $S\union T$ is linearly independent.
          Consider a linear relationship among members of 
          $S$ and $T-(S\cap T)$.        
          $c_1\vec{s}_1+\cdots+c_n\vec{s}_n
            +d_1\vec{t}_1+\cdots+d_m\vec{t}_m
            =\zero$
          Note that no $\vec{s}_i$ is equal to a $\vec{t}_j$
          so that is a combination of distinct vectors, as required by 
          \nearbylemma{le:LDIffANonTrivLinRel}.
          So the only solution 
          is the trivial one $c_1=0$, \ldots, $d_m=0$. 
          Since any vector $\vec{v}$ 
          in the intersection of the spans
          $\spanof{S}\intersection \spanof{T-(S\cap T)}$ 
          we can write 
          $\vec{v}=c_1\vec{s}_1+\cdots+c_n\vec{s}_n
            =-d_1\vec{t}_1-\cdots-d_m\vec{t}_m$,
          and it must be the zero vector because each scalar is zero.
      \end{exparts}  
    \end{answer}
  \item \label{exer:FillIndDetProofSetHasLISub}
    For \nearbycorollary{th:AlwaysAnLDSubset},
    \begin{exparts}
       \partsitem fill in the induction for the proof;
       \partsitem give an alternate proof that starts with the empty
         set and builds
         a sequence of linearly independent subsets of the given finite set
         until one appears with the same span as the given set.
    \end{exparts}
    \begin{answer}
      \begin{exparts}
         \partsitem We do induction on the number of vectors in the finite set
           \( S \).

           The base case is that $S$ has no elements.
           In this case $S$ is linearly independent and there is nothing to 
           check\Dash a subset of $S$ that has the same span as $S$ is $S$
           itself.

           For the inductive step assume that the theorem is true for all 
           sets of size $n=0$, $n=1$, \ldots, $n=k$ 
           in order to prove that it holds when \( S \) has $n=k+1$ elements.
           If the $k+1$-element set \( S=\set{\vec{s}_0,\dots,\vec{s}_{k}} \) 
           is linearly independent then the theorem is trivial,
           so assume that it is dependent.
           By \nearbycorollary{cor:LDMeansLC} there is an \( \vec{s}_i \)
           that is a linear combination of other vectors in \( S \).
           Define \( S_1=S-\set{\vec{s}_i} \) and note that 
           \( S_1 \) has the same span as \( S \) by
           \nearbycorollary{cor:OmitVecNotShrinkSpanIffVecIsDep}.
           The set \( S_1 \) has \( k \) elements and 
           so the inductive hypothesis
           applies to give that it has a linearly independent subset with the 
           same span.
           That subset of \( S_1 \) is the desired subset of \( S \).
         \partsitem Here is a sketch of the argument.
           We have left out the induction argument details.

           If the finite set \( S \) is empty then there is nothing to prove.
           If \( S=\set{\zero} \) then the empty subset will do.

           Otherwise, take some nonzero vector \( \vec{s}_1\in S \)
           and define \( S_1=\set{\vec{s}_1} \).
           If \( \spanof{S_1}=\spanof{S} \) then
           we are finished with this proof by noting that \( S_1 \) is linearly
           independent.

           If not, then there is a nonzero
           vector \( \vec{s}_2\in S-\spanof{S_1} \)
           (if every \( \vec{s}\in S \) is in \( \spanof{S_1} \) then
           \( \spanof{S_1}=\spanof{S} \)).
           Define \( S_2=S_1\union\set{\vec{s}_2} \).
           If \( \spanof{S_2}=\spanof{S} \) then 
           we are finished 
           by using \nearbytheorem{cor:LDMeansLC}
           to show that \( S_2 \) is linearly independent.

           Repeat the last paragraph until a set with a big enough
           span appears.
           That must eventually happen
           because \( S \) is finite, and
           \( \spanof{S} \) will be reached at worst when we have
           used every vector from
           \( S \).
      \end{exparts}  
    \end{answer}
  \item 
     With a some calculation we can get formulas to determine whether or
     not a set of vectors is linearly independent. 
     \begin{exparts}
       \partsitem Show that this subset of \( \Re^2 \)
         \begin{equation*}
           \set{\colvec{a \\ c},\colvec{b \\ d}}
         \end{equation*}
         is linearly independent if and only if \( ad-bc\neq 0 \).
       \partsitem Show that this subset of \( \Re^3 \)
         \begin{equation*}
           \set{\colvec{a \\ d \\ g},
                \colvec{b \\ e \\ h},
                \colvec{c \\ f \\ i}  }
         \end{equation*}
         is linearly independent iff
         \( aei+bfg+cdh-hfa-idb-gec \neq 0 \).
       \partsitem When is this subset of \( \Re^3 \)
         \begin{equation*}
           \set{\colvec{a \\ d \\ g},
                \colvec{b \\ e \\ h} }
         \end{equation*}
         linearly independent?
       \partsitem This is an opinion question:~for
         a set of four vectors from \( \Re^4 \),
         must there be a formula involving the sixteen entries 
         that determines independence of the set?
         (You needn't produce such a formula, just decide if one exists.)
    \end{exparts}
    \begin{answer}
      \begin{exparts}
        \partsitem 
          Assuming first that \( a\neq 0 \),
          \begin{equation*}
            x\colvec{a \\ c}
            +y\colvec{b \\ d}
            =\colvec{0 \\ 0}
          \end{equation*}
          gives
          \begin{equation*}
            \begin{linsys}{2}
               ax  &+  &by &=  &0  \\
               cx  &+  &dy &=  &0  
            \end{linsys}
            \grstep{-(c/a)\rho_1+\rho_2}
            \begin{linsys}{2}
               ax  &+  &by           &=  &0  \\
                   &   &(-(c/a)b+d)y &=  &0  
             \end{linsys}
          \end{equation*}
          which has a solution if and only if
          \( 0\neq-(c/a)b+d=(-cb+ad)/d \)
          (we've assumed in this case that \( a\neq 0 \), and so 
          back substitution yields a unique solution).

          The \( a=0 \) case is also not hard\Dash break it into the 
          \( c\neq 0 \) and \( c=0 \) subcases and 
          note that in these cases \( ad-bc=0\cdot d-bc \).

          \textit{Comment.}
          An earlier exercise showed that a two-vector set is linearly
          dependent if and only if either vector is a scalar multiple of the
          other.
          We could also use that to make the calculation.
        \partsitem The equation
          \begin{equation*}
            c_1\colvec{a \\ d \\ g}
            +c_2\colvec{b \\ e \\ h}
            +c_3\colvec{c \\ f \\ i}
            =\colvec{0 \\ 0 \\ 0}
          \end{equation*}
         expresses a homogeneous linear system.
         We proceed by writing it in matrix form and applying Gauss's Method.

         We first reduce the matrix to upper-triangular.
         Assume that \( a\neq 0 \).
         With that, we can clear down the first column.
         \begin{equation*}
           \grstep{(1/a)\rho_1}
           \begin{amat}{3}
              1   &b/a   &c/a  &0 \\
              d   &e     &f    &0 \\
              g   &h     &i    &0
            \end{amat}                                            
           \grstep[-g\rho_1+\rho_3]{-d\rho_1+\rho_2}
           \begin{amat}{3}
              1   &b/a           &c/a        &0   \\
              0   &(ae-bd)/a     &(af-cd)/a  &0   \\
              0   &(ah-bg)/a     &(ai-cg)/a  &0
            \end{amat}                                            
         \end{equation*}
         Then we get a $1$ in the second row, second column entry.
         (Assuming for the moment that \( ae-bd\neq 0 \), in order
         to do the row reduction step.)
         \begin{equation*}
           \grstep{(a/(ae-bd))\rho_2}
           \begin{amat}{3}
              1   &b/a           &c/a             &0  \\
              0   &1             &(af-cd)/(ae-bd) &0  \\
              0   &(ah-bg)/a     &(ai-cg)/a       &0
            \end{amat}
         \end{equation*}
         Then, under the assumptions, we perform
         the row operation $((ah-bg)/a)\rho_2+\rho_3$
         to get this.
         \begin{equation*}
           % \grstep{((ah-bg)/a)\rho_2+\rho_3}
           \begin{amat}{3}
              1   &b/a   &c/a                              &0 \\
              0   &1     &(af-cd)/(ae-bd)                   &0 \\
              0   &0     &(aei+bgf+cdh-hfa-idb-gec)/(ae-bd) &0
            \end{amat}
         \end{equation*}
         Therefore, the original system is nonsingular
         if and only if the above \( 3,3 \) entry is nonzero
         (this fraction is defined because of the \( ae-bd\neq 0 \) assumption).
         It equals zero if and only if the numerator is zero. 

         We next worry about the assumptions.
         First, if \( a\neq 0 \) but \( ae-bd=0 \) then we swap
         \begin{multline*}
           \begin{amat}{3}
              1   &b/a           &c/a        &0   \\
              0   &0             &(af-cd)/a  &0   \\
              0   &(ah-bg)/a     &(ai-cg)/a  &0
            \end{amat}                                    \\
           \grstep{\rho_2\leftrightarrow\rho_3}
           \begin{amat}{3}
              1   &b/a           &c/a        &0   \\
              0   &(ah-bg)/a     &(ai-cg)/a  &0   \\
              0   &0             &(af-cd)/a  &0
            \end{amat}
         \end{multline*}
         and conclude that the system is nonsingular if and only if either
         \( ah-bg=0 \) or \( af-cd=0 \).
         That's the same as asking that their product be zero:
         \begin{align*}
            ahaf-ahcd-bgaf+bgcd
            &=0                   \\
            ahaf-ahcd-bgaf+aegc
            &=0                   \\
            a(haf-hcd-bgf+egc)
            &=0
         \end{align*}
         (in going from the first line to the second we've applied the
         case assumption that $ae-bd=0$ by substituting $ae$ for $bd$).
         Since we are assuming that \( a\neq 0 \), 
         we have that \( haf-hcd-bgf+egc=0 \).
         With $ae-bd=0$ we can rewrite this to fit the form we need:~in
         this \( a\neq 0 \) and \( ae-bd=0 \) case, the given system
         is nonsingular when
         \( haf-hcd-bgf+egc-i(ae-bd)=0 \), as required.

         The remaining cases have the same character.
         Do the \( a=0 \) but \( d\neq 0 \) case and the \( a=0 \) and
         \( d=0 \) but \( g\neq 0 \) case by first swapping rows and
         then going on as above.
         The \( a=0 \), \( d=0 \), and \( g=0 \) case is easy\Dash a set with a
         zero vector is linearly dependent, and the formula comes out
         to equal zero.
       \partsitem It is linearly dependent if and only if either vector is a
         multiple of the other.
         That is, it is not independent iff
         \begin{equation*}
           \colvec{a \\ d \\ g}=r\cdot\colvec{b \\ e \\ h}
           \quad\text{or}\quad
           \colvec{b \\ e \\ h}=s\cdot\colvec{a \\ d \\ g}
         \end{equation*}
         (or both) for some scalars $r$ and $s$.
         Eliminating $r$ and $s$ in order to restate this condition only in
         terms of the given letters $a$, $b$, $d$, $e$, $g$, $h$, we have that 
         it is not independent\Dash it is dependent\Dash iff
         \( ae-bd=ah-gb=dh-ge \).
       \partsitem Dependence or independence is a function of the
         indices, so there
         is indeed a formula (although at first glance a person might think
         the formula involves cases: ``if the first component of the first
         vector is zero then \ldots'', this guess turns out not to be 
         correct).
      \end{exparts}  
    \end{answer}
  \recommended \item  
    \begin{exparts}
      \partsitem Prove that a set of two perpendicular 
        nonzero vectors from
        \( \Re^n \) is linearly independent when \( n>1 \).
      \partsitem What if \( n=1 \)?
        \( n=0 \)?
      \partsitem Generalize to more than two vectors.
    \end{exparts}
    \begin{answer}
      Recall that two vectors from \( \Re^n \) are perpendicular if and
      only if their dot product is zero.
      \begin{exparts}
         \partsitem Assume that \( \vec{v} \) and \( \vec{w} \) are
           perpendicular nonzero vectors in $\Re^n$, with \( n>1 \).
           With the linear relationship \( c\vec{v}+d\vec{w}=\zero \), 
           apply \( \vec{v} \) to both
           sides to conclude that \( c\cdot\norm{\vec{v}}^2+d\cdot 0=0 \).
           Because \( \vec{v}\neq\zero \) we have that \( c=0 \).
           A similar application of \( \vec{w} \) shows that \( d=0 \).
         \partsitem Two vectors in \( \Re^1 \) are perpendicular if and only if
           at least one of them is zero.

           We define \( \Re^0 \) to be a trivial space, and so both $\vec{v}$
           and $\vec{w}$ are the zero vector.
         \partsitem The right generalization is to look at a set
           \( \set{\vec{v}_1,\dots,\vec{v}_n}\subseteq\Re^k \) of vectors
           that are \definend{mutually orthogonal} 
           (also called \definend{pairwise perpendicular}):~if
           \( i\neq j \) then \( \vec{v}_i \) is perpendicular to
           \( \vec{v}_j \).
           Mimicking the proof of the first item above shows that such a set of
           nonzero vectors is linearly independent.
      \end{exparts}  
    \end{answer}
  \item 
    Consider the set of functions from the interval 
    $\openinterval{-1}{1}\subseteq \Re$  
    to $\Re$.
    \begin{exparts}
      \partsitem Show that 
         this set is a vector space under the usual operations.
      \partsitem Recall the formula for the sum of an infinite geometric 
         series: 
         \( 1+x+x^2+\cdots=1/(1-x) \) for all \( x\in(-1..1) \).
         Why does this not express a dependence inside of
         the set $\set{g(x)=1/(1-x),f_0(x)=1,f_1(x)=x,f_2(x)=x^2,\ldots}$
         (in the vector space that we are considering)?
         (\textit{Hint.}
         Review the definition of linear combination.)
      \partsitem Show that the set in the prior item is linearly independent.
    \end{exparts}
    This shows that some vector spaces exist with linearly independent subsets
    that are infinite.
    \begin{answer}
      \begin{exparts}
        \partsitem This check is routine.
        \partsitem The summation is infinite (has infinitely many summands).
          The definition of linear combination involves only finite sums.
        \partsitem No nontrivial finite sum of members of 
           \( \set{g,f_0,f_1,\ldots} \) adds to the zero object:~assume that
           \begin{equation*}
              c_0\cdot (1/(1-x))+c_1\cdot 1+\dots+c_n\cdot x^n=0
           \end{equation*}
           (any finite sum uses a highest power, here \( n \)).
           Multiply both sides by \( 1-x \) to conclude that each coefficient 
           is zero, because a polynomial describes the zero function only when 
           it is the zero polynomial.
      \end{exparts}
     \end{answer}
  \item 
    Show that, where \( S \) is a subspace of \( V \), if a subset $T$ of
    \( S \) is linearly independent in \( S \) then $T$ is also linearly
    independent in \( V \).
    Is that `only if'?
    \begin{answer}
      It is both `if' and `only if'.

      Let \( T \) be a subset of the subspace \( S \) of the vector space
      \( V \).
      The assertion that any linear relationship 
      $c_1\vec{t}_1+\dots+c_n\vec{t}_n=\zero$ among members of \( T \)
      must be the trivial relationship $c_1=0$, \ldots, $c_n=0$
      is a statement that 
      holds in \( S \) if and only if it holds in \( V \),
      because the subspace \( S \) inherits its addition and 
      scalar multiplication operations from \( V \).  
    \end{answer}
\end{exercises}
