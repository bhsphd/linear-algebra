\topic{Voting Paradoxes}
\index{voting paradox|(}
%\emph{(Optional material from this chapter is discussed here,
%but is not required to complete the exercises.)}

Imagine that a  Political Science class studying the 
American presidential process holds a mock election.
The $29$ class members
rank the Democratic Party,
Republican Party, and Third Party nominees,
from most preferred to least preferred
($>$ means `is preferred to').
\begin{center}
  \renewcommand{\arraystretch}{1.1}
  \begin{tabular}{cc}
    \multicolumn{1}{c}{\begin{tabular}[b]{@{}c@{}}
       \textit{preference order}
    \end{tabular}}
    &\multicolumn{1}{l}{{\renewcommand{\arraystretch}{1.0}
      \begin{tabular}[b]{@{}l@{}}
         \textit{number with}    \\[-0.5ex]
         \textit{that preference}
      \end{tabular}}}                \\  
    \hline
    \begin{tabular}{@{}r@{}}
      $\text{Democrat} > \text{Republican} > \text{Third}$      \\
      $\text{Democrat} > \text{Third} > \text{Republican}$      \\
      $\text{Republican} > \text{Democrat} > \text{Third}$      \\
      $\text{Republican} > \text{Third} > \text{Democrat}$      \\
      $\text{Third} > \text{Democrat} > \text{Republican}$      \\
      $\text{Third} > \text{Republican} > \text{Democrat}$   
    \end{tabular}
    &\begin{tabular}{@{}r@{}}                                                
      $5$     \\
      $4$     \\
      $2$     \\
      $8$     \\
      $8$     \\
      $2$    
    \end{tabular}    
  \end{tabular}
    \renewcommand{\arraystretch}{1}
\end{center}
What is the preference of the group as a whole?

Overall,
the group prefers the Democrat to the Republican by five votes; 
seventeen voters ranked the Democrat above the Republican
versus twelve the other way.
And the group 
prefers the Republican to the Third's nominee, 
fifteen to fourteen.
But, strangely enough, the group also 
prefers the Third to the Democrat, eighteen to eleven.
\begin{center}
  \includegraphics{voting.1}
\end{center}
This is a \definend{voting paradox}\index{voting paradox!definition},
specifically, a
\definend{majority cycle}.\index{voting paradox!majority cycle}

Mathematicians study voting paradoxes in part because of their 
implications for practical politics.
For instance, the instructor can manipulate this class
into choosing the Democrat as the overall winner 
by first asking for a vote between the Republican and
the Third, 
and then asking for a vote between the winner of that contest, who will be the
Republican, and the Democrat.
By similar manipulations
the instructor can make any of the other two candidates come out as the winner.
(We will stick to three-candidate elections but the same thing
happens in larger elections.)

Mathematicians also study voting paradoxes simply because they are
interesting.
One interesting aspect is that
the group's overall majority cycle occurs despite that  
each single voter's preference list is 
\definend{rational}\index{voting paradox!rational preference order}, in 
a straight-line order.
That is, the majority cycle seems to arise in the aggregate
without being present in the components of that aggregate, the preference lists.
However we can use linear algebra  to argue that
a tendency toward cyclic preference is actually present
in each voter's list and
that it surfaces when there is more adding of the tendency
than canceling.

For this,
abbreviating the choices as $D$, $R$, and $T$,
we can describe how
a voter with preference order $D>R>T$ contributes to the above cycle.
\begin{center}
  \includegraphics{voting.2}
\end{center}
(The negative sign is here because the arrow describes $T$ as 
preferred to $D$, but this voter likes them the other way.)
The descriptions for the other preference lists are in the table on 
page~\pageref{table:Voting}.

Now, to conduct the election we linearly combine these descriptions; 
for instance, the Political Science mock election
\begin{equation*}
  5\cdot \votinggraphic{43}
  +4\cdot\votinggraphic{44}
  +\dots+2\cdot\votinggraphic{45}
\tag*{}\end{equation*}
yields the circular group preference shown earlier.

Of course, taking linear combinations is linear algebra.
The graphical cycle notation is suggestive but inconvenient so we 
use column vectors by starting at the $D$ and
taking the numbers from the cycle in counterclockwise order.   
Thus, we represent the mock election and a single $D>R>T$ vote in this way. 
\begin{equation*}
  \votepreflist{7}{1}{5}
  \quad\text{and}\quad
  \votepreflist{-1}{1}{1}
\tag*{}\end{equation*}

We will 
decompose vote vectors into two parts, one
cyclic and the other acyclic.
For the first part, 
we say that a vector is \definend{purely cyclic} if it is in  
this subspace of $\Re^3$. 
\begin{equation*}
  C
  =\set{\votepreflist{k}{k}{k} \suchthat k \in\Re}
  =\set{k\cdot\votepreflist{1}{1}{1} \suchthat k \in\Re}
\end{equation*}
For the second part, consider the set 
of vectors that are 
perpendicular to all of the vectors in $C$.
\nearbyexercise{exer:PerpIsSubSp} shows that this is a subspace
\begin{align*}
  C^\perp &=\set{\votepreflist{c_1}{c_2}{c_3} \suchthat 
                 \colvec{c_1 \\ c_2 \\ c_3}\dotprod\colvec{k \\ k \\ k} =0
                 \text{\ for all $k\in\Re$}}             \\
          &=\set{\votepreflist{c_1}{c_2}{c_3} \suchthat 
                 c_1+c_2+c_3=0}                                         
          =\set{c_2\votepreflist{-1}{1}{0}
                +c_3\votepreflist{-1}{0}{1} \suchthat  c_2,c_3\in\Re}
\end{align*}
(read that aloud as ``$C$~perp'').
So we are led to this basis for $\Re^3$.
\begin{equation*}
 \sequence{\votepreflist{1}{1}{1},
             \votepreflist{-1}{1}{0},
             \votepreflist{-1}{0}{1}}
\tag*{}\end{equation*}
We can represent votes with respect to this basis,
and thereby decompose them into a cyclic part and an acyclic part.
\textit{(Note for readers who have covered the optional section
in this chapter:~that is, the space is the direct sum of $C$ and~$C^\perp$.)}

For example, consider the $D>R>T$ voter discussed above.
We represent that voter with respect to the basis
\begin{equation*}
  \begin{linsys}{3}
    c_1  &-  &c_2  &-  &c_3  &=  &-1 \\
    c_1  &+  &c_2  &   &     &=  &1  \\
    c_1  &   &     &+  &c_3  &=  &1  
  \end{linsys}
  \grstep[-\rho_1+\rho_3]{-\rho_1+\rho_2}
  \repeatedgrstep{(-1/2)\rho_2+\rho_3}
  \begin{linsys}{3}
    c_1  &-  &c_2  &-  &c_3      &=  &-1 \\
         &   &2c_2 &+  &c_3      &=  &2  \\
         &   &     &   &(3/2)c_3 &=  &1  
  \end{linsys}
\end{equation*}
using the coordinates $c_1=1/3$, $c_2=2/3$, and $c_3=2/3$.
Then
\begin{equation*}
  \votepreflist{-1}{1}{1}
  =\frac{1}{3}\cdot\votepreflist{1}{1}{1}
    +\frac{2}{3}\cdot\votepreflist{-1}{1}{0}
    +\frac{2}{3}\cdot\votepreflist{-1}{0}{1}
  =\votepreflist{1/3}{1/3}{1/3}
   +\votepreflist{-4/3}{2/3}{2/3}
\tag*{}\end{equation*}
gives the desired decomposition into a 
cyclic part and an acyclic part.
\begin{equation*}
  \votinggraphic{3}
=
\votinggraphic{4}
\>+\>
\votinggraphic{5}
\end{equation*}
Thus we can see that this $D>R>T$ voter's
rational preference list does have a cyclic part.

The $T>R>D$ voter is opposite to the one just considered
in that the `$>$' symbols are reversed. 
This voter's decomposition 
\begin{equation*}
\votinggraphic{6}
=
\votinggraphic{7}
\>+\>
\votinggraphic{8}
\end{equation*}
shows that these opposite preferences have decompositions that are opposite.
We say that the first voter has positive 
\definend{spin}\index{spin}\index{voting paradox!spin}
since the cycle part is with the direction 
that we have chosen for the arrows, 
while the second voter's spin is negative.

The fact that these opposite voters cancel each other is reflected in 
the fact that their vote vectors add to zero.
This suggests an alternate way to tally an election. 
We could first cancel as many opposite preference lists as possible
and then determine the outcome by adding the remaining lists. 

The table below contains the three pairs of 
opposite preference lists.
For instance, the top line contains the voters discussed above.
\begin{center}
  \def\betweenrowvspace{5.5ex}
  \makebox[\hsize][l]{%  box runs into margin; suppress overfull warn
  \begin{tabular}{@{}c@{\hspace{0em}}c@{}} 
    % \hline
    \multicolumn{1}{c}{\textit{positive spin}}
    &\multicolumn{1}{c}{\textit{negative spin}}  
     \\ \hline
    {\renewcommand{\arraystretch}{1.4}%%
     \begin{tabular}{@{}c}
      $\text{Democrat}>\text{Republican}>\text{Third}$                     \\
      $\votinggraphic{9} % \votecycle{1}{1}{-1}{3}
      =\votinggraphic{10} %\votecycle{1/3}{1/3}{1/3}{3}
      \>+\>
      \votinggraphic{11}$  %\votecycle{2/3}{2/3}{-4/3}{3}\;{}
    \end{tabular}}
    &{\renewcommand{\arraystretch}{1.4}%%
     \begin{tabular}{c@{}}
      $\text{Third}>\text{Republican}>\text{Democrat}$                     \\
      $\votinggraphic{12}  %\votecycle{-1}{-1}{1}{3}
      =\votinggraphic{13}  %\votecycle{-1/3}{-1/3}{-1/3}{3}
      \>+\>
      \votinggraphic{14}$  %\votecycle{-2/3}{-2/3}{4/3}{3}\;{}
    \end{tabular}}                  \\[\betweenrowvspace] % \hline
    {\renewcommand{\arraystretch}{1.4}%%
    \begin{tabular}{@{}c}
      $\text{Republican}>\text{Third}>\text{Democrat}$                     \\
      $\votinggraphic{15}  %\votecycle{-1}{1}{1}{3}
      =\votinggraphic{16} %\votecycle{1/3}{1/3}{1/3}{3}
      \>+\>
      \votinggraphic{17}$  %\votecycle{-4/3}{2/3}{2/3}{3}\;{}
    \end{tabular}}
    &{\renewcommand{\arraystretch}{1.4}%%
     \begin{tabular}{c@{}}
      $\text{Democrat}>\text{Third}>\text{Republican}$                     \\
      $\votinggraphic{18}  %\votecycle{1}{-1}{-1}{3}
      =\votinggraphic{19}  %\votecycle{-1/3}{-1/3}{-1/3}{3}
      \>+\>
      \votinggraphic{20}$  %\votecycle{4/3}{-2/3}{-2/3}{3}\;{}
    \end{tabular}}                  \\[\betweenrowvspace] % \hline
    {\renewcommand{\arraystretch}{1.4}%%
    \begin{tabular}{@{}c}
      $\text{Third}>\text{Democrat}>\text{Republican}$                     \\
      $\votinggraphic{21}  %\votecycle{1}{-1}{1}{3}
      =\votinggraphic{22}  %\votecycle{1/3}{1/3}{1/3}{3}
      \>+\>
      \votinggraphic{23}$  %\votecycle{2/3}{-4/3}{2/3}{3}\;{}
    \end{tabular}}
    &{\renewcommand{\arraystretch}{1.4}%%
     \begin{tabular}{c@{}}
      $\text{Republican}>\text{Democrat}>\text{Third}$                     \\
      $\votinggraphic{24}  %\votecycle{-1}{1}{-1}{3}
      =\votinggraphic{25}  %\votecycle{-1/3}{-1/3}{-1/3}{3}
      \>+\>
      \votinggraphic{26}$  %\votecycle{-2/3}{4/3}{-2/3}{3}\;{}
    \end{tabular}}                  % \\ \hline
  \end{tabular}
  }
  \label{table:Voting}
\end{center}
If we conduct the election as just described then after the cancellation
of as many opposite pairs of voters as possible there will remain three
sets of preference lists:~one set from the first row, one from the second
row, and one from the third row. 
We will finish by proving that
a voting paradox can happen only if
the spins of these three sets are in the same direction.
That is, for a voting paradox to occur the three remaining sets must all come
from the left of the table or all come from the right
(see \nearbyexercise{exer:CancelPolSci}).
This shows that there is some connection between the majority cycle and
the decomposition that we are using\Dash a 
voting paradox can happen only when the
tendencies toward cyclic preference reinforce each other.

For the proof, assume that we have canceled 
opposite preference orders and are left with one set
of preference lists for each of the three rows.
Consider the first row's remaining preference lists.
They could be from the first row's left or right 
(or between, since the lists could have canceled exactly).
We shall write
\begin{equation*}
  \votinggraphic{27}
\end{equation*}
where $a$ is an integer that is positive if the remaining lists are on the
left, where $a$ is negative if the lists are on the right, and zero
if the cancellation was perfect.
Similarly we have integers $b$ and $c$ for the second and third rows,
which can each be positive, negative, or zero.

Then the election is determined by this sum.
\begin{equation*}
  \votinggraphic{27}  %\votecycle{a}{a}{-a}{4} 
  +  
  \votinggraphic{28}  %\votecycle{-b}{b}{b}{4} 
  +  
  \votinggraphic{29}  %\votecycle{c}{-c}{c}{4}  
  =\hspace{.30in}
  \votinggraphic{30}  %\votecycle{a-b+c}{a+b-c}{-a+b+c}{4}
  \hspace{.30in}\hbox{}
\end{equation*}
A voting paradox occurs when the three numbers in the total cycle on the right,
$-a+b+c$ and $a-b+c$ and $a+b-c$, 
are all nonnegative or all nonpositive.
We will prove this occurs only when either all 
three of $a$, $b$, and~$c$ are nonnegative or all three are nonpositive.

Let the total cycle numbers be nonnegative; the other case is similar.
\begin{equation*}
  \begin{linsys}{3}
    -a &+ &b &+ &c &\geq &0 \\
     a &- &b &+ &c &\geq &0 \\
     a &+ &b &- &c &\geq &0 
  \end{linsys}
\end{equation*}
Add the first two rows to see that $c\geq 0$.
Add the first and third rows for $b\geq 0$.
And, the second and third rows together give $a\geq 0$.
Thus if the total cycle is nonnegative then in each row the remaining
preference lists are from the table's left.
That ends the proof.

This result says only that having all three spin in the same direction is a 
necessary condition for a majority cycle.
It is not sufficient; see
\nearbyexercise{exer:VoterCondNotSuff}.

Voting theory and associated topics are the subject of current research.
There are many intriguing results, notably 
those produced by K~Arrow \cite{Arrow} who won the Nobel Prize in part for
this work, showing that no voting system is entirely fair
(for a reasonable definition of ``fair'').
Some good introductory articles are \cite{Gardner70}, 
\cite{Gardner74}, \cite{Gardner80}, and \cite{NeimiRiker}.
\cite{Taylor} is a readable text.
The long list of cases from recent American political history in 
\cite{GamingVote} shows these paradoxes are 
routinely manipulated in practice.
(On the other hand, quite recent research shows that 
computing how to manipulate elections can in general be unfeasible,
but this is beyond our scope.) 
This Topic is drawn from \cite{Zwicker}.
\emph{(Author's Note:~I would like to thank Professor~Zwicker 
for his kind and illuminating discussions.)}

\begin{exercises}
  \item \label{IrrIndVote}
    Here is a reasonable way in which a voter could have a cyclic preference.
    Suppose that this voter ranks each candidate
    on each of three criteria.
    \begin{exparts}
      \partsitem Draw up a table with the rows labeled `Democrat',
         `Republican', and `Third', and the columns labeled
         `character', `experience', and `policies'.
         Inside each column, rank some 
         candidate as most preferred, 
         rank another as in the middle, and rank the remaining one
         as least preferred.
      \partsitem In this ranking, is the Democrat preferred to the Republican
         in (at least) two out of three criteria, or vice versa?
         Is the Republican preferred to the Third?
      \partsitem  Does the table that was 
         just constructed have a cyclic preference order?
         If not, make one that does.
    \end{exparts}
    So it is possible for a voter to have a cyclic preference among candidates.
    The paradox described above, however, is that even if each voter has a
    straight-line preference list, 
    a cyclic preference can still arise for the entire group. 
    \begin{answer}
      This example yields a non-rational preference order for a 
      single voter.
      \smallskip
      \begin{center}
        \begin{tabular}{l|c|c|c|}
             \multicolumn{1}{c}{\ }
                  &\multicolumn{1}{c}{\textit{character}}
                  &\multicolumn{1}{c}{\textit{experience}}
                  &\multicolumn{1}{c}{\textit{policies}}   \\ 
          \cline{2-4}
         \textit{Democrat}   &most    &middle  &least   \\
         \textit{Republican} &middle  &least   &most    \\
         \textit{Third}      &least   &most    &middle  \\
          \cline{2-4}
        \end{tabular}
      \end{center}
      \smallskip
      The Democrat is better than the Republican for character and
      experience.
      The Republican wins over the Third for character and policies.
      And, the Third beats the Democrat for experience and policies.
    \end{answer}
  \item \label{VoteCalcTable} 
    Compute the values in the table of decompositions.
    \begin{answer}
      First, compare the $D>R>T$ decomposition that was covered in the Topic
      \begin{equation*}
        \votepreflist{-1}{1}{1}
          =\frac{1}{3}\cdot\votepreflist{1}{1}{1}
           +\frac{2}{3}\cdot\votepreflist{-1}{1}{0}
           +\frac{2}{3}\cdot\votepreflist{-1}{0}{1}
      \end{equation*}
      with the decomposition of the opposite $T>R>D$ voter.
      \begin{equation*}
        \votepreflist{1}{-1}{-1}
          =d_1\cdot\votepreflist{1}{1}{1}
           +d_2\cdot\votepreflist{-1}{1}{0}
           +d_3\cdot\votepreflist{-1}{0}{1}
      \tag*{}\end{equation*}
      Obviously, the second is the negative of the first, and so
      $d_1=-1/3$, $d_2=-2/3$, and $d_3=-2/3$.
      This principle holds for any pair of opposite voters, and so we need only
      do the computation for a voter from the second row, and a voter from the 
      third row.
      For a positive spin voter in the second row,
      \begin{equation*}
        \begin{linsys}{3}
          c_1  &-  &c_2  &-  &c_3  &=  &1 \\
          c_1  &+  &c_2  &   &     &=  &1  \\
          c_1  &   &     &+  &c_3  &=  &-1  
        \end{linsys}
        \grstep[-\rho_1+\rho_3]{-\rho_1+\rho_2}
        \repeatedgrstep{(-1/2)\rho_2+\rho_3}
        \begin{linsys}{3}
          c_1  &-  &c_2  &-  &c_3      &=  &1 \\
               &   &2c_2 &+  &c_3      &=  &0  \\
               &   &     &   &(3/2)c_3 &=  &-2  
        \end{linsys}
      \tag*{}\end{equation*}
      gives $c_3=-4/3$, $c_2=2/3$, and $c_1=1/3$.
      For a positive spin voter in the third row,
      \begin{equation*}
        \begin{linsys}{3}
          c_1  &-  &c_2  &-  &c_3  &=  &1 \\
          c_1  &+  &c_2  &   &     &=  &-1  \\
          c_1  &   &     &+  &c_3  &=  &1  
        \end{linsys}
        \grstep[-\rho_1+\rho_3]{-\rho_1+\rho_2}
        \repeatedgrstep{(-1/2)\rho_2+\rho_3}
        \begin{linsys}{3}
          c_1  &-  &c_2  &-  &c_3      &=  &1 \\
               &   &2c_2 &+  &c_3      &=  &-2  \\
               &   &     &   &(3/2)c_3 &=  &1  
        \end{linsys}
        \tag*{}\end{equation*}
        gives $c_3=2/3$, $c_2=-4/3$, and $c_1=1/3$.
      \end{answer}
  \item \label{exer:CancelPolSci} 
    Perform the cancellations of opposite preference orders 
    for the Political Science class's mock election.
    Are all the remaining preferences from the left three rows of the
    table or from the right?
    \begin{answer}
      The mock election corresponds to the table on 
      page~\pageref{table:Voting} in the way shown in the first table,
      and after cancellation the result is the second table.  
     \smallskip
      \begin{center}
         % \renewcommand{\arraystretch}{1.2}
         \begin{tabular}{|c|c|} \hline
           \textit{positive spin}
           &\textit{negative spin}  \\ \hline
           \begin{tabular}{l}
             $D>R>T$        \\
             5 voters
           \end{tabular}
          &\begin{tabular}{l}
             $T>R>D$          \\
             2 voters
           \end{tabular}                  \\ \hline
          \begin{tabular}{l}
             $R>T>D$           \\
             8 voters
          \end{tabular}
         &\begin{tabular}{l}
             $D>T>R$             \\
             4 voters
           \end{tabular}                  \\ \hline
          \begin{tabular}{l}
            $T>D>R$             \\
            8 voters
          \end{tabular}
         &\begin{tabular}{l}
            $R>D>T$            \\
            2 voters
          \end{tabular}                  \\ \hline
        \end{tabular}
        \qquad
         \begin{tabular}{|c|c|} \hline
           \textit{positive spin}
           &\textit{negative spin}  \\ \hline
           \begin{tabular}{l}
             $D>R>T$        \\
             3 voters
           \end{tabular}
          &\begin{tabular}{l}
             $T>R>D$          \\
             \textit{-canceled-}
           \end{tabular}                  \\ \hline
          \begin{tabular}{l}
             $R>T>D$           \\
             4 voters
          \end{tabular}
         &\begin{tabular}{l}
             $D>T>R$             \\
             \textit{-canceled-}
           \end{tabular}                  \\ \hline
          \begin{tabular}{l}
            $T>D>R$             \\
            6 voters
          \end{tabular}
         &\begin{tabular}{l}
            $R>D>T$            \\
            \textit{-canceled-}
          \end{tabular}                  \\ \hline
        \end{tabular}
       \end{center}
       \smallskip
       All three come from the same side of the table (the left), 
       as the result from this Topic says must happen.
       Now we can tally, using the canceled numbers 
       \begin{equation*}
         % \psset{xunit=4pt,yunit=4pt,runit=4pt} 
         3\cdot \votinggraphic{31}
%          \pspicture[.4](-10.5,-4.2)(10,4) %\psgrid(-10,-10)(10,10)
%            \SpecialCoor
%            \rput(3;90){\small $D$}
%            \psarc{->}(0,0){3}{10}{70} %-rd->
%               \rput[lb](3.3;20){\scriptsize $1$ voter} 
%            \rput[B](3;210){\small $T$}
%            \psarc{->}(0,0){3}{220}{320} %-tr->
%               \rput[t](3.1;270){\scriptsize $1$ voter} 
%            \rput[B](3;330){\small $R$}
%            \psarc{->}(0,0){3}{110}{175} %-dt->
%               \rput[rb](3.3;160){\scriptsize $-1$ voter} 
%          \endpspicture
         +4\cdot\votinggraphic{32}
%          \pspicture[.4](-10,-4.2)(10,4) %\psgrid(-10,-10)(10,10)
%            \SpecialCoor
%            \rput(3;90){\small $D$}
%            \psarc{->}(0,0){3}{10}{70} %-rd->
%               \rput[lb](3.3;20){\scriptsize $-1$ voter} 
%            \rput[B](3;210){\small $T$}
%            \psarc{->}(0,0){3}{220}{320} %-tr->
%               \rput[t](3.1;270){\scriptsize $1$ voter} 
%            \rput[B](3;330){\small $R$}
%            \psarc{->}(0,0){3}{110}{175} %-dt->
%               \rput[rb](3.3;160){\scriptsize $1$ voter} 
%          \endpspicture
         +6\cdot\votinggraphic{33}
%          \pspicture[.4](-9,-4.2)(10,4) %\psgrid(-10,-10)(10,10)
%            \SpecialCoor
%            \rput(3;90){\small $D$}
%            \psarc{->}(0,0){3}{10}{70} %-rd->
%               \rput[lb](3.3;20){\scriptsize $1$ voter} 
%            \rput[B](3;210){\small $T$}
%            \psarc{->}(0,0){3}{220}{320} %-tr->
%               \rput[t](3.1;270){\scriptsize $-1$ voter} 
%            \rput[B](3;330){\small $R$}
%            \psarc{->}(0,0){3}{110}{175} %-dt->
%               \rput[rb](3.3;160){\scriptsize $1$ voter} 
%          \endpspicture
         = \votinggraphic{34}
%          \pspicture[.4](-9,-4.2)(10,4) %\psgrid(-10,-10)(10,10)
%            \SpecialCoor
%            \rput(3;90){\small $D$}
%            \psarc{->}(0,0){3}{10}{70} %-rd->
%               \rput[lb](3.3;20){\scriptsize $5$ voters} 
%            \rput[B](3;210){\small $T$}
%            \psarc{->}(0,0){3}{220}{320} %-tr->
%               \rput[t](3.1;270){\scriptsize $1$ voter} 
%            \rput[B](3;330){\small $R$}
%            \psarc{->}(0,0){3}{110}{175} %-dt->
%               \rput[rb](3.3;160){\scriptsize $7$ voters} 
%          \endpspicture
       \tag*{}\end{equation*}
       to get the same outcome.
    \end{answer}
  \item \label{exer:VoterCondNotSuff} 
    The necessary condition that a
    voting paradox can happen only if all three preference lists remaining
    after cancellation have the same spin is not also sufficient. 
    \begin{exparts}
      % \partsitem Continuing the nonnegative total cycle case 
      %   considered in the proof, 
      %   use the two inequalities $0\leq a-b+c$ and $0\leq -a+b+c$
      %   to show that $\absval{a-b}\leq c$.
      % \partsitem Also show that $c\leq a+b$, and hence that
      %   $\absval{a-b}\leq c\leq a+b$. 
      \partsitem Give an example of a vote where there is a majority cycle
        and addition of one more voter with the same spin causes the cycle to
        go away.
      \partsitem Can the opposite happen; can addition of one voter with 
        a ``wrong'' spin cause a cycle to appear?
      \partsitem Give a condition that is both necessary and sufficient
        to get a majority cycle.
    \end{exparts}
    \begin{answer}
      \begin{exparts}
        % \partsitem We can rewrite the two as $-c\leq a-b$ and $-c\leq b-a$.
        %   Either $a-b$ or $b-a$ is nonpositive and so $-c\leq-\absval{a-b}$,
        %   as required.
        % \partsitem This is immediate from the supposition that $0\leq a+b-c$.
        \partsitem A trivial example starts with the zero-voter election,
          which has a trivial majority cycle, and adds any one voter.
          A more interesting example takes the Political Science mock
          election and adds two $T>D>R$ voters
          (to satisfy the
          ``addition of one more voter'' criteria in the question
          we can add them one at a time).
          The new voters have positive spin, which is the
          spin of the votes remaining after cancellation in the original mock
          election.
          This is the resulting table of voters and next to it is 
          the result of cancellation.
      \smallskip
      \begin{center}
         % \renewcommand{\arraystretch}{1.3}
         \begin{tabular}{|c|c|} \hline
           \textit{positive spin}
           &\textit{negative spin}  \\ \hline
           \begin{tabular}{l}
             $D>R>T$        \\
             5 voters
           \end{tabular}
          &\begin{tabular}{l}
             $T>R>D$          \\
             2 voters
           \end{tabular}                  \\ \hline
          \begin{tabular}{l}
             $R>T>D$           \\
             8 voters
          \end{tabular}
         &\begin{tabular}{l}
             $D>T>R$             \\
             4 voters
           \end{tabular}                  \\ \hline
          \begin{tabular}{l}
            $T>D>R$             \\
            10 voters
          \end{tabular}
         &\begin{tabular}{l}
            $R>D>T$            \\
            2 voters
          \end{tabular}                  \\ \hline
        \end{tabular}
        \qquad
         \begin{tabular}{|c|c|} \hline
           \textit{positive spin}
           &\textit{negative spin}  \\ \hline
           \begin{tabular}{l}
             $D>R>T$        \\
             3 voters
           \end{tabular}
          &\begin{tabular}{l}
             $T>R>D$          \\
             \textit{-canceled}
           \end{tabular}                  \\ \hline
          \begin{tabular}{l}
             $R>T>D$           \\
             4 voters
          \end{tabular}
         &\begin{tabular}{l}
             $D>T>R$             \\
             \textit{-canceled-}
           \end{tabular}                  \\ \hline
          \begin{tabular}{l}
            $T>D>R$             \\
            8 voters
          \end{tabular}
         &\begin{tabular}{l}
            $R>D>T$            \\
            \textit{-canceled-}
          \end{tabular}                  \\ \hline
        \end{tabular}
       \end{center}
       \smallskip
       This is the election using the canceled numbers. 
       \begin{equation*}
         % \psset{xunit=4pt,yunit=4pt,runit=4pt} 
         3\cdot\votinggraphic{35}
%          \pspicture[.4](-10.5,-4.2)(10,4) %\psgrid(-10,-10)(10,10)
%            \SpecialCoor
%            \rput(3;90){\small $D$}
%            \psarc{->}(0,0){3}{10}{70} %-rd->
%               \rput[lb](3.3;20){\scriptsize $1$ voter} 
%            \rput[B](3;210){\small $T$}
%            \psarc{->}(0,0){3}{220}{320} %-tr->
%               \rput[t](3.1;270){\scriptsize $1$ voter} 
%            \rput[B](3;330){\small $R$}
%            \psarc{->}(0,0){3}{110}{175} %-dt->
%               \rput[rb](3.3;160){\scriptsize $-1$ voter} 
%          \endpspicture
         +4\cdot\votinggraphic{36}
%          \pspicture[.4](-10,-4.2)(10,4) %\psgrid(-10,-10)(10,10)
%            \SpecialCoor
%            \rput(3;90){\small $D$}
%            \psarc{->}(0,0){3}{10}{70} %-rd->
%               \rput[lb](3.3;20){\scriptsize $-1$ voter} 
%            \rput[B](3;210){\small $T$}
%            \psarc{->}(0,0){3}{220}{320} %-tr->
%               \rput[t](3.1;270){\scriptsize $1$ voter} 
%            \rput[B](3;330){\small $R$}
%            \psarc{->}(0,0){3}{110}{175} %-dt->
%               \rput[rb](3.3;160){\scriptsize $1$ voter} 
%          \endpspicture
         +8\cdot\votinggraphic{37}
%          \pspicture[.4](-9,-4.2)(10,4) %\psgrid(-10,-10)(10,10)
%            \SpecialCoor
%            \rput(3;90){\small $D$}
%            \psarc{->}(0,0){3}{10}{70} %-rd->
%               \rput[lb](3.3;20){\scriptsize $1$ voter} 
%            \rput[B](3;210){\small $T$}
%            \psarc{->}(0,0){3}{220}{320} %-tr->
%               \rput[t](3.1;270){\scriptsize $-1$ voter} 
%            \rput[B](3;330){\small $R$}
%            \psarc{->}(0,0){3}{110}{175} %-dt->
%               \rput[rb](3.3;160){\scriptsize $1$ voter} 
%          \endpspicture
         =\votinggraphic{38}
%          \pspicture[.4](-9,-4.2)(10,4) %\psgrid(-10,-10)(10,10)
%            \SpecialCoor
%            \rput(3;90){\small $D$}
%            \psarc{->}(0,0){3}{10}{70} %-rd->
%               \rput[lb](3.3;20){\scriptsize $7$ voters} 
%            \rput[B](3;210){\small $T$}
%            \psarc{->}(0,0){3}{220}{320} %-tr->
%               \rput[t](3.1;270){\scriptsize $-1$ voter} 
%            \rput[B](3;330){\small $R$}
%            \psarc{->}(0,0){3}{110}{175} %-dt->
%               \rput[rb](3.3;160){\scriptsize $9$ voters} 
%          \endpspicture
       \tag*{}\end{equation*}
       The majority cycle has indeed disappeared.
      \partsitem Reverse the prior exercise.
         That is, add a voter to the result of the prior exercise that
         cancels the one who made the cycle disappear. 
      \partsitem One such condition is that, after cancellation,
         all three be nonnegative or all three be nonpositive, 
         and:~$\absval{c}<\absval{a+b}$ and $\absval{b}<\absval{a+c}$ 
         and $\absval{a}<\absval{b+c}$.
         That follows from this diagram.  
         \begin{equation*}
           \votinggraphic{27}  %\votecycle{a}{a}{-a}{4} 
           +  
           \votinggraphic{28}  %\votecycle{-b}{b}{b}{4} 
           +  
           \votinggraphic{29}  %\votecycle{c}{-c}{c}{4}  
           =\hspace{.30in}
           \votinggraphic{30}  %\votecycle{a-b+c}{a+b-c}{-a+b+c}{4}
           \hspace{.30in}\hbox{}  
         \end{equation*}
%          \begin{center}
%            \votecycle{a}{a}{-a}{4} 
%            +  
%            \votecycle{-b}{b}{b}{4} 
%            +  
%            \votecycle{c}{-c}{c}{4}  
%            =\hspace{.30in}  
%            \votecycle{a-b+c}{a+b-c}{-a+b+c}{4}\hspace{.30in}\hbox{}  
%          \end{center}
      \end{exparts}
    \end{answer}
\item A one-voter election cannot have a majority cycle because of the
    requirement that we've imposed that the voter's list must be rational.
    \begin{exparts}
      \partsitem Show that a two-voter election may have a majority cycle.
        (We consider the group preference a majority cycle if all three
        group totals are nonnegative or if all three are nonpositive---that is,
        we allow some zero's in the group preference.)
      \partsitem Show that for any number of voters greater than one, there is
        an election involving that many voters that results in a majority 
        cycle.
    \end{exparts}
    \begin{answer}
      \begin{exparts}
        \partsitem A two-voter election can have a majority cycle in two ways.
          First, the two voters could be opposites,
          resulting after cancellation in the trivial election (with the 
          majority cycle of all zeroes).
          Second, the two voters could have the same spin but come from
          different rows, as here.
          \begin{equation*}
            % \psset{xunit=4pt,yunit=4pt,runit=4pt} 
            1\cdot \votinggraphic{39}
%             \pspicture[.4](-10.5,-4.2)(10,4) %\psgrid(-10,-10)(10,10)
%               \SpecialCoor
%               \rput(3;90){\small $D$}
%               \psarc{->}(0,0){3}{10}{70} %-rd->
%                  \rput[lb](3.3;20){\scriptsize $1$ voter} 
%               \rput[B](3;210){\small $T$}
%               \psarc{->}(0,0){3}{220}{320} %-tr->
%                  \rput[t](3.1;270){\scriptsize $1$ voter} 
%               \rput[B](3;330){\small $R$}
%               \psarc{->}(0,0){3}{110}{175} %-dt->
%                  \rput[rb](3.3;160){\scriptsize $-1$ voter} 
%             \endpspicture
            +1\cdot\votinggraphic{40}
%             \pspicture[.4](-10,-4.2)(10,4) %\psgrid(-10,-10)(10,10)
%               \SpecialCoor
%               \rput(3;90){\small $D$}
%               \psarc{->}(0,0){3}{10}{70} %-rd->
%                  \rput[lb](3.3;20){\scriptsize $-1$ voter} 
%               \rput[B](3;210){\small $T$}
%               \psarc{->}(0,0){3}{220}{320} %-tr->
%                  \rput[t](3.1;270){\scriptsize $1$ voter} 
%               \rput[B](3;330){\small $R$}
%               \psarc{->}(0,0){3}{110}{175} %-dt->
%                  \rput[rb](3.3;160){\scriptsize $1$ voter} 
%             \endpspicture
            +0\cdot\votinggraphic{41}
%             \pspicture[.4](-9,-4.2)(10,4) %\psgrid(-10,-10)(10,10)
%               \SpecialCoor
%               \rput(3;90){\small $D$}
%               \psarc{->}(0,0){3}{10}{70} %-rd->
%                  \rput[lb](3.3;20){\scriptsize $1$ voter} 
%               \rput[B](3;210){\small $T$}
%               \psarc{->}(0,0){3}{220}{320} %-tr->
%                  \rput[t](3.1;270){\scriptsize $-1$ voter} 
%               \rput[B](3;330){\small $R$}
%               \psarc{->}(0,0){3}{110}{175} %-dt->
%                  \rput[rb](3.3;160){\scriptsize $1$ voter} 
%             \endpspicture
            =\votinggraphic{42}
%             \pspicture[.4](-9,-4.2)(10,4) %\psgrid(-10,-10)(10,10)
%               \SpecialCoor
%               \rput(3;90){\small $D$}
%               \psarc{->}(0,0){3}{10}{70} %-rd->
%                  \rput[lb](3.3;20){\scriptsize $0$ voters} 
%               \rput[B](3;210){\small $T$}
%               \psarc{->}(0,0){3}{220}{320} %-tr->
%                  \rput[t](3.1;270){\scriptsize $2$ voters} 
%               \rput[B](3;330){\small $R$}
%               \psarc{->}(0,0){3}{110}{175} %-dt->
%                  \rput[rb](3.3;160){\scriptsize $0$ voters} 
%             \endpspicture
          \tag*{}\end{equation*}
        \partsitem There are two cases.
          An even number of voters can split half and half into opposites,
          e.g., half the voters are $D>R>T$ and half are $T>R>D$.
          Then cancellation gives the trivial election.
          If the number of voters is greater than one and odd (of the
          form $2k+1$ with $k>0$) then using the cycle diagram from the proof,
          \begin{equation*}
            \votinggraphic{27}  %\votecycle{a}{a}{-a}{4} 
            +  
            \votinggraphic{28}  %\votecycle{-b}{b}{b}{4} 
            +  
            \votinggraphic{29}  %\votecycle{c}{-c}{c}{4}  
            =\hspace{.30in}
            \votinggraphic{30}  %\votecycle{a-b+c}{a+b-c}{-a+b+c}{4}
            \hspace{.30in}\hbox{}  
          \end{equation*}
%          \begin{center}
%            \votecycle{a}{a}{-a}{4} 
%            +  
%            \votecycle{-b}{b}{b}{4} 
%            +  
%            \votecycle{c}{-c}{c}{4}  
%            =\hspace{.30in}  
%            \votecycle{a-b+c}{a+b-c}{-a+b+c}{4}\hspace{.30in}\hbox{}  
%          \end{center}
          we can take $a=k$ and $b=k$ and $c=1$.
          Because $k>0$, this is a majority cycle.
       \end{exparts}
    \end{answer}
\item \label{exer:PerpIsSubSp} 
    Let $U$ be a subspace of $\Re^3$.
    Prove that the set 
    $U^\perp=\set{\vec{v}\suchthat 
              \text{$\vec{v}\dotprod\vec{u}=0$ for all $\vec{u}\in U$}}$
    of vectors 
    that are perpendicular to each vector in $U$ is also subspace of $\Re^3$.
    Does this hold if $U$ is not a subspace?
    \begin{answer}
      It is nonempty because it contains the zero vector.
      To see that it is closed under linear combinations of two of its
      members, suppose that $\vec{v}_1$ and $\vec{v}_2$ are in $U^\perp$
      and consider $c_1\vec{v}_1+c_2\vec{v}_2$.
      For any $\vec{u}\in U$,
      \begin{equation*}
        (c_1\vec{v}_1+c_2\vec{v}_2)\dotprod\vec{u}
        =c_1(\vec{v}_1\dotprod\vec{u})+c_2(\vec{v}_2\dotprod\vec{u})
        =c_1\cdot 0+c_2\cdot 0=0         
      \tag*{}\end{equation*}
      and so $c_1\vec{v}_1+c_2\vec{v}_2\in U^\perp$.

      As to whether it holds if $U$ is a subset but not a subspace,
      the answer is yes.
    \end{answer}
\end{exercises}
\index{voting paradox|)}
\endinput



