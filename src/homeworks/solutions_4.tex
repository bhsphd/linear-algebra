\documentclass[11pt]{article}
\usepackage[margin=1in]{geometry}
\usepackage{../linalgjh}

\setlength{\parindent}{0em}
\pagestyle{empty}
\begin{document}\thispagestyle{empty}
\makebox[\linewidth]{\textbf{Homework, MA~213}\hspace*{4in}\textbf{Due: 2014-Oct-06}}

\vspace*{3ex}
\textit{You may work with others to figure out how to do questions, 
and you are welcome to look for answers in the book, online, by talking
to someone who had the course before, etc.
However, you must write 
the answers on your own.
You must also show your work (you may, of course, 
quote any result from the book).}

\begin{enumerate}
\item
Find a basis for, and the dimension of, each space.
\begin{enumerate}
\item
  \begin{equation*}
    \set{\colvec{x \\ y \\ z \\ w}\in\Re^4
         \suchthat x-w+z=0}
  \end{equation*}

Parametrize to get this description of the space.
\begin{equation*}
  \set{\colvec{w-z \\ y \\ z \\ w}
       =\colvec{0 \\ 1 \\ 0 \\ 0}y+
       \colvec{-1 \\ 0 \\ 1 \\ 0}z+
       \colvec{1 \\ 0 \\ 0 \\ 1}w
       \suchthat y,z,w\in\Re}
\end{equation*}
That gives the space as the span of the three-vector set.
To show the three vector set makes a basis we check that it is 
linearly independent.
\begin{equation*}
   \colvec{0 \\ 0 \\ 0 \\ 0}
       =\colvec{0 \\ 1 \\ 0 \\ 0}c_1+
       \colvec{-1 \\ 0 \\ 1 \\ 0}c_2+
       \colvec{1 \\ 0 \\ 0 \\ 1}c_3
\end{equation*}
The second components give that $c_1=0$, and the third and fourth 
components give that $c_2=0$ and~$c_3=0$.
So one basis is this.
\begin{equation*}
    \sequence{
       \colvec{0 \\ 1 \\ 0 \\ 0},
       \colvec{-1 \\ 0 \\ 1 \\ 0},
       \colvec{1 \\ 0 \\ 0 \\ 1} 
    } 
\end{equation*}
The dimension is the number of vectors in a basis: $3$.

\item the set of $\nbym{5}{5}$ matrices whose only nonzero entries 
   are on the diagonal (e.g., in entry $1,1$ and $2,2$, etc.) 

The natural parametrization is this.
\begin{multline*}
  \set{
    \begin{mat}
      a &0 &0 &0 &0 \\
      0 &b &0 &0 &0 \\
      0 &0 &c &0 &0 \\
      0 &0 &0 &d &0 \\
      0 &0 &0 &0 &e 
    \end{mat}
    \suchthat a,\ldots,e\in\Re}   \\
  \set{
    \begin{mat}
      1 &0 &0 &0 &0 \\
      0 &0 &0 &0 &0 \\
      0 &0 &0 &0 &0 \\
      0 &0 &0 &0 &0 \\
      0 &0 &0 &0 &0 
    \end{mat}\cdot a
    +\begin{mat}
      0 &0 &0 &0 &0 \\
      0 &1 &0 &0 &0 \\
      0 &0 &0 &0 &0 \\
      0 &0 &0 &0 &0 \\
      0 &0 &0 &0 &0 
    \end{mat}\cdot b
    +\cdots
    \suchthat a,\ldots,e\in\Re}
\end{multline*}
Checking that the five-element set is linearly independent is trivial.
So this is a basis; the dimension is $5$.
\begin{equation*}
  \sequence{
    \begin{mat}
      1 &0 &0 &0 &0 \\
      0 &0 &0 &0 &0 \\
      0 &0 &0 &0 &0 \\
      0 &0 &0 &0 &0 \\
      0 &0 &0 &0 &0 
    \end{mat},
    \begin{mat}
      0 &0 &0 &0 &0 \\
      0 &1 &0 &0 &0 \\
      0 &0 &0 &0 &0 \\
      0 &0 &0 &0 &0 \\
      0 &0 &0 &0 &0 
    \end{mat},
    \,\ldots\,,
    \begin{mat}
      0 &0 &0 &0 &0 \\
      0 &0 &0 &0 &0 \\
      0 &0 &0 &0 &0 \\
      0 &0 &0 &0 &0 \\
      0 &0 &0 &0 &1 
    \end{mat}
}
\end{equation*}


\item $\set{a_0+a_1x+a_2x^2+a_3x^3\suchthat 
            \text{$a_0+a_1=0$ and $a_2-2a_3=0$}}\subseteq\polyspace_3$ 

The restrictions form a two-equations, four-unknowns linear system.
Parametrizing that system to express the leading variables in terms of 
those that are
free gives $a_0=-a_1$, $a_1=a_1$, $a_2=2a_3$, and~$a_3=a_3$.
\begin{equation*}
  \set{-a_1+a_1x+2a_3x^2+a_3x^3\suchthat a_1,a_3\in\Re}
  =\set{(-1+x)\cdot a_1+(2x^2+x^3)\cdot a_3\suchthat a_1,a_3\in\Re}
\end{equation*}
That description shows that the space is the span of the two-element set
$\set{-1+x,x^2+x^3}$.  
We will be done if we show the set is linearly independent.
This relationship
\begin{equation*}
  0+0x+0x^2+0x^3=(-1+x)\cdot c_1+(2x^2+x^3)\cdot c_2
\end{equation*}
gives that $c_1=0$ from the constant terms, and $c_2=0$ from the 
cubic terms.
One basis for the space is $\sequence{-1+x,2x^2+x^3}$.
This is a two-dimensional space.
\end{enumerate}

\item Give a basis for the column space of this matrix.
  Give the matrix's rank.
  \begin{equation*}
    \begin{mat}
      1 &3 &-1 &2 \\
      2 &1 &1  &0 \\
      0 &1 &1  &4
    \end{mat}
  \end{equation*}

We want a basis for this span.
\begin{equation*}
  \spanof{\colvec{1 \\ 2 \\ 0},
          \colvec{3 \\ 1 \\ 1},
          \colvec{-1 \\ 1 \\ 1},
          \colvec{2 \\ 0 \\ 4}}\subseteq\Re^3
\end{equation*}
The most straightforward approach 
is to transpose those columns to rows, use Gauss's Method
to find a basis for the span of the rows, and then transpose them back to 
columns. 
\begin{equation*}
  \begin{mat}
    1 &2 &0 \\
    3 &1 &1 \\
   -1 &1 &1 \\
    2 &0 &4
  \end{mat}
  \grstep[\rho_1+\rho_3 \\ -2\rho_1+\rho_4]{-3\rho_1+\rho_2}
  \grstep[-(4/5)\rho_2+\rho_4]{(3/5)\rho_2+\rho_3}
  \grstep{-2\rho_3+\rho_4}
  \begin{mat}
    1 &2  &0 \\
    0 &-5 &1 \\
    0 &0  &8/5 \\
    0 &0  &0
  \end{mat}
\end{equation*}
Discard the zero vector as showing that there was  a redundancy among the 
starting vectors, to get this basis for the column space.
\begin{equation*}
  \sequence{
    \colvec{1 \\ 2 \\ 0},
    \colvec{0 \\ -5 \\ 1},
    \colvec{0 \\ 0 \\ 8/5}
    }
\end{equation*}
The matrix's rank is the dimension of its column space, so it is three.
(It is also equal to the dimension of its row space.)


\item Give a basis for the span of each set, in the natural vector space.
  \begin{enumerate}
  \item $\set{\colvec{1 \\ 1 \\ 3},
              \colvec{-1 \\ 2 \\ 0},
              \colvec{0  \\ 12 \\ 6}}$

Transpose the columns to rows, bring to echelon form 
(and then lose any zero rows), and transpose back to columns.
\begin{equation*}
  \begin{mat}
    1 &1  &3 \\
   -1 &2  &0 \\
    0 &12 &6
  \end{mat}
  \grstep{\rho_1+\rho_2}
  \grstep{-4\rho_2+\rho_3}
  \begin{mat}
    1 &1  &3 \\
    0 &3  &3 \\
    0 &0  &-6
  \end{mat}
\end{equation*}
One basis for the span is this.
\begin{equation*}
  \sequence{
    \colvec{1 \\ 1 \\ 3},
    \colvec{0 \\ 3 \\ 3},
    \colvec{0 \\ 0 \\ -6}
}
\end{equation*}

  \item $\set{x+x^2, 2-2x, 7, 4+3x+2x^2}$

As in the prior part we think of those as rows, to take advantage of 
the work we've done with Gauss's Method.
\begin{equation*}
  \begin{mat}
    0 &1  &1 \\
    2 &-2 &0 \\
    7 &0 &0 \\
    4 &3  &2
  \end{mat}
  \grstep{\rho_1\leftrightarrow\rho_2}
  \grstep{2\rho_1+\rho_4}{-(7/2)\rho_1+\rho_3}
  \grstep[-7\rho_2+\rho_4]{-7\rho_2+\rho_3}
  \grstep{-(5/7)\rho_3+\rho_4}
  \begin{mat}
    2 &-2  &0 \\
    0 &1 &1 \\
    0 &0  &-7 \\
    0 &0 &0
  \end{mat}
\end{equation*}
One basis for the span of that set is 
$\sequence{2-2x, x+x^2, -5x^2}$.
  \end{enumerate}
\end{enumerate}
\end{document}
