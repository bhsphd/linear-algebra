% \documentclass[9pt,t]{beamer}
\usefonttheme{professionalfonts}
\usefonttheme{serif}
\PassOptionsToPackage{pdfpagemode=FullScreen}{hyperref}
\PassOptionsToPackage{usenames,dvipsnames}{color}
% \DeclareGraphicsRule{*}{mps}{*}{}
\usepackage{linalgjh}
\usepackage{present}
\usepackage{directories}  % Define \mapdir, \mapmpdir, etc.
\usepackage{xr}\externaldocument{\mapdir map6} % read refs from .aux file
\usepackage{catchfilebetweentags}
\usepackage{etoolbox} % from http://tex.stackexchange.com/questions/40699/input-only-part-of-a-file-using-catchfilebetweentags-package
\makeatletter
\patchcmd{\CatchFBT@Fin@l}{\endlinechar\m@ne}{}
  {}{\typeout{Unsuccessful patch!}}
\makeatother

\mode<presentation>
{
  \usetheme{boxes}
  \setbeamercovered{invisible}
  \setbeamertemplate{navigation symbols}{} 
}
\addheadbox{filler}{\ }  % create extra space at top of slide 
\hypersetup{colorlinks=true,linkcolor=blue} 

\title[Projection] % (optional, use only with long paper titles)
{Three.VI Projection}

\author{\textit{Linear Algebra} \\ {\small Jim Hef{}feron}}
\institute{
  \texttt{http://joshua.smcvt.edu/linearalgebra}
}
\date{}


\subject{Projection}
% This is only inserted into the PDF information catalog. Can be left
% out. 

\begin{document}
\begin{frame}
  \titlepage
\end{frame}

% =============================================
% \begin{frame}{Reduced Echelon Form} 
% \end{frame}



% ..... Three.VI.1 .....
\section{Orthogonal Projection Into a Line}
%..........
\begin{frame}{Project a vector into a line}
\ExecuteMetaData[\mapdir map6.tex]{PictureOrthogonalProjectionIntoALine0}
\begin{center}
  \includegraphics{\mapmpdir ch3.28}
  \hspace*{0.6in}
  \includegraphics{\mapmpdir ch3.29}      
\end{center}
\pause
\ExecuteMetaData[\mapdir map6.tex]{PictureOrthogonalProjectionIntoALine1}

\pause
\ExecuteMetaData[\mapdir map6.tex]{PictureOrthogonalProjectionIntoALine2}
\end{frame}




\begin{frame}
We have decomposed $\vec{v}$ into two parts
$\vec{v}=(c_{\vec{p}}\vec{s})+(v-c_{\vec{p}}\vec{s})$.
\centergraphic{\mapmpdir ch3.29}
Intuitively, some of $\vec{v}$ lies with the line and that gives
the first part $c_{\vec{p}}\vec{s}$.
The part of $\vec{v}$ that lies with a line orthogonal to $\ell$
is $\vec{v}-c_{\vec{p}}\vec{s}$.
What's compelling about pairing these two parts 
is that they don't interact, in that the projection of one into the line
spanned by the other is the zero vector.

\pause\bigskip\noindent\textit{Note:} 
We have not given a 
definition of `angle' in spaces
other than $\Re^n$'s,
so we will stick here to those spaces.
Extending the definitions to other spaces is perfectly possible but
we don't need them here.
\end{frame}




\begin{frame}
\df[def:ProjIntoLine]
\ExecuteMetaData[\mapdir map6.tex]{df:ProjIntoLine}

\ex
The projection of this $\Re^3$ vector into the line
\begin{equation*}
  \vec{v}=\colvec{1 \\ -2 \\ 1}
  \qquad
  L=\set{c\cdot\vec{s}=c\cdot\colvec{3 \\ 1 \\ 1}\suchthat c\in\Re}
\end{equation*}
is this vector.
\begin{equation*}
  \frac{\colvec{1 \\ -2 \\ 1}\dotprod\colvec{3 \\ 1 \\ 1}}{\colvec{3 \\ 1 \\ 1}\dotprod\colvec{3 \\ 1 \\ 1}}\cdot\colvec{3 \\ 1 \\ 1}
  =\colvec{1/3 \\ 1/6 \\ 1/6}
\end{equation*}
\end{frame}
\begin{frame}
\noindent 
Because $\vec{v}$ is nearly orthogonal to 
the line~$L$
\centergraphic{asy/three_vi_3dprojtoline.pdf}
only a small part of $\vec{v}$ lies with the direction of that 
line, so the projected-to red
vector $\proj{\vec{v}}{\spanof{\vec{s}\,}}$ is quite short:
($\absval{\vec{v}}=\sqrt{6}\approx 2.45$ while
$\absval{\proj{\vec{v}}{\spanof{\vec{s}\,}}}=\sqrt{1/6}\approx 0.41$).  
\end{frame}






% ..... Three.VI.2 .....
\section{Gram-Schmidt Orthogonalization}
%..........
\begin{frame}{Mutually orthogonal vectors}
The prior subsection suggests that
projecting a vector $\vec{v}$ into the line spanned by \( \vec{s} \)
decomposes $\vec{v}$ into two parts, a part with the line and a part 
orthogonal to that.
\begin{center}  \small
  \vcenteredhbox{\includegraphics{\mapmpdir ch3.35}}
   \qquad
   $\displaystyle \vec{v}=\proj{\vec{v}}{\spanof{\vec{s}\,}}
             \,+\,\left(\vec{v}-\proj{\vec{v}}{\spanof{\vec{s}\,}}\right)$
\end{center}
Because these are orthogonal they are in some sense non-interacting.
Here we will develop that.
\pause
\df[df:MutuallyOrthogonal]
\ExecuteMetaData[\mapdir map6.tex]{df:MutuallyOrthogonal}
\ex
The vectors of the standard basis $\stdbasis_3\subset\Re^3$
are mutually orthogonal.
\begin{equation*}
  \colvec{1 \\ 0 \\ 0}, \colvec{0 \\ 1 \\ 0}, \colvec{0 \\ 0 \\ 1}
\end{equation*}
% This remains true if we rotate this basis.
\end{frame}
\begin{frame}
\ex
These two vectors in $\Re^2$ are mutually orthogonal.
\begin{center}
  $\displaystyle \colvec{1 \\ 1},\,\colvec{1 \\ -1}$
  \hspace*{.4in}
  \vcenteredhbox{\includegraphics{asy/three_vi_2dmutuallyortho.pdf}}
\end{center}
\end{frame}



\begin{frame}
The next result makes `non-interacting' precise.

\th[th:OrthoIsInd]
\ExecuteMetaData[\mapdir map6.tex]{th:OrthoIsInd}
\pause
\pf
\ExecuteMetaData[\mapdir map6.tex]{pf:OrthoIsInd}
\qed
\pause
\co[cor:OrthAndBigEnoughIsBasis]
\ExecuteMetaData[\mapdir map6.tex]{co:OrthAndBigEnoughIsBasis}

\pause
\pf
\ExecuteMetaData[\mapdir map6.tex]{pf:OrthAndBigEnoughIsBasis}
\qed

\pause
\df[df:OrthogonalBasis]
\ExecuteMetaData[\mapdir map6.tex]{df:OrthogonalBasis}
\end{frame}




\begin{frame}
\th[th:GramSchmidt]
\ExecuteMetaData[\mapdir map6.tex]{th:GramSchmidt}

\iftoggle{showallproofs}{
  \pause
  \pf
  \ExecuteMetaData[\mapdir map6.tex]{pf:GramSchmidt0}

  \pause
  \ExecuteMetaData[\mapdir map6.tex]{pf:GramSchmidt1}
}{

  \bigskip
  The book has the proof.
  We will instead illustrate.
}
\end{frame}
\iftoggle{showallproofs}{
  \begin{frame}
  \ExecuteMetaData[\mapdir map6.tex]{pf:GramSchmidt2}

  \pause
  \ExecuteMetaData[\mapdir map6.tex]{pf:GramSchmidt3}
  \end{frame}
  \begin{frame}
  \ExecuteMetaData[\mapdir map6.tex]{pf:GramSchmidt4}
  \end{frame}
  \begin{frame}
  \ExecuteMetaData[\mapdir map6.tex]{pf:GramSchmidt5}
  \qed
  \end{frame}
}{}

\begin{frame}
\ex
This basis for $\Re^2$
\begin{equation*}
  B=\sequence{\colvec{1 \\ 2},
              \colvec{1 \\ 3}}
\end{equation*}
does not have orthogonal vectors.
To derive from it a basis~$K=\sequence{\vec{\kappa}_1,\vec{\kappa}_2}$ 
that is orthogonal, start by taking the 
first vector unchanged.
\begin{equation*}
  \vec{\kappa}_1=\vec{\beta}_1=\colvec{1 \\ 2}
\end{equation*}
For $\vec{\kappa}_2$ take the part of $\vec{\beta}_2$
that does not lie with~$\vec{\kappa}_1$.
\begin{equation*}
  \vec{\kappa}_2=\vec{\beta}_2-\proj{\vec{\beta}_2}{\spanof{\vec{\kappa}_1}}
  =\colvec{1 \\ 3}
  -\frac{\colvec{1 \\ 3}\dotprod\colvec{1 \\ 2}}{%
         \colvec{1 \\ 2}\dotprod\colvec{1 \\ 2}}\colvec{1 \\ 2}
  =\colvec{-2/5 \\ 1/5}
\end{equation*}
Note that $\vec{\kappa}_1$ and~$\vec{\kappa}_2$ are indeed orthogonal.
\end{frame}
\begin{frame}
\ex  
This is a basis for $\Re^3$.
\begin{equation*}
  B=\sequence{\colvec{1 \\ 1 \\ 2}, 
    \colvec{-1 \\ 2 \\ 1}, 
    \colvec{0 \\ 3 \\ -1}}
\end{equation*}
Start the orthogonal basis with $\vec{\beta}_1$.
\begin{equation*}
    \vec{\kappa}_1=\colvec{1 \\ 1 \\ 2}
\end{equation*}
\pause
As in the prior slide, the next step is 
$\vec{\kappa}_2=\vec{\beta}_2-\proj{\vec{\beta}_2}{\spanof{\vec{\kappa}_1}}$.
\begin{equation*}
  \colvec{-1 \\ 2 \\ 1}-
  \frac{\colvec{-1 \\ 2 \\ 1}\dotprod\colvec{1 \\ 1 \\ 2}}{\colvec{1 \\ 1 \\ 2}\dotprod\colvec{1 \\ 1 \\ 2}}\cdot\colvec{1 \\ 1 \\ 2}
  =\colvec{-3/2 \\ 3/2 \\ 0}
\end{equation*}
\end{frame}
\begin{frame}
The third step is 
$\vec{\kappa}_3=\vec{\beta}_3-\proj{\vec{\beta}_3}{\spanof{\vec{\kappa}_1}}
                               -\proj{\vec{\beta}_3}{\spanof{\vec{\kappa}_2}}$.
\begin{equation*}
  \colvec{0 \\ 3 \\ -1}-
  \frac{\colvec{0 \\ 3 \\ -1}\dotprod\colvec{1 \\ 1 \\ 2}}{\colvec{1 \\ 1 \\ 2}\dotprod\colvec{1 \\ 1 \\ 2}}\cdot\colvec{1 \\ 1 \\ 2}
  -\frac{\colvec{0 \\ 3 \\ -1}\dotprod\colvec{-3/2 \\ 3/2 \\ 0}}{\colvec{-3/2 \\ 3/2 \\ 0}\dotprod\colvec{-3/2 \\ 3/2 \\ 0}}\cdot\colvec{-3/2 \\ 3/2 \\ 0}
  =\colvec{4/3 \\ 4/3 \\ -4/3}
\end{equation*}
\pause
The members of $B$ are at odd angles but the members of 
$K$ are mutually orthogonal.
\begin{center}
  \only<2>{\makebox[2in]{$B=\sequence{\colvec{1 \\ 1 \\ 2},
              \colvec{-1 \\ 2 \\ 1},
              \colvec{0 \\ 3 \\ -1}}$}}
  \only<3->{\makebox[2in]{$K=\sequence{\colvec{1 \\ 1 \\ 2},
              \colvec{-3/2 \\ 3/2 \\ 0},
              \colvec{4/3 \\ 4/3 \\ -4/3}}$}}
  \hspace{.25in}
  \vcenteredhbox{%
  \only<2>{\includegraphics{asy/three_vi_3dgramschmidt0.pdf}}%
  \only<3->{\includegraphics{asy/three_vi_3dgramschmidt1.pdf}}%
  }\pause
\end{center}
\pause
We could go on to 
make this basis even more like $\stdbasis_3$ by normalizing all
of its members to have length~$1$, making an \definend{orthonormal} basis.
\end{frame}





% % ..... Three.VI.3 .....
% \section{Projection Into a Subspace}
% %..........
% \begin{frame}
% \end{frame}




%...........................
% \begin{frame}
% \ExecuteMetaData[../gr3.tex]{GaussJordanReduction}
% \df[def:RedEchForm]
% 
% \end{frame}
\end{document}
