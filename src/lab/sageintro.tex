\chapter{\python{} and \Sage{}}

To work through the Linear Algebra in this manual
you must be acquainted with running \Sage. 
\Sage{} uses the computer language \python{} so we'll start with that.


\section{\python}
\python{} is a popular computer language,\footnote{Written by a fan of
Monty Python.} often used for scripting,
that is appealing for its simple style and powerful libraries.
The significance of `scripting' is that \Sage{} uses it in this way,
as a glue to bring together separate parts.

\python{} is Free.
If your operating system doesn't provide it then go to the home 
page \href{http://www.python.org}{\url{www.python.org}} and follow the
download and installation instructions.
Also at that site is \python's excellent tutorial.
That tutorial is thorough; 
here you will see only enough \python{} to get started.
For a more comprehensive introduction see \cite{PythonTeam12b}.

\smallskip
\textit{Comment.}
\python{} comes in two versions, \python~2 and \python~3.
The new version is better and the old version sunsets very soon, and
\Sage{} is migrating from \python~2 over to \python~3, so we will
work with \python~3. 
The main difference for us is that in \python~2 you write
\inlinecode{print 1+2} while in \python~3 it is \inlinecode{print(1+2)}. 
The examples below were produced directly 
from \python{} and \Sage{} when this document was generated so they should be 
exactly what you see, 
unless your software version is quite different than mine.
Here is my version of \python. 
\begin{pyconsole}
import sys
print(sys.version)
\end{pyconsole}

\subsection{Basics}
Start \python, for instance by entering 
\inlinecode{python}
at a command line.
You'll get a couple of lines of 
information followed by three greater-than
characters.
\begin{lstlisting}[style=python]
>>>   
\end{lstlisting}
This is a prompt.
It lets you experiment:~if you type 
\python{} code and \keyboardkey{Enter} then the system
will read your code, evaluate it, and print the result.
We will see below how to write and run whole programs
but for now we will experiment.
You can always leave the prompt with \textit{\keyboardkey{Ctrl}-D}.

Try entering these expressions (double star is exponentiation).
\begin{pyconsole}
2 - (-1)
1 + 2*3
2**3  
\end{pyconsole}

Part of \python's appeal is that doing simple things tends to be simple.
Here is how you print something to the screen.
\begin{pyconsole}
print(1, "plus", 2, "equals", 3)
\end{pyconsole}
Often you can debug just by putting in commands to print things, 
and having a straightforward print operator helps with that. 

As in any other computer language, variables give you a place to keep values.
The first line below puts one in the place called \inlinecode{i}
and the second line uses that.
\begin{pyconsole}
i = 1
i + 1
\end{pyconsole}
In some programming languages you must declare the `type' of a variable
before you use it; for instance you would have to declare 
that $i$ is an integer before you could set $i=1$.
In contrast, \python{} deduces the type of a variable 
based on what you do to it\Dash above we assigned $1$ to $i$ 
so \python{} figured that it must be an integer.
Further, we can change how we use the variable and \python{} will 
go along; here we change what is in $x$ from an integer to a string.
\begin{pyconsole}
x = 1
x
x = 'a'
x
\end{pyconsole}

\python{} lets you assign multiple values simultaneously.
\begin{pyconsole}
first_day, last_day = 0, 365
first_day
last_day   
\end{pyconsole}
\noindent
\python{} computes the right side, left to right, and then assigns 
those values to the variables on the left.
We will often use this construct.

\python{} complains by \textit{raising an exception} and giving an error 
message.
For instance, we cannot combine a string and an integer. 
\begin{pyconsole}
'a'+1
\end{pyconsole}
\noindent The error message's bottom line is the useful one.

Make a comment of the rest of the line with a hash mark \inlinecode{\#}.
\begin{pyconsole}
t = 2.2
d = (0.5) * 9.8 * (t**2)  # d in meters
d
\end{pyconsole}
\noindent (Comments are more useful in a program than at the prompt.)
Programmers often comment an entire line by starting 
that line with a hash. 

As in the listing above, we can get real 
numbers 
and even complex numbers.\footnote{Of course these aren't actually 
real numbers, instead they are floating point numbers, a system that
models the reals and is built into your computer's hardware.
In the prior example the distinction leaks through
since its bottom line ends in $000000000005$, marking where
the computer's binary approximation does not perfectly match
the real that you expect.
Similarly \protect\python's integers aren't the integers that you studied
in grade school since there is a largest one (give \python{} the 
command \protect\inlinecode{import sys} followed by 
\protect\inlinecode{sys.maxint}). 
However, while the issue of representation is fascinating\protect\Dash
see \protect\cite{PythonTeam12a} and \protect\cite{Goldberg91}\protect\Dash
we shall ignore it and just call them integers and reals.}
\begin{pyconsole}
5.774 * 3
(3+2j) - (1-4j)
\end{pyconsole}
\noindent As engineers do, \python{} uses~$j$ for the square
root of~$-1$, not~$i$ as is traditional in Mathematics.\footnote{Athough
\protect\Sage{} lets you use $i$.}

The examples above show addition, subtraction, multiplication, 
and exponentiation. 
Division has an awkward point.
\python{} was originally designed to have the division bar
\inlinecode{/} mean real number division 
when at least one of the two numbers is real.
However between two integers the division bar was taken to mean 
a quotient, as in ``$2$ goes into $5$ with quotient~$2$ and remainder~$1$.''
\begin{pyconsole}
5.2 / 2.0
5.2 / 2
5 / 2
\end{pyconsole}
Experience shows this was a mistake. 
One of the changes in \python~3
is that the quotient operation has become \inlinecode{//}
while the single-slash operator is always real division.
In the \python~2 we are using, you must make sure that at least one
number in a division is real.
\begin{pyconsole}
x = 5
y = 2
(1.0*x) / y
\end{pyconsole}
\noindent Incidentally, 
the integer remainder operation (sometimes called `modulus')  
uses a percent character:~\inlinecode{5 \% 2} returns 1.

Variables can also represent truth values; these are \textit{Booleans}.
\begin{pyconsole}
yankees_stink = True
yankees_stink
\end{pyconsole}
\noindent You need the initial capital:
\inlinecode{True}
or \inlinecode{False}, not
\inlinecode{true}
or \inlinecode{false}.
 
Above we saw a string consisting of text between single quotes.
You can use either single quotes or double quotes, as long as you use
the same at both ends of the string. 
Here \inlinecode{x} and \inlinecode{y}
are double-quoted, which makes sense because they contain apostrophes. 
\begin{pyconsole}
x = "I'm Popeye the sailor man"
y = "I yam what I yam and that's all what I yam"
x + ', ' + y
\end{pyconsole}
\noindent The \inlinecode{+} operation concatenates strings.
Inside a double-quoted string use slash-\textit{n} \inlinecode{\\n} to
get a newline.

A string marked by three sets of quotes can contain line breaks.
\begin{pyconsole}
a = """THE ROAD TO WISDOM
 
The road to wisdom?
-- Well, it's plain
and simple to express:
Err
and err
and err again
but less
and less
and less. --Piet Hein"""
\end{pyconsole}
\noindent 
The three dots at the start of lines after the first
is \python's
read-eval-print prompt telling you that what you have typed is not complete.
We'll see below that a common use for
triple-quoted strings is as documentation.

A \python{} \textit{dictionary} is a finite function.
That is, it is a finite
set of ordered pairs $\langle\textit{key},\textit{value}\rangle$ subject 
to the restriction that no key can appear twice.
Dictionaries are a simple database.
\begin{pyconsole}
english_words = {'one': 1, 'two': 2, 'three': 3}
english_words['one']
english_words['four'] = 4  
english_words
\end{pyconsole}
\noindent Don't be mislead by this example, 
the words do not always just come in the reverse of the order
in which you entered them.
A dictionary's elements will be listed in 
no apparently-sensible order.

If you assign to an existing key then that will replace the previous value. 
\begin{pyconsole}[d,0,1]
english_words = {'one': 1, 'two': 2, 'three': 3, 'four': 4}
english_words['one'] = 5
english_words
\end{pyconsole}
\noindent 
Dictionaries are central to \python, in part because looking up values 
in a dictionary is very fast.

While dictionaries are unordered, a \textit{list} is ordered.
\begin{pyconsole}
a = ['alpha', 'beta', 'gamma']
b = []
c = ['delta']
a
a+b+c
\end{pyconsole}
\noindent
Get an element from a list by specifying its index, its place in the list,
inside square brackets.
Lists are \textit{zero-offset} indexed, that is, the initial element of the
list is numbered~$0$.
Count from the back by using negative indices.
\begin{pyconsole}
a = ['alpha', 'beta', 'gamma']
a[0]
a[1]
a[-1]
\end{pyconsole}
\noindent
Specifying two indices separated by a colon gets a \textit{slice} 
of the list. 
\begin{pyconsole} 
a = ['alpha', 'beta', 'gamma']
a[1:3]
a[1:2]
\end{pyconsole}
\noindent
You can add to a list.
\begin{pyconsole}
c = ['delta']
c.append('epsilon')
c
\end{pyconsole}
\noindent
Lists can contain anything, including other lists.
\begin{pyconsole}
x = 4
a = ['alpha', [True, x]]
a
\end{pyconsole}

The function \pythoncmd{range} returns a list of numbers.
\begin{pyconsole}
range(10)
range(1,10)
\end{pyconsole}
\noindent
By default \pythoncmd{range} starts at $0$, which
is good because lists are zero-indexed.
Observe also that $9$ is the highest number in the list
given by \inlinecode{range(10)}.
This makes \inlinecode{range(10)+range(10,20)} give 
the same list as \inlinecode{range(20)}.

A \textit{tuple} is like a list in that it is ordered.
\begin{pyconsole}
a = ('fee', 'fie', 'foe', 'fum')
a
a[0]
\end{pyconsole}
\noindent
However it is unlike a list 
in that a tuple is not \textit{mutable}\Dash it cannot change.
\begin{pyconsole}[d,0,1]
a = ('fee', 'fie', 'foe', 'fum')
a[0] = 'phooey'
\end{pyconsole}
\noindent
One reason this is useful is that because of it
tuples can be keys in dictionaries while
list cannot.
\begin{pyconsole}
a = ['ke1az', 5418]
b = ('ke1az', 5418)
d = {a: 'active'}
d = {b: 'active'}
d
\end{pyconsole}

\python{} has a special value \inlinecode{None} for
when there is no sensible value for a variable.
For instance, if your program keeps track of a person's address and
includes a variable \pythoncmd{apartment} then \inlinecode{None} is
the right value for that variable when the person does not live in an
apartment.



\subsection{Flow of control}
\python{} supports the traditional ways of affecting the order of 
statement execution, with a twist.
\begin{pyconsole}
x = 4
if (x == 0):
    y = 1
else:
    y = 0

y
\end{pyconsole}
\noindent
The twist is that while many languages use braces or some other syntax to
mark a block of code, \python{} uses indentation.
(Always indent with four spaces.)
Here, \python{} executes the single-line block \inlinecode{y = 1} if $x$
equals $0$, otherwise \python{} sets $y$ to~$0$. 

Notice also that double equals \inlinecode{==} means ``is equal to.'' 
In contrast, we have already seen that single equals is the assignment
operation so that \inlinecode{x = 4} 
means ``$x$ is assigned the value~$4$.'' 

\python{} has two variants on the above \pythoncmd{if} statement.
It could have only one branch
\begin{pyconsole}
x = 4
y = 0
if (x == 0):
    y = 1

y
\end{pyconsole}
or it could have more than two branches.
\begin{pyconsole}
x = 2
if (x == 0):
    y = 1
elif (x == 1):
    y = 0
else:
    y = -1

y
\end{pyconsole}

Computers excel at iteration, looping through the same steps.
\begin{pyconsole}
for i in range(5):
    print i, "squared is", i**2

\end{pyconsole}
\noindent
A \pythoncmd{for} loop often involves a \pythoncmd{range}.
\begin{pyconsole}
x = [4, 0, 3, 0]
for i in range(len(x)):
    if (x[i] == 0):
        print "item",i,"is zero"
    else:
        print "item",i,"is nonzero"

\end{pyconsole}
\noindent
An experienced \python{} person who was not trying just to illustrate 
\pythoncmd{range}
would instead write \inlinecode{for c in x:} since
the \pythoncmd{for} loop can iterate over any sequence, not just
a sequence of integers.

A \pythoncmd{for} loop is designed to execute a certain
number of times.
The natural way to write a loop that will run an uncertain number of times
is \pythoncmd{while}.
\begin{pyconsole}
n, i = 27, 0
while (n != 1):
    if (n % 2 == 0):  # if n is even
        n = n / 2
    else:
        n = 3*n + 1
    i = i + 1
    print("i=", i)

\end{pyconsole}
\noindent
(This listing is incomplete; it takes $111$ steps to finish.)\footnote{The 
\protect\textit{Collatz conjecture} is 
that for any starting~$n$ this loop will 
terminate; no one knows if it is true.}
Note that ``not equal'' is \lstinline[style=inline]@!=@. 

The \inlinecode{break} command gets you out of a loop right away.
\begin{pyconsole}
for i in range(10):
    if (i == 3):
        break
    print("i is", i)

\end{pyconsole}

A common loop construct is to run through a list
performing an action on each entry.
\python{} has a shortcut, \textit{list comprehension}.
\begin{pyconsole}
a = [2**i for i in range(4)]
a
[i-1 for i in a]
\end{pyconsole}



\subsection{Functions}
A \textit{function} is a group of statements that executes when it is called,
and can return values to the caller.
Here is a naive version of the quadratic formula.
\begin{pyconsole}
def quad_formula(a, b, c):
    discriminant = (b**2 - 4*a*c)**(0.5)
    r1=(-1*b+discriminant) / (2.0*a)
    r2=(-1*b-discriminant) / (2.0*a)
    return (r1, r2)

quad_formula(1,0,-9)
quad_formula(1,2,1)
\end{pyconsole}
\noindent
(One way that it is naive is that it doesn't handle complex roots gracefully.)
% \begin{pyconsole}[d,0,6]
% def quad_formula(a, b, c):
%     discriminant = (b**2 - 4*a*c)**(0.5)
%     r1=(-1*b+discriminant) / (2.0*a)
%     r2=(-1*b-discriminant) / (2.0*a)
%     return (r1, r2)
%
% quad_formula(1,1,1)
% \end{pyconsole}

Functions organize code into blocks. 
These blocks of code may be 
run a number of different times, or may belong together conceptually. 
In a \python{} program most code is in functions. 

At the end of the \inlinecode{def} line, in parentheses, are
the function's \textit{parameters}. 
These can take values 
passed in by the caller.
Functions can have \textit{optional parameters} that have a default value.
\begin{pyconsole}
def hello(name="Jim"):
    print("Hello,", name)

hello("Fred")
hello()
\end{pyconsole}
\noindent
\Sage{} uses this aspect of \python{} heavily.

Functions always return something; 
if a function never executes a \inlinecode{return} then it will
return the value \inlinecode{None}.
They can also contain multiple \inlinecode{return} statements, for instance 
one for an \inlinecode{if} branch and one for an \inlinecode{else}.




\subsection{Objects and modules}
In Mathematics, the real numbers is a set associated with some operations
such as addition and multiplication.
\python{} is \textit{object-oriented}, which means that we can similarly bundle
together data and actions (in this context the functions are called 
\textit{methods}).
\begin{pyconsole}
class DatabaseRecord(object):
    def __init__(self, name, age):
        self.name = name
        self.age = age
    def salutation(self):
        print("Dear", self.name)

a = DatabaseRecord("Jim", 53)
a.name
a.age
a.salutation()
b = DatabaseRecord("Fred",109)
b.salutation()
\end{pyconsole}
\noindent
This creates two \textit{instances} of the object 
\inlinecode{DatabaseRecord}.
The \inlinecode{class} code describes what these consist of.
The above example uses the \textit{dot notation}:
to get the age data for the instance \inlinecode{a}
you write 
\inlinecode{a.age}.
(The \protect\inlinecode{self} variable 
can  be puzzling.
It is how a method refers to the instance it is part of.
Suppose that at the prompt you type \inlinecode{a.name="James"}.
Then you've used the name \inlinecode{a} 
to refer to the instance so you can
make the change.
In contrast,
inside the \inlinecode{class} description code there isn't any fixed instance. 
The \inlinecode{self} gives you a way to, for example, 
get the \inlinecode{name} attribute
of the current instance.)

You won't be writing your own classes here 
but you will be using ones from
the libraries of code that others have written, including the
code for \Sage, so you must know how to use the classes of others. 
For instance, \python{} has a library, or \textit{module}, for math.
\begin{pyconsole}
import math
math.pi
math.factorial(5)
math.cos(math.pi)
\end{pyconsole}
\noindent
The \inlinecode{import} statement gets the module and makes
its contents available.

% Another library is for random numbers.
% \begin{pyconsole}
% import random
% while (random.randint(1,10) != 1):
%     print("wrong")

% \end{pyconsole}



\subsection{Programs}
The read-eval-print loop is great for small experiments but
for more than four or five lines you 
want to put your work in a separate file and run it as a standalone program.

To write the code, use a text editor (one example is 
\texttt{Emacs}).\footnote{It may come with your operating system or 
see \protect\url{http://www.gnu.org/software/emacs}.}
Use an editor with support for \python{} such as
automatic indentation, and  
syntax highlighting, where the editor colors your code to make it easier to
read.

Here is a first example of a \python{} program.
Start your editor, open a file called \path{test.py}, and enter these lines.
Note the triple-quoted documentation string at the top of the file; 
include such documentation in everything that you write.
\begin{lstlisting}[style=python]
# test.py
"""test

A test program for Python. 
"""

import datetime
 
current = datetime.datetime.now()  # get a datetime object
print "the month number is", current.month
\end{lstlisting}
Run it through \python{} (for instance, from the command line
run \inlinecode{python test.py}) and you should see
output like \inlinecode{the month number is 9}.

Next is a small game. 
(It uses the \python{} function \inlinecode{raw_input} that prompts the user
and then collects their response.).
\begin{lstlisting}[style=python]
# guessing_game.py
"""guessing_game

A toy game for demonstration.
"""
import random
CHOICE = random.randint(1,10)

def test_guess(guess):
    """Decide if the guess is correct and print a message.
    """
    if (guess < CHOICE):
        print("  Sorry, your guess is too low")
        return False
    elif (guess > CHOICE):
        print("  Sorry, your guess is too high")
        return False
    print("  You are right!")
    return True

flag = False
while (not flag):
    guess = int(raw_input("Guess an integer between 1 and 10: "))
    flag = test_guess(guess)
\end{lstlisting}
Here is the output resulting from running this game once at the command line.
\begin{lstlisting}
$ python guessing_game.py
Guess an integer between 1 and 10: 5
  Sorry, your guess is too low
Guess an integer between 1 and 10: 8
  Sorry, your guess is too high
Guess an integer between 1 and 10: 6
  Sorry, your guess is too low
Guess an integer between 1 and 10: 7
  You are right!
\end{lstlisting}  % $

As earlier, note the triple-quoted documentation strings both for the 
file as a whole and for the function.
They provide information on how to use the \path{guessing_game.py}
program as a whole,
and how to use the function inside the program. 
Go to the directory containing \path{guessing_game.py} and start 
the \python{} read-eval-print loop.
At the \inlinecode{>>>} prompt enter \inlinecode{import guessing_game}.
You will play through a round of the game (there is a way to avoid this
but it doesn't matter here).
You are using \inlinecode{guessing_game} as a module.
Type
\inlinecode{help("guessing_game")}.
You will see the documentation, including these lines.
\begin{lstlisting}
DESCRIPTION
    A toy game for demonstration.

FUNCTIONS
    test_guess(guess)
        Decide if the guess is correct and print a message. 
\end{lstlisting}
Obviously, \python{} got this from the file's documentation strings.
In \python{}, and also in \Sage, good practice is 
to always include documentation
that is accessible with the \inlinecode{help} command.
All of \Sage's built-in routines do this.



%----------------------------------
\section{\Sage}
Learning what \Sage{} can do is the goal of much of this book 
so this is only a brief walk-through of preliminaries.
See also \citep{SageTeam12} for a more broad-based introduction.

First, if your system does not already supply it then install \Sage{} 
by following the directions at
\href{http://www.sagemath.org}{www.sagemath.org}.



\subsection{Command line}
\Sage's command line is like \python's but adapted to 
mathematical work.
First start \Sage,
for instance, enter \inlinecode{sage} into a command line window.
You get some initial text and then a prompt
(leave the prompt by typing \inlinecode{exit}
and \keyboardkey{Enter}.)
\begin{lstlisting}[style=python]
sage:  
\end{lstlisting}

Experiment with some expressions.
% \begin{lstlisting}[style=python]
% sage: 2**3                                                                           
% 8
% sage: 2^3
% 8
% sage: 3*1 + 4*2
% 11
% sage: 5 == 3+3
% False
% sage: sin(pi/3)
% 1/2*sqrt(3)
% \end{lstlisting}
\begin{sagecode}
2**3                                                                      
2^3
3*1 + 4*2
5 == 3+3
sin(pi/3)
\end{sagecode}
The second expression 
shows that \Sage{} provides a convenient shortcut for exponentiation over
\python's \inlinecode{2**3}.
The fourth expression
shows that \Sage{} sometimes returns exact results rather than an
approximation.
You can still get the approximation; here are three equivalent ways.
\begin{sagecode}
sin(pi/3).numerical_approx()
sin(pi/3).n()
n(sin(pi/3))  
\end{sagecode}
\noindent
The function \inlinecode{n()} is an abbreviation for 
\inlinecode{numerical_approx()}.


\subsection{Script}
You can group \Sage{} commands together in a file.
This way you can test the commands, 
and also reuse them without having to retype.

Create a file with the extension \path{.sage}, such as \path{sage_normal.sage}.
Enter this function and save the file.
\begin{lstlisting}[style=python]
def normal_curve(upper_limit):
    """Approximate area under the Normal curve from 0 to upper_limit.
    """
    stdev = 1.0
    mu = 0.0
    area=numerical_integral((1/sqrt(2*pi) * e^(-0.5*(x)^2)),
                             0, upper_limit)    
    print("area is", area[0])
\end{lstlisting}
Bring in the file with a \Sagecmd{load} command
and then you can use the commands defined there.
% sageoutput is a problem: http://ask.sagemath.org/question/25360/sagetexs-sage-file-gives-runfile-is-not-defined/
% Briefly, runfile won't work and load gives a "discouraged" message
\begin{lstlisting}[style=python]
sage: load("sage_normal.sage")
sage: normal_curve(1.0)   
\end{lstlisting}


\subsection{Notebook}
\Sage{} also offers a browser-based interface, where you can set up
worksheets to run alone or with other people, where you can easily
view plots integrated with the text, and many other nice features.

From the \Sage{} prompt run \inlinecode{notebook()} and
work through the tutorial.
\endinput


TODO:
  1) how to use notebook to do exercises?