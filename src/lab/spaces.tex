\chapter{Vector Spaces}

\Sage{} can operate with vector spaces, for example by finding a basis for
a space.

In this chapter vector spaces take scalars from the reals numbers $\Re$
(the book's final chapter uses the complex numbers).
You can choose from two models of the real numbers.
One is \Sagecmd{RDF}, the computer's built-in floating point
model of real numbers.\footnote{This computer uses
IEEE~754 double-precision binary floating-point numbers;  
if you have programmed then you may know this number model as binary64.}  
The other is \Sagecmd{RR}, an arbitrary precision model of 
reals.\footnote{It defaults to $53$~bits of precision to reproduce 
double-precision computations.}
The second model is useful in some circumstances but the first runs faster
and is more widely used in practice so that is what we will use here.
 




%========================================
\section{Real $n$-spaces}

\Sage{} allows us to create vector spaces.
Here is~$\Re^3$.
\begin{sagecommandline}
sage: V = VectorSpace(RDF,3)
sage: V
\end{sagecommandline}
% Omit this until the bug is fixed.
% https://groups.google.com/forum/#!topic/sage-support/ObKhoCN3KCs
% and
% http://trac.sagemath.org/ticket/17576
% ............. 
% You can test for membership.
% \begin{sagecommandline}
% sage: V = VectorSpace(RDF,3)
% sage: v1 = vector(RDF, [1, 2, 3])
% sage: v1
% sage: v1 in V
% sage: v2 = vector(RDF, [1, 2, 3, 4])
% sage: v2 in V
% \end{sagecommandline}

Next we can try subspaces.
The plane through the origin in $\Re^3$ described by the equation
$x-2y+2z=0$
is a subspace of $\Re^3$.
You can describe it as a span.
\begin{equation*}
  W=\set{\colvec{x \\ y \\ z}
    \suchthat x=2y-2z}
  =\set{\colvec{2 \\ 1 \\ 0}y+\colvec[r]{-2 \\ 0 \\ 1}z
        \suchthat y,z\in\Re}
\end{equation*}
In addition to creating that subspace,
you can test membership.
\begin{sagecommandline}
sage: V = VectorSpace(RDF,3) 
sage: v1 = vector(RDF, [2, 1, 0])
sage: v2 = vector(RDF, [-2, 0, 1])
sage: W = V.span([v1, v2])
sage: v3 = vector(RDF, [0, 1, 1])
sage: v3 in W
sage: v4 = vector(RDF, [1, 0, 0])
sage: v4 in W
\end{sagecommandline}

For a closer look at what's happening, look at this subspace of $\Re^4$.
\begin{sagecommandline}
sage: V = VectorSpace(RDF,4)
sage: V
sage: v1 = vector(RR, [2, 0, -1, 0])
sage: W = V.span([v1])
sage: W
\end{sagecommandline}
As part of creating it,
\Sage{} has identified the dimension of the subspace and found
a basis containing one vector (it prefers the vector with a leading~$1$).

As earlier, the membership set relation works here.
\begin{sagecommandline}
sage: V = VectorSpace(RDF,4)
sage: v1 = vector(RDF, [2, 0, -1, 0])
sage: W = V.span([v1])
sage: v2 = vector(RDF, [2, 1, -1, 0])
sage: v2 in W
sage: v3 = vector(RDF, [-4, 0, 2, 0])
sage: v3 in W
\end{sagecommandline}



\subsection{Basis}
\Sage{} has a command to retrieve a basis for a space.
\begin{sagecommandline}
sage: V = VectorSpace(RDF,2)
sage: v = vector(RDF, [1, -1])
sage: W = V.span([v])     
sage: W.basis()
\end{sagecommandline}
\noindent
This is the basis you will see if you ask for a description of $W$.
\begin{sagecommandline}
sage: V = VectorSpace(RDF,2)
sage: v = vector(RDF, [1, -1])
sage: W = V.span([v])     
sage: W  
\end{sagecommandline}

Of course, there are subspaces with bases of size greater than one.
\begin{sagecommandline}
sage: V = VectorSpace(RDF,3)               
sage: v1 = vector(RDF, [1, -1, 0]) 
sage: v2 = vector(RDF, [1, 1, 0]) 
sage: W = V.span([v1, v2])       
sage: W.basis()
\end{sagecommandline}
Adding a linearly dependent vector $\vec{v}_3$ to the spanning
set doesn't change the space.
\begin{sagecommandline}
sage: V = VectorSpace(RDF,3)               
sage: v1 = vector(RDF, [1, -1, 0]) 
sage: v2 = vector(RDF, [1, 1, 0]) 
sage: W = V.span([v1, v2])       
sage: v3 = vector(RDF, [2, 3, 0])
sage: W_prime = V.span([v1, v2, v3])
sage: W_prime.basis()
\end{sagecommandline}

In the prior example, notice that \Sage{} does not simply
give you back the linearly independent
vectors that you gave it.
Instead, by default \Sage{} takes the vectors from the spanning set 
as the rows of a matrix,
brings that matrix to reduced echelon form, and reports the nonzero 
rows as the members of the basis.
Because each matrix has one and only one reduced echelon form, each 
vector subspace of real $n$-space has one and only one such
basis;
this is the \textit{canonical basis} for the space.

It is this basis that \Sage{} shows when you ask for a description
of the space.
\begin{sagecommandline}
sage: V = VectorSpace(RDF,3)               
sage: v1 = vector(RDF, [1, -1, 0]) 
sage: v2 = vector(RDF, [1, 1, 0]) 
sage: W = V.span([v1, v2])       
sage: W
sage: v3 = vector(RDF, [2, 3, 0])
sage: W_prime = V.span([v1, v2, v3])
sage: W_prime
\end{sagecommandline}

If you are keen on your own basis then \Sage{} will
accommodate you.
\begin{sagecommandline}
sage: V = VectorSpace(RDF,3)
sage: v1 = vector(RDF, [1, 2, 3])
sage: v2 = vector(RDF, [2, 1, 3])
sage: W = V.span_of_basis([v1, v2])
sage: W.basis()
sage: W
\end{sagecommandline}




\subsection{Equality}

You can test whether spaces are equal.
\begin{sagecommandline}
sage: V = VectorSpace(RDF,4)
sage: v1 = vector(RDF, [1, 0, 0, 0])
sage: v2 = vector(RDF, [1, 1, 0, 0])
sage: W12 = V.span([v1, v2])
sage: v3 = vector(RDF, [2, 1, 0, 0])
sage: W13 = V.span([v1, v3])  
\end{sagecommandline}
\noindent
One way to assure yourself that the two spaces $W_{1,2}$ and $W_{1,3}$ 
are equal is to use set membership.
\begin{sagecommandline}
sage: V = VectorSpace(RDF,4)
sage: v1 = vector(RDF, [1, 0, 0, 0])
sage: v2 = vector(RDF, [1, 1, 0, 0])
sage: W12 = V.span([v1, v2])
sage: v3 = vector(RDF, [2, 1, 0, 0])
sage: W13 = V.span([v1, v3])  
sage: v3 in W12
sage: v2 in W13
\end{sagecommandline}
\noindent
Then, since obviously $\vec{v}_1\in W_{1,2}$ and $\vec{v}_1\in W_{1,3}$, the two
spans are equal.

But the more straightforward way to test equality is to just ask \Sage{}.
\begin{sagecommandline}
sage: V = VectorSpace(RDF,4)
sage: v1 = vector(RDF, [1, 0, 0, 0])
sage: v2 = vector(RDF, [1, 1, 0, 0])
sage: W12 = V.span([v1, v2])
sage: v3 = vector(RDF, [2, 1, 0, 0])
sage: W13 = V.span([v1, v3])  
sage: W12 == W13
\end{sagecommandline}
\noindent
The exercise of \inlinecode{==} equality 
would be half-hearted without a test of inequality. 
\begin{sagecommandline}
sage: V = VectorSpace(RDF,4)
sage: v1 = vector(RDF, [1, 0, 0, 0])
sage: v2 = vector(RDF, [1, 1, 0, 0])
sage: W12 = V.span([v1, v2])
sage: v3 = vector(RDF, [2, 1, 0, 0])
sage: W13 = V.span([v1, v3])  
sage: v4 = vector(RDF, [1, 1, 1, 1])
sage: W14 = V.span([v1, v4])
sage: v2 in W14
sage: v3 in W14                                 
sage: v4 in W12
sage: v4 in W13
sage: W12 == W14                                                              
sage: W13 == W14
\end{sagecommandline}

This illustrates a point about algorithms.
\Sage{} could check for equality of two spans 
by checking whether every member of the first spanning set is in the
second space and vice versa (since the two spanning sets are finite). 
But \Sage{} does something different.
For each space it maintains the canonical basis
and it checks for equality of spaces
just by checking whether they have the same canonical bases.
These two algorithms 
have the same external behavior, in that both decide whether
two spaces are equal.
But the algorithms differ internally, and as a result the second is faster.
Finding the fastest way to do jobs is an important research area of computing.


\subsection{Operations}
\Sage{} finds the intersection of two spaces.
Consider these members of $\Re^3$.
\begin{equation*}
  \vec{v}_1=\colvec{1 \\ 0 \\ 0}
  \quad \vec{v}_2=\colvec{0 \\ 1 \\ 0}
  \quad \vec{v}_3=\colvec{0 \\ 0 \\ 3}
\end{equation*}
Form two spans, the $xy$-plane $W_{1,2}=\spanof{\vec{v}_1, \vec{v}_2}$ 
and the $yz$-plane $W_{2,3}=\spanof{\vec{v}_2, \vec{v}_3}$.
The intersection of these two is the $y$-axis. 
\begin{sagecommandline}
sage: V = VectorSpace(RDF,3)
sage: v1 = vector(RDF, [1, 0, 0])
sage: v2 =  vector(RDF, [0, 1, 0])
sage: W12 = V.span([v1, v2])
sage: W12.basis()
sage: v3 = vector(RDF, [0, 0, 2])
sage: W23 = V.span([v2, v3])
sage: W23.basis()
sage: W = W12.intersection(W23)
sage: W.basis()
\end{sagecommandline}

Remember that the trivial space $\set{\zero}$ is the span of the empty set.
\begin{sagecommandline}
sage: V = VectorSpace(RDF,3)
sage: v1 = vector(RDF, [1, 0, 0])
sage: v2 =  vector(RDF, [0, 1, 0])
sage: W12 = V.span([v1, v2])
sage: v3 = vector(RDF, [1, 1, 1])
sage: W3 = V.span([v3])
sage: W3.basis()
sage: W4 = W12.intersection(W3)
sage: W4.basis()
\end{sagecommandline}

\Sage{} will also find the sum of spaces, the span of their union.
\begin{sagecommandline}
sage: V = VectorSpace(RDF,3)
sage: v1 = vector(RDF, [1, 0, 0])
sage: v2 =  vector(RDF, [0, 1, 0])
sage: W12 = V.span([v1, v2])
sage: v3 = vector(RDF, [1, 1, 1])
sage: W3 = V.span([v3])
sage: W5 = W12 + W3
sage: W5.basis()
sage: W5 == V
sage: W5
\end{sagecommandline}






%========================================
\section{Other spaces}

These computations extend to
vector spaces that aren't a subspace of some $\Re^n$
by finding a real space that is just like the one you have.

Consider this vector space of quadratic polynomials,
under the usual operations of polynomial addition and scalar multiplication.
\begin{equation*}
  \set{ a_2x^2+a_1x+a_0\suchthat a_2=a_0+a_1}           
   =\set{ (a_1+a_0)x^2+a_1x+a_0 \suchthat a_1,a_0\in\Re}
\end{equation*}
It is just like
this subspace of $\Re^3$.\footnote{The textbook's chapter on Maps Between Spaces makes 
``just like'' precise.}
\begin{equation*}
  \set{\colvec{a_1+a_0 \\ a_1 \\ a_0}\suchthat a_1,a_0\in\Re}
  =\set{\colvec{1 \\ 1 \\ 0}a_1+\colvec{1 \\ 0 \\ 1}a_0\suchthat a_1,a_0\in\Re}
\end{equation*}
\begin{sagecommandline}
sage: V = VectorSpace(RDF,3)
sage: v1 = vector(RDF, [1, 1, 0])
sage: v2 = vector(RDF, [1, 0, 1])
sage: W = V.span([v1, v2])
sage: W.basis()
\end{sagecommandline}

Similarly you can represent this space of $\nbyn{2}$ matrices
\begin{equation*}
  \set{\begin{mat}
         a  &b \\
         c  &d
       \end{mat}\suchthat \text{$a-b+c=0$ and $b+d=0$}}
\end{equation*}
by finding a real $n$-space just like it.
Rewrite the given space
\begin{equation*}
  \set{\begin{mat}
         a  &b \\
         c  &d
       \end{mat}\suchthat \text{$a=-c-d$ and $b=-d$}}
  =\set{\begin{mat}
         -1  &0 \\
          1  &0
       \end{mat}c
       +
       \begin{mat}
         -1  &-1 \\
          0  &1
       \end{mat}d
       \suchthat c,d\in\Re}
\end{equation*}
and then here is a natural matching real space.
\begin{sagecommandline}
sage: V = VectorSpace(RDF,4)
sage: v1 = vector(RDF, [-1, 0, 1, 0])
sage: v2 = vector(RDF, [-1, -1, 0, 1])
sage: W = V.span([v1, v2])
sage: W.basis()
\end{sagecommandline}
You could have gotten another matching space by going down the columns 
instead of across the rows.  
\begin{sagecommandline}
sage: V = VectorSpace(RDF,4)
sage: v1 = vector(RDF, [-1, 1, 0, 0])
sage: v2 = vector(RDF, [-1, 0, -1, 1])
sage: W = V.span([v1, v2])
sage: W.basis()
\end{sagecommandline}
\noindent
There are still other ways to produce a matching space.
They look different than each other
but the important things about the spaces, such as dimension, are 
unaffected by their look.
That is the subject of the book's third chapter.

\endinput


TODO:
