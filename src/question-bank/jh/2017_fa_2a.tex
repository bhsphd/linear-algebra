% \documentclass[11pt]{examjh}
\documentclass[11pt,answers]{examjh}
\usepackage{../../linalgjh}
\examhead{MA 213 Hef{}feron, 2017-Fall}{Exam Two}

\begin{document}
\begin{questions}

\question
Show that the map $\map{f}{\polyspace_2}{\R^2}$
\begin{equation*}
ax^2+bx+c
\mapsto
\colvec{2c+a \\ 2a+4c}  
\end{equation*}
is not onto.
\begin{solution}[1in]
The output vectors all lie on the line $y=2x$.
\end{solution}



\question
Show that the map $\map{t}{\polyspace_2}{\R^3}$ given by
$t(ax^2+bx+c)=\colvec{a \\ a+b \\ a+c }$ is an isomorphism.
\begin{solution}[3in]
To see that the map is one-to-one suppose that $t(\vec{v}_1)=t(\vec{v}_2)$,
aiming to conclude that $\vec{v}_1=\vec{v}_2$.
That is, $t(a_1x^2+b_1x+c_1)=t(a_2x^2+b_2x+c_2)$.
Then
\begin{equation*}
\colvec{a_1 \\ a_1+b_1 \\ a_1+c_1 }=\colvec{a_2 \\ a_2+b_2 \\ a_2+c_2 }
\end{equation*}
and because three-tall vectors
are equal only if they have have equal components, the same linear
conclude that 
$a_1=a_2$, that $b_1=b_2$, and that $c_1=c_2$.
Therefore $a_1x^2+b_1x+c_1=a_2x^2+b_2x+c_2$ and the function is 
one-to-one.

To see that the map is onto, we suppose that we are given a member
of the codomain.
\begin{equation*}
  \vec{w}=\colvec{p \\ q \\ r}
\end{equation*}
Observe that where $\vec{v}=px^2+(q-p)x+(r-p)$ then $t(\vec{v})=\vec{w}$.
Thus~$t$ is onto.  

To see that the map is a homomorphism we show that it respects linear 
combinations of two elements.
\begin{align*}
  t(r_1(a_1x^2+b_1x+c_1)+r_2(a_2x^2+b_2x+c_2))
  &=t((r_1a_1+r_2a_2)x^2+(r_1b_1+r_2b_2)x+(r_1c_1+r_2c_2))   \\
  &=\colvec{r_1a_1+r_2a_2 \\ (r_1a_1+r_2a_2)+(r_1b_1+r_2b_2) \\ (r_1a_1+r_2a_2)+(r_1c_1+r_2c_2)}  \\
  &=\colvec{r_1a_1 \\ r_1a_1+r_1b_1 \\ r_1a_1+r_1c_1}
  +\colvec{r_2a_2 \\ r_2a_2+r_2b_2 \\ r_2a_2+r_2c_2}  \\
  &=r_1t(a_1x^2+b_1x+c_1)+r_2t(a_2x^2+b_2x+c_2)
\end{align*}
\end{solution}



\question
Consider the map $\map{h}{\polyspace_3}{\polyspace_2}$
given by this formula
$ax^3+bx^2+cx+d \mapsto (a+b-c)x^2-(b+c)x+(b-c)$.
\begin{parts}
\item Prove it is a linear map.
\begin{solution}[2.25in]
We show that it respects combinations.
\begin{multline*}
  h(r_1(a_1x^3+b_1x^2+c_1x+d_1)+r_2(a_2x^3+b_2x^2+c_2x+d_2))  \\
  \begin{split}
  &=
    h((r_1a_1+r_2a_2)x^3+(r_1b_1+r_2b_2)x^2+(r_1c_1+r_2c_2)x+(r_1d_1+r_2d_2))  \\
    &=((r_1a_1+r_2a_2)+(r_1b_1+r_2b_2)-(r_1c_1+r_2c_2))x^2 \\
    &\quad-((r_1b_1+r_2b_2)+(r_1c_1+r_2c_2))x+(r_1b_1+r_2b_2)-(r_1c_1+r_2c_2)  \\
  &=r_1(a_1+b_1-c_1)x^2-r_1(b_1+c_1)x+r_1(b_1-c_1)  \\
    &\quad +
    r_2(a_2+b_2-c_2)x^2-r_2(b_2+c_2)x+r_2(b_2-c_2)  \\
  &=r_1h(a_1x^3+b_1x^2+c_1x+d_1)+r_2h(a_2x^3+b_2x^2+c_2x+d_2)
  \end{split}  
\end{multline*}
\end{solution}
\item Represent it with respect to these bases.
\begin{equation*}
   B=\sequence{x^3,x^3+x^2,x,x+1}
   \quad
  D=\sequence{x^2+1,x+1,1}
\end{equation*}
\begin{solution}[2.5in]
The action of the map on the domain's basis vectors is this.
\begin{equation*}
    x^3\mapsto x^2
    \quad
    x^3+x^2\mapsto 2x^2-x+1
    \quad
    x\mapsto -x^2-x-1
    \quad
    x+1\mapsto -x^2-x-1
\end{equation*}
Represent those with respect to the codomain's basis.
\begin{equation*}
  \rep{x^2}{D}=\colvec{1 \\ 0 \\ -1}_D
  \quad
  \rep{2x^2-x+1}{D}=\colvec{2 \\ -1 \\ 0}_D
  \quad
  \rep{-x^2-x-1}{D}=\colvec{-1 \\ -1 \\ 1}_D
  % \quad
  % \rep{-x^2-x-1}{D}=\colvec{-1 \\ -1 \\ 1}_D
\end{equation*}
Concatenate them together into a matrix.
\begin{equation*}
  \rep{h}{B,D}=
  \begin{mat}
    1  &2   &-1  &-1    \\
    0  &-1  &-1  &-1    \\
   -1  &0   &1   &1
  \end{mat}
\end{equation*}
\end{solution}
\end{parts}  



\question
Which of these spaces are isomorphic?  Briefly say why.
\begin{parts}
\item $\Re^n$
\item $\polyspace_n$
\item $\matspace_{2\times n}$
\begin{solution}[0.5in]
Spaces are isomorphic if and only if they have the same dimension.
The first is dimension $n$, the second is dimension~$n+1$, and the third
is dimension $2n$.
For the first two, the isomorphisms are:
$\R^1\cong \polyspace_0$, 
$\R^2\cong \polyspace_1$, 
$\R^3\cong \polyspace_2$, \ldots\@
In short, $\R^{k+1}\cong\polyspace_{k}$ for any $k\in\N$.
For the first and third, $\R^{2k}\cong\matspace_{2\times k}$ for $k\in\N^+$.
Finally, for the second and third,
$\polyspace_{2k-1}\cong\matspace_{2\times k}$ for $k\in\N^+$.
\end{solution}
\end{parts}


\question
  Consider the map $\map{h}{\Re^3}{\Re^3}$ represented by this matrix
  with respect to the standard bases.
  \begin{equation*}
    \begin{mat}
      1 &0  &1 \\
      3 &1  &1  \\
      1 &-1 &3
    \end{mat}
  \end{equation*}
  \begin{parts}
    \item Find the range space of the map.
      What is the map's rank?
\begin{solution}[2in]
Here is the Gauss-Jordan reduction.
\begin{equation*}
  \begin{amat}{3}
      1 &0 &-1 &a \\
      2 &1 &0  &b \\
      2 &2 &2  &c  
  \end{amat}
  \grstep[-2\rho_1+\rho_3]{-2\rho_1+\rho_2}
  \begin{amat}{3}
      1 &0 &-1 &a \\
      0 &1 &2  &-2a+b \\
      0 &2 &4  &-2a+c  
  \end{amat}
  \grstep{-2\rho_2+\rho_3}
  \begin{amat}{3}
      1 &0 &-1 &a \\
      0 &1 &2  &-2a+b \\
      0 &0 &0  &2a-2b+c  
  \end{amat}
\end{equation*}
The range space is the set containing all of the members of the codomain 
for which this system has a solution.
\begin{equation*}
  \rangespace{h}=\set{\colvec{b-(1/2)c \\ b \\ c}\suchthat b,c\in\Re}
\end{equation*}
The rank is 2.
\end{solution}
    \item Describe the null space of the map.
      What is the map's nullity?
\begin{solution}[1.25in]
The null space is the set of members of the domain that map to 
$a=0$, $b=0$, and~$c=0$.
\begin{equation*}
  \nullspace{h}=\set{\colvec{z \\ -2z \\ z}\suchthat z\in\Re}
\end{equation*}
The nullity is~$1$.
\end{solution}
    \item Is the map onto?  Briefly justify.
\begin{solution}[1.0in]
  The codomain has dimension~$3$ but the map's range has only dimension~$2$, 
  so the map is not onto.
\end{solution}
    \item Is the map one-to-one?  Briefly justify.
\begin{solution}[1.0in]
  The dimension of the nullspace is not~$0$, so the map is not one-to-one.
\end{solution}
  \end{parts}
  

  
\question
Give an isomorphism between $\matspace_{\nbyn{2}}$ and $\R^4$.
You need not show the verification.
\begin{solution}[0.75in]
Here is the natural one.
\begin{equation*}
\begin{mat}
a &b \\
c &d
\end{mat}
\mapsto
\colvec{a \\ b \\ c \\ d}
\end{equation*}
\end{solution}


\question
Let the matrix $H$ represent the map $\map{h}{\Re^2}{\Re^3}$
and the matrix $G$ represent $\map{g}{\Re^3}{\Re^2}$.
\begin{equation*}
H=
\begin{mat}
2  &1  \\
3  &1  \\
4  &1
\end{mat}
\qquad
G=
\begin{mat}
0  &1 &2  \\
3  &2 &1  
\end{mat}
\end{equation*}
\begin{parts}
\item What is the representation of the composition $g\circ h$?
\begin{solution}[1.25in]
The matrix product.
\begin{equation*}
  GH=
  \begin{mat}
  11 &3 \\
  16 &6
  \end{mat}
\end{equation*}
\end{solution}
\item Does the composition $h\circ g$ have a representation?
\begin{solution}[1.25in]
\begin{equation*}
  HG=
  \begin{mat}
  3 &4 &5 \\
  3 &5 &7 \\
  3 &6 &9
  \end{mat}
\end{equation*}
\end{solution}
\end{parts}





\question
Perform each matrix operation or state ``not defined.''
\begin{parts}
\item
$
  \begin{mat}
    1 &1 \\
    3 &-1 \\
  \end{mat}
  \begin{mat}
     0 &4 \\
     2 &3 \\
  \end{mat}
  \begin{mat}
    -1 &1 \\
    -1 &3 \\
  \end{mat}
  $
\begin{solution}[1.5in]
  \begin{equation*}
  \begin{mat}
  -9 &23 \\
  -7 &25
  \end{mat}
  \end{equation*}
\end{solution}    
\item
$
\begin{mat}
1 &3 &1 \\
0 &0 &2
\end{mat}
\begin{mat}
1 &0 \\
3 &0 \\
1 &2
\end{mat}
$
\begin{solution}[1.25in]
  \begin{equation*}
  \begin{mat}
  11 &2 \\
  2 &4
    \end{mat}
  \end{equation*}
\end{solution}    
\item
$
3\cdot\begin{mat}
5 &1 \\
2 &-1
\end{mat}
-2\cdot
\begin{mat}
4 &0  &0  \\
2 &-1 &6
\end{mat}
$  
\begin{solution}[1in]
This is not defined.
\end{solution}    
\end{parts}


\end{questions}
\end{document}

