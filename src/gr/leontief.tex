% Chapter 1, Topic from _Linear Algebra_ Jim Hefferon
%  http://joshua.smcvt.edu/linalg.html
%  2001-Jun-09
\topic{Input-Output Analysis}
\index{Input-Output Analysis|(}
An economy is an immensely complicated network of interdependence.
Changes in one part can ripple out to affect other parts.
Economists have struggled to be able to describe, and to make
predictions about, such a complicated object.
Mathematical models using systems of linear equations
are a key tool.
One example is Input-Output Analysis, pioneered by 
W.~Leontief,\index{Leontief, W.} 
who won the 1973 Nobel Prize in Economics.

Consider an economy with many parts, two of which are the steel industry and
the auto industry.
These two interact tightly
as they work to meet the demand for their product
from other parts of the economy, from users external to the steel and
auto sectors.
For instance, should the external demand for autos go up, that would 
increase in the auto industry's usage of steel.
Or should the external demand for steel fall, then it would lower
steel's purchase of trucks.
The model that
we consider here takes in the external demands
and predicts how the two interact to meet those demands.

We start with production and consumption statistics.
(These numbers, giving dollar values in 
millions, are from
\cite{Leontief1965} describing the 1958 U.S.\ economy.
Today's statistics would be different because of
inflation and because of technical changes in the industries.)
\begin{center}
  \begin{tabular}{r|rrrr}
         \multicolumn{1}{c}{\ }  %put the | in the right place
         &\multicolumn{1}{r}{\begin{tabular}[b]{@{}c@{}} 
                                \textit{used by} \\[-.65ex] \textit{steel} 
                             \end{tabular}}
         &\multicolumn{1}{r}{\begin{tabular}[b]{@{}c@{}} 
                                \textit{used by} \\[-.65ex] \textit{auto} 
                             \end{tabular}}
         &\multicolumn{1}{r}{\begin{tabular}[b]{@{}c@{}} 
                                \textit{used by} \\[-.65ex] \textit{others} 
                              \end{tabular}}
         &\textit{total}                                                \\
         \cline{2-5}
    \begin{tabular}{r} \textit{value of} \\[-.65ex] \textit{steel} \end{tabular}
         &$5\,395$  &$2\,664$  &     &$25\,448$                          \\
    \begin{tabular}{r} \textit{value of} \\[-.65ex] \textit{auto} \end{tabular}
         &$48$      &$9\,030$  &     &$30\,346$                          
  \end{tabular}
\end{center}
For instance, the dollar value of steel used by the auto industry in this
year is $2,664$ million.
Note that industries may consume some of their own output.

We can fill in the external demands.
This year's value of the steel used by others is $17,389$ 
and the external value of autos is $21,268$.
With that we have a complete description of
how auto and steel interact to meet their demands.

Now imagine that the external demand for steel has recently been going up by
$200$ per year and so we estimate that next year it will be $17,589$.
We also estimate that next year's 
external demand for autos will be down $25$ to $21,243$.
We wish to predict next year's total outputs.

That prediction isn't as simple as adding $200$ to this year's steel
total and subtracting $25$ from this year's auto total.
For one thing, a rise in steel will cause
that industry to have an increased demand for autos, which will mitigate
the loss in external demand for autos.
On the other hand, the drop in external demand for autos will cause
the auto industry to use less steel and so lessen somewhat the
upswing in steel's business.
In short, these two industries form a system.
We must predict where the system as a whole will settle.

Here are the equations.
\begin{align*}
  \text{next year's production of steel}
     &=\text{next year's use of steel by steel}   \\[-.8ex]
     &\quad\hbox{}+\text{next year's use of steel by auto}  \\[-.8ex]  
     &\quad\hbox{}+\text{next year's use of steel by others} \\
  \text{next year's production of autos}
     &=\text{next year's use of autos by steel}   \\[-.8ex]
     &\quad\hbox{}+\text{next year's use of autos by auto}  \\[-.8ex] 
     &\quad\hbox{}+\text{next year's use of autos by others} 
\end{align*}
For the left side let
$s$ be next years total production of steel and
let $a$ be next year's total output of autos.
For the right side, as discussed above,
our external demand estimates are $17,589$ and $21,243$.

For next year's use of steel by steel, note that this year the
steel industry used $5,395$ units of steel input to produce $25,448$ units of
steel output.
So next year, when the steel industry will produce $s$ units out, 
we guess that doing so will take $s\cdot (5\,395)/(25\,448)$
units of steel input\Dash this is simply the assumption that input is
proportional to output. 
(We are assuming that the ratio of input to output remains constant over time;
in practice, models may try to take account of trends of change in the
ratios.)

Next year's use of steel by the auto industry is similar.
This year the auto industry uses $2,664$ units of steel input to produce
$30,346$ units of auto output.
So next year, when the auto industry's total output is $a$, we expect it to 
consume $a\cdot (2\,664)/(30\,346)$ units of steel.

Filling in the other equation in the same way gives this system of linear
equations.
\begin{equation*}
  \begin{linsys}{3}
    (5\,395/25\,448)\cdot s 
      &+ &(2\,664/30\,346)\cdot a &+ &17\,589 
         &= &s \\ 
    (48/25\,448)\cdot s     
      &+ &(9\,030/30\,346)\cdot a &+ &21\,243 
         &= &a
  \end{linsys}
\end{equation*}
Moving the variables to one side and the constants to the other
\begin{equation*}
  \begin{linsys}{2}
      (20\,053/25\,448)s &- &(2\,664/30\,346)a &= &17\,589 \\ 
     -(48/25\,448)s      &+ &(21\,316/30\,346)a &= &21\,243 
  \end{linsys}
\end{equation*}
and applying Gauss's Method or using \textit{Sage}, as here
\begin{lstlisting}
sage: var('a,s')
(a, s)
sage: eqns=[(20053/25448)*s - (2664/30346)*a == 17589,  
....:        (-48/25448)*s + (21316/30346)*a == 21243] 
sage: solve(eqns, s, a)
[[s == (2745320544312/106830469), a == (6476293881123/213660938)]]
sage: n(2745320544312/106830469)
25697.9171767186
sage: n(6476293881123/213660938)
30311.0804517904
\end{lstlisting}
gives our prediction:
$s=25,698$ and $a=30,311$.

Above, we discussed that the prediction of next year's
totals isn't as simple as adding $200$ to last year's steel total and
subtracting $25$ from last year's auto total.
Comparing these predictions 
to the numbers for the current year
shows that the total production of the steel industry should rise by~$250$
while
auto's total drops by $35$.
The increase in external demand for steel causes an increase in 
internal demand by the steel industry, which is lessened somewhat by 
the drop in autos, but results in a total that is more than
$200$ higher.
Similarly, auto's total drops more than~$25$ despite the help that 
it gets from steel.

One of the advantages of having a mathematical model is that we can
ask ``What if \ldots?'' questions.
For instance, we can ask,
``What if our estimates for next year's external demands are somewhat off?''
To try to understand how much the model's predictions change in 
reaction to changes in our estimates, we can revise the estimate of
next year's external steel demand from $17,589$ down to
$17,489$, while keeping the assumption of next year's external demand for
autos fixed at $21,243$. 
The resulting system
\begin{equation*}
  \begin{linsys}{2}
        (20\,053/25\,448)s &- &(2\,664/30\,346)a &= &17\,489 \\ 
      -(48/25\,448)s      &+ &(21\,316/30\,346)a &= &21\,243 
  \end{linsys}
\end{equation*}
gives $s=25,571$ and $a=30,311$.
% sage: var('a,s')                                        
% (a, s)
% sage: eqns=[(20053/25448)*s - (2664/30346)*a == 17489, 
% ....:        (-48/25448)*s + (21316/30346)*a == 21243] 
% sage: solve(eqns, s, a)                             
% [[s == (2731759305112/106830469), a == (6476221050723/213660938)]]
% sage: n(2731759305112/106830469)                    
% 25570.9754968126
% sage: n(6476221050723/213660938)
% 30310.7395827449
This is \definend{sensitivity analysis}.
We are seeing how sensitive the predictions of our model are to the 
accuracy of the assumptions.

Naturally, we can consider larger models that detail the interactions 
among more sectors of an economy;
these models are typically solved on a computer.
Naturally also, a single model does not suit every case and 
assuring that the assumptions underlying a model
are reasonable for a particular prediction requires the judgments of experts.
With those caveats
however, this model has proven in practice to be a useful and accurate tool for
economic analysis.
For further reading, try \cite{Leontief1951} and \cite{Leontief1965}. 


\begin{exercises}
  \item 
    With the steel-auto system given above, estimate next year's
    total productions in these cases.
    \begin{exparts}
      \partsitem Next year's external demands are up $200$ from this
         year for steel and are unchanged for autos.
      \partsitem Next year's external demands are up $100$ for steel and
         are up $200$ for autos.
      \partsitem Next year's external demands are up $200$ for steel and
         are up $200$ for autos.
    \end{exparts} 
    \begin{answer}
      These answers are from \textit{Octave}.
      \begin{exparts}
        \partsitem \Sage{} gets $s\approx 25\,952$ and $a\approx 30\,312$.
\begin{lstlisting}
sage: var('s,a')
(s, a)
sage: system = [(20053/25448)*s -  (2664/30346)*a == 17789,
....:             -(48/25448)*s + (21316/30346)*a == 21243]
sage: solve(system, s,a)
[[s == (2772443022712/106830469), a == (6476439541923/213660938)]]
sage: n(2772443022712/106830469), n(6476439541923/213660938)
(25951.8005365305, 30311.7621898814)          
\end{lstlisting}
        \partsitem \Sage{} gets $s\approx 25\,857$ and $a\approx 30\,596$.
\begin{lstlisting}
sage: system = [(20053/25448)*s -  (2664/30346)*a == 17689,
....:             -(48/25448)*s + (21316/30346)*a == 21443]
sage: solve(system, s,a)
[[s == (2762271457112/106830469), a == (6537219545323/213660938)]]
sage: n(2762271457112/106830469), n(6537219545323/213660938)
(25856.5883213711, 30596.2316112410)
\end{lstlisting}
        \partsitem \Sage{} gets $s\approx 25\,984$ and $a\approx 30\,597$.
\begin{lstlisting}
sage: system = [(20053/25448)*s -  (2664/30346)*a == 17789,
....:             -(48/25448)*s + (21316/30346)*a == 21443]
sage: solve(system, s,a)
[[s == (2775832696312/106830469), a == (6537292375723/213660938)]]
sage: n(2775832696312/106830469), n(6537292375723/213660938)
(25983.5300012771, 30596.5724802865)          
\end{lstlisting}
      \end{exparts}
    \end{answer}
  \item 
    For the steel-auto system,  
    with the external demand estimates of $17,589$ and $21,243$
    discussed above, what will be the value of steel used by steel,
    the value of steel used by auto, etc.? 
    \begin{answer}
      This \textit{Sage} session
\begin{lstlisting}
sage: var('a,s')
(a, s)
sage: eqns=[(20053/25448)*s - (2664/30346)*a == 17589,
....:       (-48/25448)*s + (21316/30346)*a == 21243] 
sage: solve(eqns, s, a)
[[s == (2745320544312/106830469), a == (6476293881123/213660938)]]
sage: s_by_s = (5395/25448)*2745320544312/106830469
sage: s_by_s
582010544505/106830469
sage: n(s_by_s)
5447.98267716114
sage: s_by_a = (2664/30346)*6476293881123/213660938  
sage: n(s_by_a)
2660.93449955743
sage: a_by_s = (48/25448)*2745320544312/106830469
sage: n(a_by_s)
48.4713936058822
sage: a_by_a = (9030/30346)*6476293881123/213660938
sage: n(a_by_a)
9019.60905818451        
\end{lstlisting}
      gives this table.
      \begin{center}
        \begin{tabular}{r|rrrr}
         \multicolumn{1}{c}{\ }  %put the | in the right place
         &\multicolumn{1}{r}{\begin{tabular}[b]{@{}c@{}} 
                                \textit{used by} \\[-.65ex] \textit{steel} 
                             \end{tabular}}
         &\multicolumn{1}{r}{\begin{tabular}[b]{@{}c@{}} 
                                \textit{used by} \\[-.65ex] \textit{auto} 
                             \end{tabular}}
         &\multicolumn{1}{r}{\begin{tabular}[b]{@{}c@{}} 
                                \textit{used by} \\[-.65ex] \textit{others} 
                              \end{tabular}}
         &\textit{total}                                                \\
         \cline{2-5}
         \begin{tabular}{r} \textit{value of} \\[-.65ex] \textit{steel} \end{tabular}
           &$5\,448$  &$2\,661$  &$17\,589$     &$25\,698$                          \\
         \begin{tabular}{r} \textit{value of} \\[-.65ex] \textit{auto} \end{tabular}
           &$48$      &$9\,020$  &$21\,243$     &$30\,311$                          
        \end{tabular}
      \end{center}
      For comparison here is the original table.
      \begin{center}
        \begin{tabular}{r|rrrr}
         \multicolumn{1}{c}{\ }  %put the | in the right place
         &\multicolumn{1}{r}{\begin{tabular}[b]{@{}c@{}} 
                                \textit{used by} \\[-.65ex] \textit{steel} 
                             \end{tabular}}
         &\multicolumn{1}{r}{\begin{tabular}[b]{@{}c@{}} 
                                \textit{used by} \\[-.65ex] \textit{auto} 
                             \end{tabular}}
         &\multicolumn{1}{r}{\begin{tabular}[b]{@{}c@{}} 
                                \textit{used by} \\[-.65ex] \textit{others} 
                              \end{tabular}}
         &\textit{total}                                                \\
         \cline{2-5}
         \begin{tabular}{r} \textit{value of} \\[-.65ex] \textit{steel} \end{tabular}
           &$5\,395$  &$2\,664$  &$17\,389$     &$25\,448$                          \\
         \begin{tabular}{r} \textit{value of} \\[-.65ex] \textit{auto} \end{tabular}
           &$48$      &$9\,030$  &$21\,268$     &$30\,346$                          
        \end{tabular}
      \end{center}
    \end{answer}
  \item 
    In the steel-auto system, the ratio for the use of steel by the auto
    industry is $2,664/30,346$, about $0.0878$.
    Imagine that a new process for making autos reduces this 
    ratio to $.0500$.
    \begin{exparts}
      \partsitem How will the predictions for next year's total
        productions change compared to the first example discussed
        above (i.e., taking next year's external demands to be
        $17,589$ for steel and $21,243$ for autos)?
      \partsitem Predict next year's totals if, in addition,
        the external demand for
        autos rises to be $21,500$ because the new cars are cheaper.
    \end{exparts}
    \begin{answer}
      \begin{exparts}
        \partsitem \Sage{} gives  $s=24,244$, $a=30,307$.
\begin{lstlisting}
 sage: var('s,a')
(s, a)
sage: system = [(20053/25448)*s -  0.0500*a == 17589,
....:             -(48/25448)*s + (21316/30346)*a == 21243]
sage: solve(system, s,a)
[[s == (12951731858499/534221147), a == (32381469405615/1068442294)]]
sage: n(12951731858499/534221147), n(32381469405615/1068442294)
(24244.1392131918, 30307.1767071166)         
\end{lstlisting}
        \partsitem Sage gives $s=24,267$, $a=30,673$.
\begin{lstlisting}
sage: system = [(20053/25448)*s -  0.0500*a == 17589,
....:             -(48/25448)*s + (21316/30346)*a == 21500]
sage: solve(system, s,a)
[[s == (12964136043940/534221147), a == (16386224431390/534221147)]]
sage: n(12964136043940/534221147), n(16386224431390/534221147)
(24267.3584090448, 30673.1107957993)          
\end{lstlisting}
      \end{exparts}
    \end{answer}
  \item 
    This table gives the numbers for the auto-steel system from
    a different year, 1947 (see \cite{Leontief1951}).
    The units here are billions of 1947 dollars.
    \begin{center}
      \begin{tabular}{r|rrrr}
             \multicolumn{1}{c}{\ }  % omit the | 
             &\multicolumn{1}{r}{\begin{tabular}[b]{@{}c@{}} 
                                   \textit{used by} \\[-.65ex] \textit{steel} 
                                 \end{tabular}}
             &\multicolumn{1}{r}{\begin{tabular}[b]{@{}c@{}} 
                                   \textit{used by} \\[-.65ex] \textit{auto} 
                                 \end{tabular}}
             &\multicolumn{1}{r}{\begin{tabular}[b]{@{}c@{}} 
                                   \textit{used by} \\[-.65ex] \textit{others} 
                                 \end{tabular}}
             &\multicolumn{1}{r}{\textit{total}}             \\ \cline{2-5}
        \begin{tabular}{r} \textit{value of} \\[-.65ex] 
                           \textit{steel} 
            \end{tabular}
            &$6.90$  &$1.28$  &   &$18.69$                              \\
        \begin{tabular}{r} \textit{value of} \\[-.65ex] \textit{autos} 
            \end{tabular}
            &$0$     &$4.40$  &   &$14.27$ 
      \end{tabular}
    \end{center}
    \begin{exparts}
      \partsitem Solve for total output if next year's external
         demands are: steel's demand up $10$\% and auto's demand up
         $15$\%.
      \partsitem How do the ratios compare to those given above in the
         discussion for the 1958 economy?
      \partsitem Solve the 1947 equations with the 1958 external demands
         (note the difference in units; a 1947 dollar buys about
         what \$$1.30$ in 1958 dollars buys).
         How far off are the predictions for total output?
    \end{exparts}
    \begin{answer}
      \begin{exparts}
        \partsitem 
          These are the equations.
	  \begin{equation*}
	    \begin{linsys}{2}
	       (11.79/18.69)s   &-   &(1.28/14.27)a &= &10.51 \\ 
	       -(0/18.69)s   &+   &(9.87/14.27)a &= &9.87 
	    \end{linsys}
	  \end{equation*}
          \textit{Sage} gives this result (including an inital check).
\begin{lstlisting}
sage: var('a,s')
(a, s)
sage: eqns=[(11.79/18.69)*s - (1.28/14.27)*a == 10.51,
....:       (-0)*s         + (9.87/14.27)*a == 9.87]
sage: solve(eqns,s,a)
[[s == (1869/100), a == (1427/100)]]
sage: n(1869/100)
18.6900000000000
sage: n(1427/100)
14.2700000000000
sage: eqns=[(11.79/18.69)*s - (1.28/14.27)*a == 11.561,
....:       (-0)*s         + (9.87/14.27)*a == 11.3505]
sage: solve(eqns,s,a)
[[s == (8119559/393000), a == (32821/2000)]]
sage: n(8119559/393000)
20.6604554707379
sage: n(32821/2000)
16.4105000000000            
\end{lstlisting}
	\partsitem
          These are the 1947 ratios, to three decimal places.
	  \begin{center}
            \begin{tabular}{r|cc}
               1947   &\textit{by steel}  &\textit{by autos}  \\ \hline
               \textit{use of steel}  &$(6.90/18.69)=0.369$ &$(1.28/14.27)=0.090$  \\
               \textit{use of autos}  &$(0/18.69)=0.00$ &$(4.40/14.27)=0.308$  
            \end{tabular}
          \end{center}
          These are the 1958 ratios.
          \begin{center}
            \begin{tabular}{r|cc}
               1958           &\textit{by steel}  &\textit{by autos}  \\ \hline
               \textit{use of steel}  &$(5395/25448)=0.212$ &$(2664/30346)=0.088$  \\
               \textit{use of autos}  &$(48/25448)=0.002$ &$(9030/30346)=0.298$  
            \end{tabular}
          \end{center}
	\partsitem
          The external demands in 1958 are $17.589$ and~$21.243$,
          in billions of 1958 dollars.
          In 1948 dollars those are $(17.589/1.3)=13.53$ and
          $(21.243/1.3)=16.34$. 
          \textit{Sage} gives these numbers.
\begin{lstlisting}
sage: eqns=[(11.79/18.69)*s - (1.28/14.27)*a == 13.53,
....:       (-0)*s         + (9.87/14.27)*a == 16.34]
sage: solve(eqns,s,a)
[[s == (137466107/5541300), a == (1165859/49350)]]
sage: n(137466107/5541300)
24.8075554472777
sage: n(1165859/49350)
23.6242958459980            
\end{lstlisting}
      Converting back to 1958 dollars by multiplying by $1.3$ gives
      $32.25$ and~$30.71$.
      \end{exparts}
    \end{answer}
  \item 
    Predict next year's total productions of each of the three
    sectors of the hypothetical economy shown below
    \begin{center}
      \begin{tabular}{r|rrrrr}
             \multicolumn{1}{c}{\ }  % omit the |
             &\multicolumn{1}{r}{\begin{tabular}[b]{@{}c@{}} 
                                     \textit{used by} \\[-.65ex] \textit{farm} 
                                 \end{tabular}}
             &\multicolumn{1}{r}{\begin{tabular}[b]{@{}c@{}} 
                                    \textit{used by} \\[-.65ex] \textit{rail} 
                                 \end{tabular}}
             &\multicolumn{1}{r}{\begin{tabular}[b]{@{}c@{}} 
                                    \textit{used by} \\[-.65ex] 
                                    \textit{shipping} 
                                 \end{tabular}}
             &\multicolumn{1}{r}{\begin{tabular}[b]{@{}c@{}} 
                                   \textit{used by} \\[-.65ex] \textit{others} 
                                 \end{tabular}}
             &\textit{total}                                                \\
             \cline{2-6}
        \begin{tabular}{r} \textit{value of} \\[-.65ex] \textit{farm} 
             \end{tabular}
             &$25$  &$50$  &$100$ &     &$800$             \\
        \begin{tabular}{r} \textit{value of} \\[-.65ex] \textit{rail} 
             \end{tabular}
             &$25$  &$50$  &$50$  &     &$300$             \\
        \begin{tabular}{r} \textit{value of} \\[-.65ex] \textit{shipping} 
             \end{tabular}
             &$15$  &$10$  &$0$   &     &$500$                          
      \end{tabular}
    \end{center}
    if next year's external demands are as stated.
    \begin{exparts}
      \partsitem $625$ for farm, $200$ for rail, $475$ for shipping 
      \partsitem $650$ for farm, $150$ for rail, $450$ for shipping 
    \end{exparts}
  \item 
      This table gives the interrelationships among three segments of an 
      economy (see \cite{BangorRpt}).
      \begin{center}
        \begin{tabular}{r|rrrrr}
             \multicolumn{1}{c}{\ }  % omit the |
             &\multicolumn{1}{r}{\begin{tabular}[b]{@{}c@{}} 
                                    \textit{used by} \\[-.65ex] \textit{food} 
                                 \end{tabular}}
             &\multicolumn{1}{r}{\begin{tabular}[b]{@{}c@{}} 
                                   \textit{used by} \\[-.65ex] 
                                   \textit{wholesale} 
                                 \end{tabular}}
             &\multicolumn{1}{r}{\begin{tabular}[b]{@{}c@{}} 
                                    \textit{used by} \\[-.65ex] \textit{retail} 
                                 \end{tabular}}
             &\multicolumn{1}{r}{\begin{tabular}[b]{@{}c@{}} 
                                    \textit{used by} \\[-.65ex] \textit{others} 
                                 \end{tabular}}
             &\textit{total}                                             \\
          \cline{2-6}
          \begin{tabular}[b]{r} \textit{value of} \\[-.65ex] \textit{food} 
              \end{tabular}
              &$0$    &$2\,318$   &$4\,679$   &     &$11\,869$   \\
          \begin{tabular}[b]{r} \textit{value of} \\[-.65ex] \textit{wholesale}
              \end{tabular}
              &$393$  &$1\,089$   &$22\,459$  &     &$122\,242$   \\
          \begin{tabular}[b]{r} \textit{value of} \\[-.65ex] \textit{retail} 
              \end{tabular}
              &$3$    &$53$       &$75$       &     &$116\,041$   \\
        \end{tabular}
      \end{center}
      We will do an Input-Output analysis on this
      system.
      \begin{exparts}
        \partsitem Fill in the numbers for this year's external
          demands.
        \partsitem Set up the linear system, leaving next year's
          external demands blank.
        \partsitem Solve the system where we get next year's external demands
          by taking this year's external demands and
          inflating them $10$\%.
          Do all three sectors increase their total business by
          $10$\%?
          Do they all even increase at the same rate? 
        \partsitem Solve the system where we get next year's external demands
          by taking this year's external demands and
          reducing them $7$\%.
          (The study from which these numbers come concluded that
          because of the closing of a local military facility, overall
          personal income in the area would fall $7$\%, so this might 
          be a first guess at what would actually happen.)
      \end{exparts}
\end{exercises}
\index{Input-Output Analysis|)}
\endinput
