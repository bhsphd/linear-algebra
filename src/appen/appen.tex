\makeatletter
\newcommand{\appendsection}[1]{\vspace{2.75ex plus .15ex minus .05ex}\noindent{\addcontentsline{toc}{subsection}{#1}\large\textbf{#1}}\vspace{.5ex plus .05ex minus .05ex}\nopagebreak[4]\par\noindent\trim@spaces}
\makeatother

\markboth{}{}
\renewcommand{\thepage}{A-\arabic{page}}
\setcounter{page}{1}
%\thispagestyle{plain}
%\noindent
\chapter*{Appendix}
\addcontentsline{toc}{chapter}{Appendix}
Mathematics is made of arguments (reasoned discourse that is,
not crockery-throwing).
This section sketches the background material and 
argument techniques that we use in the book.

This section informally outlines the topics, skipping proofs.
For more, \cite{HowToProveIt} is excellent. 
Two other sources, available online, are
\cite{Hefferon} and~\cite{Beck}.




%\vskip .75in
\appendsection{Statements}
%\addcontentsline{toc}{subsection}{Propositions}
%\bigskip
%\par\noindent
Formal mathematical statements come labelled as a 
\definend{Theorem} % \index{theorem} 
for major points,
a \definend{Corollary} % \index{corollary}  
for results that follow immediately from
a prior one, or a
\definend{Lemma} %\index{lemma} 
for results chiefly used to prove others.

Statements can be complex and have many parts.
The truth or falsity of the entire statement depends both on the
truth value of the parts and on how the statement is
put together.

\startword{Not}
Where \( P \) is a proposition,
`it is not the case that \( P \)' is true provided that \( P \) is
false.
For instance, `\( n \) is not prime' is true only when \( n \) is the
product of smaller integers.

To prove that a `not \( P \)' statement holds, show that \( P \) is false.



\startword{And}
For a statement  of the form `\( P \) and \( Q \)'
to be true both halves must hold:
`\( 7 \) is prime and so is \( 3 \)' is true, while
`\( 7 \) is prime and \( 3 \) is not' is false.

To prove a `\( P \) and \( Q \)', prove each half.




\startword{Or}
A `\( P \) or \( Q \)' statement is true when either half holds:
`\( 7 \) is prime or \( 4 \) is prime' is true, while `\( 8 \) is prime
or \( 4 \) is prime' is false.
In the case that both clauses of the statement are true, 
as in `\( 7 \) is prime or \( 3 \) is prime',
we take the statement as a whole to be true.
(In everyday speech people occasionally use `or' in an 
exclusive way\Dash ``Live
free or die'' does not intend both halves to hold\Dash but
we will not use `or' in that way.)

To prove `\( P \) or \( Q \)', show that in all cases at least one
half holds (perhaps sometimes one half and sometimes the other,
but always at least one).



\startword{If-then}
\index{if-then statement}
An `if \( P \) then \( Q \)' statement may also appear as
`\( P \) implies \( Q \)' or `\( P\implies Q\)' or 
`\( P \) is sufficient to give \( Q \)' or `$Q$ if $P$'.
It 
is true unless \( P \) is true while \( Q \) is false.
Thus `if \( 7 \) is prime then \( 4 \) is not' is true 
while `if \( 7 \) is prime then \( 4 \) is also prime' is false.
(Contrary to its
use in casual speech, in mathematics `if \( P \) then \( Q \)' 
does not connote that
\( P \) precedes \( Q \) or causes \( Q \).)

Note this consequence of the prior paragraph: 
if $P$ is false then `if \( P \) then \( Q \)' is
true irrespective of the value of $Q$:
`if \( 4 \) is prime then \( 7 \) is prime' and
`if \( 4 \) is prime then \( 7 \) is not' are both true statements.
(They are \definend{vacuously true}.\index{vacuously true})
Also observe that `if \( P \) then \( Q \)' is true when $Q$ is true: 
`if $4$ is prime then $7$ is prime' and 
`if $4$ is not prime then $7$ is prime' are both true.

% (We adopt this definition of implication
% because we want statements 
% such as `if $n$ is a perfect square then $n$ is not prime'
% to be true no matter which 
% number $n$ appears in that statement.
% For instance, we want `if $5$ is a perfect square then $5$ is not prime'
% to be true so we want that if both $P$ and~$Q$ are false then
% $P\implies Q$ is true.)

There are two main ways to establish an implication.
The first way is direct:~assume that \( P \) is true and use that
assumption to prove \( Q \).
For instance,
to show `if a number is divisible by 5 then twice that
number is divisible by 10' we can assume that the number is \( 5n \) and
deduce that \( 2(5n)=10n \).
The indirect way is to prove the 
\definend{contrapositive}\index{contrapositive}
statement: `if \( Q \) is false then \( P \) is false'
(rephrased, `\( Q \) can only be false when \( P \) is also false').
Thus to show `if a natural number is prime then it
is not a perfect square' we can 
argue that if it were a square \( p=n^2 \) then it could be
factored \( p=n\cdot n \) where \( n<p \) and so wouldn't be prime
(\( p=0 \) or \( p=1 \) don't satisfy \( n<p \) but they
are nonprime).

% Note two things about this statement form.

% First, an `if \( P \) then \( Q \)' result can sometimes be improved
% by weakening \( P \) or strengthening \( Q \).
% Thus, `if a number is divisible by \( p^2 \) then its square is also
% divisible by \( p^2 \)' could be upgraded either by relaxing its
% hypothesis:~`if a number is divisible by \( p \) then its square
% is divisible by \( p^2 \)', or by tightening its conclusion:~`if
% a number is divisible by \( p^2 \) then its square is divisible by
% \( p^4 \)'.

% Second,
% after showing `if \( P \) then \( Q \)' then a good next step is to look into
% whether there are cases where \( Q \) holds but \( P \) does not.
% The idea is to better understand the relationship between \( P \) and
% \( Q \) with an eye toward strengthening the proposition.



\startword{Equivalent statements}
\index{equivalent statements}\index{propositions!equivalent}
Sometimes, not only does \( P \) imply \( Q \) but
also \( Q \) implies \( P \). 
Some ways to say this are:
`\( P \) if and only if
\( Q \)', `\( P \) iff \( Q \)', `\( P \) and \( Q \) are logically
equivalent', `\( P \) is necessary and sufficient to give \( Q \)',
`\( P\iff Q \)'.
An example is `an integer is divisible by ten if and only if  
that number ends in $0$'.

Although in simple arguments a chain like 
``\( P \) if and only if $R$, which holds if and only if $S$ \ldots''
may be practical, to prove that statements are equivalent we more often
prove the two halves
`if \( P \) then \( Q \)' and `if \( Q \) then \( P \)' separately.








\appendsection{Quantifiers}\index{quantifier}%
Compare these statements about natural numbers:
`there is a natural number \( x \) such that \( x \) is
divisible by \( x^2 \)' is true, while
`for all natural numbers \( x \), 
that \( x \) is divisible by \( x^2 \)' is false.
The prefixes `there is' and `for all' 
are \definend{quantifiers}.\index{quantifiers}

\startword{For all}
The `for all' prefix is the 
\definend{universal quantifier},\index{quantifier!universal} 
symbolized \( \forall \).

The most straightforward way to prove that 
a statement holds in all cases is to 
prove that it holds in each case.
Thus to show that `every number divisible by \( p \) has its
square divisible by \( p^2 \)', take a single number of the form
\( pn \) and square it \( (pn)^2=p^2n^2 \).
This is a \definend{typical element} proof.
(In this kind of argument be careful not to assume 
properties for that element
other than the ones in the hypothesis.
This argument is wrong:
``If \( n \) is divisible by a prime, say \( 2 \), so that \( n=2k \)
for some natural number~$k$,
then \( n^2=(2k)^2=4k^2 \) and the square of $n$ is divisible
by the square of the prime.''
That is a proof for the special case \( p=2 \) 
but it isn't a proof for all~\( p \).
Contrast it with a correct one:
``If \( n \) is divisible by a prime
so that \( n=pk \) for some natural number~$k$
then \( n^2=(pk)^2=p^2k^2 \) and so the square of $n$ is divisible
by the square of the prime.'')




\startword{There exists}
The `there exists' prefix is the 
\definend{existential quantifier},\index{quantifier!existential}
symbolized 
\( \exists \).

%This quantifier is in some ways the opposite of `for all'.
%For instance, contrast these two definitions of primality
%of an integer \( p \): (i)~for all \( n \), if
%\( n \) is not \( 1 \) and \( n \) is not \( p \) then \( n \)
%does not divide \( p \), and
%(ii)~it is not the case that there exists an \( n \)
%(with \( n\neq 1 \) and \( n\neq p \)) such that \( n \) divides \( p \).

We can prove
an existence proposition by producing something satisfying
the property: for instance, to settle the question of primality of
\( 2^{2^5}+1 \), Euler exhibited the divisor \( 641 \)\cite{Sandifer}.
But there are proofs
showing that something exists without saying how to find it;
Euclid's argument given in the next subsection
shows there are infinitely many primes without giving a formula naming them.
% In general, while demonstrating existence is better than nothing,
% giving an example is better, and an
% exhaustive list of all instances is ideal.

Finally, after
``Are there any?''\ %\spacefactor=1000 
we often ask ``How many?''
That is, the question of uniqueness often arises in conjunction
with the question of existence.
Sometimes the two arguments are simpler if separated so note that just as
proving something exists does not show that it is unique,
neither does proving that something is unique show that it exists.










\appendsection{Techniques of Proof}
\index{proof techniques|(}%
We have many ways to prove mathematical statements.
Here we will outline two techniques that we use often, and that
might not come naturally, even to a person with a technical 
turn of mind.
 

\startword{Induction}
\index{induction, mathematical}
Many proofs are iterative,
``Here's why the statement is true for the number \( 0 \), 
it then follows for \( 1 \) and from there to \( 2 \) \ldots''.
These are proofs by 
\definend{mathematical induction}.\index{mathematical induction}\index{induction}
We will see two examples.

We will first prove that \( 1+2+3+\dots+n=n(n+1)/2 \). 
That formula has a natural number variable $n$
that is free, meaning that setting $n$ to be $1$,
or $2$, etc., gives a family of cases of the statement:
first that $1=1(2)/2$, second that $1+2=2(3)/2$, etc.
Our induction proofs involve statements with one free natural number 
variable.

Each such proof has two steps. 
In the \definend{base step}\index{base step, of induction proof}
we show that the statement holds for
some intial number $i\in \N$. 
Often this step is a routine verification.
The second step, 
the \definend{inductive step},\index{inductive step, of induction proof}
is more subtle; we will show that this implication holds:
\begin{equation*}
  \begin{tabular}{l}
    \text{If the statement holds from $n=i$ up to and including~$n=k$} 
     \\
    \text{\ then the statement holds also in the $n=k+1$ case}
  \end{tabular}
  \tag{$*$}
\end{equation*}
(the first line is the 
\definend{inductive hypothesis}\index{induction!inductive hypothesis}\index{inductive hypothesis}).
Completing both steps proves that the statement is
true for all natural numbers greater than or equal to $i$.

For the sum of the initial $n$~numbers statement  
the intuition behind the principle is that first, the base step directly 
verifies the statement for the case of the initial number $n=1$. 
Then, because the inductive step verifies the implication~($*$)
for all $k$, that implication applied to $k=1$
gives that the statement is true for the case of the number
$n=2$. 
Now, with the statement established for both $1$ and $2$, 
apply ($*$) again to conclude that the statement is true for the number
$n=3$.
In this way, we bootstrap to all numbers $n\geq 1$. 

Here is a proof of \( 1+2+3+\dots+n=n(n+1)/2 \), 
with separate paragraphs for the base step 
and the inductive step.
\begin{quote}\small
For the base step we show that the formula holds when \( n=1 \).
That's easy; the sum of the first \( 1 \) natural number equals 
\( 1(1+1)/2 \).

For the inductive step, assume the inductive hypothesis that the formula holds
for the numbers \( n=1 \), \( n=2 \), \ldots, \( n=k \) with $k\geq 1$.
That is, assume  
$1=1(1)/2$, and $1+2=2(3)/2$, and $1+2+3=3(4)/2$, through  
$1+2+\cdots+k=k(k+1)/2$.
With that, the formula holds also in the \( n=k+1 \)~case:
\begin{equation*}
  1+2+\cdots+k+(k+1)
  =
  \frac{k(k+1)}{2}+(k+1)
  =
  \frac{(k+1)(k+2)}{2}
\end{equation*}
(the first equality follows from the inductive hypothesis).
\end{quote}
% The base case shows that the formula is true holds for \( n=1 \).
% The inductive step shows that,  
% because the formula holds for \( 1 \), it also holds for \( n=2 \);
% The inductive step also shows that,  
% because the formula holds for \( 1 \) and \( 2\), it holds for \( 3=2 \);
% Continuing in this way, we see that the statement holds
% for any natural number greater than or equal to \( 1 \).

Here is another example, proving
that every integer greater than or equal to \( 2 \) is a product
of primes.
\begin{quote}\small
The base step is easy: \( 2 \) is the product of a single prime.

For the inductive step assume that each of \( 2, 3,\ldots ,k \) is a
product of primes, aiming to show \( k+1 \) is also a product of
primes.
There are two possibilities.
First, if \( k+1 \) is not divisible by a number smaller than itself then it
is a prime and so is the product of primes.
The second possibility is that \( k+1 \) is divisible by a number
smaller than itself, and then  by the inductive hypothesis its
factors can be written as a product of primes.
In either case
\( k+1 \) can be rewritten as a product of primes.
\end{quote}





\startword{Contradiction}
\index{contradiction}
Another technique of proof is
to show that something is true by showing that it cannot be false.
A proof by contradiction assumes that the proposition is
false and derives some contradiction to known facts.

The classic example of proof by contradiction is Euclid's
argument that there are infinitely many primes.
\begin{quote}\small
Suppose that there are only finitely many primes \( p_1,\dots,p_k \).
Consider the number \( p_1\cdot p_2\dots p_k +1 \).
None of the primes on the supposedly exhaustive list divides this number
evenly since each leaves a remainder of \( 1 \).
But every number is a product of primes so this can't be.
Therefore there cannot be only finitely many primes.
\end{quote}

Another example is this proof that
\( \sqrt{2} \) is not a rational number.
\begin{quote}\small
Suppose that  \( \sqrt{2}=m/n \), so that $2n^2=m^2$.
Factor out any \( 2 \)'s, giving
\( n=2^{k_n}\cdot \hat{n} \)
and
\( m=2^{k_m}\cdot \hat{m} \).
Rewrite.
\begin{equation*}
  2\cdot (2^{k_n}\cdot \hat{n})^2
  =
  (2^{k_m}\cdot \hat{m})^2
\end{equation*}
The Prime Factorization Theorem says that there must be the same number of
factors of \( 2 \) on both sides, but there are an odd number of them
\( 1+2k_n \) on the left and an even number \( 2k_m \) on the right.
That's a contradiction, so a rational number with a square of
\( 2 \) is impossible.
\end{quote}

% Both of these examples aimed to prove something doesn't exist.
% A negative proposition often suggests a proof by contradiction.
\index{proof techniques|)}













\appendsection{Sets, Functions, and Relations}
The material here forms the backdrop, the vocabulary,
for all of the development that we do.

\startword{Sets}
\index{sets}
Mathematicians often work with collections. 
The most commonly-used kind of collection is a \definend{set}.\index{set} 
Sets are characterized by the Principle of Extensionality: 
two sets with the same elements are equal. 
Because of this, the order of the elements does not matter 
\( \set{2,\pi}=\set{\pi,2} \), 
and
repeats collapse \( \set{7,7}=\set{7} \).

We can describe a set using a listing between curly braces
\( \set{ 1,4,9,16 } \) (as in the prior paragraph), 
or by using set-builder notation
\( \set{x\suchthat x^5-3x^3+2=0 } \) (read ``the set of all \( x \)
such that \ldots'').
We name sets with capital roman letters; for instance the set of primes is
\( P=\set{2,3,5,7,11,\ldots\,} \) (except that a few sets 
are so important that their names are reserved, such as the
real numbers \( \Re \)
and the complex numbers \( \C \)).
To denote that something is an 
\definend{element},\index{element}\index{set!element} 
or \definend{member},\index{member}\index{set!member}) of a set we
use `\(\in \)',
so that \( 7\in\set{3,5,7} \) while \( 8\not\in\set{3,5,7} \).

We say that \( A \) is a \definend{subset} of \( B \), written
$A\subseteq B$, when $x\in A$ implies that $x\in B$.
In this book we use
`\( \subset \)' for the \definend{proper subset}\index{sets!proper subset}\index{proper subset}\index{sets!subset}\index{subset, proper} %
relationship that \( A \) is a subset of \( B \) but \( A\neq B \)
(some authors use this symbol for any kind of subset, proper or not).
An example is 
\( \set{2,\pi}\subset\set{2,\pi,7} \).
These symbols may be flipped, for instance
\( \set{ 2,\pi,5}\supset\set{2,5} \).

Because of Extensionality, to prove that two sets are equal \( A=B \)
show that they have the same members.
Often we do this by showing mutual inclusion,\index{mutual inclusion}%
\index{sets!mutual inclusion}
that both \( A\subseteq B \) and \( A\supseteq B \).
Such a proof will have a part showing that if $x\in A$ then $x\in B$,
and a second part showing that if $x\in B$ then $x\in A$.

When a set has no members then it is
the \definend{empty set}\index{empty set}\index{set!empty} \( \set{} \),
symbolized \( \emptyset \).
Any set has the empty set for a subset by the `vacuously true'
property of the definition of implication.




\startword{Diagrams}
We picture basic set operations with a
\definend{Venn diagram}.\index{Venn diagram}
This shows \( x\in P \). 
\begin{center}
  \includegraphics{appen.8}
\end{center}
The outer rectangle contains the universe $\Omega$ of all objects under 
discussion.
For instance, in a statement about real numbers, the rectangle encloses all 
members of $\R$.
The set is pictured as a circle, enclosing its members.

Here is the diagram for \( P\subseteq Q \).
It shows that if \( x\in P \) then \( x\in Q \).
\begin{center}
  \includegraphics{appen.4}
\end{center}




\startword{Set Operations}
The \definend{union}\index{union of sets}\index{set!union} of two sets is
\( P\union Q=\set{x\suchthat \text{$(x\in P)$ or $(x\in Q)$}} \).
The diagram shows that an element is in the union if it is in either of the
sets.
\begin{center}
  \includegraphics{appen.3}
\end{center}
The \definend{intersection}\index{intersection, of sets}\index{set!intersection} is
\( P\intersection Q=\set{x\suchthat \text{$(x\in P)$ and $(x\in Q)$}} \).
\begin{center}
  \includegraphics{appen.2}
\end{center}
The 
\definend{complement}\index{complement}\index{set!complement} 
of a set~\( P \) is
\( P^{\text{comp}}=\set{x\in\Omega\suchthat x\not\in P} \)
\begin{center}
  \includegraphics{appen.1}
\end{center}




\startword{Multisets}\index{multiset}
As described above, a set is a collection in which 
order does not matter, so that the sets $\set{2,\pi}$
and $\set{\pi,2}$ are equal,
and in which repeats collapse, so that the sets
$\set{7,7}$ and $\set{7}$ are equal.

A collection that is like a set in that order does not matter,
but in which repeats do not collapse, is a \definend{multiset}.
(Note that we use the same 
curly brackets notation  $\set{\ldots}$ as for sets.)
Thus the multiset $\set{1,2,2}$ differs from the multiset
$\set{1,2}$.
Because order does not matter, these multisets are all 
equal: $\set{1,2,2}$, $\set{2,1,2}$, and~$\set{2,2,1}$.
In this text we only mention multisets in a remark so going into  
how to do subsets, unions, or intersections, is beyond our scope.



\startword{Sequences}
\index{sequence}
In addition to sets and multisets,
we also use collections where order matters and where repeats do
not collapse.
These are \definend{sequences},\index{sequence} denoted with angle brackets:
\( \sequence{ 2,3,7}\neq\sequence{2,7,3} \).
A sequence of length \( 2 \) is an 
\definend{ordered pair},\index{ordered pair}\index{pair, ordered}
and is often written with parentheses: \( (\pi,3) \).
We also sometimes say `ordered triple', `ordered \( 4 \)-tuple', etc.
The set of ordered \( n \)-tuples of elements of a set \( A \) is denoted
\( A^n \).
Thus \( \Re^2 \) is the set of pairs of reals.




\startword{Functions}
\index{function}
A \definend{function}\index{function}
or \definend{map}\index{map}  $\map{f}{D}{C}$ is 
is an association between input
\definend{arguments}\index{argument, of a function}\index{function!argument} 
$x\in D$
and output
\definend{values}\index{value, of a function}\index{function!value}
$f(x)\in C$ subject to the the requirement 
that the
function must be \definend{well-defined},\index{function!well-defined}%
\index{well-defined}
that \( x \) suffices to determine \( f(x) \).
Restated, the condition is that
if \( x_1=x_2 \) then \( f(x_1)=f(x_2) \).

The set of all arguments~$D$ is \( f \)'s 
\definend{domain}\index{domain}\index{function!domain}
and the set of output values is its 
\definend{range} $\rangespace{f}$.\index{range}\index{function!range}
Often we don't work with the range and instead
work with a convenient superset, the
\definend{codomain}~$C$.\index{function!codomain}\index{codomain}
For instance, we might describe the squaring function with $\map{s}{\R}{\R}$
instead of $\map{s}{\R}{\R^+\union \set{0}}$.

We picture functions with a 
\definend{bean diagram}.\index{bean diagram}\index{function!bean diagram}
\begin{center}
  \includegraphics{appen.9}
\end{center}
The blob on the left is the domain while on the right is the
codomain.
The function associates the three points of the domain with three in the
codomain.
Note that by the definition of a function 
every point in the domain is associated with
a unique point in the codomain, but the converse needn't be true.

The association is arbitrary; no formula or algorithm is required, although in
this book there typically is one.
We often use $y$ to denote $f(x)$. 
We also use the notation \( x\mapsunder{f} 16x^2-100 \), read
`\( x \) maps under \( f \) to \( 16x^2-100 \)' or
`\( 16x^2-100 \) is the 
\definend{image}\index{image, under a function}\index{function!image} 
of \( x \)'.

A map such as \( x\mapsto \sin(1/x) \) is a
combinations of simpler maps, here
\( g(y)=\sin(y) \) applied to the image of \( f(x)=1/x \).
The \definend{composition}\index{composition}\index{function!composition} 
of \( \map{g}{Y}{Z} \) with \( \map{f}{X}{Y} \),
is the map sending
\( x\in X \) to \( g(\, f(x)\,)\in Z \).
It is denoted \( \map{\composed{g}{f}}{X}{Z} \).
This definition only makes sense if the range of \( f \) is a
subset of the domain of \( g \).

An 
\definend{identity map}\index{identity!function}\index{function!identity} 
\( \map{\identity}{Y}{Y} \) defined by
\( \identity(y)=y \) has the property that for any \( \map{f}{X}{Y} \),
the composition \( \composed{\identity}{f} \) is equal to \( f \).
So an identity map plays the same role with respect to function composition
that the number \( 0 \) plays in real number addition or that 
\( 1 \) plays in multiplication.

In line with that analogy, we define a
\definend{left inverse}\index{inverse!function!left}\index{inverse!left}\index{left inverse} of a map 
\( \map{f}{X}{Y} \) to be a
function \( \map{g}{\text{range}(f)}{X} \) such that \( \composed{g}{f} \)
is the identity map on \( X \).
A \definend{right inverse}\index{inverse!function!right}\index{inverse!right}\index{right inverse} 
of \( f \) is a
\( \map{h}{Y}{X} \) such that \( \composed{f}{h} \) is the identity.

For some $f$'s there is a map that is 
both a left and right inverse of \( f \).
If such a map exists then it is unique because if both \( g_1 \) and
\( g_2 \) have this property then
\( g_1(x)=\composed{g_1}{ (\composed{f}{g_2}) }\,(x)
         =\composed{ (\composed{g_1}{f}) }{g_2}\,(x)
         =g_2(x) \)
(the middle equality comes from the associativity of function composition)
so we call it a \definend{two-sided inverse} or just  
\definend{``the'' inverse},\index{inverse}\index{inverse!two-sided}\index{function!inverse}\index{inverse!function}\index{inverse function}\index{inversion}
and denote it \( f^{-1} \). 
For instance, the inverse of the function \( \map{f}{\Re}{\Re} \)
given by \( f(x)=2x-3 \) is the function \( \map{f^{-1}}{\Re}{\Re} \)
given by \( f^{-1}(x)=(x+3)/2 \).

The superscript notation for function inverse `\( f^{-1} \)' 
fits into a larger scheme.
Functions with the same codomain as domain $\map{f}{X}{X}$ can be iterated,
so that we can consider
the composition of $f$ with itself: \( \composed{f}{f} \), 
and \( \composed{f}{\composed{f}{f}} \), etc.
We 
write $\composed{f}{f}$ as \( f^2 \) and 
$\composed{\composed{f}{f}}{f}$ as \( f^3 \), etc.
Note that the familiar exponent rules for real numbers hold:
\( \composed{f^i}{f^j}=f^{i+j} \) and \( (f^i)^j=f^{i\cdot j} \).
Then where \( f \) is invertible,
writing \( f^{-1} \) for the inverse
and \( f^{-2} \) for the inverse of \( f^2 \), etc., gives that
these familiar exponent rules continue to hold, since we define
\( f^0 \) to be the identity map.

The definition of function requires that for every 
input there is one and only one associated output value.
If a function $\map{f}{D}{C}$ has the additional property that for every
output value there is at least one associated input value\Dash that is, 
the additional property that
$f$'s codomain equals its range \( C=\rangespace{f} \)\Dash then 
the function is
\definend{onto}.\index{function!onto}\index{onto function}
\begin{center}
  \includegraphics{appen.14}
\end{center}
A function has a right inverse if and only if it is onto. 
(The $f$ pictured above has a right inverse $\map{g}{C}{D}$ given by
following the arrows backwards, from right to left.
For the codomain point on the top, choose either one of the arrows to follow.
With that, applying $g$ first followed by $f$ takes elements
$y\in C$ to themselves, and so is the identity function.)

If a 
function $\map{f}{D}{C}$ has the property that for every
output value there is at most one associated input value\Dash that is, 
if no two arguments share an image so that
\( f(x_1)=f(x_2) \) implies  that \( x_1 = x_2 \)\Dash then 
the function is
\definend{one-to-one}.\index{function!one-to-one}\index{one-to-one function}
The bean diagram from earlier illustrates.
\begin{center}
  \includegraphics{appen.9}
\end{center}
A function has a left inverse if and only if it is one-to-one.
(In the picture define $\map{g}{C}{D}$ to follow the arrows backwards for
those $y\in C$ that are at the end of an arrow, and to send the point
to an arbitrary element in $D$ otherwise.
Then applying $f$ followed by~$g$ to elements of $D$ will act as the 
identity.)

By the prior paragraphs, a map has a two-sided inverse 
if and only if that map is both onto and one-to-one. 
Such a function is a 
\definend{correspondence}.\index{correspondence}\index{function!correspondence}
It associates one and only one element of the domain with each element of the
codomain.
Because a composition of one-to-one maps is one-to-one, and a composition
of onto maps is onto, a composition of correspondences is a
correspondence.

We sometimes want to shrink the domain of a function.
For instance, we may take the function \( \map{f}{\Re}{\Re} \) given by
\( f(x)=x^2 \) and, in order to have an inverse, limit input arguments to
nonnegative reals \( \map{\hat{f}}{\Re^+\union\set{0}}{\Re} \).
Then \( \hat{f} \) is 
the \definend{restriction}\index{function!restriction}\index{restriction} of
\( f \) to the smaller domain.








\startword{Relations}
\index{relation}
Some familiar mathematical things, such as `\( < \)' or `\( = \)',
are most naturally understood as relations between things.
A \definend{binary relation} on a set \( A \) is
a set of ordered pairs of elements of \( A \).
For example, some elements of the set that is the
relation `$<$' on the integers are
\( (3,5) \), \( (3,7) \), and \( (1,100) \).
Another binary relation on the integers is equality; this relation is
the set
\( \set{\ldots, (-1,1), (0,0),(1,1),\ldots} \).
Still another example is `closer than \( 10 \)', the set
\( \set{(x,y)\suchthat |x-y|<10 } \).
Some members of this relation are \( (1,10) \), \( (10,1) \),
and \( (42,44) \).
Neither \( (11,1) \) nor \( (1,11) \) is a member.

Those examples illustrate the generality of the definition.
All kinds of relationships (e.g., `both numbers
even' or `first number is the second with the digits reversed')
are covered.




\startword{Equivalence Relations}
\index{relation!equivalence}\index{equivalence relation}\index{equivalence}
We shall need to express that two objects are alike in some way.
They aren't identical, but they are related
(e.g., two integers that give the same remainder when divided by \( 2 \)).

A binary relation \( \set{(a,b),\ldots } \)
is an 
\definend{equivalence relation}\index{equivalence!relation}\index{relation!equivalence} 
when it satisfies
(1)~\definend{reflexivity}:\index{reflexivity, of a relation}\index{relation!reflexive} 
     any object is related to itself,
(2)~\definend{symmetry}:\index{symmetry, of a relation}\index{relation!symmetric} 
     if \( a \) is related to \( b \) then
     \( b \) is related to \( a \), and
(3)~\definend{transitivity}:\index{transitivity, of a relation}\index{relation!transitive}
     if \( a \) is related to \( b \) and \( b \) is
     related to \( c \) then \( a \) is related to \( c \).
Some examples (on the integers): `\( = \)' is an equivalence relation,
`\( < \)' does not satisfy symmetry,
`same sign' is a equivalence, while `nearer than \( 10 \)' fails transitivity.






\startword{Partitions}
\index{partition|(}
In the `same sign' relation \( \set{ (1,3),(-5,-7),(0,0),\ldots} \)
there are three kinds of pairs, pairs with both numbers positive,
pairs with both negative, and the one pair with both zero.
So integers fall into exactly one of three classes, positive, or negative,
or zero.

A \definend{partition}\index{partition} 
of a set \( \Omega \) is a collection of subsets
\( \set{S_0,S_1,S_2,\ldots} \) such that
every element of \( S \) is an element of a subset
\( S_1\union S_2\union \cdots{} = \Omega \) and overlapping parts are equal:
if \( S_i\intersection S_j\neq\emptyset \) then $S_i=S_j$.
Picture that \( \Omega \) is decomposed into non-overlapping parts.
\begin{center}
  \includegraphics{appen.10}
\end{center}
Thus the prior paragraph says that `same sign' partitions
the integers into the set of positives, the set of negatives, and 
the set containing only zero.
Similarly, the equivalence relation `=' partitions the integers into
one-element sets.

Another example is the set of strings consisting of a number, followed
by a slash,
followed by a nonzero number
$\Omega=\set{n/d\suchthat \text{$n,d\in\Z$ and $d\neq 0$}}$.
Define $S_{n,d}$ by: $\hat{n}/\hat{d}\in S_{n,d}$ if 
$\hat{n}d=n\hat{d}$.
Checking that this is a partition of $\Omega$ is routine
(observe for instance that $S_{4,3}=S_{8,6}$).
This shows some parts, listing in each a couple of its infinitely many members. 
\begin{center}
  \includegraphics{appen.11}
\end{center}

Every equivalence relation induces a partition, and every 
partition is induced by an equivalence.
(This is routine to check.)
Below are two examples.
 
Consider the equivalence relationship between two integers of 
`gives the same remainder when divided by \( 2 \)',
the set \( P=\set{ (-1,3),(2,4),(0,0),\ldots} \).
In the set $P$ are two kinds of pairs, the pairs with both members even
and the pairs with both members odd.
This equivalence induces a partition where the parts are found by: 
for each \( x \) we define the set of numbers related to
it \( S_x=\set{y\suchthat (x,y)\in P} \).
The parts are
\( \set{\ldots,-3,-1,1,3,\ldots} \) and
\( \set{\ldots,-2,0,2,4,\ldots} \).
Each part can be named in many ways; for instance,
\( \set{\ldots,-3,-1,1,3,\ldots} \) is \( S_1 \) and also is \( S_{-3} \).

Now consider the partition of the natural numbers where
two numbers are in the same part if they leave the same remainder when
divided by $10$, that is, if they have the same least significant digit.
This partition is induced by the equivalence relation $R$ defined by:
two numbers $n$, $m$ are related if they are together in the same part.
For example, $3$ is related to $33$, but $3$ is not related to $102$.
Verifying the three conditions in the definition of equivalence are 
straightforward.

We call each part of a partition an \definend{equivalence class}.%
\index{equivalence!class}\index{class!equivalence}
We sometimes pick a single element of each equivalence class to be the 
\definend{class representative}.%
\index{equivalence!representative}\index{class!representative}\index{representative!class}
\begin{center}
  \includegraphics{appen.13}
\end{center}
Usually when we pick representatives we have some natural scheme in mind.
In that case we call them the
\definend{canonical} representatives.%
\index{natural representative}\index{canonical representative}%
\index{equivalence!class!canonical representative}%
\index{representative!canonical}
An example is that
two fractions \( 3/5 \) and \( 9/15 \) are equivalent.
In everyday work we often prefer to use the `simplest form' or `reduced form'
fraction $3/5$ as the class representative.
\begin{center}
  \includegraphics{appen.12}
\end{center}
\index{partition|)}
%\end{document}
