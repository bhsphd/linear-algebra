% see: https://groups.google.com/forum/?fromgroups#!topic/comp.text.tex/s6z9Ult_zds
\makeatletter\let\ifGm@compatii\relax\makeatother 
\documentclass[10pt,t]{beamer}
\usefonttheme{professionalfonts}
\usefonttheme{serif}
\PassOptionsToPackage{pdfpagemode=FullScreen}{hyperref}
\PassOptionsToPackage{usenames,dvipsnames}{color}
% \DeclareGraphicsRule{*}{mps}{*}{}
\usepackage{../linalgjh}
\usepackage{present}
\usepackage{xr}\externaldocument{../map5} % read refs from .aux file
\usepackage{xr}\externaldocument{../map4} % 
\usepackage{catchfilebetweentags}
\usepackage{etoolbox} % from http://tex.stackexchange.com/questions/40699/input-only-part-of-a-file-using-catchfilebetweentags-package
\makeatletter
\patchcmd{\CatchFBT@Fin@l}{\endlinechar\m@ne}{}
  {}{\typeout{Unsuccessful patch!}}
\makeatother

\mode<presentation>
{
  \usetheme{boxes}
  \setbeamercovered{invisible}
  \setbeamertemplate{navigation symbols}{} 
}
\addheadbox{filler}{\ }  % create extra space at top of slide 
\hypersetup{colorlinks=true,linkcolor=blue} 

\title[Change of Basis] % (optional, use only with long paper titles)
{Three.V Change of Basis}

\author{\textit{Linear Algebra} \\ {\small Jim Hef{}feron}}
\institute{
  \texttt{http://joshua.smcvt.edu/linearalgebra}
}
\date{}


\subject{Change of Basis}
% This is only inserted into the PDF information catalog. Can be left
% out. 

\begin{document}
\begin{frame}
  \titlepage
\end{frame}

% =============================================
% \begin{frame}{Reduced Echelon Form} 
% \end{frame}



% ..... Three.V.1 .....
\section{Changing representations of vectors}
%..........
\begin{frame}{Coordinates vary with the basis}
Consider this vector $\vec{v}\in\Re^3$ and bases for the space. 
\begin{equation*}
  \vec{v}=\colvec{1 \\ 2 \\ 3}
  \qquad
  \stdbasis_3=\sequence{\colvec{1 \\ 0 \\ 0}, \colvec{0 \\ 1 \\ 0}, \colvec{0 \\ 0 \\ 1}}
  \quad
  B=\sequence{\colvec{1 \\ 1 \\ 0}, \colvec{0 \\ 1 \\ 1}, \colvec{1 \\ 0 \\ 1}}
\end{equation*}
With respect to the different bases, the coordinates of $\vec{v}$ are different.
\begin{equation*}
  \rep{\vec{v}}{\stdbasis_3}=\colvec{1 \\ 2 \\ 3}
  \qquad
  \rep{\vec{v}}{B}=\colvec{0 \\ 2 \\ 1}
\end{equation*}

In this section we will see how to convert the representation
of a vector with respect to a first basis to its representation with 
respect to a second.
\end{frame}


%..........
\begin{frame}{Change of basis matrix}
Think of translating from $\rep{\vec{v}}{B}$ to $\rep{\vec{v}}{D}$
as holding the vector constant. 
This is the arrow diagram.
\ExecuteMetaData[../map5.tex]{ChangeRepresentationOfVectorArrowDiagram}

\pause
\df[df:ChangeOfBasisMatrix]
\ExecuteMetaData[../map5.tex]{df:ChangeOfBasisMatrix}
\end{frame}



\begin{frame}
\lm[le:ChBasisMatDoesChBases]
\ExecuteMetaData[../map5.tex]{lm:ChBasisMatDoesChBases}

\iftoggle{showallproofs}{
  \pf
  \ExecuteMetaData[../map5.tex]{pf:ChBasisMatDoesChBases}
  \qed
}{
  \medskip
  The book has the proof.
}
\end{frame}



\begin{frame}
\ex To change a representation of a member of $\polyspace_2$ 
from being with respect to
$B=\sequence{1,1+x,1+x+x^2}$ to being with respect to
$D=\sequence{x^2-1,x,x^2+1}$, 
Compute $\rep{id}{B,D}$.
The identity map acting on the elements of~$B$ has no effect,
and so represent those elements with respect to~$D$. 
\begin{equation*}
  \rep{1}{D}=\colvec{-1/2 \\ 0 \\ 1/2}
  \quad
  \rep{1+x}{D}=\colvec{-1/2 \\ 1 \\ 1/2}
  \quad
  \rep{1+x+x^2}{D}=\colvec{0 \\ 1 \\ 1}
\end{equation*}
The change of basis matrix is the concatenation of those.
\begin{equation*}
  \rep{id}{B,D}=
  \begin{mat}
    -1/2 &-1/2 &0 \\
     0   &1    &1 \\
    1/2  &1/2  &1 
  \end{mat}
\end{equation*}
For example, we can translate the representation of $\vec{v}=2-x+3x^2$.
\begin{equation*}
  \rep{\vec{v}}{B}=\colvec{3 \\ -4 \\ 3}
  \qquad
  \rep{\vec{v}}{D}
  =\colvec{1/2 \\ -1 \\ 5/2}
  =\begin{mat}
    -1/2 &-1/2 &0 \\
     0   &1    &1 \\
    1/2  &1/2  &1 
  \end{mat}
  \colvec{3 \\ -4 \\ 3}
\end{equation*}
\end{frame}



\begin{frame}
\lm[le:NonSingIsChBasis]
\ExecuteMetaData[../map5.tex]{lm:NonSingIsChBasis}
\iftoggle{showallproofs}{
  \pf
  \ExecuteMetaData[../map5.tex]{pf:NonSingIsChBasis0}

  \pause
  \ExecuteMetaData[../map5.tex]{pf:NonSingIsChBasis1}
}{

  \medskip
  The book contains the proof; we will do an example.
  \medskip
  \ex
  Recall that any nonsingular matrix $M$ 
  decomposes into a product of elementary
  reduction matrices $M=R_1\cdots R_r$. 
  We can show that $M$
  changes bases by showing that each $R_i$ changes bases.
  Consider this $\nbyn{3}$ case.
  \begin{equation*}
    C_{1,3}(-4)\cdot\rep{\vec{v}}{B}
    =
    \begin{mat}
      1 &0 &0 \\
      0 &1 &0 \\
     -4 &0 &1
    \end{mat}
    \colvec{r_1 \\ r_2 \\ r_3}
    =
    \colvec{r_1 \\ r_2 \\ -4r_1+r_3}
  \end{equation*}
  Where $B=\sequence{\vec{\beta}_1,\vec{\beta}_2,\vec{\beta}_3}$,
  the right side represents the same vector~$\vec{v}$ with respect to
  $\hat{B}=\sequence{\vec{\beta}_1+4\vec{\beta}_3,\vec{\beta}_2,\vec{\beta}_3}$.
  \begin{equation*}
    r_1\cdot(\vec{\beta}_1+4\vec{\beta}_3)
      +r_2\cdot\beta_2
      +(-4r_1+r_3)\cdot\vec{\beta}_3
    =r_1\vec{\beta}_1+r_2\beta_2+r_3\vec{\beta}_3
  \end{equation*}
  Verifying that $\hat{B}$ is a basis is routine.
}
\end{frame}
\iftoggle{showallproofs}{
  \begin{frame}
  \ExecuteMetaData[../map5.tex]{pf:NonSingIsChBasis2}
  \begin{equation*}\hspace*{-1.5em}
    M_{i}(k)
      \colvec{
         c_1     \\
         \vdots  \\
         c_i    \\
         \vdots  \\
         c_n  }
    =
      \colvec{
         c_1     \\
         \vdots  \\
         kc_i    \\
         \vdots  \\
         c_n  }
    \quad
    P_{i,j}
      \colvec{
         c_1     \\
         \vdots  \\
         c_i    \\
         \vdots  \\
         c_j    \\
         \vdots  \\
         c_n  }
    =
      \colvec{
         c_1     \\
         \vdots  \\
         c_j    \\
         \vdots  \\
         c_i    \\
         \vdots  \\
         c_n  }
    \quad
    C_{i,j}(k)
      \colvec{
         c_1     \\
         \vdots  \\
         c_i    \\
         \vdots  \\
         c_j    \\
         \vdots  \\
         c_n  }
    =
      \colvec{
         c_1     \\
         \vdots  \\
         c_i    \\
         \vdots  \\
         kc_i+c_j    \\
         \vdots  \\
         c_n  }
  \end{equation*}
  \end{frame}
  \begin{frame}
  \ExecuteMetaData[../map5.tex]{pf:NonSingIsChBasis3}
  
  \pause
  \ExecuteMetaData[../map5.tex]{pf:NonSingIsChBasis4}
  \end{frame}
  \begin{frame}
  \ExecuteMetaData[../map5.tex]{pf:NonSingIsChBasis5}
  \qed
  \end{frame}
}{}


\begin{frame}
\co[co:MatrixNonsingularIffChangesBasis]
\ExecuteMetaData[../map5.tex]{co:MatrixNonsingularIffChangesBasis}
\end{frame}




% ..... Three.V.2 .....
\section{Changing map representations}
%..........
\begin{frame}
The natural next step is to see how to convert $\rep{h}{B,D}$
to $\rep{h}{\hat{B},\hat{D}}$.
Here is the arrow diagram.

\ExecuteMetaData[../map5.tex]{ChangeRepresentationOfMapArrowDiagram}
\end{frame}




\begin{frame}\vspace*{-1ex}
\ex
Consider the derivative map $\map{d/dx}{\polyspace_2}{\polyspace_2}$
as well as
$B=\sequence{1,1+x,1+x+x^2}, D=\sequence{1+x^2,x,1-x^2}$
and 
$\hat{B}=\sequence{1,x,x^2}, \hat{D}=\sequence{1+x,x+x^2,1+x^2}$.

\pause
We can find $H$ and~$\hat{H}$ using the methods we have already seen.
\begin{equation*}
  \rep{d/dx}{B,D}
  =
  \begin{mat}
    0 &1/2 &1/2 \\
    0 &0   &2 \\
    0 &1/2 &1/2
  \end{mat}
  \quad
  \rep{d/dx}{\hat{B},\hat{D}}
  =
  \begin{mat}
    0 &1/2  &1 \\
    0 &-1/2 &1 \\
    0 &1/2  &-1
  \end{mat}
\end{equation*}
These do the base changes.
\begin{equation*}
  \rep{id}{\hat{B},B}=
  \begin{mat}
    1  &-1  &0 \\
    0  &1   &-1 \\
    0  &0   &1
  \end{mat}
  \quad
  \rep{id}{D,\hat{D}}=
  \begin{mat}
    0  &1/2    &1  \\
    0  &1/2    &-1 \\
    1  &-1/2   &0
  \end{mat}
\end{equation*}
We can check this case of equation~($*$) by multiplying through.
\begin{align*}
  \rep{d/dx}{\hat{B},\hat{D}}
  &=\rep{id}{D,\hat{D}}\cdot \rep{d/dx}{B,D}\cdot \rep{id}{\hat{B},B}     \\
  &=\begin{mat}
    0  &1/2    &1  \\
    0  &1/2    &-1 \\
    1  &-1/2   &0
  \end{mat}
  \begin{mat}
    0 &1/2 &1/2 \\
    0 &0   &2 \\
    0 &1/2 &1/2
  \end{mat}
  \begin{mat}
    1  &-1  &0 \\
    0  &1   &-1 \\
    0  &0   &1
  \end{mat}  
\end{align*}
\end{frame}

\begin{frame}
\ex The map $\map{t_{\pi/6}}{\Re^2}{\Re^2}$
rotating vectors counterclockwise by $\pi/6$~radians
has this representation with respect to the standard bases. 
\begin{equation*}
  \rep{t_{\pi/6}}{\stdbasis_2,\stdbasis_2}
  =
  \begin{mat}
    \sqrt{3}/2  &-1/2 \\
    1/2         &\sqrt{3}/2
  \end{mat}
\end{equation*}
We can translate to the representatation with respect to
\begin{equation*}
  B=D=\sequence{\colvec{1 \\ 1},
                \colvec{1 \\ -1}}
\end{equation*}
using the change of basis matrices.
Here is the diagram, specialized for this case.
\begin{equation*}
  \begin{CD}
    \Re^2_{\wrt{\stdbasis_2}}             @>t_{\pi/6}>\rep{t_{\pi/6}}{\stdbasis_2,\stdbasis_2}>  \Re^2_{\wrt{\stdbasis_2}}       \\
    @V{\text{\scriptsize$\identity$}} VV                @V{\text{\scriptsize$\identity$}} VV \\
    \Re^2_{\wrt{B}}                   @>t_{\pi/6}>\hat{H}>        \Re^2_{\wrt{D}}      
  \end{CD}
\end{equation*}
To get $\hat{H}$ we move up from the upper left, across, and then down. 
\end{frame}
\begin{frame}
With respect to the standard basis real vectors represent themselves,
so the matrix representing moving up is easy.
\begin{equation*}
  \rep{id}{B,\stdbasis_2}
  =\begin{mat}
    1 &1 \\
    1 &-1
  \end{mat}
\end{equation*}
The matrix for moving down is the inverse of the prior one.
\begin{equation*}
  \rep{id}{\stdbasis_2,D}
  =(\rep{id}{D,\stdbasis_2})^{-1}
  =
  \begin{mat}
    1 &1 \\
    1 &-1
  \end{mat}^{-1}
  =\frac{-1}{2}\cdot
  \begin{mat}
    -1 &-1 \\
   -1  &1
  \end{mat}
\end{equation*}
Thus we have this.
\begin{align*}
  \rep{t_{\pi/6}}{B,D}
  &=\frac{-1}{2}\cdot
  \begin{mat}
    -1 &-1 \\
   -1  &1
  \end{mat}
  \begin{mat}
    \sqrt{3}/2  &-1/2 \\
    1/2         &\sqrt{3}/2
  \end{mat}
  \begin{mat}
    1 &1 \\
    1 &-1
  \end{mat}                         \\
  &=
  \begin{mat}
    \sqrt{3}/2 &1/2 \\
    -1/2       &\sqrt{3}/2   
  \end{mat}
\end{align*}
\end{frame}




\begin{frame}
\df[def:MatEquiv]
\ExecuteMetaData[../map5.tex]{df:MatEquiv}
\pause
\co[le:MatEqIsSameMap]
\ExecuteMetaData[../map5.tex]{co:MatEqIsSameMap}

\pause
\medskip
\nearbyexercise{exer:MatEqIsEqRel} checks that
matrix equivalence is an equivalence relation.
Thus it  partitions\index{partition!matrix equivalence classes} 
the set of matrices into matrix equivalence classes.
\begin{center}
  \raisebox{.4in}{\begin{tabular}{l}
                    All matrices:
                  \end{tabular}}
  \includegraphics{../ch3.23}
  \raisebox{.4in}{\begin{tabular}{l}
                     $H$ matrix equivalent \\ to $\hat{H}$
                  \end{tabular}}
\end{center}
\end{frame}




\begin{frame}{Canonical form for matrix equivalence}
\th[th:CanonFormForMatEquiv]
\ExecuteMetaData[../map5.tex]{th:CanonFormForMatEquiv}

\pause
\medskip
\ExecuteMetaData[../map5.tex]{BlockPartialIdentityForm}
\end{frame}

\begin{frame}
For an example, 
recall first that any nonsingular matrix $M$ can be factored into
a product 
$M=R_1\cdots R_r$ where each $R_t$ is an
elementary reduction matrix
$C_{i,j}(k)$, or $M_i(k)$, or $P_{i,j}$.    

\pause
Recall also that from the left matrices operate on rows
\begin{equation*}
  % C_{1,3}(-4)\cdot
  % \begin{mat}
  %   1 &2 &3 \\
  %   4 &5 &6 \\
  %   7 &8 &9
  % \end{mat}
  % =
  \begin{mat}
    1 &0 &0 \\
    0 &1 &0 \\
   -4 &0 &1 
  \end{mat}
  \begin{mat}
    1 &2 &3 \\
    4 &5 &6 \\
    7 &8 &9
  \end{mat}
  =
  \begin{mat}
    1 &2 &3 \\
    4 &5 &6 \\
    3 &0 &-3
  \end{mat}
\end{equation*}
while from the right they act on columns: this
does $-4\cdot\text{col}_3+\text{col}_1$.
\begin{equation*}
  \begin{mat}
    1 &2 &3 \\
    4 &5 &6 \\
    7 &8 &9
  \end{mat}
  \begin{mat}
    1 &0 &0 \\
    0 &1 &0 \\
   -4 &0 &1 
  \end{mat}
  =
  \begin{mat}
    -11 &2 &3 \\
    -20 &5 &6 \\
    -29 &8 &9
  \end{mat}
\end{equation*}

\pause
\ex This matrix has rank~$2$.
\begin{equation*}
  \begin{mat}
    1 &2 &3 \\
    4 &5 &6 \\
    7 &8 &9
  \end{mat}
\end{equation*}
We will find $P$ and~$Q$ to get the $\nbyn{3}$ canonical matrix of
rank~$2$.
\end{frame}
\begin{frame}
\noindent Row operations produce echelon form.
\begin{equation*}
  \begin{mat}
    1 &2 &3 \\
    4 &5 &6 \\
    7 &8 &9
  \end{mat}
  \grstep[-7\rho_1+\rho_3]{-4\rho_1+\rho_2}
  \grstep{-2\rho_2+\rho_3}  
  \grstep{-(1/3)\rho_2}
  \begin{mat}
    1 &2 &3 \\
    0 &1 &2 \\
    0 &0 &0
  \end{mat}
\end{equation*}
Multiplying the reduction matrices gives $P$; note the right~to~left
order.
\begin{multline*}
  \begin{mat}
    1 &0 &0 \\
    0 &-1/3 &0 \\
    0 &0 &1
  \end{mat}
  \begin{mat}
    1 &0 &0 \\
    0 &1 &0 \\
    0 &-2 &1
  \end{mat}
  \begin{mat}
    1 &0 &0 \\
    0 &1 &0 \\
    -7 &0 &1
  \end{mat}
  \begin{mat}
    1 &0 &0 \\
    -4 &1 &0 \\
    0 &0 &1
  \end{mat}                    \\
  =\begin{mat}
      1   &0    &0 \\
      4/3 &-1/3 &0 \\
      1   &-2   &1
   \end{mat}
\end{multline*}
So far we have this.
\begin{equation*}
  PHQ
  =\begin{mat}
      1   &0    &0 \\
      4/3 &-1/3 &0 \\
      1   &-2   &1
   \end{mat}
  \begin{mat}
    1 &2 &3 \\
    4 &5 &6 \\
    7 &8 &9
  \end{mat}
  Q
  =
  \begin{mat}
    1 &2 &3 \\
    0 &1 &2 \\
    0 &0 &0
  \end{mat}
  Q
\end{equation*}
\end{frame}
\begin{frame}
Column operations produce the block partial identity.
\begin{equation*}
  \begin{mat}
    1 &2 &3 \\
    0 &1 &2 \\
    0 &0 &0
  \end{mat}
  \grstep{-2\text{col}_2+\text{col}_3}
  \grstep{-2\text{col}_1+\text{col}_2}
  \grstep{\text{col}_1+\text{col}_3}
  \begin{mat}
    1 &0 &0 \\
    0 &1 &0 \\
    0 &0 &0
  \end{mat}    
\end{equation*}
Combine the reduction matrices to get $Q$.
In contrast with the construction of~$P$, here they come
left to~right.
\begin{equation*}
  \begin{mat}
    1 &0 &0 \\
    0 &1 &-2 \\
    0 &0 &1
  \end{mat}
  \begin{mat}
    1 &-2 &0 \\
    0 &1 &0 \\
    0 &0 &1
  \end{mat}
  \begin{mat}
    1 &0 &1 \\
    0 &1 &0 \\
    0 &0 &1
  \end{mat}
  =
  \begin{mat}
    1 &-2 &1 \\
    0 &1 &-2 \\
    0 &0 &1
  \end{mat}
\end{equation*}
In sum, we have this equation.
\begin{equation*}
  PHQ
  =\begin{mat}
      1   &0    &0 \\
      4/3 &-1/3 &0 \\
      1   &-2   &1
   \end{mat}
  \begin{mat}
    1 &2 &3 \\
    4 &5 &6 \\
    7 &8 &9
  \end{mat}
  \begin{mat}
    1 &-2 &1 \\
    0 &1 &-2 \\
    0 &0 &1
  \end{mat}
  =
  \begin{mat}
    1 &0 &0 \\
    0 &1 &0 \\
    0 &0 &0
  \end{mat}    
\end{equation*}  
\end{frame}
\iftoggle{showallproofs}{
  \begin{frame}
  \pf
  \ExecuteMetaData[../map5.tex]{pf:CanonFormForMatEquiv}
  \qed
  \end{frame}
}{}
\begin{frame}{Action of a canonical form matrix}
This kind of matrix is has an 
easy to understand effect.
\begin{equation*}
  \begin{mat}
    1 &0 &0 \\
    0 &1 &0 \\
    0 &0 &0
  \end{mat}
  \colvec{x \\ y \\ z}
  =
  \colvec{x \\ y \\ 0}
\end{equation*}
It is a projection\Dash
the map $\map{t}{\Re^3}{\Re^3}$ represented with respect
to the standard bases by this block partial identity matrix 
is projection from three~space to the $xy$~plane.
\end{frame}


\begin{frame}{Matrix equivalence is characterized by rank}
\co[co:MatrixEquivalentIffSameRank]
\ExecuteMetaData[../map5.tex]{co:MatrixEquivalentIffSameRank}
\pause
\pf
\ExecuteMetaData[../map5.tex]{pf:MatrixEquivalentIffSameRank}
\qed

\pause
\ex
These two matrices are not matrix equivalent 
because Gauss's Method shows
that the first has rank~$3$ while the second has rank~$2$.
\begin{equation*}
  \begin{mat}[r]
    2  &3  &0 &-1 \\
    2  &2  &1 &1  \\
    3  &1  &0 &3
  \end{mat}
  \quad
  \begin{mat}[r]
    1  &5  &1 &4 \\
    2  &0  &5 &1  \\
    3  &-5 &9 &-2
  \end{mat}
\end{equation*}
\ex
These two are matrix equivalent.
\begin{equation*}
  \begin{mat}[r]
    1  &1  &1  &1  &1  \\
   -1  &-1 &-1 &-1 &-1 \\
    2  &0  &0  &3  &0 
  \end{mat}
  \quad
  \begin{mat}
    1  &0  &0  &0  &0  \\
    0  &1  &0  &0  &0  \\
    0  &0  &0  &0  &0
  \end{mat}
\end{equation*}
\end{frame}




%...........................
% \begin{frame}
% \ExecuteMetaData[../gr3.tex]{GaussJordanReduction}
% \df[def:RedEchForm]
% 
% \end{frame}
\end{document}
